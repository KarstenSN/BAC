\chapter{Konklusion}
\label{ch:Konklusion}

Det er i projektafgrænsningen blevet bestemt at der ved projektets afslutning skulle stå en færdig prototype af 1. iteration klar. Denne prototype skal demonstrere kerne-funktionaliteten af systemet og et designkoncept der uden omfattende ændringer skal kunne masseproduceres. Ved projektets afslutning står der en prototype af 1. iteration klar som demonstrerer \textit{Must}-kravene til funktionaliteten, se Afsnit \ref{ch:Krav}: \nameref{ch:Krav} på side \pageref{ch:Krav} i rapporten eller Kapitel \ref{P-ch:kravspecifikation}: \nameref{P-ch:kravspecifikation} på side \pageref{P-ch:kravspecifikation} i dokumentationen. Dog er krav CM\_03 om en rækkevidde på 30m mellem sensor og kontrolboks ikke opfyldt da rækkevidden ved Accepttesten (Kapitel \ref{P-ch:Accepttest}: \nameref{P-ch:Accepttest} på side \pageref{P-ch:Accepttest} i Projektdokumentationen) kun strakte sig til lidt over 10m, se Afsnit \ref{hwdesign_hftransceiver} på side \pageref{hwdesign_hftransceiver} her i Rapporten. Selve designet af blokken til jordfugtmålingen har krævet en del research og der blev fundet frem til at en kapacitiv-målemetode var den bedste. Det blev forsøgt at lave en fasemåling af denne kapacitets ændring men kredsløbet viste sig at give en del udfordringer som der, i sidste ende desværre ikke var tid nok til at løse. Der blev derfor implementeret en mere simpel løsning, som dog er følsom over for støj og temperaturændringer. PSU'en på sensoren er implementeret med en boostkonverter som aktiveres når batterispændingen når under 2.8V. Herefter får kontrolboksen besked om at der er lavt batterispænding på sensoren. Dette levner dog stadig er del spænding tilbage på batteriet, denne udfordring tages der hånd om i iteration 2 med et redesign af batterimålingen. Masseproduktion af systemet vil kræve at gældende CE-krav skal overholdes. Dette er der endnu ikke blevet testet for. Dog er der foretaget en del designmæssige overvejelser som tilgodeser disse tests. En markedsundersøgelse har vist at systemet i dets nuværende form er for dyrt at producere set i forhold til en eventuel fortjeneste. Der ligger derfor stadig en del arbejde i systemet hvis det skal ende som et salgsklart produkt. Men de overordnede mål og krav for dette projekt blev opfyldt.   