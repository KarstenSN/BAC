\chapter{Konklusion}
\label{ch:Konklusion}

Det er i projektafgrænsningen blevet bestemt at der ved projektets afslutning skulle stå en færdig prototype klar. Denne prototype skal demonstrere kerne-funktionaliteten af systemet, samt et designkoncept der uden omfattende ændringer skal kunne masseproduceres. Ved projektets afslutning står der en prototype klar som demonstrerer \textit{Must}-kravene til funktionaliteten, se Afsnit \ref{ch:Krav}: \nameref{ch:Krav} på side \pageref{ch:Krav} i rapporten eller Kapitel \ref{P-ch:kravspecifikation}: \nameref{P-ch:kravspecifikation} på side \pageref{P-ch:kravspecifikation} i projektdokumentationen. Dog er krav CM\_03 om en rækkevidde på 30m mellem sensor og kontrolboks ikke opfyldt da rækkevidden ved Accepttesten (Kapitel \ref{P-ch:Accepttest}: \nameref{P-ch:Accepttest} på side \pageref{P-ch:Accepttest} i Projektdokumentationen) kun strakte sig til lidt over 10m, se Afsnit \ref{hwdesign_hftransceiver} på side \pageref{hwdesign_hftransceiver} her i Rapporten. Selve designet af blokken til jordfugtmålingen har krævet en del research og der blev fundet frem til at en kapacitiv-målemetode var den bedste. Der blev forsøgt at lave fasemåling af denne kapacitetsændring, men kredsløbet viste sig at give en del udfordringer som der, i sidste ende grundet tidspres ikke kunne løses. Det blev derfor valgt at implementeret en mere simpel løsning. Denne er dog følsom over for støj og temperaturændringer. Derfor er en færdigudvikling af fasemålingskredsløbet planlagt til implementering i 2. iteration og er estimeret til ekstra 40 ingeniørtimers arbejdes. PSU'en på sensoren er implementeret med en boostkonverter som aktiveres når batterispændingen når under 2.8V. Herefter får kontrolboksen besked om at der er lavt batterispænding på sensoren. Der er ved denne spænding dog stadig en del strøm tilbage på batteriet og dette interrupt ønskes først ved en lavere spænding, da dette vil udnytte batteriet bedre og sikre længere levetid. \newline

I forhold til at optimere nuværende komponenter, kan der med fordel tages fat i transceiveren. Der er fundet frem til at Texas Instruments har udviklet en dualband Sub-1-GHz og bluetooth transceiver CC1350 som ville kunne bruges til dette formål. Da transceiveren er dualband kan systemet drage nytte af alle fordelene ved smartphone-integration,  samt bevare stor rækkevidde imellem kontrolboks og sensoren. Transceiveren indeholder også en DC/DC boostconverter, temperaturføler, ADC samt WDT. Dette gør at der kan spares en del omkostninger på eksterne komponenter og da betjeningen sker via en smartphone, kan omkostninger til display og tastatur helt undgås. Det er derfor nødvendigt at redesigne dele af det nuværende system. Dette forventes at tage 320 ingeniørtimer, eller ca. 2 måneders arbejde. \newline

Masseproduktion af systemet vil kræve at gældende CE-krav skal være overholdt. Dette er der endnu ikke blevet testet for, dog er der foretaget en del designmæssige overvejelser som tilgodeser disse senere tests. En markedsundersøgelse har vist at systemet i dets nuværende form er for dyrt at producere set i forhold til en eventuel fortjeneste. Der ligger derfor stadig research-arbejde hvis systemet skal ende som et forretningsmæssigt bæredygtigt produkt. Men de overordnede mål og krav for dette projekt blev opfyldt.

Et estimat for udvikling af 2. iteration baserer sig derfor på de 40 timer til udvikling af jordfugtmåleren, samt 320 timer til redesign af dele af systemet i forhold til ovenstående.   