\chapter{Konklusion}
\label{ch:Konklusion}

Det var i projektafgrænsningen blevet bestemt at der ved projektets afslutning skulle stå en færdig prototype klar. Denne prototype skulle demonstrere systemet fuldendt og skulle uden større problemer kunne masseproduceres. Ved projektets afslutning var der en prototype klar som demonstrerede systemet fuldendt. Dog var kravet om 30m rækkevidde mellem sensor og kontrolboks ikke opfyldt, da rækkevidden kun strakte sig til lidt over 10m. Selve designet af blokken til målingen af jordfugtigheden havde krævet en del research og der blev fundet frem til en kapacitiv-målemetode var den bedste. Det blev forsøgt at lave en fasemåling af denne kapacitets ændring men kredsløbet viste sig at give en del problemer som der ikke var tid til at løse. Der blev derfor implementeret en simpel løsning som dog er meget støjfølsom og temperaturvarierende. Strømforsyningen på sensoren blev implementeret med en boostconverter som først aktiveres når battterispændingen når under 2.8V. Herefter får kontrolboksen besked om at der er lavt batterispænding på sensoren. Dette er dog ikke rigtigt da der stadig er en del strøm tilbage på batteriet. At masseproducere systemet vil kræve at systemet overholder kravende til CE-mærkning. Dette er der ikke blevet testet for. Dog er der blevet gjort nogle designmæssige overvejelser som gør at det ikke burde være umuligt at bestå sådan en test. En markedsundersøgelse har vist at systemet i dets nuværende form er for dyr at producere i forhold til hvad en eventuel fortjeneste vil være. Der ligger derfor stadig en del arbejde i systemet hvis det skal ende som et salgsklart produkt. Men de overordnede mål for dette projekt blev opfyldt.   