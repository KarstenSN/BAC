\chapter{Resumé} \label{ch:Resume}
Dette projekt er et afgangsprojekt for to ingeniørstuderende på Ingeniørhøjskolen i Århus. Projektet omhandler et automatisk vandingssystem som består af en sensor og en kontrolboks og som er tiltænkt at blive brugt af private personer med interesse i havearbejde. Systemet virker ved at der er en sensor placeret i jorden ved et gromedie. Sensoren måler overfladetemperatur, jordfugtighed og lysintensitet. Disse data sendes via en trådløs forbindelse til kontrolboksen, som opretholder en ønsket fugtighed i jorden, ved at åbne eller lukke for en indbygget ventil. Brugeren tilslutter kontrolboksen til en vandhane i den ene ende og i den anden ende tilsluttes en haveslange med en spreder efter brugerens eget valg. Brugeren har mulighed for at vælge hvilken fugtighed der skal opretholdes og om der skal vandes morgen, aften eller hele dagen. \\

Projektet har været udviklet helt fra bunden og der er igennem udviklingen blevet fremkaldt flere PCB's til opbygning af sensor og kontrolboks. En af hoved punkterne i projektet var at udvikle en metode til at måle jordfugt. Her blev der forsøgt med en resistiv-målemetode som viste sig at være meget upræcis. Herefter blev der førsøgt at udvikle en kapacitiv-målemetode. Denne viste sig at have gode egenskaber, men der var desværre problemer med at designe et kredsløb til at måle variationen af kapaciteten. Tilsidst pga. tidsmangel blev der implementeret et simpelt kredsløb til måling af denne, som dog er meget følsom over for temperatur-variationer. Et andet hovedpunkt var at implementere en løsning til at kommunikere mellem sensor og kontrolboks. Dette blev klaret med en 433MHz transceiver som viste sig velegnet til batteri-applikationer. Ved projektets afslutning var der et næsten fuldt funktionelt prototype klar. Der var kun nogle enkelte mindre væsentlige funktionaliteter som manglede. 
