\chapter{Resumé} \label{ch:Resume}
Dette projekt er et afgangsprojekt for to ingeniørstuderende ved Ingeniørhøjskolen i Århus. Projektet omhandler et automatisk vandingssystem bestående af en sensor og en kontrolboks og er tiltænkt private personer med interesse i havearbejde. Systemet virker ved at en sensor placeres i jorden hvor der ønskes vanding. Sensoren måler overfladetemperatur, jordfugtighed og lysintensitet. Disse data transmitteres via en trådløs forbindelse til kontrolboksen, som opretholder en ønsket fugtighed i jorden ved at åbne eller lukke for en indbygget ventil. Brugeren tilslutter kontrolboksen til en vandhane i den ene ende, og i den anden tilsluttes en haveslange med en spreder efter brugerens eget valg. Brugeren har mulighed for at vælge hvilken fugtighed der skal opretholdes og om der skal vandes morgen, aften eller hele dagen. \newline

Projektet er udviklet helt fra bunden og der er igennem udviklingen blevet fremstillet flere PCB's til opbygning af hhv. sensor og kontrolboks. Èt af hovedpunkterne i projektet har været at udvikle en metode til at måle jordfugtighed. Her blev der forsøgt med en resistiv-målemetode som viste sig at være meget upræcis. Herefter blev der forsøgt at udvikle en kapacitiv-målemetode. Denne viste sig at have gode måleegenskaber, men der var desværre en del udfordringer med at designe et velfungerende og stabilt målekredsløb til at måle variationen i kapaciteten. Tilsidst blev der pga. tidsmangel implementeret et simpelt kredsløb til måling af kapaciteten, som dog er meget følsom over for temperatur-variationer. Et andet hovedpunkt var at implementere en løsning til at kommunikere mellem sensor og kontrolboks. Dette blev klaret med en 433MHz transceiver som viste sig velegnet til batteri-applikationer. Ved projektets afslutning er der et næsten fuldt funktionelt prototype klar. Der er kun nogle enkelte mindre væsentlige funktionaliteter som mangler til iteration.