\chapter{Projektformulering}  \label{ch:projektformulering}
Som befolkningen vokser bliver det nødvendigt at bruge jordens ressourcer mere effektivt. Drikkevand er en ressource som i nogle områder er truet pga. forurening og overforbrug. Ved kun at bruge den nødvendige mængde af de nuværende drikkevandsressourcer vil der blive mere vand til vores efterkommer og vi vil have større mulighed for stadig at sikre rent drikkevand i hanerne. I Danmark vandes der i mange haver med rent drikkevand og her er der en mulighed for at kunne optimere brugen og samtidig sikre optimale grobetingelser for planterne. Ved at installere et automatisk fugtbaseret havevandingsanlæg sikres der at der ikke overvandes men samtidig også at planterne ikke mangler vand. Det automatiske fugtbaseret havevandingsanlæg består samlet af en sensor, en kontrolboks og en motorventil.
Systemet virker ved at sensoren er placeret ved et gromedie og sender et trådløst signal indeholdende jordfugtigheden samt en overfladetemperatur, batteristatus og lysintensitet til kontrolboksen. Kontrolboksen åbner eller lukker herefter for den givne ventil afhængig af hvilken fugtighed kontrolboksen er præindstillet til og brugeren har mulighed for at aflæse fugtigheden og temperatur på kontrolboksen. Brugeren skal selv tilslutte en haveslange til ventilen og kan derfor selv bestemme hvilken type af havevander der skal tilkobles. En anden stor fordel ved systemet er at brugeren nu også har mulighed for at være væk fra sin bopæl eller fritidshus igennem længere tid, uden at skulle have andre til at tilse sin have.  

\section{Projektafgrænsning} \label{sec:projektafgraensning}

Målet for dette projekt er at ende ud med en prototype som demonstrer systemet. Prototypen skal overholde alle must-kravende fra tabel \ref{tbl:funk_krav} i sektion \ref{ch:Krav}. Er der tid til overs skal det forsøges at implementere alle should-kravende. Prototypen skal designes således at det vil være muligt at lave en masseproduktion uden at skulle designe systemet forfra. Det vil sige at der skal tænkes produktionsomkostninger med under udviklingen. Til slut skal der laves en acceptest som demonstrer at kravende er opfyldt.  

