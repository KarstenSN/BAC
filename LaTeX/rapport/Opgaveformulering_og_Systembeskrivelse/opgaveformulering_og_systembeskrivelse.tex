\chapter{Projektformulering}  \label{ch:projektformulering}
Som befolkningen vokser bliver det nødvendigt at bruge jordens ressourcer mere effektivt. Drikkevand er en ressource som i nogle områder er truet pga. forurening og overforbrug. Ved kun at bruge den nødvendige mængde af de nuværende drikkevandsressourcer vil der blive mere vand til vores efterkommer og vi vil have større mulighed for stadig at sikre rent drikkevand i hanerne. I Danmark vandes der i mange haver med rent drikkevand og her er der en mulighed for at kunne optimere brugen og samtidig sikre optimale grobetingelser for planterne. Ved at installere et automatisk fugtbaseret havevandingsanlæg sikres der at der ikke overvandes men samtidig også at planterne ikke mangler vand. Det automatiske fugtbaseret havevandingsanlæg består samlet af en sensor, en kontrolboks og en motorventil.
Systemet virker ved at sensoren er placeret ved et gromedie og sender et trådløst signal indeholdende jordfugtigheden samt en overfladetemperatur, batteristatus og lysintensitet til kontrolboksen. Kontrolboksen åbner eller lukker herefter for den givne ventil afhængig af hvilken fugtighed kontrolboksen er præindstillet til og brugeren har mulighed for at aflæse fugtigheden og temperatur på kontrolboksen. Brugeren skal selv tilslutte en haveslange til ventilen og kan derfor også selv bestemme hvilken type af havevander der skal tilkobles. En anden stor fordel ved systemet er at brugeren nu også har mulighed for at være væk fra sin bopæl eller fritidshus igennem længere tid, uden at skulle have andre til at tilse sin have.  

\section{Projektafgrænsning} \label{sec:projektafgraensning}

Målet for dette afgangsprojekt har fra dag 1 været at ende med ud et produkt egnet til privat-salg. Derfor er der naturligt en afgrænsning i forbindelse med hvor meget der kan tages med indenfor den tidsramme der er blevet stillet. Det arbejde og de overvejelser der ligger forude er beskrevet i Afsnit \ref{ch:Fremtidigt_arbejde}: \nameref{ch:Fremtidigt_arbejde} på side \pageref{ch:Fremtidigt_arbejde}. Afgrænsning af projektet har i første omgang baseret sig på at kunne stå med 1. iteration af en færdig prototype der opfylder alle \textit{Must-kravene} fra tabel \ref{tbl:funk_krav} i afsnit \ref{ch:Krav} på side \pageref{ch:Krav}. Der har derfor været fuld fokus på disse krav og dermed er valget også taget om, at de resterende funktionelle krav ville blive implementeret hvis der var tid og ressourcer til det. Der har som udgangspunkt også været en afgrænsning omkring de salgsrelaterede overvejelser og hele den forretningsmæssig del af et salgsprojekt, da fokus er holdt på udvikling af selve produktet og dokumentation heraf. Dog er der foretaget en markedsundersøgelse for at danne sig et billede af hvad produktet er oppe imod. For mere info henvises til Afsnit \ref{sec:marked}: \nameref{sec:marked} på side \pageref{sec:marked}. I forbindelse med udvikling er der fra starten tænkt high-volumen produkt ind i designet, således at det vil kunne egne sig til masseproduktion. Disse overvejelser er uddybet i Kapitel \ref{ch:produktion} under Afsnit \ref{sec:produktion} \nameref{sec:produktion} på side \pageref{sec:produktion}.

\clearpage