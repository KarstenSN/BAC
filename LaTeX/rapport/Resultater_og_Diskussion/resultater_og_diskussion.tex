\chapter{Resultater} \label{ch:resultater}

Målet for dette projekt har været at ende ud med 1. iteration af en funktionsdygtig prototype der opfylder de opstille \textit{Must}-krav i 
afsnit \ref{P-sec:funktionelle_krav}: \nameref{P-sec:funktionelle_krav} og \ref{P-sec:ikke_funktionelle_krav}: \nameref{P-sec:ikke_funktionelle_krav} på side \pageref{P-sec:funktionelle_krav}  og \pageref{P-sec:ikke_funktionelle_krav} i Projektdokumentationen. Langt de fleste krav blev opfyldt, men der er dog enkelt som ikke blev opfyldt, hovedsageligt grundet mangel på tid til at realisere alle funktionaliteter.
En detaljeret gennemgang af testforløbet til disse krav kan ses i Kapitel \ref{P-ch:Accepttest}: \nameref{P-ch:Accepttest} på side \pageref{P-ch:Accepttest} projektdokumentationen.

Den primære udfordring i udførelsen af dette projekt har været at udtænke, designe og redesigne kredsløbet til måling af jordfugt. Der viste sig at være store udfordring med at få kredsløbet til at give en stabil og korrekt måling af variationen af målekapaciteten. Det blev prøvet adskillige løsninger som ses dokumentet i afsnit \ref{P-sec:HWimp_jordfugt}: \nameref{P-sec:HWimp_jordfugt} på side \pageref{P-sec:HWimp_jordfugt} i Projektdokumentationen.


% indsæt /implementering/hw/sensor/jordfugt

\clearpage