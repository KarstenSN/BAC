\section{Hardware Design og Implementering} \label{ch:HWdesign}

På Figur \ref{fig:bddsystem} ses et blokdiagram over systemet. Det ses at systemet indeholder en kontrolboks og en sensor. Det ses også hvad kontrolboksen og sensoren indeholder af hardware. 

\begin{figure}[H]
  \centering
    \includegraphics[width=0.7\textwidth]{../fig/Systemarkitektur/BDDsystem.pdf}
    \caption{BDD over systemet}
    \label{fig:bddsystem}
\end{figure}

\subsection{Kontrolboks}

\begin{figure}[H]
  \centering
    \includegraphics[width=1\textwidth]{../fig/Systemarkitektur/BDDkontrolboks.pdf}
    \caption{BDD over kontrolboksen}
    \label{fig:bddkontrol}
\end{figure}

På Figur \ref{fig:bddkontrol} ses en uddybning af de blokke der tilsammen udgør kontrolboksen. Herunder beskrives hvordan disse enkelte blokke er blevet implementeret. For uddybning af designovervejelserne henvises til sektion \ref{P-sec:hardwaredesign}: \nameref{P-sec:hardwaredesign} på side \pageref{P-sec:hardwaredesign} i projektdokumentationen.

\subsubsection{Design}



\textbf{Mikroprocessor}\\
Mikroprocessoren til kontrolboksen er valgt til at være en Atmel ATMEGA8L i TQFP pakke. Samme mikroprocessor er valgt til sensoren. Valget stod mellem Microchip's PIC serie eller Atmels ATMEGA serie og da der var størst kendskab til programmering af Atmel var det den producent der blev valgt. Fordelen ved Atmel er samtidig at programmet til at programmere mikroprocessoren i, AtmelStudio, er gratis og at der er nogle gode biblioteker som er godt dokumenteret. Microchip kan derimod godt være svære at gennemskue med hensyn til hvad der kræves betaling for.  

Af perifere enheder på kontrolboksen bruges der SPI til transceiveren, I2C til displayet samt en del analoge pins til kontrol af motorventil og tastatur. På sensoren bruges der også den indbyggede 10-bit SAR ADC. 

\textbf{Motorventil}\\
Der blev hos den kinesiske producent Flow-Control bestilt nogle motorventiler hjem. Disse blev i første omgang valgt grundet deres lave pris, omkring 150kr stykket og har en tilslutning som fylder kravet om den skal kunne monteres på et 1/2" vandrør. Ventilen blev adskilt og det sås at der sat et lille print med et relæ til styring af ventilen. Styringen var lavet således at når den enten var i fuldt lukket eller åben position trak den intet strøm.  \ref{fig:motorventilSamlet} ses motorventilen som den kommer fra producenten. På Figur \ref{fig:delt} ses motorventilen adskilt, hvor de grønne ledninger er til posotions-kontakterne og de røde er til motoren.

\begin{figure}[H]
  \centering
  \begin{minipage}[b]{0.4\textwidth}
    \includegraphics[width=1\textwidth]{../fig/Motorventil/samlet.JPG}
    \caption{Motorventil}
    \label{fig:motorventilSamlet}
  \end{minipage}
  \hfill
  \begin{minipage}[b]{0.385\textwidth}
    \includegraphics[width=1\textwidth]{../fig/Motorventil/delt.JPG}
    \caption{Motorventil adskilt}
    \label{fig:delt}
  \end{minipage}
\end{figure}

Der blev dog i første omgang designet en relæstyring af ventilen. Men den viste sig at være dyr at producere samtidig med at relæet træk en del strøm når det var sluttet. Herefter blev der designet en H-bro som bruger minimal tomgangsstrøm.
 
\begin{figure}[ht]
  \centering
    \includegraphics[width=0.4\textwidth]{../fig/Motorventil/hbro.png}
    \caption{Diagram over H-bro'en}
    \label{fig:hbro}
\end{figure}

\textbf{Display}\\
Af krav til displayet var at det skulle køre på 3.3V, kommunikere over I2C bus og have 16X2 karakterer og have baggrundsbelysning. Ved undersøgelse af flere displays blev Newheaven's display \textit{NHD-C0216CiZ-FSW-FBW-3V3} udvalgt. Dette skyldes primært at display'et har indbygget DC-DC step-up converter således at displayet kan forsynes ned til 2.8V. Normalt kører LCD displays på 5V, hvilket er over forsyningskravet på 3.3V. Opsætning og kommunikation sker via I2C som kravet foreskriver. Diagrammet for display'et ses på Figur \ref{fig:display_diagram}.

\begin{figure}[ht]
	\centering
	\includegraphics[width=0.7\textwidth]{../fig/display/display_diagram_kontrol2.png}
	\caption{Diagram over I2C Display}
	\label{fig:display_diagram}
\end{figure}

D11 sidder for at sænke forsyningsspændingen til omkring 3V da forsyningen ikke må overskride 3.3V. C22 sidde over pin5 og pin6 som udgangskondesator for Step-up konverteren, C23 sidder over pin7 og pin8 som en integreret del af step-up konverteren og disse værdier er valgt ud fra databladet anbefalinger. Derudover benyttes Q13 for at kontrollere LED'en til baggrundsbelysningen hvormed dette kan styres fra software. SDA og SCL linjerne har pull-up modstande påsat som en del af standardopsætning for I2C-bussen.

\textbf{HF Transceiver}\\
Der er valgt at benytte en transceiveren MRFX49A fra Microchip, denne er valgt da den opfylder designkravene. Derudover skal denne kreds ikke forprogrammeres som mange andre RF-kredse skal, den kan derimod sættes op ved initialisering fra mikroprocessoren som krævet af systemet. Chippen er som skræddersyet til dette system da den er designet til batteri-applikationer. Den har en indbygget "low battery voltage detection" som afgiver et interrupt til mikroprocessoren når batterispændingen når under et forudindstillet threshold som sættes internt i transceiveren. Dette interrupt bliver brugt på sensoren til at enable boostkonverteren, mere info kan finde i implementeringsafsnittet for transceiveren på sensoren i dokumentationen. Derudover har den også en indbygget wake-up timer således at chippen kan sættes i sleep-mode, herved minimeres strømforbruget til nogle få $\mu A$, den kan herefter vækkes op igen efter en forudindstillet tidsperiode. Transceiveren kan kommunikere ved 433/868/915 MHz og ligger derfor i et frekvens område som ofte benyttes i consumer-elektronik. Transceiveren benytter sig af FSK modulation.

På Figur \ref{fig:mrfx49a_diagram_kontrolboks} ses implementering af transceiveren på kontrolboksen.

\begin{figure}[H]
	\centering
	\includegraphics[width=0.8\textwidth]{../fig/transciver/mrfx49a_diagream2_kontrolboks.PNG}
	\caption{Diagram af MRF49XA på kontrolboksen}
	\label{fig:mrfx49a_diagram_kontrolboks}
\end{figure}

Som udgangspunkt er databladets anbefalinger fulgt. Balun-kredsløbet er designet på baggrund af referencedesignet i databladet. Her er det vigtigt at holde sig til at lavtolerance-komponenter for at holde sig tæt på de 50 $\Omega$ udgangsimpedans som det kræves af antennen. Det skal dog nævnes at der i databladet er byttet om på værdierne til balun-kredsløbet ved 433MHz og 868MHz i tabellerne under diagrammerne. Det er derfor svært at vide hvilke værdier der er de rigtige. Fejlen blev først opdaget tilsidst i projektet da rækkevidden ikke kunne komme længere op end ca 10m. Det er derfor nærliggende at tro det er de forkerte værdier som er blevet monteret.
\newpage

\textbf{Antenne}\\
Der er på kontrolboksen valgt at bruge en SMD-antenne. Dette skyldes at der var begrænset plads i kontrolboksen og at andre løsninger enten ville være for dyre eller mekanisk ustabil. Her var der overvejet at trække en metalwire rundt i hele kontrolboksen som antenne eller at lave et ekstra print med en PCB-antenne. SMD-antennen er af producenten \texttt{Johanson Technology} med  navn \texttt{0433AT62A0020}. 

\textbf{Strømforsyning}\\
Kontrolboksen forsynes med en ekstern strømforsyning på 5V denne har en tolerence på 10\%. Da kontrolboksen internt skal køre på 3.3V er derfor nødvendigt at konvertere spændingen ned. Til dette er brugt en spændingsregulator. Samtidig er der lavet beskyttelse på indgangen således der beskyttes for omvendt polarisation, overspænding samt et for stort strømtræk. Se Figur \ref{fig:stromforsyning_digram1} 

\begin{figure}[H]
	\centering
	\includegraphics[width=0.8\textwidth]{../fig/stromforsyning/stromforsyning_diagram1.png}
	\caption{Diagram over strømforsyning}
	\label{fig:stromforsyning_digram1}
\end{figure}  

Spændingsregulatoren kan levere op til 500mA hvilket er mere end hvad der er brug for. Allerede sikringen vil sprænge ved et strømforbrug på 250mA. Beregninger som findes i projektdokummentationen i afsnittet strømforsyning på kontrolboks, viser at der trækkes maksimalt omkring 40mA. Spændingsregulatoren kan derfor vælges til en mindre type som samtidig vil være billigere. 

\textbf{Tastatur}\\
Systemet betjenes via et tastatur der monteres på kontrolboksen. Tastaturet består af 4 knapper som brugeren kan benytte til at navigere rundt i menuen. En sketchup af tastaturet kan ses på Figur \ref{fig:keypad_UI}. \texttt{SET} knappen benyttes til at skifte mellem menupunkterne. \texttt{Pil-op} og \texttt{Pil-ned} benyttes hhv.til at inkrementere eller dekrementere en given værdi. \texttt{OK} benyttes til at gemme indstillingen. For et fuldt overblik over menuen henvises til usecases i sektion \ref{P-ch:kravspecifikation} i projektdokumentationen. Tastaturet indeholder desuden et vindue til displayet, samt en rød LED der tændes hvis systemet registrerer en fejl. Eks. hvis batterispændingen bliver for lav. I det færdige produkt forventes det at implementere en specialbygget keypad, men til prototypeformål er her benyttet en 4 knappers membrane switch keypad som tastatur.

\begin{figure}[H]
  \centering
  \begin{minipage}[b]{0.45\textwidth}
	\includegraphics[width=1\textwidth]{../fig/Keypad/keypad.png}
	\caption{Skitse af keypad'en som den ønskes i en færdig produktion}
	\label{fig:keypad_UI}
  \end{minipage}
  \hfill
  \begin{minipage}[b]{0.45\textwidth}
	\includegraphics[width=1\textwidth]{../fig/Keypad/diagram.png}
	\caption{Diagram over keypad'en}
	\label{fig:keypad_diagram}
  \end{minipage}
\end{figure}

\subsection{Sensor}



\begin{figure}[H]
  \centering
    \includegraphics[width=1\textwidth]{../fig/Systemarkitektur/BDDsensor.pdf}
    \caption{BDD over sensoren}
    \label{fig:bddsensor}
\end{figure}

På Figur \ref{fig:bddsensor} ses en uddybning af de blokke der tilsammen udgør sensoren. Herunder beskrives hvordan disse enkelte blokke er blevet implementeret. For uddybning af designovervejelserne henvises til sektion \ref{P-sec:hardwaredesign}: \nameref{P-sec:hardwaredesign} på side \pageref{P-sec:hardwaredesign} i projektdokumentationen.

\textbf{Mikroprocessor}\\
Som mikroprocessor er der valgt samme type som på kontrolboksen. Her er der dog små ændringer i brugen da den interne ADC bruges til at sample værdierne fra både jordfugtmåleren og lyssensoren. 

\textbf{Jordfugtmåler}\\
Hovedformålet med jordfugtmåleren er som det fremgår af navnet; at måle jordfugtigheden. Her er der i videnskabelige artikler fundet frem til fire metoder at gøre dette på. En resistiv målemetode \cite{lib:MultiSensorSystem}, en kapasitiv målemetode\cite{lib:MultiSensorSystem}, en optisk målemetode\cite{lib:optical} og en metode hvor der bruges en varmepuls\cite{lib:DPHP}. Den resistive målemetode er den simpleste men desværre også den mest upræcise. Der er igennem projektet blevet undersøgt om denne målemetode kunne bruges trods dens begrænsninger og om der kunne findes en løsning der på. Da det viste sig at være svært at forbedre den resistive målemetode blev der herefter undersøgt hvordan der kunne laves en kapasitiv måling. Skalaen som der er brugt at at bestemme jordfugten er den volumetriske skala\cite{lib:wiki_waterContent}. 

\begin{equation}
	\theta = \frac{V_w}{V_s + V_w +V_a} \cdot 100\%
	\label{equ:fugtighed}
\end{equation}

Hvor $\theta$ er fugtigheden i procent. $V_w$ er volumen af væsken, $V_s$ er volumen af jorden og $V_a$ er volumen af luften indeholdende i jorden. For at gøre det nemmere at bestemme en jordfugtighed afvejes først 1 liter jord, som der sammenpresses nok til at der ikke er noget luft i prøven. Herefter noteres vægten som $m_{s1L}$ og nu kan ligning \ref{equ:fugtighed} omskrives til ligning \ref{equ:fugtighed2} 

\begin{equation}
	\theta = \frac{m_w}{(m_{wet}-m_w)\cdot \eta + m_w} \cdot 100\% 
	\label{equ:fugtighed2}
\end{equation}

Hvor $m_w$ er massen af væske, $m_{wet}$ er massen af den fugtige jord og $\eta$ er udregnet via vægten af 1 liter jord.

\begin{equation}
	\eta = \frac{1}{m_{s1L}}
\end{equation}

Det er derfor nu muligt at bestemme en jordfugtighed uden at skulle måle volumen af prøven. Til vejning af jordprøver er der blevet brugt en vægt af producenten \texttt{Sartorius} model \texttt{QUINTIX2102-1S} med en præcision på $\pm0.01g$.

\textit{Resistiv måling} \\
Den resistive målemetode er en metode til at måle den ohmske værdi af jorden. Jo mere vand jorden indeholder jo mere ledende vil jorden være og den ohmske værdi vil herefter falde. Der er dog mange faktor som kan være med til at give en måleusikkerhed, disse usikkerheder er betinget af selve næringsindholdet i jorden, om jorden er forurenet med metaller eller om proben som der måles med, har været udsat for kraftig tæring. Disse betingelser kunne der kalibreres for, men en kalibrering vil være besværlig for en almindelig bruger af systemet, da det vil kræve at brugeren ovntørrer en jordprøve og herefter tilsætter en præcis mængde vand for at vide med sikkerhed hvilken værdi der skal kalibreres ind efter. Det viste sig dog at der sammen med den resistive måling af jorden også opstod en kapasitiv virkning, som funktion af en stigetid. Ved hjælp af denne kunne der opstilles et ækvivalent-diagram som ses på Figur \ref{fig:akvj1}. 

\begin{figure}[H]
	\centering
	\includegraphics[width=0.5\textwidth]{../fig/Jordfugt/akvivalentjord.PNG}
	\caption{Ækvivalentdiagram af jorden}
	\label{fig:akvj1}
\end{figure} 

Systemet blev moduleret som et første ordenssystem med overføringsfunktionen:
\begin{equation}
	System = \dfrac{\alpha \cdot \beta}{S + \alpha}
\end{equation}
Der blev i alt lavet målinger af 3 typer jord. To fra en have og en fra en plantesæk. Både $\beta$ værdien og $\alpha$ værdien varierede med fugtigheden af jorden og det var derfor nærliggende at tro at ved brug af begge, ville der kunne opnås en større præcision. Det viste sig dog at $\alpha$ værdien varierede med tiden siden sidste måling og at den derfor var uegnet til udregning af fugtigheden. $\beta$ værdien var noget mere præcis, men den varierede meget for hvilken jordtype der var tale om og er derfor hellere ikke egnet til udregning af fugtighed. Se Figur \ref{fig:funcfit} for regression over $\beta$ værdien i MATLAB. Regressionen blev bestemt til at være:

\begin{equation}
	\theta = a \cdot exp(b\cdot x) + c \cdot exp(d\cdot x)
\end{equation}

Hvor a=2.219$\cdot 10^{7}$ b=-39.8 c=151.9 d=-3.473
 
\begin{figure}[H]
	\centering
	\includegraphics[width=0.8\textwidth]{../fig/Jordfugt/functionfit.PNG}
	\caption{Functionfit i MATLAB}
	\label{fig:funcfit}
\end{figure} 


\textit{Kapacitiv måling} \\
Den kapacitive målemetode benytter sig af et jordspyd stukket ned i jorden\cite{lib:SoilModeling} \cite{lib:MultiSensorSystem} \cite{lib:DPHP} \cite{lib:cap_soil}. På jordspyddet er der to kobberplader som vil danne en kapacitet der vil variere afhængig af fugtigheden af den jord det befinder sig i. Se Figur \ref{fig:kap1} og \ref{fig:kap2}. Kobberbanerne skal dog forinden belægges med en elektrisk isolerende coating for at undgå kortslutning og korodering.

\begin{figure}[H]
  \centering
  \begin{minipage}[b]{0.4\textwidth}
    \includegraphics[width=1\textwidth]{../fig/Jordfugt/printkapasitet.png}
    \caption{Tværsnit af jordspyd}
    \label{fig:kap1}
  \end{minipage}
  \hfill
  \begin{minipage}[b]{0.4\textwidth}
    \includegraphics[width=1\textwidth]{../fig/Jordfugt/printkapasitet_vad.png}
    \caption{Tværsnit af jordspyd i vand}
    \label{fig:kap2}
  \end{minipage}
\end{figure}

For en kapacitet gælder:

\begin{equation}
	C = \frac{\epsilon_0\cdot \epsilon_r \cdot A}{d}
\end{equation}

Hvor $\epsilon_0 = 8.854\cdot 10^{-12}F/m$ er permativiteten af vaccum, $\epsilon_r$ er permativiteten af materialet mellem kobberet som er 1 for luft og omkring 80 for vand. A er arealet af det kobber som kan se hinanden og d er diameteren mellem banerne. 

Det vil sige at kapaciteten vil blive 80 gange større når jordspyddet er stukket ned i rent vand end hvis det var i tør luft. Hvis vi antager at der er 0.3mm mellem banerne, at de har en længde på 5cm, at kobbertykkelsen er 35$\mu m$ og at jordspyddet befinder sig i vaccum. Vil der være en kapacitet på:

\begin{equation}
	C = \frac{8.854\cdot 10^{-12}\cdot 1 \cdot 35 \cdot 10^{-6} \cdot 5 \cdot 10^{-2}}{0.3 \cdot 10^{-3}} = 0.05pF
\end{equation} 

Antager vi at jordspyddet er stukket ned i vand vil vi have en kapacitet 80 gange større:

\begin{equation}
	C = \frac{8.854\cdot 10^{-12}\cdot 80 \cdot 35 \cdot 10^{-6} \cdot 5 \cdot 10^{-2}}{0.3 \cdot 10^{-3}} = 4.00pF
\end{equation} 

Der er blevet designet 4 typer af jordspyd som er blevet brugt til test. Det viste sig at kapaciteten lå højere end det forventede. Dette skyldes at der er udstråling fra banerne som der er svære at tage med i beregningerne. Se figur \ref{fig:spyd1},\ref{fig:spyd2},\ref{fig:spyd3},\ref{fig:spyd4}. Type 1 og 2 er meget ens. De danner begge en kapacitet mellem stelplanet og midterbanen. Størstedelen af kapaciten vil ligge for enden af spyddet hvor banen er tykkest på Type 1 og gaffelformet på Type 2. Dog er det ikke ligegyldigt hvor dybt spyddes stikkes ned i jorden da der også vil dannes en kapacitet mellem fødebanen og stelplan'et. For at undgå at spyddet skal stikkes ned i en bestemt højde er Type 2 og 3 blevet lavet. Type 3 fører en bane på oversiden af printet hvor der ligger en identisk bane på bagsiden. På den måde skulle kapaciteten ikke ændre sig når der kommer vand længere op end til "diamanten". Test viste dog at kapaciteten alligevel varierede. Det blev lagt et stelplan uden på banerne for at indkapsle dem, ved at skære et andet print til, og lime det på således der laves et 4-lags print. Dette viste sig at have gode egenskaber og derfor skal jordspyddet også laves i 4-lag hvis der en gang skal laves en produktion. Type 4 er delt med et stelplan således at banerne ikke kan se hinanden. Dette print var dog det der performede dårligst da det kun virkede når der lå vand mellem banerne. Kom der først vand ud på stelplan'et blev her dannet en kapacitet som var sammenligning med det der var mellem pladerne og virkningen blev herved reduceret. Der blev herefter testet i en så og prikle jord med Type 1,2 og 3 for sammenhæng mellem kapacitet og fugtighed. Se Figur \ref{fig:graffugt}.

\begin{figure}[H]
  \centering
  \begin{minipage}[b]{0.4\textwidth}
  \centering
    \includegraphics[width=0.4\textwidth]{../fig/Jordfugt/type1.jpg}
    \caption{Type 1 jordspyd}
    \label{fig:spyd1}
  \end{minipage}
  \hfill
  \begin{minipage}[b]{0.4\textwidth}
  \centering
    \includegraphics[width=0.31\textwidth]{../fig/Jordfugt/type2.jpg}
    \caption{Type 2 jordspyd}
    \label{fig:spyd2}
  \end{minipage}
\end{figure}


\begin{figure}[H]
  \centering
    \begin{minipage}[b]{0.4\textwidth}
    \centering
    \includegraphics[width=0.35\textwidth]{../fig/Jordfugt/type3.jpg}
    \caption{Type 3 jordspyd}
    \label{fig:spyd3}
  \end{minipage}
  \hfill
    \begin{minipage}[b]{0.4\textwidth}
    \centering
    \includegraphics[width=0.34\textwidth]{../fig/Jordfugt/type4.jpg}
    \caption{Type 4 jordspyd}
    \label{fig:spyd4}
  \end{minipage}
\end{figure}

\begin{figure}[H]
	\centering
	\includegraphics[width=0.8\textwidth]{../fig/Jordfugt/capVsSoil.PNG}
	\caption{Kapacitet vs. jordfugtighed. Her er målt i en så og priklejord}
	\label{fig:graffugt}
\end{figure}  

Det ses på Figur \ref{fig:graffugt} at Type 1 og type 2 er meget varierende i deres udslag, men at Type 3 har et pænt lineært forhold op til omkring 40\%. Afvigelsen skyldes at ved 40\% bliver jorden så våd at der begynder at ligge mere vand på spyddet end hvad der er i jorden. Derfor bliver kapaciteten højere end hvad den burde. Der kan nu opstilles en formel til beregning af fugtigheden af jorden hvis kapaciteten af spyddet er kendt for både tør og våd. 

\begin{equation}
	\theta = \frac{C_{weat} - C_{dry}}{100} \cdot (C-C_{dry}) \cdot \%
\end{equation}

\textit{Målekredsløb}\\
Det blev overvejet hvordan kapaciteten kunne måles præcist uden at det skulle blive et alt for dyrt kredsløb. Det blev overvejet at lave en amplitude måling af kapaciteten. Men denne viste sig at være meget støjfyldt og meget temperatur afhængig. Herefter blev der designet et kredsløb til måling af faseændringen, som ved hjælp at et båndpass-filter med et højt Q dæmpede støjen betragteligt. Se Figur \ref{fig:j9}.

\begin{figure}[H]
	\centering
	\includegraphics[width=0.8\textwidth]{../fig/Jordfugt/diagram4.PNG}
	\caption{Målekredsløb ver.2}
	\label{fig:j9}
\end{figure}  

Kredsløbet virker ved at der fra mikroprocessoren kommer en firkantspænding på 100kHz igennem jordspydet som virker som en kondensator. Denne kondensator bliver belastet af R22 således at der totalt opnåes en faseændring på 55$^\circ$ fra 0\% fugtighed til 100\% fugtighed. Herefter føres firkantspændingen i et båndpasfilter og vil ligge som en faseforskudt sinus på udgangen af U12A. Da U12 er en inverterende forstærker monteres U8A som en komparator der både vender fasen til 0$^\circ$ igen og samtidig sørger for at konvertere sinusen om til en firkant igen, således indgangen af U10A skifter hurtigt. U10A er en XOR-gate som her bruges tom fasedetektion. På indgang B kommer den faseforskudte firkant og på indgang A er en forskudt frekvens fra mikroprocessoren som kalibreres ind ved at kalde en kalibrerings-funktion således der ligger 0V på C26 ved 0\% fugtighed. Alt afhænging af fugtigheden i jorden vil der nu ligge en DC på C26. Denne forstærkes som sidste led op af U12B således der gived et bedre dynamisk range til mikroprocessorens ADC. Q11 er blevet monteret for at kunne slukke totalt for målekredsløbet når det er inaktivt for at spare strøm. \\

Desværre viste kredsløbet sig at volde store problemer. I første øjekast så det ud til at virke, men ved nærmere iagttagelse sås der at det gav en for stor faseforskydelse. Det viste sig at faseforskydelsen opstod i U12A. Der blev testet med en tonegenerator og det blev opserveret at der ved en amplitude ændring på indgangen opstod en faseforskydelse på udgangen. Filterets komponentværdier blev ændret til et lavere Q men dette gav intet bedre resultat. Komponent-værdierne blev også skaleret således udgangen ikke blev belastet så hårdt. Dette hjalp hellere ikke. Til sidst blev det forsøgt at udskifte operationsforstærkeren til en anden type. Dette gjorde dog at båndpassfilteret begyndte at oscillere.
Da det begyndte at knibe med tiden blev der derfor implementeret en løsning som i starten var valgt fra pga. dens dårlige støj og temperatur forhold. Se Figur \ref{fig:jdig1}.

\begin{figure}[H]
	\centering
   \includegraphics[width=0.5\textwidth]{../fig/Jordfugt/diagram1.png}
   \caption{Amplitudemåling af kondensator}
   \label{fig:jdig1}
\end{figure}  

På Figur \ref{fig:j20} ses udgangsspændingen vs. kapaciteten. Regressionen er blevet implementeret i mikroprocessoren og en kalibrerings-funktion bliver kaldt når brugeren trykker på connect-knappen på sensoren.  

\begin{figure}[H]
	\centering
	\includegraphics[width=0.7\textwidth]{../fig/Jordfugt/funch_dc_charge.PNG}
	\caption{Kapacitet vs. udgangspænding i forhold til forsyningsspændingen}
	\label{fig:j20}
\end{figure} 

\textbf{Temperaturføler}\\
Kravet til temperaturføleren var at den skulle have en opløsning på minimalt 1 decimal og at den skulle have en præcision på 0.5$^\circ$C og kommunikere over I2C bussen. Dette krav blev opfyldt med valg af \texttt{TMP75B }som ses på Figur \ref{fig:dtemp}. \texttt{TMP75B} har en intern 12-bit ADC som kan give en opløsning på 0.0625$^\circ$C og en præcision på 0.5$^\circ$C ved 25$^\circ$C. \texttt{TMP75B} opsættes til at køre i single-shot mode hvilket vil sige at den ligger i sleep-mode det meste af tiden. Kun når temperaturen aflæses laves der en enkelt konvertering. Dette er med til at sikre et lavt strømforbrug.  

\begin{figure}[H]
  \centering
  \includegraphics[width=0.4\textwidth]{../fig/Temperatur/temp_i2c.png}
  \caption{Diagram over den digitale temperaturføler}
  \label{fig:dtemp}
\end{figure}

\textbf{HF-Transceiver}
Transceiveren er valgt til \texttt{MRF49XA} som er den samme som på kontrolboksen.

\textbf{Antenne}\\
Som antenne på sensoren er der valgt en helical antenne. Denne er valgt da den har et relativt lavt tab på 2.9dBi.  Antennen er samtidig også meget isotopisk hvilket vil sige at den udstråler lige meget energi i alle retninger, hvilket gør at brugeren ikke skal tænke over hvilken vej sensoren pejer i forhold til kontrolboksen. Antennen fylder dog en del og det har derfor ikke været muligt at få plads til den i kontrolboksen.  

\textbf{Lysmåler}\\
Som lysmåler er der blevet brugt en fototransistor af typen BPW96B. Diagram ses på Figur \ref{fig:lysdiagram}. Når der indstråles lys på fotodioden vil spændingen over C32 stige. Det blev valgt at designe kredsløbet sådan at der ved skumringstid vil ligge 1VDC på C32. Dette gøres fordi at der ved måling fra mikroprocessorens ADC køres på den interne 2.56V reference. R20 blev bestemt til at være 47k$\Omega$. Når spændingen måles midles der over 20ms for at minimere 50Hz støj. På Figur \ref{fig:test_måling4} ses der lysmålerens respons ved solopgang og solnedgang. Den røde kurve er strøm og den blå er spænding. Den striplede linje er 1V. 

\begin{figure}[H]
	\centering
	\includegraphics[width=0.25\textwidth]{../fig/Lysmaeler/diagram.png}
	\caption{Diagram over lysmåleren}
	\label{fig:lysdiagram}
\end{figure} 

\begin{figure}[H]
	\centering
	\includegraphics[width=1\textwidth]{../fig/Lysmaeler/testgraf.png}
	\caption{Dataopsamling fra lyssensor: Dato: 10-11/04, kl.15.00-08.00, Vejr: klart, Solnedgang + solopgang}
	\label{fig:test_måling4}
\end{figure}

\textbf{PSU}\\
Som strømforsyning til sensoren er der blevet valgt en boostconverter af typen \texttt{MCP1640B} fra \texttt{Microchip}. Se Figur \ref{fig:boostkonverter}. Boostconverteren bliver først enablet når batterispændingen når under 2.8V. Dette måles af transceiveren som ved denne spænding giver et interrupt til mikroprocessoren, som derefter tænder for boostconverteren.  

\begin{figure}[H]
	\centering
	\includegraphics[width=0.8\textwidth]{../fig/psu/boostkonverter.png}
	\caption{Diagram over boostkonverter}
	\label{fig:boostkonverter}
\end{figure}