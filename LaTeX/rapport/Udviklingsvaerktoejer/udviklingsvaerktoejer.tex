\chapter{Udviklingsprocess} \label{ch:udviklingsprocess}
%%%%%%%%%%%%%%%%%%%%%%%%%%%%%%%%%%%%%%%%  UDVIKLINGSMODELLER %%%%%%%%%%%%%%%%%%%%%%%%%%%%%%%%%%%%%%%%
\section{Udviklingsmodeller}
\subsection{V-Model og ASE-Model}
Under udvikling af dette projekt er det valgt at følge V-Modellen \cite{lib:V_model}. Denne model definere forløbet for udviklingsprojektet og egner sig godt til et iterativt projekt, da der let kan køres flere iterationer under implementeringsfasen. Se Figur \ref{fig:udv_vmodel} for denne model. 

\begin{figure}[ht]
	\centering
	\includegraphics[width=0.5\textwidth]{../fig/rapport/v_model.png}
	\caption{V-model}
	\label{fig:udv_vmodel}
\end{figure} 

I dette projet er modul- og process-integrationstest dog udeladt, og projektet har holdt sig til modul samt accepttests. Derudover er der, som det også fremgår af den iterative beskrivelse i projektdokumentationen udført løbende modultests under implementering af hardware og software. \newline
Ud over V-Modellen, er ASE-Modellen \cite{lib:V_model} taget i brug som en vejledning til gennemførelse af projektet. ASE-modellen kobler hver fase i udviklingsprojektet med et dokument, disse dokumenter samles herefter til projektdokumentationen. Derudover kan der med denne model efter defineret Systemarkitektur udvikles hardware og software hver for sig i et givent antal iterationer, dette samles så til sidst i Accepttesten for hele systemet. Oversigt over ASE-modellen ses på figur \ref{fig:udv_asemodel} på \pageref{fig:udv_asemodel}. 

\begin{figure}[ht]
	\centering
	\includegraphics[width=0.8\textwidth]{../fig/rapport/ase_model.png}
	\caption{ASE-model}
	\label{fig:udv_asemodel}
\end{figure}


\textbf{SysML} \newline
Projektet har anvendt SysML primært i systemarkitektur-fasen, for at beskrive systemet bedst muligt
ud industristandarden for opbygning logiske blokke. Muligheden for derefter at beskrive overordnede blokke har givet mulighed for forskellige abstraktionsniveauer i udviklingen af projektet.
Veldefinerede grænseflader mellem blokke ved hjælp af SysML har også bidraget til en mere klart defineret designfase og implementeringsfase.


\textbf{Versionsstyring} \newline
Der er anvendt versionshistorik på dokumenter i projektdokumentationen samt projektrapport via Github, og for resten af projektet i form af Dropbox. Væsentlige ændringer i eks. design har givet anledning til versions-ændring, hvilket hjælper med at holde styr på hvilke større ændringer projektet har gennemgået.


%%%%%%%%%%%%%%%%%%%%%%%%%%%%%%%%%%%%%%%%  PROJEKTSTYRING %%%%%%%%%%%%%%%%%%%%%%%%%%%%%%%%%%%%%%%%
\section{Projektstyring}
Da projektgruppen kun består af 2 personer og at den primære udvikling er sket i samarbejde, er det valgt at udarbejde en specifik projektstyring til dette projet. Der er derfor benyttet forskellige modeller, hvor gruppen har udvalgt de dele som tilsammen har givet den mest optimale styring. Der er eksempelvis taget udgangspunkt i scrum-framework'et at arbejde efter, hvor udviklingen inddeles i mindre "sprints" for overskueligheds skyld. Dette er gjort at for at holde projektet og udviklingen så agilt som muligt. Men der er eksempelvis ikke benyttet de typiske SCRUM-roller. Dertil kommer at gruppen for hele tiden at fastholde fokus og enighed om en opgave, har holdt korte orienteringsmøder "standup møder" hver anden dag. Dette har også hjulpet til at få ressourcerne allokeret korrekt igennem hele forløbet. Kravene til systemet har således sat den Backlog som der blev arbejdet ud fra i hver "sprint". Hele Sprint-metoden og den iterative tilgang til udviklingen har passet godt med hvordan projektet var ønsket gennemført. Til projektstyring har Microsoft Project været benyttet, primært været til at fastholde fokus på tidsplanen, samt hele tiden at ha overblik over projektfaserne og allokering af ressourcer. På Figur \ref{fig:udv_gantt} ses strukturen i projektstyringen.

\begin{figure}[ht]
	\centering
	\includegraphics[width=1\textwidth]{../fig/rapport/ganttchart.png}
	\caption{Gantt-chart}
	\label{fig:udv_gantt}
\end{figure}

\textbf{Tidsplan} \newline
Der blev ved projektet opstart fastlagt en overordnet tidsplan for at få et overblik over de givne faser i projektet, og for at sikre sig at der blev allokeret tid nok til disse allerede fra start. Denne blev med overlæg lavet konservativ således at der undervejs var indbyggede buffere hvis noget skulle vise sig at tage længere tid end forventet. Tidsplanen var udarbejdet med en færdig iteration 1 i tankerne, hvor alle \textit{Must}-krav ønskes opfyldt. Tidsplanen som den ser ud d.d ses på Figur \ref{fig:udv_tidsplan}, det har været minimalt hvad der har været af ændring til de overordnede faser, da ingen af dem overskred den estimerede tidsramme. Vejledermøder er blevet tilføjet undervejs, ligesom reviews. 

\begin{figure}[ht]
	\centering
	\includegraphics[width=1\textwidth]{../fig/rapport/tidsplan.png}
	\caption{Tidsplan}
	\label{fig:udv_tidsplan}
\end{figure}

%%%%%%%%%%%%%%%%%%%%%%%%%%%%%%%%%%%%%%%%  UDVIKLIGNSVÆRKTØJER %%%%%%%%%%%%%%%%%%%%%%%%%%%%%%%%%%%%%

\section{Udviklingsværktøjer}
I dette afsnit vil de forskellige udviklingsværktøjer, som er blevet anvendt under dette projekts design-, implementerings- og integrationsproces, blive gennemgået.

\textbf{Cadence OrCAD Capture} \\
Cadence OrCAD Capture er brugt til at designe alle hardwarediagrammer til kontrolboks og sensor. Programmet blev valgt da dette er industristandard, samt at der i projektgruppen er et rigtig godt kendskab til dette arbejdsmiljø. Der har ligeledes været mulighed for at oprette et stort antal af egene komponenter der er brugt til designs. Derudover Cadence pSpice-modulet blevet brugt i forbindelse med alle simuleringer. OrCAD integrer derudover fuldt ud med det valgte layout-tool. 


\textbf{Cadence Allegro} \\
Cadence Allegro er valgt som Layout-tool da der i projektgrupper er stor erfaring med dette program. Derudover integrerer det fuldt ud med det brugte design-tool, og mulighed derudover crossprobing som markant nedsætter arbejdstiden.  


\textbf{Atmel Studio 7.0} \\
Atmel Studio 7.0 er benyttet som udviklignsmiljø til al software. Dette miljø er skab af fabrikanten til den valgte mikrokontroller. Derfor var det oplagt at skrive i dette tool.  


\textbf{AWR Design Environment 11} \\
AWR Design Environment 11 designet af native Instruments er lavet til beregner på højfrekvente kredsløb, her er det benyttet til at lave beregninger på mikrostrip-linjerne til antennerne.


\textbf{WaveForms} \\
WaveForms er brugt til at teste de forskellige hardwareenheder samt indput/output porte af de forskellige embeddede enheder. WaveForms er et program der integrerer med Analog Discovery-enheden, som begge er produceret af Analog Devices, og er et multi-funktions instrument med mulighed for bland andet at agere oscilloskop og funktionsgenerator. Fordelen med Analog Discovery er den mobilitet der gives under udviklingen.


\textbf{PTC MathCad Prime} \\
PTC MathCad Prime er benyttet til at lave diverse beregninger.


\textbf{Maplesoft Maple} \\
Maplesoft Maple er benyttet til at lave diverse beregninger.


\textbf{MathWorks MATLAB} \\
MathWorks MATLAB er benyttet til at lave diverse beregninger samt diverse plot og dataanalyser.


\textbf{Git} \\
Git er et versionsstyringsværktøj til vedligeholdelse af diverse dokumenter. Som repository host er der valgt Github grundet stabilitet og projektgruppens tidligere arbejde med dette miljø. Projektgruppen har valgt at lægge kildekoden til bl.a. denne rapport og Projektdokumentationen på Git for netop at opnå en kraftfuld versionsstyring af hele projektet.


\textbf{Microsoft Project 2016} \\
Project er blevet benyttet som projektstyringsværktøj for at kunne holde fokus på tidsplan og de individuelle projektfaser samt sikre sig at alle ressourcer er allokeret optimalt. det er med vilje blevet valgt at håndtere krav-delen ved selvstændigt fra dette værktøj. 