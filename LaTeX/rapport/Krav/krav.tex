\chapter{Kravspecifikation}\label{ch:Krav} %Please start me on a left page :)

I dette afsnit er de opstillede krav for Autovandingssystemet beskrevet. Disse krav er opstillet ud fra projektformuleringen og ud fra forestillinger om relevante test af systemet. Funktionelle krav er opstillet ud fra Use Cases og ikke-funktionelle krav er opstillet ud fra relevante målbare størrelser. På Figur \ref{fig:use_cases} på side \pageref{fig:use_cases} ses Use Cases for systemet samt en kort beskrivelse af dem. 
De funktionelle krav for systemet er udledt ud fra disse og beskrevet herunder. 
For at se den komplette beskrivelse af kravene for systemet henvises til Kapitel \ref{P-ch:kravspecifikation}: \nameref{P-ch:kravspecifikation} på side \pageref{P-ch:kravspecifikation} i dokumentationen.

\begin{itemize} \itemsep0mm
	\item \textbf{UC1: Tænd system}\\ Brugeren tænder systemet.
	\item \textbf{UC2: Aflæs fugtighed og temperatur}\\ Giver brugeren mulighed for at aflæse fugtighed og temperatur ved sensoren
	\item \textbf{UC3: Indstil ønsket fugtig}\\ Giver brugeren mulighed for at indstille den ønskede fugtighed, i det medie hvori sensoren er placeret
	\item \textbf{UC4: Indstil åbningstid}\\ Giver brugeren mulighed for at indstille hvor længe der skal vandes ad gangen, når der måles for lav fugtighed
	\item \textbf{UC5: Indstil vandingsinterval}\\ Giver brugeren mulighed for at indstille en ren tidsbaseret vandingssekvens 
	\item \textbf{UC6: Åbn/luk for ventil manuelt}\\ Giver brugeren mulighed for at åbne og lukke ventilen manuelt
	\item \textbf{UC7: Indstil vandingstidspunkt}\\ Giver brugeren mulighed for at indstille hvornår på dagen der ønskes at vande
	\item \textbf{UC8: Par sensor og kontrolboks}\\ Giver brugeren mulighed for parre sensor og kontrolboks
\end{itemize}

I tabel \ref{tbl:funk_krav} opstilles de funktionelle krav til systemet, disse krav skal være opfyldt for at systemet betragtes som fuldt funktionelt. Kravene listes med et unik ID således de kan spores tilbage igennem hele dokumentationen. Kravenes ID er givet ved:

\begin{itemize} \itemsep0mm
	\item	\texttt{MC}: \textit{Mechanical Contraints:} Dette er de mekaniske krav til systemet
	\item	\texttt{EL}: \textit{Electrical Contraints:} Dette er de elektriske krav til systemet
	\item	\texttt{CM}: \textit{Communications Contraints:} Dette er de krav der vedrører kommunikation i systemet
	\item 	\texttt{UX}: \textit{User Experience:} Dette er de bruger-relaterede krav til systemet
\end{itemize}

\begin{table}[H]
	\begin{tabularx}{\textwidth}{| m{1.5 cm} | Z | L{1.9 cm}|} \hline
	\textbf{ID:} 		& \textbf{Krav:}		& \textbf{Prioritet:} 	 \\ \hline
		MC\_01 	& Systemet skal bestå af en sensor samt en kontrolboks											& \textit{Must}	 \\ \hline
		MC\_02	& Sensoren skal måle temperatur, jordfugtighed samt lysintensitet								& \textit{Must}  \\ \hline
		MC\_03	& Kontrolboksen skal vise jordfugtighed og temperatur på et display								& \textit{Would} \\ \hline
		MC\_04	& Kontrolboksen skal indeholde et tastatur														& \textit{Would} \\ \hline
		MC\_05	& Kontrolboksen skal drive en indbygget motorventil til at åbne og lukke for vandet				& \textit{Would} \\ \hline
		MC\_06	& Systemet skal kunne pre-indstilles til en specifik afgrøde 									& \textit{Could} \\ \hline
		MC\_07  & Systemet skal kunne måle PH-værdien i jorden													& \textit{Could} \\ \hline
		MC\_08	& Systemet skal kunne betjenes fra en smartphone-applikation									& \textit{Would} \\ \hline
		MC\_09	& Systemet skal kunne indeholde flere sensor/kontrolboks par									& \textit{Would} \\ \hline
		EL\_01	& Sensoren skal være batteridrevet																& \textit{Must}	 \\ \hline
		EL\_02	& Kontrolboksen skal kunne forsynes med en 3. parts 5V AC/DC adaptor							& \textit{Must}	 \\ \hline
		CM\_01	& Kontrolboksen skal kunne udveksle data med sensoren via en trådløs forbindelse				& \textit{Must}	 \\ \hline
		CM\_02	& Sensor og kontrolboks skal parres manuelt 													& \textit{Must}	 \\ \hline
		CM\_03	& Kommunikation fra sensor til kontrolboks skal virke ved minimum 30m 							& \textit{Should}\\ \hline
		UX\_01	& Brugeren skal have mulighed for at åbne/lukke for ventilen manuelt på kontrolboksen			& \textit{Must}	 \\ \hline
		UX\_02	& Brugeren skal kunne indstille en ønsket jordfugtighed på kontrolboksen						& \textit{Must}	 \\ \hline
		UX\_03	& Brugeren skal kunne indstille en åbningstid for ventilen										& \textit{Must}	 \\ \hline
		UX\_04	& Brugeren skal kunne vælge automatisk vandingstidsrum											& \textit{Must}	 \\ \hline
		UX\_05	& Kontrolboksen skal kunne måle om slangen er sprunget fra, og give fejlmelding					& \textit{Should}\\ \hline
		UX\_06	& Brugeren skal kunne indstille en tidsbaseret vandingssekvens									& \textit{Should}\\ \hline		
	\end{tabularx}
	\caption{Funktionelle krav}
	\label{tbl:funk_krav}
\end{table}

De ikke-funktionelle krav findes i Kapitel \ref{P-ch:kravspecifikation}: \nameref{P-ch:kravspecifikation} på side \pageref{P-ch:kravspecifikation} i dokumentationen.

\clearpage

\begin{figure}[H]
\centering
\includegraphics[width=\textwidth]{../fig/Systemarkitektur/Usecases}
\caption{Use Cases for autovandingssystem}
\label{fig:use_cases}
\end{figure}

\clearpage