\section{Markedsundersøgelse} \label{sec:marked}

For at undgå at udvikle et produkt som i forvejen eksisterer, eller som på anden vis er umulig at sælge til en pris der kan give overskud, foretages der derfor en markedsundersøgelse. Markedsundersøgelsen baserer sig på søgeord som vandingscomputer, automatisk vandingssystem, jordfugt sensor osv. Kärcher SensoTimer ST6 Er det produkt som kommer nærmest dette projekt. SensoTimer er i sin funktionalitet meget simpel. Den måler jordfugtigheden med en trådløs sensor og vandingscomputeren som er tilsluttet direkte på vandhanen kan derfor indstilles til en bestemt fugtighed. Det er dog ikke muligt at aflæse fugtigheden i procent på displayet, men blot som 5 punkter hvor det 5. punkt er fuld fugtighed. Om det så er 50\% eller 100\% er der ingen information om. Det er hellere ikke muligt at indstille vandingstidspunktet til morgen eller aften baseret på lysintensiteten. Det er derimod muligt at indstille et fast vandingstidspunkt. Der findes ingen information om batterilevetid eller hvilken type af ventil vandingscomputeren bruger. Det er derfor nærtliggende at tro produktet bruger en magnetventil og derfor har en sparsom batterilevetid.  Produktet virker til at være lavet i en god kvalitet men er meget sparsom på sin funktionalitet og information. Prisen ligger på 785kr og kan købes hos Toolworld \cite{lib:toolworld} 

\begin{figure}[H]
	\centering
	\includegraphics[width=0.5\textwidth]{../fig/rapport/markedsundersogelse/sensotimer.png}
	\caption{SensoTimer}
	\label{fig:BMC}
\end{figure}

Et andet produkt er GARDENA MasterControl vandingscomputer, som kan tilsluttes to typer af sensorer, herunder regnsensor og fugtsensor. Dens funktion er meget begrænset da den ligesom  SensoTimer kun baserer sig på fugtighed og tid. Sensorerne købes separat og monteres med en 5m lang ledning. På trods af dette skal der batterier i både sensor og vandingscomputer. GARDENA lover en batterilevetid på 1 år. Jordfutgtigheden kan ikke aflæses nogen steder, men kan indstilles via en dreje-knap på sensoren. Skalaen er ikke angivet i procent og det er derfor svært at vide hvilken fugtighed der indstilles til. Sensoren bruger en varmepuls til udregning af fugtighed. Produktet virker meget gammeldags og billig i dens plastik. Vandingscomputer og sensor skal købes hver for sig hvor vandingscomputeren har en pris på 899kr og sensoren 409kr altså en samlet pris på 1308kr. Produktet kan købes hos Conrad-Elektronik \cite{lib:conradelektronik}

\begin{figure}[H]
	\centering
	\includegraphics[width=0.6\textwidth]{../fig/rapport/markedsundersogelse/mastercontrol.png}
	\caption{Gardena mastercontrol}
	\label{fig:BMC}
\end{figure}

Gardena har dog også lavet den mere moderne udgave, Smart System, som bruger en trådløs gateway således hele systemet kan styrres via en app på ens smartphone. Systemet måler jordfugtighed, lysentensitet og temperatur. Systemet har dog en voldsom pris på 5899kr og kan købes hos byghjemme \cite{lib:byghjemme}. Systemet virker meget gennemført og hvis det ikke var pga. den høje pris var det svært at producere noget som kunne slå dette.

\begin{figure}[H]
	\centering
	\includegraphics[width=0.8\textwidth]{../fig/rapport/markedsundersogelse/smartsystem.png}
	\caption{Gardena Smart System}
	\label{fig:BMC}
\end{figure}

Det konkluderes at det kan lade sig gøre at lave et produkt der både er bedere og billigere end hvad der i forvejen er på markedet. Det anslås at et produktet ikke må koste mere end 500kr for at have en reel interesse. Samtidig skal det også være af god kvalitet og være præcis i sin måling af jordfugt. 
