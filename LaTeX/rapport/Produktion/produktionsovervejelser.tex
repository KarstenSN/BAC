\section{Udviklingsomkostinger og mulig intjening}

Formålet med projektet har som udgangspunkt været at skabe en forretning. Da det er blevet mere moderne med automatisering i hjemmet synes det automatiske vandingssystem at være et oplagt produkt. En markedsundersøgelse har vist at denne type produktet findes i forvejen, men ofte er de allerede eksisterende løsninger dyre og upræcise i målingen af jordfugt. Det har derfor været formålet at gøre systemet mere præcis og samtidig også billigere end konkurrenterne. Se sektion \ref{sec:marked} for markedsundersøgelse. For at skabe en forretning kræves der både tid, knowhow og penge. Se Figur \ref{fig:plc} for Product Life Cycle \cite{lib:plc} denne model beskriver økonomien for et produkt, fra udviklingsfasen til det går af mode og tages af markedet. I den periode et produkt udvikles er der ingen indtjening men der kræves investering af både tid og penge. Når produktet bliver introduceret på markedet vil der gå noget tid før det er modnet og salget for alvor tager fat. På et tidspunkt vil produktet blive forældet, eller der vil være kommet andre konkurrenter og salget vil aftage. Det er derfor vigtigt at sikre sig at udviklings-omkostningerne holdes så lave som muligt da det er utroligt svært at vide hvor stort et salgspotentiale der er for produktet. Da projektet er et afgangsprojekt har det været minimalt hvad der er blevet investeret økonomisk. Udgifter til løn, lokaler osv. har været nul og der har været gratis ekspertise til rådighed.   

\begin{figure}[H]
	\centering
	\includegraphics[width=0.8\textwidth]{../fig/rapport/forretningsmodel/plc.png}
	\caption{Product Life Cycle. Projektet befinder sig på nuværende tidspunkt under Product development.}
	\label{fig:plc}
\end{figure}

Skal produktet sælges kræves det af EU at det skal leve op til kravet om CE-mærkning. Det vil sige at som producent er man ansvarlig for at  sikre at systemet overholder alle gældende krav. Disse krav relaterer sig hovedsageligt til EMC og personsikkerhed samt afskaffelse og brug af miljøforurenende kemikalier. Som producent kunne man blot vælge at sætte et CE-mærke på sit produkt uden reelt at teste det. Men det vil betyde en høj risiko hvor der i sidste ende kan blive tale om at alle produkterne der er solgt, skal trækkes tilbage. Dette kan ende som en stor udgift som i sidste ende kan lukke virksomheden. Der findes ingen billig eller nem måde at sikre sig at produktet overholder alle gældende krav til CE-mærkning og det må medregnes at en udgift til dette vil øge udgifterne inden at der kommer en reel indtjening. 

\textbf{Reduktion af produktionsomkostninger}\\
Det har igennem udviklingen været et krav at systemet skulle designes billigst muligt, samtidig med at en god kvalitet skulle opretholdes. For at få en lav produktionsomkostning i forbindelse med printfremstilling er der i størst muligt omfang forsøgt at undgå håndmontage og derfor er flest mulige komponenter implementeret som SMD. Dog er der få komponenter som ikke kan monteres som SMD og derfor skal håndmonteres. Her er der tale om display'et, tastaturet og selve ventilen som skal påmonteres kontrolboksen. Selve teknologien til jordfugtmåling har også en betydning for prisen. Den resistive målemetode benytter to spyd stukket ned i jorden. Den kapacitive derimod kan blot laves ved at forlænge printpladen og herved spare omkostninger til montage og produktion af jordspyd.  

\section{Forretningsmodel}
Som forretningsmodel tages der udgangspunkt i Business model canvas \cite{lib:bmc}. Modellen er god til at opstille en række kritiske spørgsmål der klarlægger ideen til konceptet som herved enten styrkes eller viser sig uholdbar. 

\begin{figure}[H]
	\centering
	\includegraphics[width=0.8\textwidth]{../fig/rapport/forretningsmodel/Business-Model-Canvas.jpg}
	\caption{Business Model Canvas}
	\label{fig:BMC}
\end{figure}

\textbf{Value Proposition}\\
Hvad er det produktet gør? Produktet sparer ressourcer af både vand, tid og penge, samtidig med at det optimerer planternes grobetingelser. Eksempelvis er der plantet en ny hæk, kunden skal efterfølgende på ferie i 14 dage og kan derfor ikke sikre sig at hækken ikke tørrer ud i mellemtiden. I denne situation vil produktet spare kunden for bekymringer på sin ferie. Det vil sikre at ingen planter visner og derfor spare kunden for at skulle købe nye planter der kræver udgifter til både tid og penge. Produktet kan også bruges i en urtehave som kunden ellers normalt ville vande med en haveslange. Her sørger produktet for at optimere således der ikke vandes mere end det nødvendige.

\textbf{Customer Segment}\\
Hvem henviser produktet sig til? Et marked kan deles op i flere segmenter. Dette gør det lettere at føre en målrettet kampagne og herved nå ud til flere potentielle købere. Segmentet til dette produkt er som udgangspunkt alle personer som har en have. Dette kan både være ældre som har svært ved at løfte en vandkande eller unge mennesker som ikke har nogen anelse om hvor ofte en have skal vandes.

\textbf{Customer Relationship}\\
Hvordan kommunikeres produktet ud til kunden? Dette gøres via online kampagner, eksempelvis \texttt{google adwords} hvor markedssegmentet identificeres. Her kan der køres målrettede kampagner som når direkte ud til den potentielle kunde via søgeord som "legusterhæk", "såning af græs", "gardner" eller simpelthen blot "automatisk vandingssystem. Herefter bliver kunden ledt ind på webshop'en.

\textbf{Distribution Channels}\\
Hvordan leveres produktet? Dette gøres der ved, at kunden køber produktet på en webshop som efterfølgende leveres med post. Her har kundeoplevelsen og tilfreds højeste prioritet. Ofte er folk ikke hjemme når posten kommer og at hente pakken i et givent posthus åbningstid kan ofte være en besværlig process. Derfor er det vigtigt at give kunden muligheden får at få leveret pakken i en Swip-Box eller lignende, hvor kunden kan hentet pakken når det passer ind i deres tidsplan. Fragten skal være gratis.

\textbf{Key Activities}\\
Hvordan gøres det? Det gøres ved at lave et produkt som er billigere end konkurrenternes samtidig med det er bedre. Dertil kommer at kunde service skal have høj prioritet, således at kunden får en god oplevelse med handlen selvom der mod forventning skulle gå noget galt. Eksempelvis, hvis en kunde modtager et defekt produkt skal det, uden omkostninger kunne returneres og et nyt fremsendes uden beregning. Produktet tænkes solgt igennem en selvejet webshop, hvilket vil minimerer omkostningerne til diverse mellemled i salget. 

\textbf{Key Partners}\\
Hvem er samarbejdspartnere? Dette kan være eksempelvis investorer og producenter af 3.parts moduler hvis dette i fremtiden skal implementeres.  

\textbf{Key Resources}\\
Hvad er der brug for? Der er brug for faglige kompetencer samt finansiering af den første produktion og CE-mærkning. Derudover er brug for en webshop. På sigt kan der blive brug for medarbejdere til at håndtere webshoppen, pakke og sende. Lokaler, evt. et mindre lager. 

\textbf{Cost Structure}\\
Der er i markedsundersøgelsen blevet antaget at et fuldt system maksimalt må koste 500kr med moms. Det giver 400kr total. I Bilag \ref{P-appendix:prisliste}: \nameref{P-appendix:prisliste} på side \pageref{P-appendix:prisliste} i projektdokumentationen er der fundet frem til at en samlet materialepris ligger på omkring 273kr quotet ved 1000 stk, dertil kommer omkostninger til printfremstilling, montage og samling af produktet. Hertil kommer omkostninger til webshop, indpakning, fragt samt løn. 
   
\textbf{Revenue Stream}\\
Hvad er fortjenesten?  I produktes nuværende form er det begrænset hvad der er af fortjeneste. Det ses klart og tydeligt, at skal forretningen løbe rundt kræves der en produktionspris som er lavere end den nuværende. Skal produktet sælges for 400kr kræves der en produktionspris på omkring 100kr. Hvilket pt. kan være svært at indfri. 


% DRC og Contraints
% PCB layout guidelines: følge standerder:
%	IPC_2221A 	Generic Standard on PCB
% 	IPC_7351 	Generic Requirements for SMD and Land Pattern Standard
%	IPC-7251 	Generic Requirements for Through-Hole Design and Land Pattern Standard