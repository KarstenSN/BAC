\chapter{Fremtidigt Arbejde} \label{ch:Fremtidigt_arbejde}

Eftersom at denne prototype kun er 1. iteration af det samlede system er der en del planer for fremtidigt arbejde, det er derfor valgt at dele op i enkelte afsnit afhængig af hvad det omhandler. 

\textbf{Low Battery detect} \newline
Den nuværende indikation af lav batterispænding på sensoren fungerer ved at Transceiveren afgiver et interrupt når dens forsyningsspænding går under 2.8V. Herefter tændes boostconverteren og kontrolboksen får besked om at der er lavt batteri på sensoren. Dog er der stadig meget strøm tilbage på batteriet ved denne spænding og det vil derfor være en fordel at lave en direkte måling af batterispændingen med mikroprocessorens ADC for at alt energi bliver brugt på batteriet.

\textbf{Clockfrekvens} \newline
Yderligere strømbesparelse vil kunne opnås ved at nedsætte clockfrekvensen på mikroprocessoren fra de nuværende 8MHz til den lavest mulige, som stadig kan opretholde de timings- og performancekrav fra Kapitel \ref{P-ch:kravspecifikation}: \nameref{P-ch:kravspecifikation} fra side \pageref{P-ch:kravspecifikation} i projektdokumentation. Dette er planlagt til implementering i iteration 2.

\textbf{Transceiver} \newline
Valget at den implementerede transceiver er bevidst holdt konservativt i 1. iteration. Der kan dog med fordel tænkes i at implementere en nyere tranceiver, eks. af CC1350 fra Texas Instruments. Denne er bevidst valgt til iteration 2, bla. for dens dual band transceiverog indbyggede bluetooth modul. Det vil i sidste ende kunne spare en seperat tranceiver, da der også findes typer af IC'er med indbygget transceiver. 

\textbf{App baseret kontrolinterface} \newline
Det har starten af projektet også være en ønske om at gøre systemet kontrollerbart fra en smartphone via en specialudviklet App. På denne måde spares der omkostninger til display og tastetur på kontrolboksen og kunden kan indstille systemet på afstand. Dette er planlagt til implementering i iteration 2. 
 
\textbf{Specifik afgrøde} \newline
Et sted hvor systemet yderligere kan optimeres er muligheden for at forprogrammere systemet til den specifikke afgrøde hvor sensoren er placeret. Således at systemet overtage vandingen baseret på info om hvor meget vand den gældende afgrøde skal have i den faser sat i forhold til temperatur, pH og jordfugtighed.

\textbf{pH Måling} \newline
Det har også været et ønske at implementere en pH måler i sensoren, således at kunden vil kunne få information omkring eksempelvis hvilke typer planter den givne jord egner sig til, eller kunne sætte alarmer hvis jorden når en kritisk surhed i forhold til den plante der måles ved.
Udvidelserne af systemet med pH-måler og specifikke afgrøder kræver at der programmeres en database til at holde det information som systemet skal basere sine valg på. Dette er derfor ikke planer om at implementere på nuværende tidspunkt.

\textbf{Fejlhåndtering ved slangeafkobling} \newline
Til iteration 2 er det også valgt at implementere en fejlhåndtering således at systemet selv lukke for ventilen hvis slange er faldet af, således at der ikke unødvendigt spildes vand.