\section{Keypad} 

\begin{figure}[H]
	\centering
	\includegraphics[width=0.6\textwidth]{../fig/Klassebeskrivelser/keypad/keypad.png}
	\caption{Klassediagram over keypad}
	\label{fig:keypad}
\end{figure} 

\textbf{Funktionsbeskrivelser}
\begin{table}[H]
\begin{tabularx}{\textwidth}{| l | Z |} \hline
Funktion & \texttt{getKeyPressed} \\\hline
Parametre &  Ingen \\\hline
Returværdi &  char \\\hline
Beskrivelse &  Læser om der bliver trykket på en knap på keypad'en, bliver der trykket på en knap vil funktionen vente indtil knappen er sluppet igen, for at undgå kontaktprel. Funktionen kan returnere \texttt{KEY1},\texttt{KEY2}, \texttt{KEY3} og \texttt{KEY4}.\\\hline
\end{tabularx}
\caption{Metodebeskrivelse for \texttt{getKeyPressed}}
\end{table}

\begin{table}[H]
\begin{tabularx}{\textwidth}{| l | Z |} \hline
Funktion & \texttt{ledOn} \\\hline
Parametre &  Ingen \\\hline
Returværdi &  void \\\hline
Beskrivelse &  Tænder for LED'en på keypad'en \\\hline
\end{tabularx}
\caption{Metodebeskrivelse for \texttt{ledOn}}
\end{table}

\begin{table}[H]
\begin{tabularx}{\textwidth}{| l | Z |} \hline
Funktion & \texttt{ledOff} \\\hline
Parametre &  Ingen \\\hline
Returværdi &  void \\\hline
Beskrivelse &  Slukker for LED'en på keypad'en \\\hline
\end{tabularx}
\caption{Metodebeskrivelse for \texttt{ledOff}}
\end{table}

\textbf{Ekstern funktion}
\begin{table}[H]
\begin{tabularx}{\textwidth}{| l | Z |} \hline
Funktion & \texttt{motorvalveInit} \\\hline
Parametre &  \_motorvalve: struct motorvalve* \\\hline
Returværdi &  void \\\hline
Beskrivelse &  Giver funktionspointere fra .c filen videre til structen. \\\hline
\end{tabularx}
\caption{Metodebeskrivelse for \texttt{motorvalveInit}}
\end{table}

