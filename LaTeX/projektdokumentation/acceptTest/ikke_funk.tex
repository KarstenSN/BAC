I denne sektion testes de ikke-funktionelle krav. Kravende testes individuelt og opstilles med deres ID.

\textbf{Brugervenlighed}

\textbf{US\_01} Tastaturet skal bestå af 4 taktile trykknapper. \\
Der kigges på tastaturet og det konstateres at det består af 4 knapper. Kravet er opfyldt.

\textbf{US\_02} Tastaturet skal indeholde 1 rød LED til indikation af fejl. \\
Der kigges på kontrolboksen og det konstateres at der er monteret en rød LED ved siden af displayet. Kravet er delvist opfyldt da LED'en ikke sidder på tastaturet.

\textbf{US\_03} Tastaturet skal have en størrelse på B:70mm H:23mm. \\
Tastaturet måles med en skydelærer og kravet er opfyldt.

\textbf{US\_04} Displayet skal kunne vise jordfugtighed samt temperatur. \\
Kravet er testet via UC2. Kravet er opfyldt

\textbf{US\_05} Displayet skal kune vise systemets menuer. \\
Kravet er testet via UC3 - UC8. Kravet er opfyldt.

\textbf{US\_06} Displayet skal have en størrelse på B:53mm H:22mm. \\
Displayet måles med en skydelærer og kravet er opfyldt.

\textbf{US\_07} Displayet skal være baggrundsbelyst. \\
Der trykkes på \texttt{SET} knappen på tastaturet og baggrundslyset tændes. Kravet er opfyldt.

\textbf{US\_08} Motorventilen skal kunne tilsluttes 1/2" tomme vandrør. \\
Det forsøges at montere motorventilen på et 1/2" vandrør. Kravet er opfyldt.

\textbf{US\_09} Printplader til kontrolboks og sensor skal monteres i et plasthus med en størrelse på B:83mm H:59mm D:34mm \\
Der måles med en skydelærer på både sensor og kontrolboks. Kravet er opfyldt.

\textbf{Driftsikkerhed}

\textbf{RI\_01} Lav batterispænding (1.8V) skal indikeres med en rød LED på tastaturet. \\
Batterispændingen sænkes til 1.8V og LED'en tændes og teksten "Low Bat" fremkommer på displayet. Kravet er delvist opfyldt da der ikke er indikation af andre fejl.

\textbf{RI\_02} Lav forsyningsspænding (4.5V) skal indikeres med en rød LED på tastaturet. \\
forsyningsspændingen sænkes til 4.5V og LED'en tændes og teksten "Low Power" fremkommer på displayet. Kravet er delvist opfyldt da der ikke er indikation af andre fejl.

\textbf{Performance}

\textbf{PF\_01} Sensorens batteritype skal være 2 stk AAA. \\
Der kigges på sensoren og der ses at der er plads til montage af 2 stk AAA batterier. Kravet er opfyldt.

\textbf{PF\_02} Sensorens batterilevetid skal være minimum 1 år. \\
Batterilevetiden testes ved at lodde en 10$\Omega$ modstand i serie med en ekstern strømforsyning til sensoren. Der måles med et oscilloskop således at sensorens strømtræk kan udregnes via en konvention. Det antages at et standard AAA batteri har en kapacitet på 800mA. 

\begin{equation}
 i = \dfrac{V_{measured}}{10\Omega}
\end{equation}

\begin{figure}[H]
	\centering
	\includegraphics[width=0.7\textwidth]{../fig/acceptest/scope_2_current.png}
	\caption{Spændingen over 10$\Omega$ modstanden ved en forsyningsspænding på 3.3V}
	\label{fig:currentdraw_345}
\end{figure} 

Det ses på Figur \ref{fig:currentdraw_345} at det første der sker er at mikroprocessoren vågner op. Herefter tændes for måle-GND-planet og \texttt{sensorPackage} sendes til kontrolboksen. Dette ses via det skiftende strømtræk. Herefter går der et stykke tid og mikroprocessoren lægges i sleep-mode igen. Det antages at spændingen totalt over hele perioden er 180mV (Proben er sat i x10). Periodetiden er 380ms. Dette bliver omregnet til 18mA. Sænkes forsyningsspændingen til under  2.8V enables boostconverteren og der vil derfor trækkes mere strøm. Se Figur \ref{fig:currentdraw_237}, her trækkes der 40mA i 380ms. Det ses på Figur \ref{fig:currentdraw_3} og \ref{fig:currentdraw_4} at når boost converteren er enable'et vil der også trækkes strøm selvom at mikroprocessoren er i power-down mode. Dette skyldes at boost-converteren hele tiden skal opretholde en spænding på 3.3V. Det antages at der i gennemsnit trækkes 30mA i 380ms hvert kvarter hvilket bliver 1.52 sek hver time. Der vælges at se bort fra boost-converterens strømtræk da dette kun vil være i slutningen af batteriets levetid den vil være enable'et. Batterilevetiden kan nu approximeres:

\begin{equation}
lifetime = \dfrac{800mA}{30mA\cdot 1.52s / 60s\cdot 60min}\cdot \dfrac{1}{24hour\cdot 360days} = 7.3 year
\end{equation} 

Det skal dog siges at det er en grov approximation og at beregningerne kan laves mere præcis. Men dette viser at kravet om 1 års leve tid er opfyldt. 

\begin{figure}[H]
	\centering
	\includegraphics[width=0.7\textwidth]{../fig/acceptest/scope_3_current.png}
	\caption{Spændingen over 10$\Omega$ modstanden ved en forsyningsspænding på 1.8V}
	\label{fig:currentdraw_237}
\end{figure}  

\begin{figure}[H]
	\centering
	\includegraphics[width=0.7\textwidth]{../fig/acceptest/scope_5_current.png}
	\caption{Strømtræk fra boost-converteren}
	\label{fig:currentdraw_3}
\end{figure}  

\begin{figure}[H]
	\centering
	\includegraphics[width=0.7\textwidth]{../fig/acceptest/scope_4_current.png}
	\caption{Strømtræk fra boost-converteren}
	\label{fig:currentdraw_4}
\end{figure}  

\textbf{PF\_03} Sensoren skal fungere ned til 1.8V batterispænding.\\
Her henvises til modultesten i sektion \ref{sec:psu_modultest} fra side \pageref{sec:psu_modultest}. Kravet er opfyldt

\textbf{PF\_04} Minimum 10 instanser af systemet skal kunne køre samtidig. \\
Kravet er opfyldt da sensoren og kontrolboksen skal parres og herved får tildelt et tilfældigt ID mellem 0 og 255. 

\textbf{PF\_05} Temperatursensoren skal kunne måle i området fra $-20^\circ$ til $-70^\circ$. \\
Sensoren ligges i en fryser og det konstateres at sensoren fryser i dens software når temperaturen kommer under 0 grader. Kravet er ikke opfyldt.

\textbf{PF\_06} Jordfugtigheden skal kunne måles med en præcision på minimum 5\%. \\
Kravet testes ikke da målekredsløbet til jordfugtigheden ikke er blevet færdigdesignet. Der er i stedet blevet implementeret et midlertidigt kredsløb som ikke vil leve op til kravet. Kavet er derfor ikke opfyldt.
 
\textbf{PF\_07} Kommunikation mellem sensor og kontrolboks skal ske på under 1 sekund. \\
Her henvises til krav \textbf{PF\_02}. Det ses på Figur \ref{fig:currentdraw_345} at sensoren sender sin pakke til kontrolboksen som modtages igen af sensoren. Hele periodetiden er aflæst til 380ms og kravet er derfor opfyldt.

\textbf{PF\_08} Jordfugtighed og temperatur skal kunne vises med 1 decimal på kontrolboksen. \\
Der kigges på displayet og fugtighed og temperatur vises med 1 decimal. Kravet er opfyldt.

\textbf{PF\_09} Sensoren og kontrolboks skal være funktionsdygtig i temperaturområdet $0^\circ$ til $-70^\circ$. \\
Det antages at kravet er opfyldt da alle komponenter på både kontrolboks og sensor opfylder kravet.

\textbf{Vedligeholdelse}

\textbf{SP\_01} Systemet skal kun vedligeholdes i forbindelse med batteriskift. \\
Kravet er opfyldt da der ikke skal vedligeholdes på andre tidspunker.

\textbf{SP\_02} Batteriet på sensoren skal være let tilgængelig. \\
Kravet er delvist opfyldt da der skal skrues et låg af for at kunne tilgå batterierne.

\textbf{SP\_03} Brugeren skal selv kunne udskifte batteriet på sensoren. \\
Kravet er opfyldt