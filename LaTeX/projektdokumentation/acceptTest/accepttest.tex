\chapter{Accepttest} \label{ch:Accepttest}

\section{Funktionelle Krav}

For at teste de funktionelle krav tages der udgangspunkt i use-casene i sektion \ref{sec:use_cases} side \pageref{sec:use_cases}. Visse Use Cases kan dog kun testes hvis andre Use Cases er blevet udført og indstillet til en bestemt indstilling. Dette vil fremgå af selve accepttesten under forudsætning. Under test af en Use Case vil der samtidig kunne testes om de funktionelle krav fra Sektion \ref{sec:funktionelle_krav} side \pageref{sec:funktionelle_krav} er implementeret. Sammenhæng mellem disse og Use Casene ses i Tabel \ref{tbl:kravTabel}. Det er dog ikke alle krav der kan testes i Use Casene og disse vil derfor blive testet separat. 
\todo[inline]{opdater funktionelle ikke funktionelle krav}
\begin{table}[H]
\centering
\begin{tabularx}{\textwidth-5cm}{| C{2 cm} | X | X | X | X | X | X | X | X | X | X | X | X |}
\hline
Use-case/krav &\begin{turn}{90}{UC1}\end{turn} % tænd system
			  &\begin{turn}{90}{UC2}\end{turn} % aflæs data
			  &\begin{turn}{90}{UC3}\end{turn} % indstil fugtighed
			  &\begin{turn}{90}{UC4}\end{turn} % indstil åbningstid
			  &\begin{turn}{90}{UC5}\end{turn} % indstil vandingsinterval
			  &\begin{turn}{90}{UC6}\end{turn} % abn/luk manuelt
			  &\begin{turn}{90}{UC7}\end{turn} % indstil vandingstidspunkt
			  &\begin{turn}{90}{UC8}\end{turn} \\ \hline % par sensor og kontrolboks
	   MC\_01 & X & X & X & X & X & X & X & X  \\ \hline
	   MC\_02 &   &   &   &   &   &   &   &    \\ \hline
	   MC\_03 &   &   &   &   &   &   &   &    \\ \hline
	   MC\_04 &   &   &   &   &   &   &   &    \\ \hline
	   MC\_05 &   &   &   &   &   &   & X &    \\ \hline
	   MC\_06 &   &   &   &   &   &   &   &    \\ \hline
	   MC\_07 &   &   &   &   &   &   &   &    \\ \hline
	   MC\_08 & X &   &   &   &   &   &   &    \\ \hline
	   MC\_09 &   &   &   &   &   &   &   &    \\ \hline
	   MC\_10 &   &   &   &   &   &   &   &    \\ \hline
	   EL\_01 & X &   &   &   &   &   &   &    \\ \hline
	   EL\_02 &   &   & X & X & X & X & X &    \\ \hline
	   EL\_03 & X &   &   &   &   &   &   &    \\ \hline
	   CM\_01 &   &   & X & X & X & X & X & X  \\ \hline
	   CM\_02 &   &   &   &   &   &   &   &    \\ \hline
	   CM\_03 &   &   &   &   &   &   &   & X  \\ \hline
	   UX\_01 & X & X & X & X & X & X & X & X  \\ \hline
	   UX\_02 &   &   &   &   &   &   &   &    \\ \hline
	   UX\_03 &   &   &   &   &   & X &   &    \\ \hline
	   UX\_04 &   &   &   & X &   &   &   &    \\ \hline
	   UX\_05 &   &   & X &   &   &   &   &    \\ \hline
	   UX\_06 &   &   &   &   & X &   &   &    \\ \hline
	   UX\_07 &   &   &   &   &   &   & X &    \\ \hline

\end{tabularx}
\caption{Tabel som viser hvilket krav der testes via hvilke Use Cases}
\label{tbl:kravTabel}
\end{table}

\textbf{MC\_02} \textit{Could} Systemet skal kunne pre-indstilles til en specifik afgrøde \\
Er ikke blevet implementeret. 

\textbf{MC\_03} \textit{Would} Systemet skal kunne betjenes via en smartphone-applikation \\
Er ikke blevet implementeret.

\textbf{MC\_04} \textit{Would} Systemet skal kunne tilsluttes flere sensorer \\
Er ikke blevet implementeret.


%Her vælges bredde på hele tabellen samt hvilken FIL selve tabellen defineres i.
\subsubsection{Use Case 1: Tænd system}
\LTXtable{\textwidth}{accepttest/uc1}
\clearpage
\subsubsection{Use Case 2: Aflæs fugtighed og temperatur}
\LTXtable{\textwidth}{accepttest/uc2}
\clearpage
\subsubsection{Use Case 3: Indstil ønsket fugtighed i jorden}
\LTXtable{\textwidth}{accepttest/uc3} 
\clearpage
\subsubsection{Use Case 4: Indstil åbningstid}
\LTXtable{\textwidth}{accepttest/uc4}
\clearpage
\subsubsection{Use Case 5: Indstil vandingsinterval}
\LTXtable{\textwidth}{accepttest/uc5} 
\clearpage
\subsubsection{Use Case 6: Abn/luk for ventil manuelt}
\LTXtable{\textwidth}{accepttest/uc6}
\clearpage
\subsubsection{Use Case 7: Indstil vandingstidspunkt}
\LTXtable{\textwidth}{accepttest/uc7}
%\clearpage
\subsubsection{Use Case 8: Par sensor og kontrolboks}
\LTXtable{\textwidth}{accepttest/uc8}
\clearpage

\clearpage

\section{Ikke-funktionelle krav}
\LTXtable{\textwidth}{Accepttest/ikke_funk}

\clearpage