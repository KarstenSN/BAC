%%%%%%%%%%%%%%%%%%%%%%%%%%%%%%%%%%%%%%%%%%%%%%%%%%%%%%%%%%%%%%%%%%%%%%%%%%%%%%%%%%%%%%%%%%%%%%%%%%%%%%%%%%%%%%%%%%
%
%							Software Design - Kontrolboks
%
%%%%%%%%%%%%%%%%%%%%%%%%%%%%%%%%%%%%%%%%%%%%%%%%%%%%%%%%%%%%%%%%%%%%%%%%%%%%%%%%%%%%%%%%%%%%%%%%%%%%%%%%%%%%%%%%%%

\section{Softwaredesign} \label{sec:softwaredesign}
\subsection{Kontrolboks}

\begin{figure}[H]
	\centering
	\includegraphics[width=0.9\textwidth]{../fig/Systemarkitektur/flowDiagramKontrolboks.pdf}
	\caption{Flowdiagram over kontrolboksen}
	\label{fig:FlowdiagramKontrolboks}
\end{figure}

På Figur \ref{fig:FlowdiagramKontrolboks} ses flowdiagram over softwaren i kontrolboksen. Ved efter power-up vises hovedmenuen, hvor der vises "No data" hvis der endnu ikke er forbindelse til en sensor. Når der er opnået forbindelse vises de seneste sensordata på display'et. Efter start-up er det muligt for brugeren at opsætte systemet som ønsket, samt tilgå systemets indstillinger som beskrevet i Use Case'ne i afsnit \ref{sec:use_cases} fra sidde \pageref{sec:use_cases}. Herefter entres en while-løkke hvor der søges efter ny data fra sensoren. På baggrund af det modtaget data kan kontrolboksen åbne eller lukke for ventilen. systemet befinder sig i denne løkke indtil at brugeren vælger at frakoble forsyningsspændende til kontrolboksen. 


\begin{figure}[H]
	\centering
	\includegraphics[width=0.9\textwidth]{../fig/Systemarkitektur/klassediagramKontrolboks.pdf}
	\caption{Klassediagram over kontrolboksen}
	\label{fig:cdKontrolboks}
\end{figure}

\textbf{Controller: mikroprocessor} \\
\texttt{mikroprocessor}: Klasse der behandler data fra de resterende klasser. \newline
Den opdaterer data i \texttt{controlboxPackage} som herefter sendes til \texttt{transceiver}. \texttt{mikroprocessor} kommunikerer direkte med \texttt{transceiver} i forbindelse med initiering og parring. \texttt{mikroprocessor} kommunikerer herudover med \texttt{keypad}, \texttt{display}, \texttt{motorvalve}, samt \texttt{clock}, \texttt{isr} og \texttt{settings}. \\


\textbf{Boundary: transceiver} \\
\texttt{transceiver}: Driver der skal håndtere kommunikation med \texttt{transceiveren}. \newline
Den sender \texttt{controlboxPackage} og modtager \texttt{sensorPackage} fra Sensor'en. Den initeres af \texttt{mikroprocessoren} ved start up. Her skal opsættes kommunikationsfrekvens, krystal-load, datarate samt low battery-thredshold.


\textbf{Boundary: keypad} \\
\texttt{keypad}: Driver der skal håndtere brugerinterfacet til tastaturets 4 trykknapper, samt LED til fejlindikation. \\


\textbf{Boundary: i2cDisplay} \\
\texttt{i2cDisplay}: Driver der skal håndtere kommunikation med I2C Display'et, herunder også baggrundsbelysning. \\


\textbf{Boundary: motorvalve} \\
\texttt{motorvalve}: Driver der skal kunne åbne og lukke for motorventilen samt aflæse status for positionen.


\textbf{Domain: clock} \\
\texttt{clock}: Data-struct som indeholder intern tidsreference. Denne clock startes når kontrolboksen tilsluttes forsyningsspænding og benyttes eks. som refernce til \textit{UC5: Indtil vandingsinterval}, samt de generelle tider i system. 

\textbf{Domain: isr} \\
\texttt{isr}: Interrupt-rutine der skal behandle interrupts fra transceiveren. Når der modtages en pakke fra sensoren skal rutinen modtage en byte af gangen som lægges i \texttt{sensorPackage}. Modtages der interrupt om for lav forsyningsspænding skal der reageres herpå. Endvidere skal rutinen tælle \texttt{clock} op hvert sekund.


\textbf{Domain: settings} \\
\texttt{settings}: Data-struct som skal indeholde brugerens opsætning af systemet fra Use cases herunder, fugtighed, åbningstid og tidsbaseret vandingssekvens.   


\textbf{Domain: controlboxPackage} \\
\texttt{Controlbox Package}: Datapakke på 5 bytes som skal indeholde følgende: 2 startbytes, Kontrolboksens ID samt 2 bytes checksum. \\


\textbf{Domain: sensorPackage} \\
\texttt{Sensor Package}: Datapakke på 11 bytes som skal indeholde følgende: 2 startbytes, Sensor ID, batteristatus, 2 temperaturbytes, 2 fugtighedsbytes, en byte til lysstyrke, samt 2 bytes checksum.

\todo[inline]{Definer EKSTERNE variable (interface) til SW-blokke i Kontrolboks}

\newpage


%%%%%%%%%%% %%%%%%%%%%%%%%%%%%%%%%%%%%%%%%%%%%%%%%%%%%%%%%%%%%%%%%%%%%%%%%%%%%%%%%%%%%%%%%%%%%%%%%%%%%%%%%%%%%%%%%%
%
%							Software Design - Sensor
%
%%%%%%%%%%%%%%%%%%%%%%%%%%%%%%%%%%%%%%%%%%%%%%%%%%%%%%%%%%%%%%%%%%%%%%%%%%%%%%%%%%%%%%%%%%%%%%%%%%%%%%%%%%%%%%%%%%

\subsection{Sensor}

\begin{figure}[H]
	\centering
	\includegraphics[width=0.4\textwidth]{../fig/Systemarkitektur/flowDiagramSensor.pdf}
	\caption{Flowdiagram over sensoren}
	\label{fig:FlowdiagramSensor}
\end{figure}

På Figur \ref{fig:FlowdiagramSensor} ses flowdiagram over softwaren i sensoren. Ved start initieres de interne indstillinger samt de enheder som der skal kommunikeres med. Derefter sættes en wakeUp timer der holder styr på hvornår sensoren skal vækkes for at foretage målinger. Dernæst sætte en Watch Dog Timer for at forhindre at softwaren ender i deadlock. Herefter foretages en indhentning af data fra hhv. temperatursensoren, lyssensoren samt jordfugtmåleren. Denne datapakke transmitteres til kontrolboksen, skulle dette mislykkes første gang, forsøges der i alt at transmittere datapakken 10 gange. Efter endt transmission (eller 10 x mislykket transmission) disables watch Dog Timeren og softwaren lægges i sleepmode. Softwaren vækkes fra sleepmode af wakeUp Timeren. Denne sekvens forsætter indtil brugeren vælger at frakoble batterierne, eller at batterierne selv løber tør.  


\begin{figure}[H]
	\centering
	\includegraphics[width=0.9\textwidth]{../fig/Systemarkitektur/klassediagramSensor.pdf}
	\caption{Klassediagram over sensor}
	\label{fig:cdSensor}
\end{figure}

\textbf{Controller: microprocessor} \\
\texttt{mikroprocessor}: Klasse der behandler data fra de resterende klasser. Den opdaterer data i \texttt{sensorPackage} som herefter sendes til \texttt{transceiver}. \texttt{mikroprocessor} kommunikerer direkte med \texttt{transceiver} i forbindelse med initiering og parring. \texttt{mikroprocessor} kommunikerer herudover med \texttt{soilProbe}, \texttt{temperatureSensor} og \texttt{lightSensor}. \\


\textbf{Boundary: transceiver} \\
\texttt{transceiver}: Driver der styrer kommunikation imellem kontrolboks og sensor. Den håndterer afsendelse af hhv. \texttt{controlboxPackage} og \texttt{sensorPackage}


\textbf{Boundary: soilSensor} \\
\texttt{soilSensor}: Driver der styrer hardwarekredsløbet til jordfugtmåling. \\



\textbf{Boundary: temperatureSensor} \\
\texttt{temperatureSensor}: Driver til temperatursensor \\


\textbf{Boundary: lightSensor} \\
\texttt{lightSensor}: Driver der styrer hardwarekredsløbet til lysmåling. \\


\textbf{Domain: isr} \\
\texttt{isr}: Interrupt-rutine der skal behandle interrupts fra transceiveren. Når der modtages en pakke fra kontrolboksen skal rutinen modtage en byte af gangen som lægges i \texttt{controlboxPackage}. Modtages der interrupt om for lav batterispænding skal der reageres herpå, ved at sætte \texttt{batteryStatus} til 1 i \texttt{sensorPackage}.


\textbf{Domain: controlboxPackage} \\
\texttt{controlboxPackage}: Datapakke på 5 bytes som indeholder følgende: 2 startbytes, Kontrolboksens ID samt 2 bytes checksum \\


\textbf{Domain: sensorPackage} \\
\texttt{sensorPackage}: Datapakke på 11 bytes som indeholder følgende: 2 startbytes, Sensorens ID, batteristatus, 2 temperaturbytes, 2 fugtighedsbytes, lysstyrke samt checksum.


\todo[inline]{Definer EKSTERNE variable (interface) til SW-blokke i Sensor}
\newpage