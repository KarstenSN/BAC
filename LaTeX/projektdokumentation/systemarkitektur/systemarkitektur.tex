\chapter{Systemarkitektur} \label{sec:sysark}

\begin{figure}[H]
\centering
\includegraphics[width=0.8\textwidth]{../fig/Systemarkitektur/BDDsystem.pdf}
\caption{Blokdiagram af det samlede system}
\label{fig:BDDsystem}
\end{figure}

I Figur \ref{fig:BDDsystem} ses et Blok diagram over det samlede system. Det ses at det der betegnes som systemet er en sensor og en kontrolboks. Se blokdiagram over sensoren og kontrolboksen i figur \ref{fig:BDDsensor} og \ref{fig:BDDkontrolboks}.
\raggedbottom

\begin{figure}[H]
\centering
\includegraphics[width=1.0\textwidth]{../fig/Systemarkitektur/BDDkontrolboks.pdf}
\caption{Blokdiagram af kontrolboksen}
\label{fig:BDDkontrolboks}
\end{figure}

I Figur \ref{fig:BDDkontrolboks} ses der et blokdiagram over kontrolboksen. \texttt{HF-transceiver} er blokken som står for kommunikation mellem sensor og kontrolboks. Kommunikationen sker ved en trådløs forbindelse og derfor indeholder blokken også en antenne. \texttt{Mokroprocessor} tager sig af at behandle data fra alle de andre blokke og give brugeren besked via \texttt{Display} om nuværende status samt mulighed for at navigere rundt i indstillingerne. \texttt{Tastatur} bruges af brugeren til at indstille systemet til hans ønsker. \texttt{Motorventil} åbner og lukker for vandtilførslen og styrres via \texttt{Mikroprocessor}. \texttt{Plasthus} omkranser hele elektronikken således det bliver beskyttet for både vand og støv. \texttt{DC-connector} bruges til at koble en spændingsforsyning til kontrolboksen. \texttt{Batteri} forsynet sensoren med strøm og \texttt{Plasthus} er en vandtæt plastkasse som omkranser elektronikken.  

\begin{figure}[H]
\centering
\includegraphics[width=1.0\textwidth]{../fig/Systemarkitektur/BDDsensor.pdf}
\caption{Blokdiagram af sensoren}
\label{fig:BDDsensor}
\end{figure}

I Figur \ref{fig:BDDsensor} ses der et blokdiagram over sensoren. \texttt{Jordfugtmåler} måler fugten i jorden og denne blok er monteret på \texttt{Plasthus} således at den kan stikkes i jorden, uden at de andre blokke bliver udsat for vand og støv. \texttt{Mikroprocessor} behandler data fra de andre blokken og sende det til kontrolboksen via \texttt{HF-transceiver}. \texttt{Temperatur-føler} måler overflade temperaturen og \texttt{Lysmåler} måler lysintensiteten således \texttt{Mikroprocessor} kan udregne om det er morgen eller aften.   
