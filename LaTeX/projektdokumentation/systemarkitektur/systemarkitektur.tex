\chapter{Systemarkitektur} \label{sec:sysark}

\begin{figure}[H]
	\centering
	\includegraphics[width=0.8\textwidth]{../fig/Systemarkitektur/BDDsystem.pdf}
	\caption{Blokdiagram af det samlede system}
	\label{fig:BDDsystem}
\end{figure}

I Figur \ref{fig:BDDsystem} ses et blok definitions diagram over det samlede system. Det samlede system indeholder en sensor og en kontrolboks. Seperate blokdiagramer over sensoren og kontrolboksen ses i figur \ref{fig:BDDsensor} og \ref{fig:BDDkontrolboks}.

\raggedbottom

\section{Kontrolboks}

Herunder opstilles der kravene til de enkelte blokke som udgør kontrolboksen.

\begin{figure}[H]
	\centering
	\includegraphics[width=1.0\textwidth]{../fig/Systemarkitektur/BDDkontrolboks.pdf}
	\caption{Blokdiagram af kontrolboksen}
	\label{fig:BDDkontrolboks}
\end{figure}

\textbf{Mikroprocessor} \\
Mikroprocessoren skal indeholde perifere enheder til at kommunikere over SPI og I2C. Den skal have minimum 3 analoge I/O-porte samt en analog reset pin. Den skal forsynes med 3.3VDC $ \pm 10\% $ og kunne programmeres eksternt. 

\textbf{Motorventil} \\
Motorventilen skal bestå af en motor som forsynes med 5VDC $ \pm 10\%$ og have to kontakter der afbrydes når ventilen er i hhv. en fuldt lukket eller åben position. Den skal kunne tilsluttes 1/2" vandrør og være lavet af et solidt materiale. Motorventilen skal monteres på kontrolboksen. 

\textbf{Display} \\
Displayet skal forsynes med 3VDC $ \pm 10\%$ og skal kommunikere via I2C. Displayet skal bestå af 2x16 karakter og være i størrelsen 50mmx20mm $\pm 10\%$

\textbf{HF-Transceiver} \\
HF-Transceiveren skal kommunikere med mikroprocessoren via SPI. Den skal inkludere en antenne som skal indbygges i kontrolboksen. Transceiverens indstillinger skal initieres af mikroprocessoren ved opstart. 

\textbf{Strømforsyning} \\
Strømforsyningen skal inkludere en DC-connector, monteret på kontrolboksen. Den skal have ESD beskyttelse og beskytte mod omvendt polarisation og strøm-overforbrug. Den skal kunne acceptere 5VDC $\pm 10\%$ inputspænding og herved indeholde en spændingsregulator som giver 3.3VDC $\pm 10\%$. 

\textbf{Tastatur} \\
Tastaturet skal bestå af 4 trykknapper samt en rød LED til indikation af fejl.  


\section{Sensor}

Herunder opstilles der kravende til de enkelte blokke som udgør sensoren i Figur \ref{fig:BDDsensor}.

\begin{figure}[H]
	\centering
	\includegraphics[width=1.0\textwidth]{../fig/Systemarkitektur/BDDsensor.pdf}
	\caption{Blokdiagram af sensoren}
	\label{fig:BDDsensor}
\end{figure}

\textbf{Mikroprocessor} \\
Mikroprocessoren skal indeholde perifere enheder til at kommunikere over SPI og I2C. Den skal have minimum 3 analoge porte, have en analog reset pin og den skal kunne programmeres eksternt. Den skal forsynes med 3.3VDC $ \pm 10\%$ samt indeholde en min. 10bits analog til digital konverter. 

\textbf{Jordfugt måler} \\
Jordfugt måleren skal inkludere et jordspyd som stikkes ned i jorden for at måle jordfugtigheden. Den skal give en analog spænding ud mellem 0-3.3VDC som indikerer fugtigheden. 

\textbf{Temperaturføler}
Temperaturføleren skal kommunikere via I2C og skal have en præcision på minimum $\pm 0.5^\circ$C og en decimalafstand på minimum $0.1^\circ$C.

\textbf{HF-Transceiver} \\
HF-Transceiveren skal kommunikere med mikroprocessoren via SPI. Den skal inkludere en antenne som skal indbygges i kontrolboksen. Transceiverens indstillinger skal initieres af mikroprocessoren ved opstart. 

\textbf{Batteriholder} \\
Batteriholderen skal kunne indeholde 2 batterier af typen AAA med en spænding på 1.5VDC hver.  

\textbf{Lysmåler} \\
Lysmåleren skal kunne detektere nat eller dag. Lysintensitet ved tusmørke fastlægges ved test og denne sættes til  referencespænding på 2VDC $\pm 50\%$.


\begin{figure}[H]
	\centering
	\includegraphics[width=1.0\textwidth]{../fig/Systemarkitektur/IBDkontrolboks.pdf}
	\caption{Internt blokdiagram af kontrolboksen}
	\label{fig:IBDkontrolboks}
\end{figure}

\textbf{Signalbeskrivelser}
\begin{table}[H]
	\begin{tabularx}{\textwidth}{| l | l | Z | p{3 cm} |} \hline
	\textbf{Signal} & \textbf{Type} & \textbf{Beskrivelse} & \textbf{Tolerance} \\ \hline
	5VDC 	
		& P1 	
		& Forsyningspænding fra adaptor								
		& 5VDC $\pm 10\%$					\\ \hline
			
	3.3VDC 	
		& P2 	
		& Forsyningsspænding fra strømforsyning $ V_{dd} $						
		& 3.3VDC $\pm 10\%$ 				\\ \hline
			
	SPI 
		& K1 	
		& Kommunikation: HF-transceiver og mikroprocesor 	
		& CMOS-logic: 						\newline 
			Low:  $0V - 1/3V_{dd}$ 			\newline
			High: $2/3V_{dd} - V_{dd} $		\\ \hline
			 
	I2C	 	
		& K3 	
		& Kommunikation: Display og mikroprocessor 		
		& CMOS-logic: 						\newline 
			Low:  $ 0V-1/3V_{dd} $ 			\newline
			High: $ 2/3V_{dd}-V_{dd} $		\\ \hline
	touch 	
		& B1 	
		& Bus indeholdende 5 forbindelser til tastaturet 			
		& $ 0-V_{dd} \pm 10\%$				\\ \hline
			
	openBreak 
		& A1  
		& Signal der åbner motorventilen. A2 holdes lav. 
		  For at bremse motoren er begge signaler høje 			
		& Analog logic:					 	\newline 
		 	Low:  $ 0V-0.6V_{dd} $ 			\newline
			High: $ 1V_{dd}-V_{dd} $		\\ \hline
	closeBreak 
		& A2 
		& Signal der lukker for motorventilen.Det kræves at A1 samtidig går lav. 						
		& Analog logic:					 	\newline 
		 	Low:  $ 0V-0.6V_{dd} $ 			\newline
			High: $ 1V_{dd}-V_{dd} $		\\ \hline
	Open 	
		& A3 	
		& Signal som er høj når motorventilen er åben				
		& $ 0-V_{dd} \pm 10\%$				\\ \hline
	Close 	
		& A4 	
		& Signal som er høj når motorventilen er lukket				
		& $ 0-V_{dd} \pm 10\%$				\\ \hline
	HF 		
		& HF1 	
		& Luftbåret, højfrekvent signal	
		& \\ \hline
	IRQ 	
		& A3 	
		& Interrupt fra HF-transceiveren 
		& $ 0-V_{dd} \pm 10\%$				\\ \hline
	Data 	
		& A4 	
		& Analogt data-signal til HF-transceiveren
		& $ 0-V_{dd} \pm 10\%$				\\ \hline
	Reset 	
		& A5 	
		& Reset reset-signal til HF-transceiveren
		& $ 0-V_{dd} \pm 10\%$				\\ \hline
	RESET 	
		& K4 	
		& Reset signal til mikroprocessoren
		& $ 0-V_{dd} \pm 10\%$				\\ \hline
	PROG 	
		& K3 	
		& Programmerings interface til mikroprocessoren
		& $ 0-V_{dd} \pm 10\%$				\\ \hline
	\end{tabularx}
	\caption{Signalbeskrivelse for \texttt{kontrolboks}}
\end{table}

\begin{figure}[H]
	\centering
	\includegraphics[width=1.0\textwidth]{../fig/Systemarkitektur/IBDsensor.pdf}
	\caption{Internt blokdiagram af sensoren}
	\label{fig:IBDsensor}
\end{figure}

\textbf{Signalbeskrivelser}
\begin{table}[H]
	\begin{tabularx}{\textwidth}{| l | l | Z | p{4 cm} |} \hline
	\textbf{Signal} & \textbf{Type} & \textbf{Beskrivelse} & \textbf{Tolerance} \\ \hline
		3VDC 	
			& P1 
			& Batterispænding $ V_{dd} $												
			& 3.3V $ \pm 10\%$					\\ \hline
		SPI  	
			& K1 
			& Kommunikation: HF-transceiver og Mikroprocesor
			& CMOS-logic: 						\newline 
				Low:  $ 0V - 1/3V_{dd} $ 		\newline
				High: $ 2/3V_{dd} - V_{dd} $	\\ \hline
		I2C	 	
			& K2 
			& Kommunikation: Temperaturføler og Mikroprocessor
			& CMOS-logic: 						\newline 
				Low:  $ 0V - 1/3V_{dd} $ 		\newline
				High: $ 2/3V_{dd} - V_{dd} $	\\ \hline
		light 	
			& A1 
			& Spændingsindikation af lysintensitet på kontrolboksen
			& $ 0-V_{dd} $ 	\\ \hline
		humidt 	
			& A2 
			& Spændingsindikation jordfugtighed		
			& $ 0-V_{dd} $ 	\\ \hline
		HF 		
			& H1 
			& Luftbåret, højfrekvent signal			
			&  			\\ \hline
		IRQ 	
			& A3
			& Interrupt fra HF-transceiveren				
			& $ 0-V_{dd} $ 	\\ \hline
		Data 	
			& A4 
			& Analogt signal til HF-transceiveren				
			& $ 0-V_{dd} $ 	\\ \hline
		Reset 	
			& A5 
			& Reset signal til HF-transceiveren				
			& $ 0-V_{dd} $ 	\\ \hline
		RESET 	
			& K4 
			& Reset signal til mikroprocessoren				
			& $ 0-V_{dd} $ 	\\ \hline
		PROG 	
			& K3 
			& Programmerings interface til mikroprocessoren
			& $ 0-V_{dd} $ 	\\ \hline
	\end{tabularx}
	\caption{Signalbeskrivelse for \texttt{Sensor}}
\end{table}