\chapter{Systemarkitektur} \label{sec:sysark}

\begin{figure}[H]
\centering
\includegraphics[width=0.8\textwidth]{../fig/Systemarkitektur/BDDsystem.pdf}
\caption{Blokdiagram af det samlede system}
\label{fig:BDDsystem}
\end{figure}

I Figur \ref{fig:BDDsystem} ses et Blok diagram over det samlede system. Det ses at det der betegnes som systemet er en sensor og en kontrolboks. Se blokdiagram over sensoren og kontrolboksen i figur \ref{fig:BDDsensor} og \ref{fig:BDDkontrolboks}.
\raggedbottom

\begin{figure}[H]
\centering
\includegraphics[width=1.0\textwidth]{../fig/Systemarkitektur/BDDkontrolboks.pdf}
\caption{Blokdiagram af kontrolboksen}
\label{fig:BDDkontrolboks}
\end{figure}

I Figur \ref{fig:BDDkontrolboks} ses der et blokdiagram over kontrolboksen. \texttt{HF-transceiver} er blokken som står for kommunikation mellem sensor og kontrolboks. Kommunikationen sker ved en trådløs forbindelse og derfor indeholder blokken også en antenne. \texttt{Mokroprocessor} tager sig af at behandle data fra alle de andre blokke og give brugeren besked via \texttt{Display} om nuværende status samt mulighed for at navigere rundt i indstillingerne. \texttt{Tastatur} bruges af brugeren til at indstille systemet til hans ønsker. \texttt{Motorventil} åbner og lukker for vandtilførslen og styrres via \texttt{Mikroprocessor}. \texttt{Plasthus} omkranser hele elektronikken således det bliver beskyttet for både vand og støv. \texttt{DC-connector} bruges til at koble en spændingsforsyning til kontrolboksen. \texttt{Batteri} forsynet sensoren med strøm og \texttt{Plasthus} er en vandtæt plastkasse som omkranser elektronikken.  

\begin{figure}[H]
\centering
\includegraphics[width=1.0\textwidth]{../fig/Systemarkitektur/BDDkontrolboks.pdf}
\caption{Blokdiagram af kontrolboksen}
\label{fig:BDDkontrolboks}
\end{figure}

I Figur \ref{fig:BDDkontrolboks} ses der et blokdiagram over sensoren. \texttt{Jordfugtmåler} måler fugten i jorden og denne blok er monteret på \texttt{Plasthus} således at den kan stikkes i jorden, uden at de andre blokke bliver udsat for vand og støv. \texttt{Mikroprocessor} behandler data fra de andre blokken og sende det til kontrolboksen via \texttt{HF-transceiver}. \texttt{Temperatur-føler} måler overflade temperaturen og \texttt{Lysmåler} måler lysintensiteten således \texttt{Mikroprocessor} kan udregne om det er morgen eller aften.

\begin{figure}[H]
\centering
\includegraphics[width=1.0\textwidth]{../fig/Systemarkitektur/IBDkontrolboks.pdf}
\caption{Internt blokdiagram af kontrolboksen}
\label{fig:IBDkontrolboks}
\end{figure}

\textbf{Signalbeskrivelser}
\begin{table}[H]
		\begin{tabularx}{\textwidth}{| l | l | Z | l |} \hline
			\textbf{Signal} & \textbf{Type} & \textbf{Beskrivelse} 							& \textbf{Tolerance} 	\\ \hline
			3.3VDC 	& P1 	& Forsyningsspænding 											& 3.2V-3.4V \\ \hline
			SPI 	& K1 	& SPI kommunikation mellem HF-transceiver og mikroprocesor 		& 			\\ \hline
			I2C	 	& K3 	& I2C kommunikation mellem display og mikroprocessoren 			&  			\\ \hline
			touch 	& B1 	& Bus indeholdende 5 forbindelser til tastaturet 				& 0-3.3V 	\\ \hline
			openBreak 	& A1 & Signal der åbner motorventilen. Det kræves at A2 samtidig går lav. 
								For at bremse motoren er begge signaler høje 				& 0-3.3V 	\\ \hline
			closeBreak 	& K2 & Signal der lukker for motorventilen. 
								Det kræves at A1 samtidig går lav. 							& 0-3.3V	\\ \hline
			Open 	& A3 	& Signal som er høj når motorventilen er åben					& 0-3.3V 	\\ \hline
			Close 	& A4 	& Signal som er høj når motorventilen er lukket					& 0-3.3V 	\\ \hline
			HF 		& HF1 	& Højfrekvent signal som er luftbåret 							&  			\\ \hline
			IRQ 	& A3 	& Interrupt fra HF-transceiveren 								& 0-3.3V  	\\ \hline
			Data 	& A4 	& Analogt signal til HF-transceiveren 							& 0-3.3V  	\\ \hline
			Reset 	& A5 	& Reset signal til HF-transceiveren 							& 0-3.3V  	\\ \hline
			RESET 	& K4 	& Reset signal til mikroprocessoren 							& 0-3.3V	\\ \hline
			PROG 	& K3 	& Programmerings interface til mikroprocessoren 				& 0-3.3V 	\\ \hline
		\end{tabularx}
		\caption{Signalbeskrivelse for \texttt{kontrolboks}}
\end{table}

\begin{figure}[H]
\centering
\includegraphics[width=1.0\textwidth]{../fig/Systemarkitektur/IBDsensor.pdf}
\caption{Internt blokdiagram af sensoren}
\label{fig:IBDsensor}
\end{figure}

\textbf{Signalbeskrivelser}
\begin{table}[H]
		\begin{tabularx}{\textwidth}{| l | l | Z | l |} \hline
			\textbf{Signal} & \textbf{Type} & \textbf{Beskrivelse} & \textbf{Tolerance} 				\\ \hline
			3VDC 	& P1 & Batterispænding 													& 2.7V-3.3V \\ \hline
			SPI  	& K1 & SPI kommunikation mellem HF-transceiver og mikroprocesor 		& 			\\ \hline
			I2C	 	& K2 & I2C kommunikation mellem temperaturføleren og mikroprocessoren 	&  			\\ \hline
			light 	& A1 & Spænding som indikerer hvor meget lys der er på kontrolboksen 	& 0-3.3V 	\\ \hline
			humidt 	& A2 & Spænding som indikerer fugtighed af jorden 						& 0-3.3V 	\\ \hline
			HF 		& H1 & Højfrekvent signal som er luftbåret 								&  			\\ \hline
			IRQ 	& A3 & Interrupt fra HF-transceiveren 									& 0-3.3V  	\\ \hline
			Data 	& A4 & Analogt signal til HF-transceiveren 								& 0-3.3V  	\\ \hline
			Reset 	& A5 & Reset signal til HF-transceiveren 								& 0-3.3V  	\\ \hline
			RESET 	& K4 & Reset signal til mikroprocessoren 								& 0-3.3V	\\ \hline
			PROG 	& K3 & Programmerings interface til mikroprocessoren 					& 0-3.3V 	\\ \hline
		\end{tabularx}
		\caption{Signalbeskrivelse for \texttt{Sensor}}
\end{table}


  
