\section{Hardwaredesign} \label{sec:hardwaredesign}

\begin{figure}[H]
	\centering
	\includegraphics[width=0.8\textwidth]{../fig/Systemarkitektur/BDDsystem.pdf}
	\caption{Blokdiagram over det samlede system}
	\label{fig:BDDsystem}
\end{figure}

På Figur \ref{fig:BDDsystem} ses BDD over det samlede system indeholdende sensor og kontrolboks. Separate blokdiagrammer over sensoren og kontrolboksen ses i figur \ref{fig:BDDsensor} og \ref{fig:BDDkontrolboks}.

\raggedbottom

%%%%%%%%%%%%%%%%%%%%%%%%%%%%%%%%%%%%%%%%%%%%%%%%%%%%%%%%%%%%%%%%%%%%%%%%%%%%%%%%%%%%%%%%%%%%%%%%%%%%%%%%%%%%%%%%%%
%
%							Hardware Design - Kontrolboks
%
%%%%%%%%%%%%%%%%%%%%%%%%%%%%%%%%%%%%%%%%%%%%%%%%%%%%%%%%%%%%%%%%%%%%%%%%%%%%%%%%%%%%%%%%%%%%%%%%%%%%%%%%%%%%%%%%%%

\subsection{Kontrolboks}

I dette underafsnit opstilles kravene til de enkelte blokke der udgør kontrolboksen på Figur \ref{fig:BDDkontrolboks}.

\begin{figure}[H]
	\centering
	\includegraphics[width=0.8\textwidth]{../fig/Systemarkitektur/BDDkontrolboks.pdf}
	\caption{Blokdiagram over \texttt{Kontrolboks}}
	\label{fig:BDDkontrolboks}
\end{figure}

\textbf{Mikroprocessor} \label{designkrav:mikroprocessor} \\
Det er som udgangspunkt valgt at benytte samme type og model Mikroprocessor på både kontrolboks og sensor, dette valg er truffet primært af software-grunde, da kontrolboks og sensor-enhed benytter mange af de samme funktioner og derfor vil det være optimalt at benytte samme processor, dertil kommer stykprisen hvis der påtænkes et high-volumen-produkt. Derfor kommer mikroprocessoren på kontrolboksen til at indeholde nogle interne moduler som ikke vil blive udnyttet, men som derimod skal bruges i forbindelse med sensorens funktionalitet. Af hensyn til overblikket er alle krav til mikroprocessorens interne moduler beskrevet her. Mikroprocessoren skal indeholde perifere enheder til at kommunikere både over SPI og I2C. Det ønskes at RF transceiveren kommunikerer via SPI, da denne kommunikationsbus egner sig godt til højhastigheds dataoverførsel. Kontrolboksens Display og Temperaturmåleren på Sensoren ønskes at kommunikere over I2C-bussen, da denne egner sig godt til lavhastigheds dataoverførsel. 
Mikroprocessoren skal indeholde min. 2 separate ADC-kanaler til sampling af hhv. jordfugtmåling og måling af lysintensitet. ADC'en skal min. have en samplingsfrekvens på 5KHz, da dette giver 100 målinger på 20ms, hvilket svare til 50Hz. Dette krav sættes for at kunne midle over evt. 50Hz støj. Grundet den lave samplingsfrekvens vælges en SAR-ADC. For at opfylde krav PF\_08 om 0.1 decimals præcision, skal der min. vælges en 10-bits ADC, da dette giver en opløsning på 1024 der lever op til kravet. \\
Mikroprocessoren skal indeholde en Watch-Dog timer for at sikre at systemet ikke ender i deadlock. Dertil kommer en Counter, der skal benyttes til at genere 2 firkant-spændinger der skal benyttes i målekredsløbet til jordfugtighed. 
Mikroprocessoren skal vælges ud fra et ønske om en forsyningsspænding på 3.3VDC $ \pm 10\% $, da dette egner sig godt til kravet EL\_01 om en batteridrevet sensor. Derudover skal der vælges en processor med så lavt strømforbrug som muligt, dertil ønskes mulighed for power down/sleep-mode for yderligere strømbesparelse og dermed forlængelse af batterilevetiden. Mikroprocessoren skal kunne programmeres eksternt, så der er mulighed for at ændre dens funktionalitet.


\textbf{Motorventil} \label{designkrav:motorventil} \\
Motorventilen skal bestå af en DC-motor der ønskes forsynet med 5VDC $\pm 10\%$ for at opfylde krav EL\_03 om 5V forsyning af kontrolboksen. Derudover skal motorventilen have to separate kontakter til afbrydelse når ventilen er i hhv. fuldt lukket eller åben position. krav MC\_10	skal være opfyldt. DC-motoren ønskes styret via en H-Bro da dette er det mests optimale i forhold til strømbesparelse. Motorventilen skal monteres på kontrolboksen. 


\textbf{Display} \label{designkrav:display} \\
Displayet skal vælges til en forsyningsspænding på 3VDC $ \pm 10\%$ og skal kunne kommunikere via I2C-protokollen. For at opfylde kravene US\_03 og US\_04 skal der vælges et display der kan vise 2x16 karakterer. Derudover skal størrelseskravet MC\_07 være opfyldt for at passe i formfaktor til kontrolboksen, krav MC\_06. Endeligt ønskes et baggrundsbelyst display, krav MC\_08 således at brugeren let kan aflæse display'et både i fuldt sollys samt om natten.


\textbf{HF-Transceiver} \label{designkrav:transceiver} \\
Det er naturligvis valgt at benytte samme type og model HF Transceiver på både kontrolboks og sensor. Transceiveren skal vælge således at den kan kommunikere med mikroprocessoren via SPI-bussen. Denne bus er valgt primært at 2 grunde. Dels at bussen er seriel, hvilket egner sig til at transmittere den type data som skal kommunikeres imellem kontrolboks og sensor. Data-rate'en for systemet skal opfylde krav PF\_07. Det ønskes at implementere en transceiver der egner specielt godt til batteriapplikationer, for bedst muligt at udnytte batterilevetiden på sensoren. Derudover ønskes transceiver'en at indeholde en lavspændings-detektion, således at den kan give besked til mikroprocessoren når batterispændingen på sensoren bliver for lav, dette signal kan så udnyttes til at enable en boost-konverter på sensoren. Transceiver'en skal vælges ud fra et ønske om en forsyningsspænding på 3.3VDC $ \pm 10\% $, da dette egner sig godt til kravet EL\_01 om en batteridrevet sensor, samt kan indgå i forsyningskæden på kontrolboksen. Desuden ønskes en transceiver der kan initieres fra en ekstern mikroprocessor, således at den ikke skal forprogrammeres men kan sættes op ved start-up. 


\textbf{Strømforsyning} \label{designkrav:stromforsyning} \\
Det er valgt at systemet skal kunne benytte 3.parts AC/DC 5V Power Supply, jvf. EL\_03. Følgende krav skal opfyldes til denne eksterne power Supply. Den skal kunne levere en udgangsspænding på 5VDC $\pm 10\%$, samt en udgangsstrøm på min. 1A. Derudover skal Sekundærside-konnekteren skal være en 6.50Ø $\pm 2mm$ DC jack-hun-konnekter med min. 2.50Ø mm $\pm 2mm$ center hul \\
Strømforsyningen på kontrolboksen skal inkludere en 6.50Ø $\pm 2mm$ DC jack-konnekter med 2.5 mm center pin. Denne skal have en maksimum rated input current på min. 2A. For at sikre kontrolboksen optimalt skal strømforsyningen indeholde ESD-beskyttelse, beskytte mod omvendt-polarisation samt strøm-overforbrug. Strømforsyning skal kunne acceptere 5VDC $\pm 10\%$ inputspænding fra den eksterne power supply. Derudover skal den indeholde en spændingsregulator ned til 3.3VDC $\pm 10\%$ på udgangssiden. 


\textbf{Tastatur} \label{designkrav:tastetur} \\
Tastaturet skal bestå af 4 trykknapper samt en rød LED til indikation af fejl, samt ESD-beskyttelse. Tastaturet skal være at matrix-tastatur således at det kan sættes op via 4 pins + 1 GND, samt 2 pins til LED'en. Derudover skal tastaturet opfylde krav MC\_09 om størrelse. 

\newpage

%%%%%%%%%%%%%%%%%%%%%%%%%%%%%%%%%%%%%%%%%%%%%%%%%%%%%%%%%%%%%%%%%%%%%%%%%%%%%%%%%%%%%%%%%%%%%%%%%%%%%%%%%%%%%%%%%%
%
%							IBD - Kontrolboks
%
%%%%%%%%%%%%%%%%%%%%%%%%%%%%%%%%%%%%%%%%%%%%%%%%%%%%%%%%%%%%%%%%%%%%%%%%%%%%%%%%%%%%%%%%%%%%%%%%%%%%%%%%%%%%%%%%%%

På Figur \ref{fig:IBDkontrolboks} ses IBD af kontrolboksen og herefter følger signalbeskrivelse i tabel \ref{tbl:SignalerKontrolboks}. De interne signaler i kontrolboksen er typebenævnt efter følgende konvention: Tallene efter typebetegnelsen er fortløbende: 

\begin{itemize} \itemsep0em
	\item	\texttt{P}: Dette er systemets power-signaler. 
	\item 	\texttt{K}: Dette er signaler der relaterer sig til kommunikation. 
	\item 	\texttt{B}: Dette er Bus-signaler. 
	\item 	\texttt{A}: Dette er analoge spændinger. 
	\item 	\texttt{H}: Dette er luftbåret højfrekvente signaler.
\end{itemize}

\begin{figure}[H]
	\centering
	\includegraphics[width=1\textwidth]{../fig/Systemarkitektur/IBDkontrolboks.pdf}
	\caption{Internt blokdiagram over \texttt{Kontrolboks}}
	\label{fig:IBDkontrolboks}
\end{figure}
\newpage

\textbf{Signalbeskrivelser}
\begin{table}[H]
	\begin{tabularx}{\textwidth}{| l | l | Z | p{3 cm} |} \hline
	\textbf{Signal} & \textbf{Type} & \textbf{Beskrivelse} & \textbf{Tolerance} \\ \hline
	5VDC 	
		& P1 	
		& Forsyningspænding fra ekstern power supply								
		& 5VDC $\pm 10\%$					\\ \hline
			
	3.3VDC 	
		& P2 	
		& Forsyningsspænding fra strømforsyning $ V_{dd} $						
		& 3.3VDC $\pm 10\%$ 				\\ \hline
		
	5VDC 	
		& P3 	
		& Forsyningsspænding fra strømforsyning $ V_{dd} $						
		& 5VDC $\pm 10\%$ 					\\ \hline
			
	SPI 
		& K2 	
		& Kommunikation: HF-transceiver og mikroprocesor 	
		& CMOS-logic: 						\newline 
			Low:  $0V - 1/3V_{dd}$ 			\newline
			High: $2/3V_{dd} - V_{dd} $		\\ \hline
			 
	I2C	 	
		& K3 	
		& Kommunikation: Display og mikroprocessor 		
		& CMOS-logic: 						\newline 
			Low:  $ 0V-1/3V_{dd} $ 			\newline
			High: $ 2/3V_{dd}-V_{dd} $		\\ \hline
	touch 	
		& B1 	
		& Bus indeholdende 5 forbindelser til tastaturet 			
		& $ 0-V_{dd} \pm 10\%$				\\ \hline
			
	openBreak 
		& A1  
		& Signal der åbner motorventilen. A2 holdes lav. 
		  For at bremse motoren er begge signaler høje 			
		& Analog logic:					 	\newline 
		 	Low:  $ 0V-0.6V_{dd} $ 			\newline
			High: $ 1V_{dd}-V_{dd} $		\\ \hline
	closeBreak 
		& A2 
		& Signal der lukker for motorventilen.Det kræves at A1 samtidig går lav. 						
		& Analog logic:					 	\newline 
		 	Low:  $ 0V-0.6V_{dd} $ 			\newline
			High: $ 1V_{dd}-V_{dd} $		\\ \hline
	Open 	
		& A3 	
		& Signal som er høj når motorventilen er åben				
		& $ 0-V_{dd} \pm 10\%$				\\ \hline
	Close 	
		& A4 	
		& Signal som er høj når motorventilen er lukket				
		& $ 0-V_{dd} \pm 10\%$				\\ \hline
	HF 		
		& H1 	
		& Luftbåret, højfrekvent signal	
		& $ 433MHz $						\\ \hline
	IRQ 	
		& A3 	
		& Interrupt fra HF-transceiveren 
		& $ 0-V_{dd} \pm 10\%$				\\ \hline
	Data 	
		& A4 	
		& Analogt data-signal til HF-transceiveren
		& $ 0-V_{dd} \pm 10\%$				\\ \hline
	Reset 	
		& A5 	
		& Reset reset-signal til HF-transceiveren
		& $ 0-V_{dd} \pm 10\%$				\\ \hline
	CONNECT 	
		& K4 	
		& Connect signal til mikroprocessoren
		& $ 0-V_{dd} \pm 10\%$				\\ \hline
	PROG 	
		& K3 	
		& Programmerings interface til mikroprocessoren
		& $ 0-V_{dd} \pm 10\%$				\\ \hline
	\end{tabularx}
	\caption{Signalbeskrivelse for \texttt{Kontrolboks}}
	\label{tbl:SignalerKontrolboks}
\end{table}
\newpage

%%%%%%%%%%%%%%%%%%%%%%%%%%%%%%%%%%%%%%%%%%%%%%%%%%%%%%%%%%%%%%%%%%%%%%%%%%%%%%%%%%%%%%%%%%%%%%%%%%%%%%%%%%%%%%%%%%
%
%							Hardware Design - Sensor
%
%%%%%%%%%%%%%%%%%%%%%%%%%%%%%%%%%%%%%%%%%%%%%%%%%%%%%%%%%%%%%%%%%%%%%%%%%%%%%%%%%%%%%%%%%%%%%%%%%%%%%%%%%%%%%%%%%%

\subsection{Sensor}
I dette underafsnit opstilles kravene til de enkelte blokke der udgør sensoren på Figur \ref{fig:BDDsensor}.

\begin{figure}[H]
	\centering
	\includegraphics[width=0.9\textwidth]{../fig/Systemarkitektur/BDDsensor.pdf}
	\caption{Blokdiagram over \texttt{Sensor}}
	\label{fig:BDDsensor}
\end{figure}

\textbf{Mikroprocessor} \\ \label{designkrav:mikroprocessor_sensor}
Mikroprocessoren er den samme som i kontrolboksen, derfor henvises til afsnit \ref{designkrav:mikroprocessor} på side \pageref{designkrav:mikroprocessor} for nærmere info.


\textbf{Jordfugtmåler} \\ \label{designkrav:jordfugt}
Jordfugt måleren skal inkludere et jordspyd som stikkes ned i jorden for at måle jordfugtigheden. Den skal give en analog spænding ud mellem 0-3.3VDC som indikerer fugtigheden, hvor 0V indikerer $0 \%$ og 3.3V indikerer $100 \%$ Jordfugtmåleren skal opfylde krav PF\_06  og krav PF\_08 om målepræcision.


\textbf{Temperaturføler} \\ \label{designkrav:temperaturfoler}
Temperaturføleren skal kommunikere via I2C og skal have en præcision på minimum $\pm 0.5^{\circ}C$ og en decimalafstand på minimum $0.1^{\circ}C$, jfv. krav PF\_06 . Temperaturmåleren skal kunne være operationsdygtig i intervallet $ -20 - 70^{\circ}C$ for at opfylde krav PF\_05. Derudover ønskes en temperaturføler der benytter 3.3VDC forsyningsspænding for at opfylde krav EL\_01. 


\textbf{HF-Transceiver} \\ \label{designkrav:transceiver_sensor}
HF-Transceiveren er den samme som i kontrolboksen, derfor henvises til afsnit \ref{designkrav:transceiver} på side \pageref{designkrav:transceiver} for nærmere info.


\textbf{Lysmåler} \\ \label{designkrav:lyssensor}
Lysmåleren skal kunne detektere nat eller dag. Lysintensitet ved tusmørke fastlægges ved test og denne sættes til  referencespænding på 1VDC $\pm 50\%$. Således fås en Vout tæt på 0VDC om natten, og 3.3VDC ved højlys dag. 


\textbf{PSU} \\ \label{designkrav:psu}
PSU'en består af en batteriholder der skal kunne rumme 2 batterier af typen AAA med en spænding på 1.5VDC hver (1.8VDC - 3.4 VDC). Derudover indeholder PSU'en en boostkonverter der, når batterispændingen bliver tilstrækkelig lav fastholder en udgangsspænding på 3.3VDC $\pm$ 10\% samt levere en udgangstrøm på min. 100mA til de resterende kredsløb på sensoren. Boostkonverteren skal enables når batterispændingen når ned på 3.0VDC, for at sikre krav de restende  40\% power der er tilbage på batterierne udnyttes optimalt før brugerne skal udskifte dem. Boostkonverter sørger her for, fuldt ud at udnytte batterierne ned til en spænding på 1.8 VDC jfv. krav PR\_03. Transceiverens IRO-pin (low battery detect )benyttes som boostkonverter enable. PSU'en skal sørge for en batterilevetid på min. 1 år. jfv. krav PF\_02.

%%%%%%%%%%%%%%%%%%%%%%%%%%%%%%%%%%%%%%%%%%%%%%%%%%%%%%%%%%%%%%%%%%%%%%%%%%%%%%%%%%%%%%%%%%%%%%%%%%%%%%%%%%%%%%%%%%
%
%							IBD - Sensor
%
%%%%%%%%%%%%%%%%%%%%%%%%%%%%%%%%%%%%%%%%%%%%%%%%%%%%%%%%%%%%%%%%%%%%%%%%%%%%%%%%%%%%%%%%%%%%%%%%%%%%%%%%%%%%%%%%%%
\newpage

På Figur \ref{fig:IBDsensor} ses Internt blok diagram af sensoren og herefter følger signalbeskrivelse i tabel \ref{tbl:SignalerSensor}. De interne signaler i sensoren er typebenævnt efter følgende konvention: Tallene efter typebetegnelsen er fortløbende: 

\begin{itemize} \itemsep0em
	\item	\texttt{P}: Dette er systemets power-signaler. 
	\item 	\texttt{K}: Dette er signaler der relaterer sig til kommunikation. 
	\item 	\texttt{B}: Dette er Bus-signaler. 
	\item 	\texttt{A}: Dette er analoge spændinger. 
	\item 	\texttt{H}: Dette er luftbåret højfrekvente signaler.
\end{itemize}

\begin{figure}[H]
	\centering
	\includegraphics[width=0.9\textwidth]{../fig/Systemarkitektur/IBDsensor.pdf}
	\caption{Internt blokdiagram over \texttt{Sensor}}
	\label{fig:IBDsensor}
\end{figure}
\newpage

\textbf{Signalbeskrivelser}
\begin{table}[H]
	\begin{tabularx}{\textwidth}{| l | l | Z | p{4 cm} |} \hline
	\textbf{Signal} & \textbf{Type} & \textbf{Beskrivelse} & \textbf{Tolerance} \\ \hline
		3.3VDC 	
			& P1 
			& Batterispænding $ V_{dd} $												
			& 3.3V $ \pm 10\%$					\\ \hline
		SPI  	
			& K1 
			& Kommunikation: HF-transceiver og Mikroprocesor
			& CMOS-logic: 						\newline 
				Low:  $ 0V - 1/3V_{dd} $ 		\newline
				High: $ 2/3V_{dd} - V_{dd} $	\\ \hline
		I2C	 	
			& K2 
			& Kommunikation: Temperaturføler og Mikroprocessor
			& CMOS-logic: 						\newline 
				Low:  $ 0V - 1/3V_{dd} $ 		\newline
				High: $ 2/3V_{dd} - V_{dd} $	\\ \hline
		light 	
			& A1 
			& Spændingsindikation af lysintensitet på kontrolboksen
			& $ 0-V_{dd} $ 	\\ \hline
		humid 	
			& A2 
			& Spændingsindikation jordfugtighed		
			& $ 0-V_{dd} $ 	\\ \hline
		HF 		
			& H1 
			& Luftbåret, højfrekvent signal			
			&  			\\ \hline
		IRQ 	
			& A3
			& Interrupt fra HF-transceiveren				
			& $ 0-V_{dd} $ 	\\ \hline
		Data 	
			& A4 
			& Analogt signal til HF-transceiveren				
			& $ 0-V_{dd} $ 	\\ \hline
		Reset 	
			& A5 
			& Reset signal til HF-transceiveren				
			& $ 0-V_{dd} $ 	\\ \hline
		CONNECT 	
			& K4 
			& Connect signal til mikroprocessoren				
			& $ 0-V_{dd} $ 	\\ \hline
		PROG 	
			& K3 
			& Programmerings interface til mikroprocessoren
			& $ 0-V_{dd} $ 	\\ \hline
	\end{tabularx}
	\caption{Signalbeskrivelse for \texttt{Sensor}}
	\label{tbl:SignalerSensor}
\end{table}
\newpage