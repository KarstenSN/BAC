\section{Hardwaredesign} \label{sec:hardwaredesign}

\begin{figure}[H]
	\centering
	\includegraphics[width=0.8\textwidth]{../fig/Systemarkitektur/BDDsystem.pdf}
	\caption{Blokdiagram over det samlede system}
	\label{fig:BDDsystem}
\end{figure}

På Figur \ref{fig:BDDsystem} ses BDD over det samlede system indeholdende sensor og kontrolboks. Separate blokdiagrammer over sensoren og kontrolboksen ses i figur \ref{fig:BDDsensor} og \ref{fig:BDDkontrolboks}.

\raggedbottom

\subsection{Kontrolboks}

I dette underafsnit opstilles kravene til de enkelte blokke der udgør kontrolboksen på Figur \ref{fig:BDDkontrolboks}.

\begin{figure}[H]
	\centering
	\includegraphics[width=0.8\textwidth]{../fig/Systemarkitektur/BDDkontrolboks.pdf}
	\caption{Blokdiagram over \texttt{Kontrolboks}}
	\label{fig:BDDkontrolboks}
\end{figure}

\textbf{Mikroprocessor} \label{designkrav:mikroprocessor} \\
Det er som udgangspunkt valgt at ville benytte samme type og model Mikroprocessor til både kontrolboks og sensor, dette valg er truffet primært af software-grunde, da kontrolboks og sensor-enhed benytter mange af de samme funktioner og derfor vil det være optimalt at benytte samme processor, dertil kommer stykprisen hvis der påtænkes et high-volumen-produkt. Derfor kommer mikroprocessoren på kontrolboksen til at indeholde nogle interne moduler som ikke vil blive udnyttet, men som derimod skal bruges i forbindelse med sensorens funktionalitet. Af hensyn til overblikket er alle krav til mikroprocessorens interne modulerne beskrevet her. Mikroprocessoren skal indeholde perifere enheder til at kommunikere både over SPI og I2C. SPI er valgt at benytte da RF transceiveren benytter sig af denne kommunikationsprotokol. I2C er valgt at benytte for at kunne kommunikere med hhv. den ønskede temperatursensor, samt display'et på kontrolboksen. Da alle blokke der kommunikeres imellem sidder på samme print og afstanden er derfor er yderst begrænset, kan det retfærdiggøres at benytte sig af disse 2 protokoller, da de er designet til at fungere over korte afstande. Mikroprocessoren skal indeholde min. 2 separate ADC-kanler til sampling af hhv. jordfugtmålinger og målinger for lysintensitet.  Der skal være minimum 6 analoge I/O-pins. 2 af disse benyttes til de 2 ADC-kanaler, 1 pin er valgt at bruge til transceiveren low battery detect-funktion, samt 1 pin til kontrol af det separate stelplan til målekredsløbet. Derudover skal der kunne sendes 2 separate firkant-spænding ud fra mikroprocessoren, disse kræver ligeledes hver 1 pin. Derudover skal 1 analog pin kunne kobles til CONNECT-knappen. 
Mikroprocessoren skal vælges udfra et ønske om en forsyningsspænding på 3.3VDC $ \pm 10\% $, da dette egner sig godt til kravet EL\_01 om en batteridrevet sensor. Derudover skal der vælges en processor med så lavt strømforbrug som muligt, dertil ønskes mulighed for power down/sleep-mode for yderligere strømbesparelse og dermed forlængelse af batterilevetiden. Processoren skal kunne programmeres eksternt, så der er mulighed for at ændre dens funktionalitet. Kravene til Mikroprocessoren kan samles til følgende: 

\begin{itemize}
	\setlength\itemsep{0em}
	\item 	Mikroprocessoren skal kunne forsynes fra både Strømforsyning og PSU'en med 2 x AAA alkaline batterier (1.8VDC - 3.4 VDC)
	\item	Mikroprocessoren skal via SPI og I2C kunne kommunikere med transceiveren, temperaturføleren samt Diplay'et.
	\item	Mikroprocessoren skal via analog pins kunne kommunikere med Motorventilen, Tasteturet, Jordfugtmåleren samt lysmåleren. 
	\item	Mikroprocessoren skal opfylde kravet om lavest muligt strømforbrug
	\item	Mikroprocessoren skal have mulighed for power down/sleep mode
	\item	Mikroprocessoren skal kunne programmeres eksternt
\end{itemize}

\textbf{Motorventil} \label{designkrav:motorventil} \\
Motorventilen skal bestå af en DC-motor der skal forsynes med 5VDC $ \pm 10\%$ og have to kontakter der afbrydes når ventilen er i hhv. en fuldt lukket eller åben position. Den skal kunne tilsluttes 1/2" vandrør og være lavet af et solidt materiale. Motorventilen skal monteres på kontrolboksen. 

\textbf{Display} \label{designkrav:display} \\
Displayet skal vælges til en forsyningsspænding på 3VDC $ \pm 10\%$ og skal kunne kommunikere via I2C. For at opfylde kravene US\_03 og US\_04 skal der vælges et display der kan vise 2x16 karakterer. Derudover skal størrelseskravet MC\_07 være opfyldt for at passe i formfaktor til kontrolboksen, krav MC\_06. Endeligt ønskes et baggrundsbelyst display, krav MC\_08 således at brugeren let kan aflæse display'et både i fuldt sollyst samt om natten. 

\textbf{HF-Transceiver} \label{designkrav:transceiver} \\
HF-Transceiveren skal kommunikere med mikroprocessoren via SPI. Den skal inkludere en antenne som skal indbygges i kontrolboksen. Transceiverens indstillinger skal initieres af mikroprocessoren ved opstart. 

\textbf{Strømforsyning} \label{designkrav:stromforsyning} \\
Strømforsyningen skal inkludere en DC-connector, monteret på kontrolboksen. Den skal have ESD beskyttelse og beskytte mod omvendt polarisation og strøm-overforbrug. Den skal kunne acceptere 5VDC $\pm 10\%$ inputspænding og herved indeholde en spændingsregulator som giver 3.3VDC $\pm 10\%$. 

\textbf{Tastatur} \label{designkrav:tastetur} \\
Tastaturet skal bestå af 4 trykknapper samt en rød LED til indikation af fejl.


På Figur \ref{fig:IBDkontrolboks} ses IBD af kontrolboksen og herefter følger signalbeskrivelse i tabel \ref{tbl:SignalerKontrolboks}.

\begin{figure}[H]
	\centering
	\includegraphics[width=1\textwidth]{../fig/Systemarkitektur/IBDkontrolboks.pdf}
	\caption{Internt blokdiagram over \texttt{Kontrolboks}}
	\label{fig:IBDkontrolboks}
\end{figure}
\newpage

\textbf{Signalbeskrivelser}
\begin{table}[H]
	\begin{tabularx}{\textwidth}{| l | l | Z | p{3 cm} |} \hline
	\textbf{Signal} & \textbf{Type} & \textbf{Beskrivelse} & \textbf{Tolerance} \\ \hline
	5VDC 	
		& P1 	
		& Forsyningspænding fra adaptor								
		& 5VDC $\pm 10\%$					\\ \hline
			
	3.3VDC 	
		& P2 	
		& Forsyningsspænding fra strømforsyning $ V_{dd} $						
		& 3.3VDC $\pm 10\%$ 				\\ \hline
			
	SPI 
		& K1 	
		& Kommunikation: HF-transceiver og mikroprocesor 	
		& CMOS-logic: 						\newline 
			Low:  $0V - 1/3V_{dd}$ 			\newline
			High: $2/3V_{dd} - V_{dd} $		\\ \hline
			 
	I2C	 	
		& K3 	
		& Kommunikation: Display og mikroprocessor 		
		& CMOS-logic: 						\newline 
			Low:  $ 0V-1/3V_{dd} $ 			\newline
			High: $ 2/3V_{dd}-V_{dd} $		\\ \hline
	touch 	
		& B1 	
		& Bus indeholdende 5 forbindelser til tastaturet 			
		& $ 0-V_{dd} \pm 10\%$				\\ \hline
			
	openBreak 
		& A1  
		& Signal der åbner motorventilen. A2 holdes lav. 
		  For at bremse motoren er begge signaler høje 			
		& Analog logic:					 	\newline 
		 	Low:  $ 0V-0.6V_{dd} $ 			\newline
			High: $ 1V_{dd}-V_{dd} $		\\ \hline
	closeBreak 
		& A2 
		& Signal der lukker for motorventilen.Det kræves at A1 samtidig går lav. 						
		& Analog logic:					 	\newline 
		 	Low:  $ 0V-0.6V_{dd} $ 			\newline
			High: $ 1V_{dd}-V_{dd} $		\\ \hline
	Open 	
		& A3 	
		& Signal som er høj når motorventilen er åben				
		& $ 0-V_{dd} \pm 10\%$				\\ \hline
	Close 	
		& A4 	
		& Signal som er høj når motorventilen er lukket				
		& $ 0-V_{dd} \pm 10\%$				\\ \hline
	HF 		
		& HF1 	
		& Luftbåret, højfrekvent signal	
		& \\ \hline
	IRQ 	
		& A3 	
		& Interrupt fra HF-transceiveren 
		& $ 0-V_{dd} \pm 10\%$				\\ \hline
	Data 	
		& A4 	
		& Analogt data-signal til HF-transceiveren
		& $ 0-V_{dd} \pm 10\%$				\\ \hline
	Reset 	
		& A5 	
		& Reset reset-signal til HF-transceiveren
		& $ 0-V_{dd} \pm 10\%$				\\ \hline
	CONNECT 	
		& K4 	
		& Connect signal til mikroprocessoren
		& $ 0-V_{dd} \pm 10\%$				\\ \hline
	PROG 	
		& K3 	
		& Programmerings interface til mikroprocessoren
		& $ 0-V_{dd} \pm 10\%$				\\ \hline
	\end{tabularx}
	\caption{Signalbeskrivelse for \texttt{Kontrolboks}}
	\label{tbl:SignalerKontrolboks}
\end{table}
\newpage

\subsection{Sensor}
I dette underafsnit opstilles kravende til de enkelte blokke der udgør sensoren på Figur \ref{fig:BDDsensor}.

\begin{figure}[H]
	\centering
	\includegraphics[width=0.9\textwidth]{../fig/Systemarkitektur/BDDsensor.pdf}
	\caption{Blokdiagram over \texttt{Sensor}}
	\label{fig:BDDsensor}
\end{figure}

\textbf{Mikroprocessor} \\
Mikroprocessoren skal indeholde perifere enheder til at kommunikere over SPI og I2C. Den skal have minimum 3 analoge porte, have en analog reset pin og den skal kunne programmeres eksternt. Den skal forsynes med 3.3VDC $ \pm 10\%$ samt indeholde en min. 10bits analog til digital konverter. 

\textbf{Jordfugt måler} \\
Jordfugt måleren skal inkludere et jordspyd som stikkes ned i jorden for at måle jordfugtigheden. Den skal give en analog spænding ud mellem 0-3.3VDC som indikerer fugtigheden. 

\textbf{Temperaturføler}
Temperaturføleren skal kommunikere via I2C og skal have en opløsning på minimum $\pm 0.5^\circ$C og en decimalafstand på minimum $0.1^\circ$C samt en præcision på 5 $\%$ jfv. krav PF\_06.

\textbf{HF-Transceiver} \\
HF-Transceiveren skal kommunikere med mikroprocessoren via SPI. Den skal inkludere en antenne som skal indbygges i kontrolboksen. Transceiverens indstillinger skal initieres af mikroprocessoren ved opstart.

\textbf{Lysmåler} \\
Lysmåleren skal kunne detektere nat eller dag. Lysintensitet ved tusmørke fastlægges ved test og denne sættes til  referencespænding på 2VDC $\pm 50\%$.Dette er den reference som mikroprocessoren arbejder med når der skal detekteres dag eller nat. 

\textbf{PSU} \\
PSU'en består af en batteriholder der skal kunne rumme 2 batterier af typen AAA med en spænding på 1.5VDC hver. Derudover indeholder PSU'en en boostkonverter der, når batterispændingen bliver tilstrækkelig lav fastholder en udgangsspænding på 3.3VDC til de resterende kredsløb på sensoren. Da mikroprocessoren kun opererer ned til 2.7VDC og Transceiveren kun kan opererer ned til 2.2VDC er der stadig omkring 30\% power tilbage på batterierne når brugerne må udskifte dem. Boostkonverter sørger her for, fuldt ud at udnytte batterierne ned til en spænding på 1.8 VDC jfv. krav PR\_03. Kravene til PSU'en kan samles til følgende: 

\begin{itemize}
	\setlength\itemsep{0em}
	\item 	PSU'en skal forsynes med 2 x AAA alkaline batterier (1.8VDC - 3.4 VDC)
	\item	PSU'en skal booste batterispændingen til at drive mikroprocessoren, transceiveren, temperaturføleren, jordfugtmålerkredsløbet samt lysmåleren ned til 1.8 VDC batterispænding. Transceiverens IRO-pin benyttes som boostkonverter enable.
	\item	PSU'en skal sørge for en batterilevetid på min. 1 år. jfv. krav PF\_02.
	\item	Derudover skal PSU'en skal levere en udgangsspænding på 3.3VDC $\pm$ 10\% og kunne levere min. 100mA ned til en batterispænding på 1.8VDC. jvf. krav PR\_03.
\end{itemize}


På Figur \ref{fig:IBDsensor} ses Internt blok diagram af sensoren og herefter følger signalbeskrivelse i tabel \ref{tbl:SignalerSensor}.

\begin{figure}[H]
	\centering
	\includegraphics[width=0.9\textwidth]{../fig/Systemarkitektur/IBDsensor.pdf}
	\caption{Internt blokdiagram over \texttt{Sensor}}
	\label{fig:IBDsensor}
\end{figure}

\textbf{Signalbeskrivelser}
\begin{table}[H]
	\begin{tabularx}{\textwidth}{| l | l | Z | p{4 cm} |} \hline
	\textbf{Signal} & \textbf{Type} & \textbf{Beskrivelse} & \textbf{Tolerance} \\ \hline
		3.3VDC 	
			& P1 
			& Batterispænding $ V_{dd} $												
			& 3.3V $ \pm 10\%$					\\ \hline
		SPI  	
			& K1 
			& Kommunikation: HF-transceiver og Mikroprocesor
			& CMOS-logic: 						\newline 
				Low:  $ 0V - 1/3V_{dd} $ 		\newline
				High: $ 2/3V_{dd} - V_{dd} $	\\ \hline
		I2C	 	
			& K2 
			& Kommunikation: Temperaturføler og Mikroprocessor
			& CMOS-logic: 						\newline 
				Low:  $ 0V - 1/3V_{dd} $ 		\newline
				High: $ 2/3V_{dd} - V_{dd} $	\\ \hline
		light 	
			& A1 
			& Spændingsindikation af lysintensitet på kontrolboksen
			& $ 0-V_{dd} $ 	\\ \hline
		humid 	
			& A2 
			& Spændingsindikation jordfugtighed		
			& $ 0-V_{dd} $ 	\\ \hline
		HF 		
			& H1 
			& Luftbåret, højfrekvent signal			
			&  			\\ \hline
		IRQ 	
			& A3
			& Interrupt fra HF-transceiveren				
			& $ 0-V_{dd} $ 	\\ \hline
		Data 	
			& A4 
			& Analogt signal til HF-transceiveren				
			& $ 0-V_{dd} $ 	\\ \hline
		Reset 	
			& A5 
			& Reset signal til HF-transceiveren				
			& $ 0-V_{dd} $ 	\\ \hline
		CONNECT 	
			& K4 
			& Connect signal til mikroprocessoren				
			& $ 0-V_{dd} $ 	\\ \hline
		PROG 	
			& K3 
			& Programmerings interface til mikroprocessoren
			& $ 0-V_{dd} $ 	\\ \hline
	\end{tabularx}
	\caption{Signalbeskrivelse for \texttt{Sensor}}
	\label{tbl:SignalerSensor}
\end{table}
\newpage