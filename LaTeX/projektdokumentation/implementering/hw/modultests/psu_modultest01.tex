%%%%%%%%%%%%%%%%%%%%%%%%%%%%%%%%%%%%%%%%%%%%%%%%%%%%%%%%%%%%%%%%
%
%			MODULTEST: PSU - iteration 1
%
%%%%%%%%%%%%%%%%%%%%%%%%%%%%%%%%%%%%%%%%%%%%%%%%%%%%%%%%%%%%%%%%

Første iteration af PSU'en blev opbygget med standardværdier fra databladets anbefalinger for at verificere at grundopsætningen og udlægget fungerer inden der optimeres herpå. 
Følgende data er gældende for denne test: 

\begin{table}[H]
\begin{tabularx}{\textwidth}{| L{3.3 cm} | Z |} \hline
	\textbf{Test:} 					& Modultest 01 af Boostkonverter	\\ \hline
	\textbf{Dato \& tid:}			& 16-03-2017: 10.00-13.00			\\ \hline
	\textbf{Udført af:}				& KE \& KSN							\\ \hline
	\textbf{SW ver:}				& NA								\\ \hline
	\textbf{Benyttet udstyr:} 		& 									
		\begin{packed_enum}
			\item  Oscilloskop: Keysight DSOX2002A: ID-D22401
			\item  Spændingsforsyning: TTI EL302Tv: ID-1-C2\_9
			\item  Digitalt multimeter: Fluke 189
			\item  Decadebox (Variable modstand)
		\end{packed_enum}	 		 									\\ \hline										
\end{tabularx}
\caption{Data for PSU-modultest 01}
\label{tbl:psu_modultest1}
\end{table}

Testopstilling ses på Figur \ref{fig:psu_modultest_opstilling}.

\begin{figure}[H]
	\centering
	\includegraphics[width=0.9\textwidth]{../fig/psu/modultest/psu_opstilling.jpg}
	\caption{Testopstilling til modultest af boostkonverter}
	\label{fig:psu_modultest_opstilling}
\end{figure}

Kanal 1 på Spændingsforsyningen emulerer batteripakken, og kanal 2 forsyner EN-pin'en med 3.3V. Den variable modstand sættes som load og indstilles til 33 $\Omega$ for at simulerer max. load på 100mA, da:

\begin{align}
	R_{load_{max}} = \frac{V_{OUT}}{I_{OUT}} = \frac{3.3V}{100mA} = 33\Omega  
\end{align}

Den variable modstand indstilles ligeledes til 330 $\Omega$ for at simulerer min. load på 10mA, samt til 66 $\Omega$ for at simulerer middel. load på 50mA.

Herefter er følgende målinger udført: 

\begin{packed_enum}
			\item  Kontinuert sekvens ved $ R_{load}=100mA $:
			\begin{packed_enum}
				\item 	$V_{IN} = 3.0V $
				\item 	$V_{IN} = 2.5V $
				\item 	$V_{IN} = 2.0V $
				\item 	$V_{IN} = 1.8V $
			\end{packed_enum}
			\item  Kontinuert sekvens ved $ R_{load}=10mA $:
			\begin{packed_enum}
				\item 	$V_{IN} = 3.0V $
				\item 	$V_{IN} = 2.5V $
				\item 	$V_{IN} = 2.0V $
				\item 	$V_{IN} = 1.8V $
			\end{packed_enum}
			\item  EN-sekvens ved $ R_{load}=100mA $:
			\begin{packed_enum}
				\item 	$V_{IN} = 3.0V $
				\item 	$V_{IN} = 2.5V $
				\item 	$V_{IN} = 2.0V $
				\item 	$V_{IN} = 1.8V $
			\end{packed_enum}
			\item  EN-sekvens ved $ R_{load}=10mA $:
			\begin{packed_enum}
				\item 	$V_{IN} = 3.0V $
				\item 	$V_{IN} = 2.5V $
				\item 	$V_{IN} = 2.0V $
				\item 	$V_{IN} = 1.8V $
			\end{packed_enum}
			\item  DIS-sekvens ved $ R_{load}=100mA $:
			\begin{packed_enum}
				\item 	$V_{IN} = 3.0V $
				\item 	$V_{IN} = 1.8V $
			\end{packed_enum}
			\item  DIS-sekvens ved $ R_{load}=10mA $:
			\begin{packed_enum}
				\item 	$V_{IN} = 3.0V $
				\item 	$V_{IN} = 1.8V $
			\end{packed_enum}
			\item  Middelbelastning ved $ R_{load}=50mA $::
			\begin{packed_enum}
				\item 	$V_{IN} = 2.5V $
			\end{packed_enum}
		\end{packed_enum}

Dokumentation for samtlige målinger kan ses i appendix \ref{appendix:psu_modultest01}.

Her er det interessant at se på målingerne ved $V_{IN}<2.8V $ da det primært er her at boostkonverteren kommer til at arbejde, samt at kravet PF\_03 skal være opfyldt.På Figur \ref{fig:modultest01_scope_05} på side \pageref{fig:modultest01_scope_05_kopi} ses at der ved $ R_{load}=100mA $ stadig står $3.3V$ på udgangen af konverteren.

\begin{figure}[H]
	\centering
	\includegraphics[width=0.9\textwidth]{../fig/appendix/psu/scope_5.png}
	\caption{Kontinuert-mode: $R_{load}=100mA$,$V_{IN} = 1.8V$}
	\label{fig:modultest01_scope_05_kopi}
\end{figure}