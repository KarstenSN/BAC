\subsubsection{HF Transceiver} \label{sec:HWimp_transceiver_kontrolboks}
Der er valgt at benytte en transceiver af typen MRFX49A\cite{lib:trans_MRFX49A_datasheet} fra Microchip, da denne model er SPI-kompatibel og ikke behøver at blive forprogrammeret som mange andre RF-kredse skal. 
Den initialiseres i stedet via SPI-kommunikationen. Chippen er som skræddersyet til dette system da den er designet til batteri-applikationer. Den har en indbygget "low battery voltage detect" som afgive et interrupt til mikroprocessoren når batterispændingen når under et forudbestemt niveau som sættes internt i transceiveren. Den har også en indbygget wake-up timer så chippen kan sættes i sleep-mode og herved minimere strømforbruget til nogle få $\mu A$ indtil den vækkes op igen efter en forudbestemt tid. Transceiveren kan kommunikere ved 433/868/915 MHz og ligger derfor i et frekvens område som ofte bruges i consumer elektronik. Dog er 915MHz mere rettet til det Amerikanske marked.   

\begin{figure}[ht]
	\centering
	\includegraphics[width=0.8\textwidth]{../fig/transciver/mrfx49a_circuit.PNG}
	\caption{Application circuit af MRF49XA}
	\label{fig:mrfx49a_cir}
\end{figure}

\newpage