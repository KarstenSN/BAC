\subsubsection{Display} \label{sec:HWimp_display}
Af krav til displayet var at det skulle køre på 3.3V, kommunikere over I2C bus og have 16X2 karakterer, samt opfylde størrelsekrav \textbf{MC\_07}. Derudover skal \textbf{MC\_08} være opfyldt vedrørende baggrundbelysning. Ved undersøgelse af flere displays blev Newheaven's display \textit{NHD-C0216CiZ-FSW-FBW-3V3} \cite{lib:display_datasheet} udvalgt. Dette skyldes primært at display'et har indbygget DC-DC step-up converter således at displayet kan forsynes ned til 2.8V. Normalt kører LCD displays på 5V, hvilket er over forsyningskravet i afsnit \ref{designkrav:display} på side \pageref{designkrav:display}. Opsætning og kommunikation sker via I2C som kravet foreskriver. Diagrammet for display'et ses på Figur \ref{fig:display_diagram}.

\begin{figure}[H]
	\centering
	\includegraphics[width=0.8\textwidth]{../fig/display/display_diagram_kontrol2.png}
	\caption{Diagram over I2C Display}
	\label{fig:display_diagram}
\end{figure}

D11 sidder for at sænke forsyningsspændingen til omkring 3V da forsyningen ikke må overskride 3.3V. C22 sidder over pin5 og pin6 som udgangskondesator for Step-up konverteren, C23 sidder over pin7 og pin8 som en integreret del af step-up konverteren og disse værdier er valgt ud fra databladet anbefalinger. Derudover benyttes Q13 for at kontrollere LED'en til baggrundsbelysningen hvormed dette kan styres fra software. SDA og SCL linjerne har pull-up modstande påsat som en del af standardopsætning for I2C-bussen.