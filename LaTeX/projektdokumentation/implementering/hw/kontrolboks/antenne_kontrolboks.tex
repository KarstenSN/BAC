\newpage
\subsubsection{Antenne} \label{sec:HWimp_antenne_kontrolboks}


Valget af antennetype og transmissionsfrekvens afhænger af hvilken effekt der ønskes ved receiver-antennen og dermed hvor lang rækkevidde der kan opnås, 
dette sættes i forhold til den effekt transmitter-antennen fødes med. Sammenhængen imellem disse parametre fås ved \textit{Friis transmissions formel} 
2-117 på side 95 i \cite{lib:balanis}

\begin{align} \label{eq:ant_friis}
	\frac{P_{r}}{P_{t}} = e_{t}e_{r} \frac{\lambda^{2}D_{t}(\theta_{t},\phi_{t})D_{r}(\theta_{r},\phi_{r})}{(4 \pi R)^{2}}
\end{align}

\begin{table}[H]
	\begin{tabularx}{\textwidth}{ m{1.5 cm}  m{8 cm}  Z } \\
		$ e_{r} $	& Radiation efficiency of receiver		&	\\
		$ e_{t} $	& Radiation efficiency of transmitter	&	\\
		$\lambda$ 	& Wavelength in Meters					&	\\
		$ P_{r} $ 	& Received Power in W 					&	\\
		$ P_{t} $	& Transmitted Power in W				&	\\
		$ D_{t} $	& Directivity of Transmitter			&	\\
		$ D_{r} $	& Directivity of Receiver				&	\\
		$ R 	$ 	& Distance between Antennas in Meters	&	\\
	\end{tabularx}
	\caption{Friis parametre}
	\label{tbl:friis_param}
\end{table}

Følgende parametre kan opstilles for systemet, her antages det at udstrålingseffektiviteten $ e_{t} $ og $ e_{r} $ for transmitter og receiver sættes til $ 0.1 $. Dette er optimistisk sat, men det er grundet et ønske om at få en målestok for antennens maksimale rækkevidde. 

\begin{table}[H]
	\begin{tabularx}{\textwidth}{ m{1.5 cm}  m{4 cm}  Z } 	\\										
		$ P_{t}	 $ 	& $ 5 \times 10^{-3} 		$	& Transmit Power in $ W $					\\
		$ D_{t}  $ 	& $ 1 						$	& Directivity of transmitter 				\\
		$ D_{r}  $ 	& $ 1 						$	& Directivity of receiver 					\\
		$ e_{r}  $	& $ 1 \times 10^{-3}  		$	& Radiation efficiency of receiver			\\
		$ e_{t}  $	& $ 1 \times 10^{-3}  		$	& Radiation efficiency of transmitter		\\
		$ f  	 $	& $ 433						$	& Transmit frequency in $ Hz $				\\
		$ v_{0}  $	& $ 2.99 \times 10^8 m/s 	$	& Free-space velocity of light in $ m/s $	\\
		
	\end{tabularx}
	\caption{system parametre}
	\label{tbl:system_param}
\end{table}

Først omregnes transmissionsfrekvensen til bølgelængder: 

\begin{align} \label{eq:ant_lambda}
	\lambda = \frac{v_{0}}{f} \Rightarrow \frac{2.99 \times 10^8 m/s}{f} = 0,69053 m
\end{align}

Herefter benyttes $ \lambda $ til at beregne antennens maksimum effektiv areal $ A_{r} $.\\
Da transmitter og receiver er af samme type gælder denne parameter for begge antenner. 

\begin{align} \label{eq:ant_ar}
	A_{r} = e_{r} D_{r}(\theta_{r},\phi_{r}) \frac{\lambda^{2}}{4 \pi} \Rightarrow  0.9 \times 1 \times \frac{0,6905^{2}}{4 \pi} = 0.03415
\end{align}

Nu kendes faktoren for hvor meget af den udstrålede effekt der kan omsættes og $ P_{r} $ kan nu beregnes ved at benytte formel 2-114 og 2-116 på side 95 i \cite{lib:balanis}

\begin{align} \label{eq:ant_Wt}
	W_{t} = e_{t} \frac{P_{t} D_{t}(\theta_{t},\phi_{t})}{4 \pi R^{2}}
\end{align}

Nu fås den modtagne effektiv som ligning \ref{eq:ant_Wt} ganget på ligning \ref{eq:ant_ar}

\begin{align} \label{eq:ant_Pr}
	P_{r} = e_{t} \frac{P_{t} D_{t}(\theta_{t},\phi_{t})}{4 \pi R^{2}} \times e_{r} D_{r}(\theta_{r},\phi_{r}) \frac{\lambda^{2}}{4 \pi}
\end{align}

Denne ligning kan nu løses for $ R $ for at finde den maksimale rækkevidde for antennen.  

\begin{align} \label{eq:ant_R}
	R = \sqrt{ e_{t}   \frac{P_{t} D_{t}(\theta_{t},\phi_{t})} {4 \pi P_{r}} \times e_{r} D_{r}(\theta_{r},\phi_{r}) \frac{\lambda^{2}}{4 \pi}}= 38,86m
\end{align}

af ligning \ref{eq:ant_R} ses det at ved den givne transmissionsfrekvens kan det forventes at opnå en rækkevidde på 38m og dette opfylder krav \textit{CM2} om rækkevidde.
\newpage

\textbf{Valg af Antenne} \\

For at finde den optimale antenne til systemet er der foretaget en teknologiundersøgelse af forskellige antennetyper samt deres performance. Der er ydermere foretaget en række beregninger i forbindelse med disse antenner, dette kan ses i Appendix \ref{appendix:antenne}. Tre typer antenner blev undersøgt: TH-monteret Helical antenne, microstrip antenne, samt SMD-monteret antenne. Derudover blev der også eksperimenteret med at implementere en loopantenne på indersiden af plasthuset, sidstnævnte blev dog hurtigt afskrevet pga. dens krævende monteringsprocess samt manglende mekaniske understøttelse der vil føre til et ustabilt produkt og øget omkostninger. 

Til kontrolboksen er valgt at benytte en SMD-monteret 433 MHz ISM Antenna fra \textit{Johanson Technology} se datablad i bilag \cite{lib:ant_jt_datasheet}. Dette valg skyldes primært pladskravene til antennen på PCB'et, hvorfor en større antenne er udelukket. Den valgte antenne har en Max Gain på $ G_{max}=-4.0 dBi$ jvf. databladet.
Antennes karakteristisk impedans er 50 $\Omega$ hvilket sætter kravet til den microstripline der føder antennen. Ved at benytte formel 9-39 på side 528 i \cite{lib:balanis} kan banebredden beregnes når følgende parametre for PCB'et kendes: 

\begin{align} \label{eq:ant_trace}
	Z_{c} = \frac{87}{\varepsilon_{r}+1.41} \times ln{\bigg( \frac{5.98h}{0.8w+t} \bigg)}
\end{align}

\begin{table}[H]
	\begin{tabularx}{\textwidth}{ m{1.5 cm}  m{8 cm}  Z } \\
		$ Z_{c} $			& Karakteristisk impedans af microstripline	i $\Omega$ 	&	\\
		$ \varepsilon_{r} $	& Epsilon-relativ for substratet						&	\\
		$ h $ 				& Højde af substratet i $mm$							&	\\
		$ w $ 				& Bredde af substratet i $mm$							&	\\
		$ t $ 				& Tykkelse af microstripline i $mm$						&	\\
	\end{tabularx}
	\caption{Microstripline parametre}
	\label{tbl:antTrace_param}
\end{table}

De givne parametre indsættes og ligning \ref{eq:ant_trace} løses for $w$

\begin{align} \label{eq:ant_traceParam}
	50 \Omega = \frac{87}{4.4+1.41} \times ln{\bigg( \frac{5.98 \times 1.57mm}{0.8\times w+ 0.035mm} \bigg)} = 2.89mm
\end{align}

Hermed kendes banebredde der skal føde antennen for at få optimal matching.


Til sensoren er valgt at anvende en through-hole monteret helical antenne af typen W3127 ISM 433MHz fra \textit{Pulse Electronics}. Denne antenne er den mest isotropiske af kandidaterne med en Max Gain på $ G_{max}=-2.9 dBi$. jvf. Fig 3 i databladet i bilag. \cite{lib:ant_W3127_datasheet}. Dette valgt baserer sig på at udnytte pladsen på sensor-PCB'et mest muligt for at kunne implementere den bedst mulige antenne, samt at denne antenne ved dens placering  og orientering på PCB'et når PCB'ets er placeret i jorden gør at der ydes optimal udstråling. 
Antennes karakteristisk impedans er som på kontrolboksen 50 $\Omega$ hvorfor samme banebredde er benytte til at føde antennen. 