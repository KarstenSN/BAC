\subsubsection{Tastatur} \label{sec:HWimp_tastetur}
Systemet betjenes via et tastatur der moneres på kontrolboksen. Tastaturet består af 4 knapper som brugeren kan benytte til at navigere rundt i menuen. En sketchup af tastaturet kan ses på Figur \ref{fig:keypad_UI}. \texttt{SET} knappen benyttes til at skifte mellem menupunkterne. \texttt{Pil-op} og \texttt{Pil-ned} benyttes hhv.til at inkrementere eller dekrementere en given værdi. \texttt{OK} benyttes til at gemme indstillingen. For et fuldt overblik over menuen henvises til usecases i sektion \ref{ch:kravspecifikation}. Tastaturet indeholder desuden et vindue til displayet, samt en rød LED der tændes hvis systemet registrerer en fejl. Eks. hvis batterispændingen bliver for lav. I det færdige produkt forventes det at implementere en specialbygget keypad, men til prototypeformål er her benyttet en 4 knappers membrane switch keypad som tastatur.

\begin{figure}[H]
  \centering
  \begin{minipage}[b]{0.45\textwidth}
	\includegraphics[width=1\textwidth]{../fig/Keypad/keypad.png}
	\caption{Skitse af keypad'en som den ønskes i en færdig produktion}
	\label{fig:keypad_UI}
  \end{minipage}
  \hfill
  \begin{minipage}[b]{0.45\textwidth}
	\includegraphics[width=1\textwidth]{../fig/Keypad/diagram.png}
	\caption{Diagram over keypad'en}
	\label{fig:keypad_diagram}
  \end{minipage}
\end{figure}


\todo[inline]{opdater billede af tastetur}

\newpage