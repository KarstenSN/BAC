\subsubsection{Mikroprocessor}
Mikroprocessoren som der er blevet valgt at bruge til systemet er en ATMEGA8 i TQFP pakke fra Atmel. Den kan forsynes fra 2.7V og optil 5.5V. Vores forsyningsspænding er sat til 3.3V. Af perifere enheder skal vi bruge SPI, I2C og analog til digital converter. ATMEGA8 er et godt valg da den er billig (omkring 15kr), den kommer med nogle gode biblioteker skrevet i C og så har den en intern RC-oscillator som gør at der ikke behøves at bruges et krystal. Der kunne også være valgt en PIC processor fra Mircochip, men da der var mest erfaring med Atmel, var det, det der blev valgt. 

\begin{figure}[ht]
	\centering
	\includegraphics[width=0.5\textwidth]{../fig/Mikroprocessor/mikroprocessor.jpg}
	\caption{Overblik over portene på ATMEGA8}
	\label{fig:atmega8}
\end{figure} 