\subsubsection{Mikroprocessor} \label{sec:HWimp_mikroprocessor_kontrolboks}
Som mikroprocessoren i kontrolboks og sensor er er valgt at benytte en ATMEGA8L i TQFP pakke fra Atmel \cite{lib:mikro_ATMEGA8L_datasheet}. Denne er valgt da den opfylder designkravene i afsnit \ref{designkrav:mikroprocessor} på side \pageref{designkrav:mikroprocessor} samt kravene i afsnit \ref{ch:kravspecifikation} fra side \pageref{ch:kravspecifikation}. \texttt{L}-typen er valgt da den kan forsynes i intervallet fra 2.7VDC  - 5.5VDC hvilket både egner sig til adaptor-forsyning på kontrolboksen, samt batteriforsyning på sensoren. Processoren kan køre med en clockfrekvens fra 0-8MHz. Her er valgt at kører med 8MHz til de første iterationer, grundet SPI og I2C kommunikationen, men denne frekvens kan med fordel nedjusteres til 4MHz for ydeligere strømbesparelse iht. sensorens batterilevetid.

Af perifere enheder skal der på kontrolboksen benyttes SPI, I2C, samt et antal analoge pins til kontrol af hhv. motorventil og tastatur, herunder error-LED. ATMEGA8L rummer derudover en del forprogrammerede biblioteker skrevet i C således at disse ikke skal skrives fra ny.
Desuden har mikroprocessoren en intern RC-oscillator som gør at der ikke behøves et eksternt krystal. Mikroprocessorens pinout samt den benyttede pakke kan ses på Figur \ref{fig:atmega8L_pinout} herunder. 

\begin{figure}[H]
	\centering
	\includegraphics[width=0.4\textwidth]{../fig/Mikroprocessor/mikroprocessor.png}
	\caption{ATMEGA8L pinout diagram}
	\label{fig:atmega8L_pinout}
\end{figure}

TQFP-pakken er valgt da PCB'et til disse prototyper håndloddes og denne pakkes benytter "gull wing"-ben som letter denne process. Der vil der med fordel kunne vælges MLF-pakken til en serieproduktion da denne kan opnå bedre mekanisk stabilisering via en underliggende GND-pad. Dette kræver dog en anelse rettelser til PCB layoutet fra den nuværende iteration grundet denne GND-pad der skal loddes eller limes fast på PCB'et. Hvorfor de nuværende baner skal omarrangeres.

Det grundlæggende diagram for opkobling af mikroprocessoren kan ses på Figur \ref{fig:atmega8L_kontrolboks_diagram}. Opkobling af de perifere kredsløb vil blive diskuteret under deres respektive afsnit. 

\begin{figure}[H]
	\centering
	\includegraphics[width=0.6\textwidth]{../fig/Mikroprocessor/kontrolboks_diagram_2.png}
	\caption{ATMEGA8L Kontrolboks diagram}
	\label{fig:atmega8L_kontrolboks_diagram}
\end{figure}

Som det ses på Figur \ref{fig:atmega8L_kontrolboks_diagram} er mikroprocessoren afkoblet med C11 på $1 \mu F$. Denne værdier er valgt efter databladets anbefaling. Selvom ADC'en ikke benyttes i kontrolboksen er databladet anbefaling om at forbindelse \texttt{AVCC} til \texttt{VCC} og afkoble denne blevet fuldt, deraf C12 på $1 \mu F$. 