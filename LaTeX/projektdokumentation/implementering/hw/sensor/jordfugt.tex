\subsubsection{Jordfugtmåler}  \label{sec:HWimp_jordfugt}
Hovedformålet med jordfugt sensoren er som det fremgår af navnet; at måle jordfugtigheden. Her er der i videnskabelige artikler fundet frem til fire metoder at gøre dette på. En resistiv målemetode \cite{lib:MultiSensorSystem}, en kapasitiv målemetode\cite{lib:MultiSensorSystem}, en optisk målemetode\cite{lib:optical} og en metode hvor der bruges en varmepuls\cite{lib:DPHP}. Den resistive målemetode er den simpleste men desværre også den mest upræcise. Der er igennem projektet blevet undersøgt om denne målemetode kunne bruges trods dens begrænsninger og om der kunne findes en løsning der på. Da det viste sig at være svært at forbedre den resistive målemetode blev der herefter undersøgt hvordan der kunne laves en kapasitiv måling. Skalaen som der er brugt at at bestemme jordfugten er den volumetriske skala\cite{lib:wiki_waterContent}. 

\begin{equation}
	\theta = \frac{V_w}{V_s + V_w +V_a} \cdot 100\%
	\label{equ:fugtighed}
\end{equation}

Hvor $\theta$ er fugtigheden i procent. $V_w$ er volumen af væsken, $V_s$ er volumen af jorden og $V_a$ er volumen af luften indeholdende i jorden. For at gøre det nemmere at bestemme en jordfugtighed afvejes først 1 liter jord, som der sammenpresses nok til at der ikke er noget luft i prøven. Herefter noteres vægten som $m_{s1L}$ og nu kan ligning \ref{equ:fugtighed} omskrives til ligning \ref{equ:fugtighed2} 

\begin{equation}
	\theta = \frac{m_w}{(m_{wet}-m_w)\cdot \eta + m_w} \cdot 100\% 
	\label{equ:fugtighed2}
\end{equation}

Hvor $m_w$ er massen af væske, $m_{wet}$ er massen af den fugtige jord og $\eta$ er udregnet via vægten af 1 liter jord.

\begin{equation}
	\eta = \frac{1}{m_{s1L}}
\end{equation}

Det er derfor nu muligt at bestemme en jordfugtighed uden at skulle måle volumen af prøven. Til vejning af jordprøver er der blevet brugt en vægt af producenten \texttt{Sartorius} model \texttt{QUINTIX2102-1S} med en præcision på $\pm0.01g$.

\textbf{Resistiv måling} \\
Den resistive målemetode er en metode til at måle den ohmske værdi af jorden. Jo mere vand jorden indeholder jo mere ledende vil jorden være og den ohmske værdi vil herefter falde. Der er dog mange faktor som kan være med til at give en måleusikkerhed, disse usikkerheder er betinget af selve næringsindholdet i jorden, om jorden er forurenet med metaller eller om proben som der måles med, har været udsat for kraftig tæring. Disse betingelser kunne der kalibreres for, men en kalibrering vil være besværlig for en almindelig bruger af systemet, da det vil kræve at brugeren ovntørrer en jordprøve og herefter tilsætter en præcis mængde vand for at vide med sikkerhed hvilken værdi der skal kalibreres ind efter. Det viste sig dog at der sammen med den resistive måling af jorden også opstod en kapasitiv virkning, som funktion af en stigetid. Se Figur\ref{fig:maling1}. Det blev derfor undersøgt om det ved hjælp af denne kunne gøre målingen mere præcis.


\begin{figure}[H]
	\centering
	\includegraphics[width=0.8\textwidth]{../fig/Jordfugt/maling1.PNG}
	\caption{Impedans måling af jorden ved 17\% fugtighed}
	\label{fig:maling1}
\end{figure} 

Udfra Figur \ref{fig:maling1} kan der opstilles et ækvivalentdiagram som ses på Figur \ref{fig:akvj1}. I det øjeblik der bliver sat en spænding på proben, vil C1 være totalt afladet og det vil derfor være muligt at aflæse værdien af R1 ved brug af spændingsdeler-formlen. C1 vil herefter begynde at oplade og derfor træder R2 mere og mere i kraft. Det blev  besluttet at modellere jorden som et førsteordens system med overføringsfunktionen:
\begin{equation}
	System = \dfrac{\alpha \cdot \beta}{S + \alpha}
\end{equation}
Ved brug af denne overføringsfunktion ses R1 som værende 0 ohm. 
 
\begin{figure}[H]
	\centering
	\includegraphics[width=0.5\textwidth]{../fig/Jordfugt/akvivalentjord.PNG}
	\caption{Ækvivalentdiagram af jorden}
	\label{fig:akvj1}
\end{figure} 

I matlab blev der skrevet et script, som kan findes i bilagende, \todo[inline]{henvisning} til at udregne overføringsfunktionen. Der blev i alt foretaget målinger på tre typer jord, en taget på Helgenæs, en i Randers samt en så og plantejord fra en plantesæk. Aflæsningen blev foretaget ved 3$\tau$ og ikke ved 1 $\tau$ som ellers er normalen. Dette skyldes at R1 er sat til 0 ohm. 

\begin{figure}[H]
	\centering
	\includegraphics[width=0.8\textwidth]{../fig/Jordfugt/maelingafsystem.PNG}
	\caption{Måling af overføringsfunktion i MATLAB}
	\label{fig:msystem}
\end{figure} 

På figur \ref{fig:msystem} ses hvordan den approksimerende overføringsfunktion er blevet målt. Den grønne streg er start tidspunktet og den gule streg er stop tidspunket. i Intervallet herimellem aflæses tiden af tre tidskonstanter. Disse divideres med 3 således at $\alpha$ står tilbage. $\beta$ aflæses blot som forholdet mellem inputtet og outputtet lige før spændingen går negativ. Grunden til at spændingen går negativ skyldes at der med tiden vil opstå slitage på proben, denne slitage vendes således at hver enkelt spyd vil slides lige meget. Det ses at når spændingen fjernes fra proben aflades C1 langsomt. Der blev noteret overføringsfunktioner for alle tre typer jord ved forskellige fugtigheder og tilsidst blev der lavet en regression til at bestemme fugtigheden ud fra $\beta$ og $\alpha$. $\alpha$ viste sig desværre afhænge meget af tiden siden den foregående måling og det var derfor ikke en fordel at inkludere den i en måling. På Figur\ref{fig:funcfit} ses regressionen af $\beta$ og på Figur\ref{fig:resplot} ses et residual plot over regressionen. Regressionen blev bestemt via matlabs indbyggede funktion \textit{fit} til at være:

\begin{equation}
	\theta = a \cdot exp(b\cdot x) + c \cdot exp(d\cdot x)
\end{equation}

Hvor a=2.219$\cdot 10^{7}$ b=-39.8 c=151.9 d=-3.473
 
\begin{figure}[H]
	\centering
	\includegraphics[width=0.8\textwidth]{../fig/Jordfugt/functionfit.PNG}
	\caption{Functionfit i MATLAB}
	\label{fig:funcfit}
\end{figure} 

\begin{figure}[H]
	\centering
	\includegraphics[width=0.8\textwidth]{../fig/Jordfugt/residualplot.PNG}
	\caption{Plot over residualerne}
	\label{fig:resplot}
\end{figure} 
 
Residualplottet viser at der er helt op til 8\% afvigelse på regressionen, hvilket må siges ikke at leve op til kravet om 5\% præcision. 


\textbf{Kapacitiv måling} \\
Den kapacitive målemetode benytter sig af et jordspyd stukket ned i jorden\cite{lib:SoilModeling} \cite{lib:MultiSensorSystem} \cite{lib:DPHP} \cite{lib:cap_soil}. På jordspyddet er der to kobberplader som vil danne en kapacitet der vil variere afhængig af fugtigheden af den jord det befinder sig i. Se Figur \ref{fig:kap1} og \ref{fig:kap2}.

\begin{figure}[H]
  \centering
  \begin{minipage}[b]{0.4\textwidth}
    \includegraphics[width=1\textwidth]{../fig/Jordfugt/printkapasitet.png}
    \caption{Tværsnit af jordspyd}
    \label{fig:kap1}
  \end{minipage}
  \hfill
  \begin{minipage}[b]{0.4\textwidth}
    \includegraphics[width=1\textwidth]{../fig/Jordfugt/printkapasitet_vad.png}
    \caption{Tværsnit af jordspyd som der er i kontakt med vand. Her skal kobberbanerne dog være belagt med en elektrisk isolerende coating.}
    \label{fig:kap2}
  \end{minipage}
\end{figure}

For en kondensator gælder:

\begin{equation}
	C = \frac{\epsilon_0\cdot \epsilon_r \cdot A}{d}
\end{equation}

Hvor $\epsilon_0 = 8.854\cdot 10^{-12}F/m$ er permativiteten af vaccum, $\epsilon_r$ er permativiteten af materialet mellem kobberet som er 1 for luft og omkring 80 for vand. A er arealet af det kobber som kan se hinanden og d er diameteren mellem banerne. 

Det vil sige at kapaciteten vil blive 80 gange større når jordspyddet er stukket ned i rent vand en hvis den var i tør luft. Hvis vi antager at der er 0.3mm mellem banerne, at de har en længde på 5cm, at kobbertykkelsen er 35$\mu m$ og at jordspyddet befinder sig i vaccum. Vil der være en kapacitet på:

\begin{equation}
	C = \frac{8.854\cdot 10^{-12}\cdot 1 \cdot 35 \cdot 10^{-6} \cdot 5 \cdot 10^{-2}}{0.3 \cdot 10^{-3}} = 0.05pF
\end{equation} 

Antager vi at jordspyddet er stukket ned i vand vil vi have en kapacitet 80 gange størrer:

\begin{equation}
	C = \frac{8.854\cdot 10^{-12}\cdot 80 \cdot 35 \cdot 10^{-6} \cdot 5 \cdot 10^{-2}}{0.3 \cdot 10^{-3}} = 4.00pF
\end{equation} 

Der er blevet designet 4 typer af jordspyd som er blevet brugt til test. Det viste sig at kapaciteten lå højere end det forventede. Dette skyldes at der er udstråling fra banerne som der er svære at tage med i beregningerne. Se figur \ref{fig:spyd1},\ref{fig:spyd2},\ref{fig:spyd3},\ref{fig:spyd4}. Type 1 og 2 er meget ens. De danner begge en kapacitet mellem stelplanet og midterbanen. Størstedelen af kapaciten vil ligge for enden af spyddet hvor banen er tykkest på Type 1 og gaffelformet på Type 2. Dog er det ikke ligegyldigt hvor dybt spyddes stikkes ned i jorden da der også vil dannes en kapacitet mellem fødebanen og stelplanet. For at undgå at spyddet skal stikkes ned i en bestemt højde er Type 2 og 3 blevet lavet. Type 3 fører en bane på oversiden af printet hvor der ligger en identisk bane på bagsiden. På den måde skulle kapaciteten ikke ændre sig når der kommer vand længere op end til "diamanten". Test viste dog at kapaciteten alligevel varierede. Det blev lagt et stelplan uden på banerne for at indkapsle dem, ved at skære et andet print til, og lime det på således der laves et 4-lags print. Dette viste sig at have gode egenskaber og derfor skal jordspyddet også laves i 4-lag hvis der en gang skal laves en produktion. Type 4 er delt med et stelplan således at banerne ikke kan se hinanden. Dette print var dog det der performede dårligst da det kun virkede når der lå vand mellem banerne. Kom der først vand ud på stelplanet blev her dannet en kapacitet som var sammenligning med det der var mellem pladerne og virkningen blev herved reduceret. Der blev herefter testet i en så og prikle jord med Type 1,2 og 3 for sammenhæng mellem kapacitet og fugtighed. Se Figur \ref{fig:graffugt}.

\begin{figure}[H]
  \centering
  \begin{minipage}[b]{0.4\textwidth}
  \centering
    \includegraphics[width=0.4\textwidth]{../fig/Jordfugt/type1.jpg}
    \caption{Type 1 jordspyd}
    \label{fig:spyd1}
  \end{minipage}
  \hfill
  \begin{minipage}[b]{0.4\textwidth}
  \centering
    \includegraphics[width=0.31\textwidth]{../fig/Jordfugt/type2.jpg}
    \caption{Type 2 jordspyd}
    \label{fig:spyd2}
  \end{minipage}
\end{figure}


\begin{figure}[H]
  \centering
    \begin{minipage}[b]{0.4\textwidth}
    \centering
    \includegraphics[width=0.35\textwidth]{../fig/Jordfugt/type3.jpg}
    \caption{Type 3 jordspyd}
    \label{fig:spyd3}
  \end{minipage}
  \hfill
    \begin{minipage}[b]{0.4\textwidth}
    \centering
    \includegraphics[width=0.34\textwidth]{../fig/Jordfugt/type4.jpg}
    \caption{Type 4 jordspyd}
    \label{fig:spyd4}
  \end{minipage}
\end{figure}

\begin{figure}[H]
	\centering
	\includegraphics[width=0.6\textwidth]{../fig/Jordfugt/moleopstilling.jpg}
	\caption{Måleopstilling. Alle målinger er foretaget med LCR meter ESCORT ELC-133A ved 10kHz}
	\label{fig:graffugt}
\end{figure}  

\begin{figure}[H]
	\centering
	\includegraphics[width=0.8\textwidth]{../fig/Jordfugt/capVsSoil.PNG}
	\caption{Kapacitet vs. jordfugtighed. Her er målt i en så og priklejord}
	\label{fig:graffugt}
\end{figure}  

Det ses på Figur \ref{fig:graffugt} at Type 1 og type 2 er meget varierende i deres udslag, men at Type 3 har et pænt lineært forhold op til omkring 40\%. Afvigelsen skyldes at ved 40\% bliver jorden så våd at der begynder at ligge mere vand på spyddet end hvad der er i jorden. Derfor bliver kapaciteten højere end hvad den burde. Der kan nu opstilles en formel til beregning af fugtigheden af jorden hvis kapaciten af spyddet er kendt for både tør og våd. 

\begin{equation}
	\theta = \frac{C_{weat} - C_{dry}}{100} \cdot (C-C_{dry}) \cdot \%
\end{equation}


Det er derfor nu muligt at måle jordfugten ved at måle variationen af kondensatoren. Til det er der blevet overvejet to muligheder. Enten at måle amplitude\cite{lib:cap_meas} ændringen når der sættes et AC signal igennem eller at måle faseændringen.  

\textbf{Målemetoder} \\

\begin{figure}[H]
  \centering
  \begin{minipage}[b]{0.4\textwidth}
    \includegraphics[width=1.2\textwidth]{../fig/Jordfugt/diagram1.png}
    \caption{Amplitudemåling af kondensator}
    \label{fig:jdig1}
  \end{minipage}
  \hfill
  \begin{minipage}[b]{0.4\textwidth}
    \includegraphics[width=1.2\textwidth]{../fig/Jordfugt/diagram2.png}
    \caption{Amplitudemåling af kondensator med instrumentationsforstærker}
    \label{fig:jdig2}
  \end{minipage}
\end{figure}

På Figur \ref{fig:jdig1} og \ref{fig:jdig2} ses der hvordan amplitudemålingen kan foretages. På Figur \ref{fig:jdig1} ses der et meget simpelt kredsløb som vil oplade en DC på C1 som hvis spænding afhænger af størrelsen af C\_soil. Kredsløbet er dog meget støjfølsomt og er derfor ugenet til dette formål. Se Sektion \ref{sec:test_amplitudemoling} for bevis af denne påstand. Kredsløbet kan forbedres med en instrumentationsforstærker som på Figur \ref{fig:jdig2}. Desværre er diodeforspændingen meget varierende med temperatur og strømtræk. Derfor vil dette kredsløb hellere ikke være ideel. Derimod er en måling af fasen mere præcis men samtid også mere avanceret. 

\textbf{Målerkredsløb ver.01} \\
Der er designet et kredsløb der konverterer faseændringen om til en DC-spænding som herefter samples af mikroprocessorens ADC. Dette kredsløb ses på Figur \ref{fig:jf2}. Kredsløbet fungerer ved at der påtrykkes en firkant-spænding på 100kHz over jordspyddet som bliver loaded af R25. Dette efterfølges af en spændingsfølger således R25 kan vælges uden forbehold for det efterfølgende kredsløb. Firkant-spændingen filtreres herefter af et båndpass-filter således at der ligger en sinus på udgangen af U12A. Denne sinus er $180^\circ$ faseforskudt da U12A er koblet som en inverterende forstærkter, derfor sættes der en komparator på udgangen af U12A for at returnere fasen til $0^\circ$ og samtisig sikre et hurtigt skift i XOR-gaten U10A. Til orientering, vil der her blive introduceret et propagation delay som vil bidrage en smule til faseforskydelsen, der forventes i omegnen af $80ns$ \cite{lib:comp_MCP6561_datasheet}. Ved at køre frekvensen fra mikroprocessoren samt udgangen af U8A igennem en XOR-gate (U10A). Bliver faseforskellen af de 2 signaler konverteret til et PWM-signal som efterfølgende filtreres således der ligger en spænding på C26 varierende fra 0V-Vcc/2. For at få et bedere dynamisk range er der indsat en inverterende forstærker med en forstærkning på 2.  

\begin{figure}[H]
	\centering
	\includegraphics[width=1\textwidth]{../fig/Jordfugt/diagram3.PNG}
	\caption{Diagram over kredsløb til måling af faseændring ver.01.}
	\label{fig:jf2}
\end{figure}

\textbf{Målerkredsløb beregninger} \\

\textit{Båndpasfilter} \\
Overføringsfunktionen for båndpass-filteret opstilles.
 
\begin{equation}
	G(s) = -\dfrac{R2 C1 R3}{C1 C2 R1 R2 R3 s^2 + C1 R1 R2 s + C2 R1 R2 s + R1+ R2}
	 \label{equ:j2}
\end{equation} 

Hvor R22 = R1, R21 = R2, R16 = R3, C33 = C1, C34 = C2. Det ses at filteret er et 2. ordens og har derfor en dæmpningsfaktor som er tæt relateret til quality-factor eller Q. Q siger noget om hvor snæver båndbredden er i systemet. Komponentværdierne er bestemt via ligning  \ref{equ:j3}, \ref{equ:j4}, \ref{equ:j5}. Hvor $A$ er forstærkningen, $\omega_0 = 2\pi f$ hvor $f$ er frekvensen og C er værdien af C1 og C2.

\begin{equation}
	R1 = \dfrac{Q}{A\cdot \omega _0 \cdot C}
	 \label{equ:j3}
\end{equation} 

\begin{equation}
	R2 = \dfrac{Q}{(2Q^2-A) \cdot \omega_0 \cdot C}
	 \label{equ:j4}
\end{equation} 

\begin{equation}
	R3 = \dfrac{2Q}{\omega _0 \cdot C}
	 \label{equ:j5}
\end{equation} 
  
Sættes forstærkningen til $A=2$, $Q=10$, $f=100kHz$ og $C=470pF$. Fås komponentværdierne som på Figur \ref{fig:jf2}. På Figur \ref{fig:j6} ses der at forstærkningen ligger på 2 ved 100kHz som forventet og på Figur \ref{fig:j7} ses at der er en faseændring på 0 grader, her må dog forventes en faseændringen på $180^\circ$ da U12A er koblet inverterende.

\begin{figure}[H]
  \centering
  \begin{minipage}[b]{0.4\textwidth}
    \includegraphics[width=1.3\textwidth]{../fig/Jordfugt/system_respons.png}
    \caption{Forstærkning vs. frekvens}
    \label{fig:j6}
  \end{minipage}
  \hfill
  \begin{minipage}[b]{0.4\textwidth}
    \includegraphics[width=1.3\textwidth]{../fig/Jordfugt/system_faserespons.png}
    \caption{Fasedrej vs. frekvens}
    \label{fig:j7}
  \end{minipage}
\end{figure}


\textit{Load af jordspyd} \\

Jordspyddet har en kapacitet varierende fra 18pF til omkring 220pF. Opstilles overføringsfunktionen for et 1. ordens RC-højpasled kan den optimale load af jordspyddet udregnes. 

\begin{equation}
	G(s) = \dfrac{R}{\dfrac{1}{s\cdot C} \cdot R}
	 \label{equ:j1}
\end{equation}  

På Figur \ref{fig:jf1} ses der et plot af fase ændringen med en modstand udregnet til R25 = 22k$\Omega$ og en frekvens på 100kHz. Det ses at fasen ændres totalt med omkring $55^\circ$. 

\begin{figure}[H]
	\centering
	\includegraphics[width=0.9\textwidth]{../fig/Jordfugt/fase_vs_cap.PNG}
	\caption{Faseændring versus kapaciteten}
	\label{fig:jf1}
\end{figure} 

På Figur \ref{fig:j8} ses et plot af udgangsspændingen vs. kapaciteten af C26.    

\begin{figure}[H]
	\centering
	\includegraphics[width=1\textwidth]{../fig/Jordfugt/udgangsvolt.PNG}
	\caption{Udgangsspænding vs. kapacitet}
	\label{fig:j8}
\end{figure}

\textit{U12} \\
U12 er blevet valgt til typen TL972 da den har en båndbredde på 12MHz. Det skal dog nævens at ved valg af denne var der 
usikkerhed på om målingen skulle foretages 1MHz \cite{lib:cap_meas} eller lavere. Pga. den store båndbredde bruger den desværre også meget tomgangs strøm som typisk ligger på 2mA pr forstærker. 

\textit{U10} \\
U10 er valgt til typen SN74AHC86 da den har en lav pris og samtidig har et lavt propagation-delay på 8ns.

\textit{U8} \\
U8 er valgt til typen MCP6562 da den har en push-pull udgang og en lav tomgangsstrøm på 130uA pr. forstærker, samt et lavt propagation-delay på 80ns og risetime på 20nS. 

\textit{D5 og D6} \\
Er beskyttelsdioder til ESD. Er blevet valgt til typen BAT54 da de er billige og er meget brugte til netop dette formål.


\subsubsection{Modultest - Målerkredsløb ver.01}
Kredsløbet blev designet på PCB sensor\_3.brd som bruger type 4 jordspyd. Grunden til det var Type 4 jordspyd var at der på dette tidspunkt ikke var lavet test af disse. Ved måling stod det hurtigt klart at D5 var ødelæggende for signalet, da denne klipper den negative del af pulsen under -0.6V. U8B var fejlagtigt en komparator og duede derfor ikke som spændingsfølger, dette ses tydeligt på Figur \ref{fig:scopeU8B}. Endvidere var C35 monteret forkert da denne bør sidde parallelt med R28 som båndbredde-begrænsning. Det viste sig ydermere at faseforsinkelsen igennem kredsløbet var så stor, at indputtet til U1A ben A skulle have sin egen separate puls hvor forsinkelsen kan styres via software. Endvidere var forstærkningen i båndpass-filteret for stort således at signalet klippede ved en kapacitet på 160pF. 

\begin{figure}[H]
  \centering
  \begin{minipage}[b]{0.4\textwidth}
    \includegraphics[width=1\textwidth]{../fig/Jordfugt/scopeR25.jpg}
    \caption{Måling over R25 ved tørt print}
    \label{fig:scopeR25}
  \end{minipage}
  \hfill
  \begin{minipage}[b]{0.4\textwidth}
    \includegraphics[width=1\textwidth]{../fig/Jordfugt/scopeU8B.jpg}
    \caption{Måling på udgang af U8B (øverste) og over R25. Støj skyldes der holdes en finger på jordspyddet}
    \label{fig:scopeU8B}
  \end{minipage}
\end{figure}

\begin{figure}[H]
	\centering
	\includegraphics[width=0.4\textwidth]{../fig/Jordfugt/scope_klip.jpg}
	\caption{Målt på udgangen af U12A. Her ses klipning ved 160pF. Scopet står i dual channel men kanal 2 er ikke koblet til noget}
	\label{fig:j8}
\end{figure}

\textbf{Målerkredsløb ver.02} \\
Efter modultesten af Målerkredsløb ver.01 blev der fortaget en del rettelser til designet hvilket resulterede i et nyt diagram da der samtidig blev designet nyt printudlæg til sensoren (sensor\_4.brd) grundet andre designfejl. I det nye design af målerkredsløbet benyttes der Type 3 jordspyd, designet ses på Figur \ref{fig:j9}.  

\begin{figure}[H]
	\centering
	\includegraphics[width=1.0\textwidth]{../fig/Jordfugt/diagram4.PNG}
	\caption{Målekredsløb ver.2}
	\label{fig:j9}
\end{figure}  

Her er indført en MOSFET-transistor, Q11 således at mikroprocessoren kan tænde og slukke for et separat måle-groundplan.  Herved kan der lukkes ned for hele målerkredsløbet når sensoren lægges i sleep-mode. I mikroprocessoren er der en automatisk kalibreringsfunktion som køres ved tryk på Connect-knappen. Spændingsfølgeren på indgangen er fjernet da den viste sig overflødig. Selvom at R22 ikke er 22k$\Omega$ giver 33k$\Omega$ ikke den store forskel. R22 er ændret til 33k$\Omega$ da forstærkningen er sænket til 1. 

\textbf{Modultest - Målerkredsløb ver.02} \\
Kredsløbet viste sig dog stadig at volde problemer. Det første der blev iagttaget var at den ikke-inverterende forstærker U12B var inverterende men ikke gav udslag som den ville hvis benene 5 og 6 var vendt forkert. Der blev kigget i databladet for U12 og der burde ikke være nogle design fejl. Forstærkeren blev afmonteret således resten af kredsløbet kunne blive testet. Her viste det sig at kredsløbet allerede gav fuldt udslag ved 160pF hvilket er 50\% af fuld kapacitet og at udgangsspændingen herefter faldt en smule. Fuldt udslag burde ikke kunne opstå da der maksimalt burde være en faseforskydelse på omkring $55^\circ$ og at fuldt udslag opstår ved $90^\circ$. På Figur \ref{fig:j10} ses der et scopebillede ved 0\% fugtighed. Den automatiske kalibrerings-funktion er blevet kørt og der ligger 357mV rms på C26. At spændingen ikke kan komme ned på 0V skyldes propagation-delay'et i U8A som gør at pulsen bliver bredere og at mikroprocessoren derfor ikke kan forskyde 100kHz\_delayed så den rammer præcis.   

\begin{figure}[H]
	\centering
	\includegraphics[width=0.9\textwidth]{../fig/Jordfugt/scope_2.PNG}
	\caption{Scopebilled ved 0\% fugtighed. Rød er målt på C26, Grøn U10A ben 4, Gul 100kHz delayed og Grøn er U10A ben 2.}
	\label{fig:j10}
\end{figure}  

\begin{figure}[H]
	\centering
	\includegraphics[width=0.9\textwidth]{../fig/Jordfugt/scope_3.PNG}
	\caption{Scopebilled ved jordspyd stukket ca. halvt ned i vand. Her ses allerede fuldt udslag}
	\label{fig:j11}
\end{figure}  

Det blev nu undersøgt om det kunne skyldes at R22 var udregnet forkert. Derfor blev overføringsfunktionen opskrevet endnu en gang for båndpass-filteret dog inkluderet med kapaciteten fra jordspydet. 

\begin{footnotesize}
\begin{equation}
	G(s) = \frac{s^2 C_{soil} R2 C1 R3}{C1 C2 C_{soil} R1 R2 R3 S^3 + C1 C2 R2 R3 S^2 + C1 C_{soil} R1 R2 s^2 
	+ C1 R2 s+ C2 R2 s + C_{soil} R1 s+ C_{soil} R2 s +1 }
\end{equation} 
\end{footnotesize}

Ved første øjekast stemte det fint overens med det forventede i Figur \ref{fig:j6} og \ref{fig:j7}. Men blev frekvensen sat til 87kHz skete der noget som måske kunne forklare det observerede. Se Figur \ref{fig:j12}. Her ses der et pludseligt faseskift på $90^\circ$ ved omkring 160pF. Faseskiftet vil ikke kunne ses på udgangsspændingen da der ikke gives nogen indikation af positiv eller negativ forskydelse. Det passer fint med at spændingen falder ved en kapacitet større end 160pF. Herefter blev frekvensen fra mikroprocessoren målt og denne lå på 86kHz. Herefter blev knækfrekvensen i båndpass-filteret målt og denne lå på 88 kHz. Se figur \ref{fig:j13}. Afvigelsen skyldes at C33 og C34 var monteret med 10\% tolerance kondensatorer. Men da knækfrekvensen stemte fint overens med frekvensen fra mikroprocessoren kunne fejlen ikke være det. Det blev forsøgt at ændre Q-værdien både op og ned i filteret, men lige meget hjalp det.  

\begin{figure}[H]
	\centering
	\includegraphics[width=0.9\textwidth]{../fig/Jordfugt/system_respons_87khz.PNG}
	\caption{Faseændring ved 87kHz vs. kapaciteten i jordspyddet}
	\label{fig:j12}
\end{figure} 

\begin{figure}[H]
	\centering
	\includegraphics[width=0.9\textwidth]{../fig/Jordfugt/scope_8.PNG}
	\caption{Måling af knækfrekvens i båndpass-filteret}
	\label{fig:j13}
\end{figure} 

Det eneste sted der kunne være en faseforskydelse var i båndpasfilteret, selvom at overføringsfunktionen viste at faseforskydelsen i skulle ligge fast på $0^\circ$ blev det alligevel testes med en spændingsfølger på indgangen, koblet ligesom U8B på Figur \ref{fig:jf2}, dog blev den båndbredde-begrænset til 500kHz. Spændingsfølgeren blev forsynet med både en positiv og en negativ spænding på 3.3V og var af typen LM357. Se Figur \ref{fig:j14} for scopebillede af indgang og udgang. Herefter blev der målt på udgangen af spændingsfølgeren og udgangen af båndpasfilteret (U12A ben 1). Her blev det konstateret at ved 20pF var der en faseforskydelse på $90^\circ$ og ved en kapacitet på 120pF var der en faseforskydelse på $180^\circ$. Dette ses tydeligt på Figur \ref{fig:j15} og \ref{fig:j16}. Holdes dette resultet op med opservationen om at U12B koblet som inverterende forstærker gav en positiv forstærkning er der ingen tvivl om at operationsforstærkerne er i en ustabil kobling. Ses der i databladet for TL972 ses der at fasemarginen ved en load på udgangen ved en load på 250pF ligger omkring $40^\circ$. Ses der på diagrammet for båndpass-filteret kan der ses at udgangen af U12A bliver loadet med 470pF og 171$\Omega$ serielt. Dette kan muligvis resultere i ustabilitet. Filteret blev skaleret ned så kondensatorene C23 og C34 var 47pF. Dette gav desværre ikke noget bedere resultat. Tilbage var kun at skifte operationsforstærkeren ud til en anden type.  

\begin{figure}[H]
	\centering
	\includegraphics[width=0.9\textwidth]{../fig/Jordfugt/scope_14.PNG}
	\caption{Måling af ind og udgang af spændingsfølger. Gul er indgang og Grøn er udgang. Her ses at de skarpe kanter er filtreret fra.}
	\label{fig:j14}
\end{figure} 

\begin{figure}[H]
	\centering
	\includegraphics[width=0.9\textwidth]{../fig/Jordfugt/scope_15.PNG}
	\caption{Faseforskydelse af indgang og udgang i båndpass-filteret ved 20pF. Grøn er indgang og gul er udgang. Her ses ca. $90^\circ$ faseforskydelse. }
	\label{fig:j15}
\end{figure} 

\begin{figure}[H]
	\centering
	\includegraphics[width=0.9\textwidth]{../fig/Jordfugt/scope_16.PNG}
	\caption{Faseforskydelse af indgang og udgang i båndpass-filteret ved 120pF. Her ses ca. $180^\circ$ faseforskydelse.}
	\label{fig:j16}
\end{figure} 

\begin{figure}[H]
	\centering
	\includegraphics[width=0.5\textwidth]{../fig/Jordfugt/datablad1.PNG}
	\caption{Fasemargin vs forsyningsspænding. Operationsforstærkeren bliver loadet med 470pF og 171$\Omega$ serielt.}
	\label{fig:j17}
\end{figure} 

\textit{U12} \\
Der er valgt at erstatte den hidtil benyttede TL972 Operationsfortærker med modellen MCP6102t fra Microchip \cite{lib:opamp_MCP6102t_datasheet}. Denne nye opamp er valgt på baggrund af dens mere passende båndbredde på $1Mhz$. Dertil har den lavere tomgangs-strøm på $85 \mu A$.Derudover har den en slew rate på $0.6V/ \mu s$ der opfylder kravet til systemet som ses i ligning \ref{equ:slewrate}.  

\begin{equation}
	Slew Rate = 2 \times \pi \times f \times V_{max} = 2 \times \pi \times 10kHz \times 3.3V = 0.207V/ \mu s   
	 \label{equ:slewrate}
\end{equation}

\textbf{Test af kredsløb til amplitudemåling af kondensator}  \\ \label{sec:test_amplitudemoling}
 For at bevise påstanden om at kredsløbet i Figur \ref{fig:jdig1} er meget støjfølsomt, er det blevet opbygget på et Type 3 jordspyd. Komponentværdierne er følgende: R1 = 100k D1 = BAT54 C1 = 100nF R2=1M$\Omega$. Der er blevet målt at diode tabet er ca 0.2V. Det ses på Figur \ref{fig:j18} at der ved en printkapacitet på 220pF (fuldt dyppet i vand) ligger 80mVpkpk 50Hz støj. Denne støj kan dog minimeres ved at gøre C1 større, men dette vil også gøre tidskonstanten større således at der skal ventes længere tid før en måling kan foretages. Se Figur \ref{fig:j19} for scope billede af stigetiden fra 20pF til 220pF. Dette vil kræve mere strøm og batteriet vil derfor hurtigere blive afladet. Samtidig er der også problemet med D1 da diodetabet vil ændres med temperatur. 
 
\begin{figure}[H]
	\centering
	\includegraphics[width=0.6\textwidth]{../fig/Jordfugt/scope_19.PNG}
	\caption{Spænding målt over C1 ved en printkapacitet på 220pF. Her ses en stor mængde 50Hz støj}
	\label{fig:j18}
\end{figure} 

\begin{figure}[H]
	\centering
	\includegraphics[width=0.6\textwidth]{../fig/Jordfugt/scope_17.PNG}
	\caption{Stigetiden fra 20pF til 220pF aflæses til 37ms}
	\label{fig:j19}
\end{figure} 
 
\clearpage