\newpage
\subsubsection{Antenne} \label{sec:HWimp_antenne_sensor}

Til sensoren er valgt at anvende en through-hole monteret helical antenne af typen W3127 ISM 433MHz fra \textit{Pulse Electronics}\cite{lib:ant_W3127_datasheet}. Denne antenne er den mest isotropiske af kandidaterne med en Max Gain på $ G_{max}=-2.9 dBi$. jvf. Figur \ref{fig:ant_maxGain_sensor}. 

\begin{figure}[H]
	\centering
	\includegraphics[width=0.5\textwidth]{../fig/antenner/datablad_fig2.png}
	\caption{W3127 MaxGain}
	\label{fig:ant_maxGain_sensor}
\end{figure}


Valgtet af denne antenne og antennetype baserer sig på at udnytte pladsen på sensor-PCB'et bedst muligt for at kunne implementere den bedst mulige antenne. Dog skal det bemærkes  at denne antenne ved dens placering  og orientering på PCB'et når sensoren er placeret i jorden, fås at sensorboksen gerne skal rettes imod kontrolboksens placering for at opnå optimale gain. jfv. Figur \ref{fig:ant_maxGain_sensor}.

\begin{figure}[H]
	\centering
	\includegraphics[width=0.5\textwidth]{../fig/antenner/datablad_fig3_sensor.png}
	\caption{W3127 Gain i ZY-planet}
	\label{fig:ant_GainZY_sensor}
\end{figure}

Den optimale placering af antennen på PCB'et ville være horisontalt jfv. Fig 3 på side i databladet, da dette giver den optimale udstråling langs langs med jorden (XY-planet). Men desværre muliggør pladsen på sensoren ikke dette. Derfor er antennen placeret hvor den er i denne iteration af sensoren.

Antennes karakteristisk impedans er som på kontrolboksen 50 $\Omega$, dog er denne microstrip-bane beregnet som coplaner, da den delvist er omgivet af GND-plan på begge side. Derfor bliver banebredden anderledes end på kontrolboksen, beregning er lavet i AWR mircrowave office og ses på Figur \ref{fig:ant_cpw}

\begin{figure}[H]
	\centering
	\includegraphics[width=0.7\textwidth]{../fig/antenner/cpw.png}
	\caption{Beregning af coplanar transmissionlinie i Microwave Office}
	\label{fig:ant_cpw}
\end{figure}

\newpage