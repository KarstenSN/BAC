\subsubsection{Jordfugtmåler} \label{sec:SWimp_jordfugtmoler}

\begin{figure}[H]
	\centering
	\includegraphics[width=0.8\textwidth]{../fig/Klassebeskrivelser/soilsensor/soilsensor.pdf}
	\caption{Klassediagram over \texttt{soilsensor}}
\end{figure} 

\textit{Variable-beskrivelser} \\
\begin{table}[H]
		\begin{tabularx}{\textwidth}{| L{2.5cm} | l | Z |} \hline
			Navn 					& Type 				& Beskrivelse \\ \hline
			\texttt{adcMin}	& \texttt{uint8\_t}	& Indeholder den minimale værdi som kan samples af ADC'en når sensoren er placeret i tør luft. \\ \hline
		\end{tabularx}
		\caption{Beskrivelse af variablen \texttt{adcMin}}
\end{table}

\begin{table}[H]
		\begin{tabularx}{\textwidth}{| L{2.5cm} | l | Z |} \hline
			Navn 					& Type 				& Beskrivelse \\ \hline
			\texttt{adcMax}	& \texttt{uint8\_t}	& Indeholder den maksimale værdi som kan samples af ADC'en når sensoren er placeret rent vand. \\ \hline
		\end{tabularx}
		\caption{Beskrivelse af variablen \texttt{adcMax}}
\end{table}


\textit{Funktions-beskrivelse} \\

\begin{table}[H]
	\begin{tabularx}{\textwidth}{| l | Z |} \hline
		Funktion 	& \texttt{enableCounter100khz} \\\hline
		Parametre 	& Ingen \\\hline
		Returværdi 	& void \\\hline
		Beskrivelse & Enabler \texttt{TIMER1} til at give en firkant-spænding ud på både \texttt{PB1} og \texttt{PB2}. Firkanten på \texttt{PB2} er forskudt $90\circ$ i forhold til \texttt{PB1} som standard. Denne forsinkelse kan justeres af funktionen \texttt{calibrate()}.    \\\hline
	\end{tabularx}
	\caption{Metodebeskrivelse for \texttt{enableCounter100khz}}
\end{table}

\begin{table}[H]
	\begin{tabularx}{\textwidth}{| l | Z |} \hline
		Funktion 	& \texttt{disableCounter100khz} \\\hline
		Parametre 	& Ingen \\\hline
		Returværdi 	& void \\\hline
		Beskrivelse & Disabler \texttt{TIMER1} således firkant-spændingen stopper.  \\\hline
	\end{tabularx}
	\caption{Metodebeskrivelse for \texttt{disableCounter100khz}}
\end{table}


\begin{table}[H]
	\begin{tabularx}{\textwidth}{| l | Z |} \hline
		Funktion 	& \texttt{adcGetResult} \\\hline
		Parametre 	& Ingen \\\hline
		Returværdi 	& uint16\_t \\\hline
		Beskrivelse & Initierer ADC kanal 2 med en prescale på 128. Herefter tages der 100 målinger som efterfølgende midles og returneres.   \\\hline
	\end{tabularx}
	\caption{Metodebeskrivelse for \texttt{adcGetResult}}
\end{table}

\begin{table}[H]
	\begin{tabularx}{\textwidth}{| l | Z |} \hline
		Funktion 	& \texttt{getHumidity} \\\hline
		Parametre 	& \_humidity1: uint8\_t*, \_humidity2: uint8\_t* \\\hline
		Returværdi 	& void \\\hline
		Beskrivelse & returnerer fugtigheden i procent. \_humidity1 er heltallet og \_humidity 2 er kommatallet. Fugtigheden udregnes ved brug af formlen
		$a\cdot exp(b\cdot x) + c\cdot exp(d\cdot x)$ hvor x er resultatet fra funktionen \texttt{adcGetResult()}. a, b, c og d er statiske variabler.   \\\hline
	\end{tabularx}
	\caption{Metodebeskrivelse for \texttt{getHumidity}}
\end{table}

\begin{table}[H]
	\begin{tabularx}{\textwidth}{| l | Z |} \hline
		Funktion 	& \texttt{calibrate} \\\hline
		Parametre 	& Ingen \\\hline
		Returværdi 	& uint8\_t \\\hline
		Beskrivelse & Kalibrer \texttt{TIMER1} således at faseforskellen mellem \texttt{PB1} og \texttt{PB2} giver den mindst mulige værdi der kan samples af ADC'en. Funktionen opdaterer \texttt{adcMin}. Funktionen forudsætter at sensoren er placeret i tør luft.  \\\hline
	\end{tabularx}
	\caption{Metodebeskrivelse for \texttt{calibrate}}
\end{table}

\begin{table}[H]
	\begin{tabularx}{\textwidth}{| l | Z |} \hline
		Funktion 	& \texttt{soilSensorInit} \\\hline
		Parametre 	& \_soilSensor: struct soilSensor* \\\hline
		Returværdi 	& void \\\hline
		Beskrivelse & Giver funktionspointere videre og kalder funktionen \texttt{enableCounter100khz()}.  \\\hline
	\end{tabularx}
	\caption{Metodebeskrivelse for \texttt{soilSensorInit}}
\end{table}



