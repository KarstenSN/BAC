\subsubsection{lysmåler} \label{sec:SWimp_lysmoler}

\begin{figure}[H]
	\centering
	\includegraphics[width=0.6\textwidth]{../fig/Klassebeskrivelser/lysmoler/lysmoler.pdf}
	\caption{Klassediagram over \texttt{lightdet}}
\end{figure}

\textit{Variable-beskrivelse}
\begin{table}[H]
		\begin{tabularx}{\textwidth}{| L{2.5cm} | l | Z |} \hline
			Navn 					& Type 					& Beskrivelse \\ \hline
			\texttt{oldIntensity}	& \texttt{uint18\_t}	& Indeholder værdien af intensiteten som blev målt sidste gang funktionen \texttt{getIntensity} blev kaldt \\ \hline
		\end{tabularx}
		\caption{Beskrivelse af variablen \texttt{oldIntensity}}
\end{table}

\textit{Funktions-beskrivelse}

\begin{table}[H]
	\begin{tabularx}{\textwidth}{| l | Z |} \hline
		Funktion 	& \texttt{getIntensity} \\\hline
		Parametre 	& Ingen 				\\\hline
		Returværdi 	& uint8\_t 				\\\hline
		Beskrivelse & Opstætter ADC kanal 3 og måler intensiteten fra lysdioden. Er spændingen over 1V antages det at være dag. Er spændingen under 1V antages det at være nat. Er \texttt{oldIntensity} lig med \texttt{LIGHTDET\_DAYTIME} og bliver der målt at det nu er nat returneres \texttt{LIGHTDET\_MORNING}. Har det været nat og bliver det nu dag returneres \texttt{LIGHTDET\_MORNING}. Ellers returneres \texttt{LIGHTDET\_DAYTIME} eller \texttt{LIGHTDET\_NIGHTTIME}.  \\\hline
	\end{tabularx}
	\caption{Metodebeskrivelse for \texttt{getIntensity}}
\end{table}

\begin{table}[H]
	\begin{tabularx}{\textwidth}{| l | Z |} \hline
		Funktion 	& \texttt{lightDetInit} \\\hline
		Parametre 	& \_lightDet: struct lightDet* \\\hline
		Returværdi 	& void \\\hline
		Beskrivelse & Giver funktionspointere videre til structen.  \\\hline
	\end{tabularx}
	\caption{Metodebeskrivelse for \texttt{lightDetInit}}
\end{table}

\newpage