\subsubsection{Mikroprocessor} \label{sec:SWimp_mikroprocessor_kontrolboks}

\begin{figure}[H]
	\centering
	\includegraphics[width=0.5\textwidth]{../fig/Klassebeskrivelser/mikroprocessor/mikroprocessor_kontrol.pdf}
	\caption{Klassediagram over mikroprocessor}
	\label{fig:class_mikroprocessor}
\end{figure} 

\textit{Variable-beskrivelser}

\begin{table}[H]
		\begin{tabularx}{\textwidth}{| L{3.8cm} | l | Z |} \hline
			Navn 				& Type 				& Beskrivelse \\ \hline
			\texttt{display}	& \texttt{i2cDisp}	& Driver til displayet som er monteret på kontrolboksen \\ \hline
		\end{tabularx}
		\caption{Beskrivelse af structen \texttt{display}}
\end{table}

\begin{table}[H]
		\begin{tabularx}{\textwidth}{| L{3.8cm} | l | Z |} \hline
			Navn 			& Type 					& Beskrivelse \\ \hline
			\texttt{valve}	& \texttt{motorvalve}	& Driver til motorventilen \\ \hline
		\end{tabularx}
		\caption{Beskrivelse af structen \texttt{valve}}
\end{table}

\begin{table}[H]
		\begin{tabularx}{\textwidth}{| L{3.8cm} | l | Z |} \hline
			Navn 			& Type 				& Beskrivelse \\ \hline
			\texttt{keypad}	& \texttt{kaypad}	& Driver til tastaturet \\ \hline
		\end{tabularx}
		\caption{Beskrivelse af structen \texttt{keypad}}
\end{table}

\begin{table}[H]
		\begin{tabularx}{\textwidth}{| L{3.8cm} | l | Z |} \hline
			Navn 					& Type 					& Beskrivelse \\ \hline
			\texttt{transceiver}	& \texttt{transceiver}	& Driver til transceiveren \\ \hline
		\end{tabularx}
		\caption{Beskrivelse af structen \texttt{transceiver}}
\end{table}

\begin{table}[H]
		\begin{tabularx}{\textwidth}{| L{3.8cm} | l | Z |} \hline
			Navn 					& Type 						& Beskrivelse \\ \hline
			\texttt{sensorPackage}	& \texttt{sensor\_package}	& struct som indeholder data fra sensoren \\ \hline
		\end{tabularx}
		\caption{Beskrivelse af structen \texttt{sensorPackage}}
\end{table}

\begin{table}[H]
		\begin{tabularx}{\textwidth}{| L{3.8cm} | l | Z |} \hline
			Navn 						& Type 							& Beskrivelse \\ \hline
			\texttt{controlBoxPackage}	& \texttt{controlbox\_package}	& struct som indeholder data fra kontrolboksen \\ \hline
		\end{tabularx}
		\caption{Attributter for klassen \texttt{controlBoxPackage}}
\end{table}

\begin{table}[H]
		\begin{tabularx}{\textwidth}{| L{3.8cm} | l | Z |} \hline
			Navn 				& Type 					& Beskrivelse \\ \hline
			\texttt{settings}	& \texttt{settingsage}	& struct som indeholder indstillinger lavet af brugeren. Disse sætted ved at kalde funktionen \texttt{menu()}. \\ \hline
		\end{tabularx}
		\caption{Attributter for klassen \texttt{settings}}
\end{table}

\begin{table}[H]
		\begin{tabularx}{\textwidth}{| L{3.8cm} | l | Z |} \hline
			Navn 			& Type 				& Beskrivelse \\ \hline
			\texttt{clock}	& \texttt{clock}	& struct som indeholder system klokken. Denne tælles op hvert sekund af \texttt{ISR(TIMER1\_COMPA\_vect)}. \\ \hline
		\end{tabularx}
		\caption{Attributter for klassen texttt{clock}}
\end{table}


%%%%%%%%%%%%%%%%%%%%%%%%%%%%%%%%%%%%%%%%%%%%%%%%%%%%%%%%%%%%%%%%%%%%%Funktions beskrivelser%%%%%%%%%%%%%%%%%%%%%%%%%
\textit{Funktions-beskrivelser}

\begin{table}[H]
		\begin{tabularx}{\textwidth}{| l | Z |} 		\hline
			Funktion 	& \texttt{delaySec} 			\\ \hline
			Parametre 	& uint8\_t sec					\\ \hline
			Returværdi 	& void							\\ \hline
			Beskrivelse & venter i x antal sekunder angivet at \texttt{sec}.   \\ \hline
		\end{tabularx}
		\caption{Metodebeskrivelse for \texttt{delaySec}}
\end{table}

\begin{table}[H]
		\begin{tabularx}{\textwidth}{| l | Z |} \hline
			Funktion 	& \texttt{enableCounter1Sek} 	\\ \hline
			Parametre 	& ingen							\\ \hline
			Returværdi 	& void							\\ \hline
			Beskrivelse & Enabler \texttt{TIMER1} til at give et interrupt hvert sekund.   \\ \hline
		\end{tabularx}
		\caption{Metodebeskrivelse for \texttt{enableCounter1Sek}}
\end{table}

\begin{table}[H]
		\begin{tabularx}{\textwidth}{| l | Z |} \hline
			Funktion 	& \texttt{updateDataToDisp} 	\\ \hline
			Parametre 	& ingen							\\ \hline
			Returværdi 	& void							\\ \hline
			Beskrivelse & Opdaterer hoved-menuen på kontrolboksen. Fugtighed og temperatur udskrives således brugeren kan aflæse det.   \\ \hline
		\end{tabularx}
		\caption{Metodebeskrivelse for \texttt{updateDataToDisp}}
\end{table}

\begin{table}[H]
		\begin{tabularx}{\textwidth}{| l | Z |} \hline
			Funktion 	& \texttt{findSensorAndPair} 	\\ \hline
			Parametre 	& ingen							\\ \hline
			Returværdi 	& char							\\ \hline
			Beskrivelse & Funktionen tildeler et tilfældigt ID til sensoren. Se Figur \ref{fig:parring} for overblik over parringssekvensen. Når denne funktion eksekveres vil kontrolboksen stoppe \texttt{TIMER2} som blev startet i det øjeblik brugeren trykkede på en tast på tastaturet og give dens værdi som Id til \texttt{controlBoxPackage}. Herefter sendes \texttt{controlBoxPackage} til sensoren og sensoren svarer tilbage med samme ID i \texttt{sensorPackage}. Går parringen godt returneres 1 ellers returneres 0. \\ \hline
		\end{tabularx}
		\caption{Metodebeskrivelse for \texttt{findSensorAndPair}}
\end{table}

\begin{table}[H]
		\begin{tabularx}{\textwidth}{| l | Z |} \hline
			Funktion 	& \texttt{menu} 				\\ \hline
			Parametre 	& ingen 						\\ \hline
			Returværdi 	& void							\\ \hline
			Beskrivelse & Kaldes når brugeren trykker på en vilkårlig tast på tastaturet. Funktionen giver mulighed for brugeren at navigere rundt i systemets Use Cases og herved indstille systemet til det ønskede.    \\ \hline
		\end{tabularx}
		\caption{Metodebeskrivelse for \texttt{menu}}
\end{table}

\begin{table}[H]
		\begin{tabularx}{\textwidth}{| l | Z |} \hline
			Funktion 	& \texttt{control} 				\\ \hline
			Parametre 	& ingen 						\\ \hline
			Returværdi 	& void							\\ \hline
			Beskrivelse & Kaldes hvert minut af \texttt{ISR(TIMER1\_COMPA\_vect)}. Funktionen aflæser \texttt{settings} og     sørger for at åbne eller lukke for \texttt{valve} når det passer med brugerens indstillinger.\\ \hline
		\end{tabularx}
		\caption{Metodebeskrivelse for \texttt{control}}
\end{table}

\begin{table}[H]
		\begin{tabularx}{\textwidth}{| l | Z |} \hline
			Funktion 	& \texttt{main} 				\\ \hline
			Parametre 	& void 							\\ \hline
			Returværdi 	& void							\\ \hline
			Beskrivelse & Funktionen som først sørger for at initiere alt hardware og give give funktionspointere videre via \texttt{init}. Efterfølgende læses brugerens indstillinger fra EEPROM'en. Er det første gang funktionen kører læses der ikke fra EEPROM'en. Herefter kaldes funktionen \texttt{enableCounter1Sek()} og der ventes i en uendelig while-løkke hvor der aflæses om der trykkes på tastaturet eller om der er modtaget data fra sensoren. Er der trykket på tastaturet startes \texttt{TIMER1} og funktionen \texttt{menu()} kaldes. Er der modtaget nyt data kaldes funktionen \texttt{updateDataToDisp()}.   		\\ \hline
		\end{tabularx}
		\caption{Metodebeskrivelse for \texttt{main}}
\end{table}

\begin{table}[H]
		\begin{tabularx}{\textwidth}{| l | Z |} \hline
			Funktion 	& \texttt{ISR(INT0\_vect)} 		\\ \hline
			Parametre 	& INT0\_vect					\\ \hline
			Returværdi 	& void							\\ \hline
			Beskrivelse & Interrupt som kaldes når der eksterne interrupt INT0 forekommer. Rutinen aflæser statusregisteret i \texttt{transceiver}. Er der interrupt om lavt batteri informeres brugeren herom, eller er der nyt data klar modtages der en byte i \texttt{sensorPackage}.									\\ \hline
		\end{tabularx}
		\caption{Metodebeskrivelse for \texttt{ISR(INT0\_vect)}}
\end{table}

\begin{table}[H]
		\begin{tabularx}{\textwidth}{| l | Z |} \hline
			Funktion 	& \texttt{ISR(TIMER1\_COMPA\_vect)} 	\\ \hline
			Parametre 	& TIMER1\_COMPA\_vect			\\ \hline
			Returværdi 	& void							\\ \hline
			Beskrivelse & Interrupt som kaldes på \texttt{TIMER1} compare-match. Dette sker hvert sekund. \texttt{clock} tælles én op og hvert minut kaldes \texttt{control()}.						\\ \hline
		\end{tabularx}
		\caption{Metodebeskrivelse for \texttt{ISR(TIMER1\_COMPA\_vect)}}
\end{table}

\begin{figure}[H]
	\centering
	\includegraphics[width=0.5\textwidth]{../fig/Kommunikation/parring1.pdf}
	\caption{Parringssekvens}
	\label{fig:parring}
\end{figure}