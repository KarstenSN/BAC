\subsubsection{SensorPakke} \label{sec:SWimp_sensorPakke}

Som det ses på Figur \ref{fig:sensorpakke1} skal sensorpakken bestå af i alt 11 bytes. To bytes som tilsammen udgør et startword til \texttt{transceiver} samt 1 byte til hhv. ID og batteristatus,2 temperatur-bytes, 2 fugtigheds-bytes, 1 byte til lystyrke samt 2 Cheksum-bytes. 

\begin{figure}[H]
	\centering
	\includegraphics[width=0.5\textwidth]{../fig/Kommunikation/sensorpakke.pdf}
	\caption{Sensorpakke}
	\label{fig:sensorpakke1}
\end{figure}

Implementeringen af sensorpakken ses på Figur \ref{fig:sensorpakke2}. 

\begin{figure}[H]
	\centering
	\includegraphics[width=0.6\textwidth]{../fig/Klassebeskrivelser/packages/sensorpackage.pdf}
	\caption{Klassediagram over \texttt{sensorPackage}}
	\label{fig:sensorpakke2}
\end{figure}

\textit{Variable-beskrivelser}
\begin{table}[H]
		\begin{tabularx}{\textwidth}{| L{3.8cm} | l | Z |} \hline
			Navn 					& Type 				& Beskrivelse \\ \hline
			\texttt{startWord}	& \texttt{uint16\_t}	& Variabel som indeholder startword'et til \texttt{transceiver}. \\ \hline
		\end{tabularx}
		\caption{Beskrivelse af variablen \texttt{startWord}}
\end{table}

\begin{table}[H]
		\begin{tabularx}{\textwidth}{| L{3.8cm} | l | Z |} \hline
			Navn 					& Type 				& Beskrivelse \\ \hline
			\texttt{checksum}	& \texttt{uint16\_t}	& Variabel som indeholder checksummen på hele pakken. \\ \hline
		\end{tabularx}
		\caption{Beskrivelse af variablen \texttt{checksum}}
\end{table}

\begin{table}[H]
		\begin{tabularx}{\textwidth}{| L{3.8cm} | l | Z |} \hline
			Navn 					& Type 				& Beskrivelse \\ \hline
			\texttt{receivedChecksum}	& \texttt{uint16\_t}	& Variabel som indeholder checksummen på den modtagende pakke. \\ \hline
		\end{tabularx}
		\caption{Beskrivelse af variablen \texttt{receivedChecksum}}
\end{table}

\begin{table}[H]
		\begin{tabularx}{\textwidth}{| L{3.8cm} | l | Z |} \hline
			Navn 					& Type 				& Beskrivelse \\ \hline
			\texttt{batteryStatus}	& \texttt{char}	& Variabel som indeholder batteri-status. Er der for lavt batteri sættes variablen = 1. \\ \hline
		\end{tabularx}
		\caption{Beskrivelse af variablen \texttt{batteryStatus}}
\end{table}

\begin{table}[H]
		\begin{tabularx}{\textwidth}{| L{3.8cm} | l | Z |} \hline
			Navn 					& Type 				& Beskrivelse \\ \hline
			\texttt{temp1}	& \texttt{char}	& Variabel som indeholder heltallet af temperaturen \\ \hline
		\end{tabularx}
		\caption{Beskrivelse af variablen \texttt{temp1}}
\end{table}

\begin{table}[H]
		\begin{tabularx}{\textwidth}{| L{3.8cm} | l | Z |} \hline
			Navn 					& Type 				& Beskrivelse \\ \hline
			\texttt{temp2}	& \texttt{char}	& Variabel som indeholder kommatallet af temperaturen \\ \hline
		\end{tabularx}
		\caption{Beskrivelse af variablen \texttt{temp2}}
\end{table}

\begin{table}[H]
		\begin{tabularx}{\textwidth}{| L{3.8cm} | l | Z |} \hline
			Navn 					& Type 				& Beskrivelse \\ \hline
			\texttt{Id}	& \texttt{char}	& Variabel som indeholder Id'et af pakken \\ \hline
		\end{tabularx}
		\caption{Beskrivelse af variablen \texttt{Id}}
\end{table}

\begin{table}[H]
		\begin{tabularx}{\textwidth}{| L{3.8cm} | l | Z |} \hline
			Navn 					& Type 				& Beskrivelse \\ \hline
			\texttt{humidity1}	& \texttt{char}	& Variabel som indeholder heltallet af jordfugtigheden \\ \hline
		\end{tabularx}
		\caption{Beskrivelse af variablen \texttt{humidity1}}
\end{table}

\begin{table}[H]
		\begin{tabularx}{\textwidth}{| L{3.8cm} | l | Z |} \hline
			Navn 					& Type 				& Beskrivelse \\ \hline
			\texttt{humidity2}	& \texttt{char}	& Variabel som indeholder kommatallet af jordfugtigheden \\ \hline
		\end{tabularx}
		\caption{Beskrivelse af variablen \texttt{humidity2}}
\end{table}

\begin{table}[H]
		\begin{tabularx}{\textwidth}{| L{3.8cm} | l | Z |} \hline
			Navn 					& Type 				& Beskrivelse \\ \hline
			\texttt{light}	& \texttt{char}	& Variabel som indeholder status på om det er nat eller dag. Er det nat er variablen 0 er det dag er den 1. \\ \hline
		\end{tabularx}
		\caption{Beskrivelse af variablen \texttt{light}}
\end{table}

\begin{table}[H]
		\begin{tabularx}{\textwidth}{| L{3.8cm} | l | Z |} \hline
			Navn 					& Type 				& Beskrivelse \\ \hline
			\texttt{newDataReady}	& \texttt{uint16\_t}	& Variabel som sættes til 1 når \texttt{transceiver} har modtaget en fuld pakke. \\ \hline
		\end{tabularx}
		\caption{Beskrivelse af variablen \texttt{newDataReady}}
\end{table}


\textit{Funktions-beskrivelse}

\begin{table}[H]
	\begin{tabularx}{\textwidth}{| l | Z |} \hline
		Funktion 	& \texttt{testChecksum} \\\hline
		Parametre 	& \_package: struct sensor\_package* \\\hline
		Returværdi 	& void \\\hline
		Beskrivelse & Udregner checksummen i den nuværende pakke og sammenligner den med \texttt{receivedChecksum}. Er de ens returneres 1 ellers returneres 0. \\\hline
	\end{tabularx}
	\caption{Metodebeskrivelse for \texttt{testChecksum}}
\end{table}

\begin{table}[H]
	\begin{tabularx}{\textwidth}{| l | Z |} \hline
		Funktion 	& \texttt{updateChecksum} \\\hline
		Parametre 	& \_package: struct sensor\_package* \\\hline
		Returværdi 	& void \\\hline
		Beskrivelse & Opdaterer \texttt{checksum} \\\hline
	\end{tabularx}
	\caption{Metodebeskrivelse for \texttt{updateChecksum}}
\end{table}
