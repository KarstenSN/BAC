\subsubsection{Transceiver} \label{sec:SWimp_transceiver_kontrolboks}

\todo[inline]{første afsnit skal lige rettes til da det er kopieret direkte fra HW-imp (frem til klasse diagram)}
MRF49XA har i alt 16 indbyggede registre som der benyttes til at konfigurere IC'en med. For et overblik over disse henvises der til databladet\cite{lib:trans_MRFX49A_datasheet}. Herunder gennemgås hvordan der skrives til og læses fra et enkelt register. \texttt{STSREG} er statusregistret og er det eneste register der kan læses fra. Når statusregistret læses er der oftest blevet givet et eksternt interrupt til mikroprocessoren, dette interrupt kan være forsaget af mange hændelser og disse aflæses i statusregistret. Hændelser der forsager interruptet kan eks. være, at der er for lav batterispænding på sensoren, at der er wake-up timer overflow, eller at der er modtaget en byte fra senderen osv. Overblik over \texttt{STSREG} kan ses i Figur \ref{fig:mrfx49a_stsreg} og strukturen for en fuld læsning ses i Figur \ref{fig:mrfx49a_read}. Hvis bit 10 er "1" når status registret læses betyder det "Low Battery Detected", for yderligere info om statusregistret henvises til databladet \cite{lib:trans_MRFX49A_datasheet}.

\begin{figure}[H]
	\centering
	\includegraphics[width=0.8\textwidth]{../fig/transciver/mrfx49a_stsreg.PNG}
	\caption{Overblik over \texttt{STSREG}}
	\label{fig:mrfx49a_stsreg}
\end{figure}

\begin{figure}[H]
	\centering
	\includegraphics[width=0.8\textwidth]{../fig/transciver/mrfx49a_read.PNG}
	\caption{Læsning af \texttt{STSREG}}
	\label{fig:mrfx49a_read}
\end{figure}

Ved opstart skrives der først til \texttt{GENCREG}-registret som er det generelle konfigurations register. Her vælges hvilken frekvens transceiveren skal kommunikere ved, hvilken load kapacitet krystallet skal bruge, om TX-data registret skal aktiveres og om FIFO-registret skal aktiveres. Overblik over \texttt{GENCREG} kan ses på Figur \ref{fig:mrfx49a_gencreg}.

\begin{figure}[H]
	\centering
	\includegraphics[width=0.8\textwidth]{../fig/transciver/mrfx49a_gencreg.PNG}
	\caption{Overblik over \texttt{GENCREG}-registret}
	\label{fig:mrfx49a_gencreg}
\end{figure}

\texttt{CBB<15:8>} indeholder den første byte som skal sendes til receiveren. Denne byte angiver hvilket register der ønskes skrevet til. For \texttt{GENCREG} er andressen \texttt{0x80H}. Funktionaliteten sættes i bits 7-0. Ønskes der at loade krystallet med $10pF$, at kommunikere ved 433MHz, aktivering af FIFO samt TX registret, skal denne byte indeholde \texttt{0xD3H}. En komplet skrivning skal derfor indeholde \texttt{0x80D3H}. Den eneste forskel fra læsesekvensen i Figur \ref{fig:mrfx49a_read} er at \texttt{SDI} nu indeholder data i stedet for \texttt{SDO}. 

\begin{figure}[H]
	\centering
	\includegraphics[width=0.8\textwidth]{../fig/Klassebeskrivelser/rfTransciver/rfTransciver.pdf}
	\caption{Klassediagram over RF-Transciveren}
	\label{fig:class_rftransciver}
\end{figure} 

På Figur \ref{fig:class_rftransciver} ses klassediagram over \texttt{transceiver}.Koden er skrevet i C og det er derfor svært at give en korrekt standardiseret beskrivelse af klassen som en traditionel C++ klasse. Funktionerne er i stedet implementeret i en en struct. Det vil sige at struct'en oprettes med funktionspointere til alle public funktioner, disse er markeret med + foran funktionen. Ved private members er der sat - foran funktionen. Alle funktioner er implementeret i en .c fil hvor struct'en er implementeret i en .h fil. Det er derfor kun muligt at se de funktioner som struct'en initieres med i init-funktionen. Denne funktionen er en ekstern funktion da den ligger uden for struct'en og bliver kaldt med \texttt{transceiver} som parameter. Funktionen giver pointere fra .c filen videre til struct'en. Private members er funktioner som kun kan ses af selve .c filen selv og disse funktioner kan derfor ikke ses i .h filen.

\textbf{Variabel beskrivelse}
	\begin{table}[H]
		\begin{tabularx}{\textwidth}{| L{2.5cm} | l | Z |} \hline
			Navn 					& Type 				& Beskrivelse \\ \hline
			\texttt{mrf\_Register}	& \texttt{struct}	& Bliver brugt til at gemme værdierne af i alt 17 registre, som er konfigurerbare på transceiveren. Disse register sættes 																og skrives til transceiveren i initierings-funktionen \texttt{rfTransceiverInit}. Når der skal ændres på nogle 																		funktioner i transceiveren er det vigtigt at vide hvad der i forvejen står i et register, for at undgå at overskrive en 															indstilling. Defor opdateres denne struct hver gang et register ændres. \\ \hline
		\end{tabularx}
		\caption{Attributter for klassen MainWindow}
	\label{table:attr_distancesensor}
\end{table}


\textbf{Funktionsbeskrivelser}
	\begin{table}[H]
		\begin{tabularx}{\textwidth}{| l | Z |} \hline
			Funktion 	& \texttt{SendByte} 	\\ \hline
			Parametre 	& \_byte: uint16\_t 	\\ \hline
			Returværdi 	& void					\\ \hline
			Beskrivelse & Sender en enkelt byte til \texttt{TXBREG} i transceiveren. Som efterfølgende moduleres ud på antennen. \\ \hline
		\end{tabularx}
		\caption{Metodebeskrivelse for \texttt{Sendbyte}}
\end{table}


\begin{table}[H]
	\begin{tabularx}{\textwidth}{| l | Z |} \hline
		Funktion 	& \texttt{enableReceiver} \\\hline
		Parametre 	& Ingen \\\hline
		Returværdi 	& void\\\hline
		Beskrivelse & Enabler modtageren i transceiveren. \\\hline
	\end{tabularx}
	\caption{Metodebeskrivelse for \texttt{enableTransceiver}}
\end{table}


\begin{table}[H]
	\begin{tabularx}{\textwidth}{| l | Z |} \hline
		Funktion 	& \texttt{disableReceiver} \\\hline
		Parametre 	& Ingen \\\hline
		Returværdi 	& void\\\hline
		Beskrivelse & Disabler modtageren i transceiveren. \\\hline
		\end{tabularx}
	\caption{Metodebeskrivelse for \texttt{disableTransceiver}}
\end{table}


\begin{table}[H]
	\begin{tabularx}{\textwidth}{| l | Z |} 	\hline
		Funktion 	& \texttt{getFifo} 			\\ \hline
		Parametre 	& Ingen 					\\ \hline
		Returværdi 	& uint8\_t					\\ \hline
		Beskrivelse & Læser hvad der står i \texttt{FIFOREG}\\\hline
		\end{tabularx}
	\caption{Metodebeskrivelse for \texttt{getFifo}}
\end{table}


\begin{table}[H]
	\begin{tabularx}{\textwidth}{| l | Z |} 	\hline
		Funktion 	& \texttt{writeRegister} 	\\\hline
		Parametre 	& \_value: uint16\_t  		\\\hline
		Returværdi 	& void						\\\hline
		Beskrivelse & Bruges til at skrive til et register i transceiveren. Registret OR'ed sammen med dets settings bliver givet med som parametre. \\\hline
		\end{tabularx}
	\caption{Metodebeskrivelse for \texttt{writeRegister}}
\end{table}


\begin{table}[H]
	\begin{tabularx}{\textwidth}{| l | Z |} 	\hline
		Funktion 	& \texttt{getStatus} 		\\\hline
		Parametre 	& Ingen  					\\\hline
		Returværdi 	& uint16\_t 				\\\hline
		Beskrivelse & Kaldes når transceiveren har afgivet interrupt og læser herefter \texttt{STATUSREG}, således mikroprocessoren ved hvad der forårsagede interruptet. \\\hline
		\end{tabularx}
	\caption{Metodebeskrivelse for \texttt{writeRegister}}
\end{table}

\begin{table}[H]
	\begin{tabularx}{\textwidth}{| l | Z |} 	\hline
		Funktion 	& \texttt{resetOn} 			\\\hline
		Parametre 	& Ingen 					\\\hline
		Returværdi 	& void 						\\\hline
		Beskrivelse & Sætter reset porten på transceiveren lav. \\\hline
		\end{tabularx}
	\caption{Metodebeskrivelse for \texttt{resetOn}}
\end{table}

\begin{table}[H]
	\begin{tabularx}{\textwidth}{| l | Z |} 	\hline
		Funktion 	& \texttt{sleep} 			\\\hline
		Parametre 	& Ingen 					\\\hline
		Returværdi 	& void 						\\\hline
		Beskrivelse & Ligger transceiveren i sleep mode \\\hline
		\end{tabularx}
	\caption{Metodebeskrivelse for \texttt{sleep}}
\end{table}

\begin{table}[H]
	\begin{tabularx}{\textwidth}{| l | Z |} 	\hline
		Funktion 	& \texttt{wakeup} 			\\\hline
		Parametre 	& Ingen 					\\\hline
		Returværdi 	& void 						\\\hline
		Beskrivelse & Sæter transceiveren i IDLE-mode \\\hline
		\end{tabularx}
	\caption{Metodebeskrivelse for \texttt{wakeup}}
\end{table}

\begin{table}[H]
	\begin{tabularx}{\textwidth}{| l | Z |} 	\hline
		Funktion 	& \texttt{wakeUpTimerOn} 			\\\hline
		Parametre 	& \_WTEV:uint8\_t, \_WTMV:uint8\_t 					\\\hline
		Returværdi 	& void 						\\\hline
		Beskrivelse & Enabler wake-up-timeren på transceiveren. \_WTEV og \_WTMV skrives direkte til \texttt{WTSREG} i transceiveren. Se Figur \ref{fig:wake-up} for beregning af disse. \\\hline
		\end{tabularx}
	\caption{Metodebeskrivelse for \texttt{wakeUpTimerOn}}
\end{table}

\begin{figure}[H]
	\centering
	\includegraphics[width=1\textwidth]{../fig/Klassebeskrivelser/rfTransciver/wake-up.PNG}
	\caption{Beregning af \_WTEV:uint8\_t, \_WTMV:uint8\_t}
	\label{fig:wake-up}
\end{figure}


\begin{table}[H]
	\begin{tabularx}{\textwidth}{| l | Z |} 	\hline
		Funktion 	& \texttt{wakeUpTimerOff} 			\\\hline
		Parametre 	& ingen					\\\hline
		Returværdi 	& void 						\\\hline
		Beskrivelse & Disabler wake-up-timeren. \\\hline
		\end{tabularx}
	\caption{Metodebeskrivelse for \texttt{wakeUpTimerOff}}
\end{table}

\begin{table}[H]
	\begin{tabularx}{\textwidth}{| l | Z |} 	\hline
		Funktion 	& \texttt{resetOff} 		\\\hline
		Parametre 	& Ingen  					\\\hline
		Returværdi 	& void 						\\\hline
		Beskrivelse & Sætter reset porten på transceiveren høj. \\\hline
		\end{tabularx}
	\caption{Metodebeskrivelse for \texttt{resetOff}}
\end{table}

\begin{table}[H]
	\begin{tabularx}{\textwidth}{| l | Z |} 	\hline
		Funktion 	& \texttt{oscillatorOn} 	\\\hline
		Parametre 	& Ingen 					\\\hline
		Returværdi 	& void 						\\\hline
		Beskrivelse & Enabler krystal oscillatoren på transceiveren  \\\hline
		\end{tabularx}
	\caption{Metodebeskrivelse for \texttt{oscillatorOn}}
\end{table}

\begin{table}[H]
	\begin{tabularx}{\textwidth}{| l | Z |} \hline
		Funktion 	& \texttt{oscillatorOff} 	\\\hline
		Parametre 	& Ingen  					\\\hline
		Returværdi 	& void 						\\\hline
		Beskrivelse & Disabler krystal oscillatoren på transceiveren  \\\hline
		\end{tabularx}
	\caption{Metodebeskrivelse for \texttt{oscillatorOff}}
\end{table}

\begin{table}[H]
	\begin{tabularx}{\textwidth}{| l | Z |} 	\hline
		Funktion 	& \texttt{lowBatDetOn} 		\\\hline
		Parametre 	& voltage: char  			\\\hline
		Returværdi 	& void 						\\\hline
		Beskrivelse & Enabler batterispændings overvågnings kredsløbet i transceiveren. Den medgivende parameter er spændingen som der sættes for grænsen af hvad batterispændingen minimum må være. Når spændingen kommer under dette niveau giver transceiveren et interrupt som kan læses i \texttt{STSREG} \\\hline	
	\end{tabularx}
	\caption{Metodebeskrivelse for \texttt{lowBatDetOn}}
\end{table}

\begin{table}[H]
	\begin{tabularx}{\textwidth}{| l | Z |} 	\hline
		Funktion 	& \texttt{lowBatDetOff} 	\\\hline
		Parametre 	& Ingen  					\\\hline
		Returværdi 	& void 						\\\hline
		Beskrivelse & Disabler batterispændings overvågnings kredsløbet i transceiveren. \\\hline
		\end{tabularx}
	\caption{Metodebeskrivelse for \texttt{lowBatDetOff}}
\end{table}


\begin{table}[H]
	\begin{tabularx}{\textwidth}{| l | Z |} \hline
		Funktion 	& \texttt{extClockOn} 		\\\hline
		Parametre 	& char  					\\\hline
		Returværdi 	& void 						\\\hline
		Beskrivelse & Enabler den eksterne clock udgang på transceiveren. Den ønskede frekvens gives med som parameter. \\\hline
		\end{tabularx}
	\caption{Metodebeskrivelse for \texttt{extCloclOn}}
\end{table}


\begin{table}[H]
	\begin{tabularx}{\textwidth}{| l | Z |} 	\hline
		Funktion 	& \texttt{extClockOff} 		\\\hline
		Parametre 	& Ingen  					\\\hline
		Returværdi 	& void 						\\\hline
		Beskrivelse & Disabler den eksterne clock udgang på transceiveren. \\\hline
		\end{tabularx}
	\caption{Metodebeskrivelse for \texttt{extCloclOff}}
\end{table}


\begin{table}[H]
	\begin{tabularx}{\textwidth}{| l | Z |} \hline
		Funktion 	& \texttt{sendControlBoxPackage} \\\hline
		Parametre 	& \_package: struct control\_package*   \\\hline
		Returværdi 	& void 									\\\hline
		Beskrivelse & Sender kontrolboks-pakken til sensoren. Funktionen sender en byte og venter herefter på \texttt{TXRXFIFO} interruptet fra transceiveren. Når dette er blevet giver sendes den næste byte. Dette gentages indtil hele pakken er sendt.   \\\hline
		\end{tabularx}
	\caption{Metodebeskrivelse for \texttt{sendControlPackage}}
\end{table}

\begin{table}[H]
	\begin{tabularx}{\textwidth}{| l | Z |} \hline
		Funktion 	& \texttt{receiveSensorPackage} \\\hline
		Parametre 	& \_package: struct sensor\_package* \\\hline
		Returværdi 	& void \\\hline
		Beskrivelse & Modtager pakken fra sensoren. Funktionen modtager en byte af gangen og returnerer herefter. Funktionen skal derfor kaldes i en interrupt-rutine. \\\hline
		\end{tabularx}
	\caption{Metodebeskrivelse for \texttt{receiveSensorPackage}}
\end{table}


\begin{table}[H]
	\begin{tabularx}{\textwidth}{| l | Z |} \hline
		Funktion 	& \texttt{mrf\_SS} \\\hline
		Parametre 	& Ingen \\\hline
		Returværdi 	& void \\\hline
		Beskrivelse & Sætter SS porten på transciveren lav. Bruges under SPI kommunikation. \\\hline
	\end{tabularx}
	\caption{Metodebeskrivelse for \texttt{mrf\_SS}}
\end{table}


\begin{table}[H]
	\begin{tabularx}{\textwidth}{| l | Z |} \hline
		Funktion 	& \texttt{mrf\_DS} \\\hline
		Parametre 	& Ingen \\\hline
		Returværdi 	& void \\\hline
		Beskrivelse & Sætter SS porten på transciveren høj. Bruges under SPI kommunikation. \\\hline
	\end{tabularx}
	\caption{Metodebeskrivelse for \texttt{mrf\_DS}}
\end{table}


\begin{table}[H]
	\begin{tabularx}{\textwidth}{| l | Z |} \hline
		Funktion 	& \texttt{mrf\_dataLow} \\\hline
		Parametre 	& Ingen \\\hline
		Returværdi 	& void \\\hline
		Beskrivelse & Sætter data porten på transciveren lav. Bruges under SPI kommunikation når der skal læses fra \texttt{FIFOREG} \\\hline
	\end{tabularx}
	\caption{Metodebeskrivelse for \texttt{mrf\_dataLow}}
\end{table}


\begin{table}[H]
	\begin{tabularx}{\textwidth}{| l | Z |} \hline
		Funktion 	& \texttt{mrf\_dataHigh} \\\hline
		Parametre 	& Ingen \\\hline
		Returværdi 	& void \\\hline
		Beskrivelse & Sætter data porten på transciveren høj. Bruges under SPI kommunikation når der skal læses fra \texttt{FIFOREG} \\\hline
	\end{tabularx}
	\caption{Metodebeskrivelse for \texttt{mrf\_dataHigh}}
\end{table}


\begin{table}[H]
	\begin{tabularx}{\textwidth}{| l | Z |} \hline
		Funktion 	& \texttt{spiInit} \\\hline
		Parametre 	& Ingen \\\hline
		Returværdi 	& void \\\hline
		Beskrivelse & Initierer de perifere enheder på mikroprocessoren der har med SPI at gøre \\\hline
	\end{tabularx}
	\caption{Metodebeskrivelse for \texttt{spiInit}}
\end{table}


\begin{table}[H]
	\begin{tabularx}{\textwidth}{| l | Z |} \hline
		Funktion 	& \texttt{spiSend8} \\\hline
		Parametre 	& \_data: uint8\_t \\\hline
		Returværdi 	& uint8\_t \\\hline
		Beskrivelse & Sender en byte til transceiveren. \\\hline
	\end{tabularx}
\caption{Metodebeskrivelse for \texttt{spiSend8}}
\end{table}


\begin{table}[H]
	\begin{tabularx}{\textwidth}{| l | Z |} \hline
		Funktion 	& \texttt{spiSend} \\\hline
		Parametre 	& \_data: uint16\_t \\\hline
		Returværdi 	& uint16\_t \\\hline
		Beskrivelse & Sender et word til transceiveren. \\\hline
	\end{tabularx}
	\caption{Metodebeskrivelse for \texttt{spiSend}}
\end{table}


\begin{table}[H]
	\begin{tabularx}{\textwidth}{| l | Z |} \hline
		Funktion 	& \texttt{mrf\_resetSynchByteRecognition} \\\hline
		Parametre 	& ingen \\\hline	
		Returværdi 	& void \\\hline	
		Beskrivelse & Resetter den synkrone byte genkendelse når der er blevet modtaget en pakke. På den måde ved transceiveren at den skal have et start word inden den skal modtage noget igen. \\\hline
	\end{tabularx}
	\caption{Metodebeskrivelse for \texttt{mrf\_resetSynchByteRecognition}}
\end{table}


\textbf{Ekstern funktion }
\begin{table}[H]
	\begin{tabularx}{\textwidth}{| l | Z |} \hline
		Funktion 	& \texttt{rfTransceiverInit} \\\hline
		Parametre 	& \_transceiver: struct rfTransceiver* \\\hline
		Returværdi 	& void \\\hline
		Beskrivelse & Initierer \texttt{transceiver} structen og giver funktionspointere fra .c filen videre. Funktionen initierer også selve transciveren og sætter den op til 433MHz. Funktionen sætter også krystal-load-kapaciteterne til 10pF, båndbredden til 270kHz, FIFO interrupt ved 8 bit og datarate til 1200bps.  \\\hline
	\end{tabularx}
	\caption{Metodebeskrivelse for \texttt{rfTransceiverInit}}
\end{table}

\vfill