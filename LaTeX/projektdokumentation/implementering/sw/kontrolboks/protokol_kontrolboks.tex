\subsubsection{Trådløs protokol} \label{sec:SWimp_protokol_kontrolboks}

\todo[inline]{hvor placerer vi bedst dette afsnit?}
Det første der skal ske når sensoren er placeret i jorden og kontrolboksen tændes, er at sensoren og kontrolboksen skal parres, dette sker automatisk jfv. \textit{CM\_03}. For at det kan lade sig gøre skal sensoren og kontrolboks som udgangspunkt have ID 0. Når kontrolboksen tændes første gang vil den søge efter en sensor med ID 0. Er der en sensor i nærheden som ikke før har været parret vil denne svare på forespørgslen. Når kontrolboksen har modtaget en respons fra sensoren vil den generere et tilfældigt ID som herefter transmitteres til sensoren. Sensoren svarer tilbage med accept. Kontrolboks og Sensor har således nu samme ID og er parret. Ønskes der at parre sensor og kontrolboks igen skal dette ske ved udtagelse af batteri, jfv. \textit{CM\_04}, sensor opstarte herefter med ID 0. Se Figur \ref{fig:parring} for parringsprocess. Når parringen er gennemført vil kontrolboksen herefter forespørge om data fra sensoren. Dette sker ved udveksling af datapakker som ses på Figur \ref{fig:udvdata}. Kontrolboksen indeholder den korteste pakke på 4 bytes. De 2 første bytes er et startword som benyttes af transceiveren til at starte transmission af data. Den 3. byte er kontrolboksens ID, og det 4. er checksum for pakken. Se Figur \ref{fig:kbpakke} for opbygning af datapakke fra kontrolboks. Ved modtagelse af Kontrolboks-pakke sendes sensorens data retur i en Sensor-pakke med opbygning som på Figur \ref{fig:sensorpakke}. Denne pakke består ligeledes af startword, ID og checksum. Men derudover indeholder den også batteristatus, temperatur, jordfugtighed samt lysstyrke. Temperatur og jordfugtighed transmitteres med hver 2 bytes. Den første byte indeholder heltallet eks. 25, hvis temperaturen er 25.2 grader, den næste byte inderholder 2 for selve decimaltallet. Se Figur \ref{fig:udvdata} for sekvens over udveksling af datapakker.
\clearpage

\begin{figure}[H]
	\centering
	\includegraphics[width=0.5\textwidth]{../fig/Kommunikation/parring1.pdf}
	\caption{Parringssekvens for kontrolboks og sensor}
	\label{fig:parring}
\end{figure}      


\begin{figure}[H]
	\centering
	\includegraphics[width=0.5\textwidth]{../fig/Kommunikation/parring2.pdf}
	\caption{Udveksling af data imellem kontrolboks og sensor}
	\label{fig:udvdata}
\end{figure}      

\vfill

\begin{figure}[H]
	\centering
	\includegraphics[width=0.5\textwidth]{../fig/Kommunikation/kontrolbokspakke.pdf}
	\caption{Kontrolbokspakke}
	\label{fig:kbpakke}
\end{figure}


\begin{figure}[H]
	\centering
	\includegraphics[width=0.5\textwidth]{../fig/Kommunikation/sensorpakke.pdf}
	\caption{Sensorpakke}
	\label{fig:sensorpakke}
\end{figure}

\vfill