\subsubsection{Kontrolboks Pakke} \label{sec:SWimp_kontrolboksPakke}

Som det ses på Figur \ref{fig:kbpakke1} skal kontrolboks-pakken bestå af i alt 5 bytes. To byte som tilsammen udgør et startword til \texttt{transceiver}, et ID og et cheksum-word. 

\begin{figure}[H]
	\centering
	\includegraphics[width=0.5\textwidth]{../fig/Kommunikation/kontrolbokspakke.pdf}
	\caption{Kontrolbokspakke}
	\label{fig:kbpakke1}
\end{figure}

Implementeringen ses på Figur \ref{fig:kbpakke2}. 

\begin{figure}[H]
	\centering
	\includegraphics[width=0.7\textwidth]{../fig/Klassebeskrivelser/packages/controlbokspackage.pdf}
	\caption{Klassediagram over \texttt{controlboxPackage}}
	\label{fig:kbpakke2}
\end{figure}


\textit{Variable-beskrivelser} \\

\begin{table}[H]
		\begin{tabularx}{\textwidth}{| L{2.5cm} | l | Z |} \hline
			Navn 					& Type 				& Beskrivelse \\ \hline
			\texttt{startWord}	& \texttt{uint16\_t}	& Variabel som indeholder start-word'et til transceiveren. \\ \hline
		\end{tabularx}
		\caption{Beskrivelse af variablen \texttt{startWord}}
\end{table}

\begin{table}[H]
		\begin{tabularx}{\textwidth}{| L{2.5cm} | l | Z |} \hline
			Navn 					& Type 				& Beskrivelse \\ \hline
			\texttt{Id}	& \texttt{uint8\_t}	& Variabel som indeholder Id'et på kontrolboksen. \\ \hline
		\end{tabularx}
		\caption{Beskrivelse af variablen \texttt{Id}}
\end{table}

\begin{table}[H]
		\begin{tabularx}{\textwidth}{| L{2.5cm} | l | Z |} \hline
			Navn 					& Type 				& Beskrivelse \\ \hline
			\texttt{checksum}	& \texttt{uint16\_t}	& Variabel som indeholder checksummen af hele pakken. \\ \hline
		\end{tabularx}
		\caption{Beskrivelse af variablen \texttt{checksum}}
\end{table}

\begin{table}[H]
		\begin{tabularx}{\textwidth}{| L{2.5cm} | l | Z |} \hline
			Navn 					& Type 				& Beskrivelse \\ \hline
			\texttt{receivedChecksum}	& \texttt{uint16\_t}	& Variabel som indeholder checksummen af den modtagende pakken. \\ \hline
		\end{tabularx}
		\caption{Beskrivelse af variablen \texttt{receivedChecksum}}
\end{table}


\begin{table}[H]
		\begin{tabularx}{\textwidth}{| L{2.5cm} | l | Z |} \hline
			Navn 					& Type 				& Beskrivelse \\ \hline
			\texttt{newDataReady}	& \texttt{uint16\_t}	& Variabel som sættes til 1 når \texttt{transceiver} har modtaget en fuld pakke. \\ \hline
		\end{tabularx}
		\caption{Beskrivelse af variablen \texttt{newDataReady}}
\end{table}


\textit{Funktions-beskrivelse} \\

\begin{table}[H]
	\begin{tabularx}{\textwidth}{| l | Z |} \hline
		Funktion 	& \texttt{testChecksum} \\\hline
		Parametre 	& \_package: struct controlBox\_package* \\\hline
		Returværdi 	& void \\\hline
		Beskrivelse & Udregner checksummen i den nuværende pakke og sammenligner den med \texttt{receivedChecksum}. Er de ens returneres 1 ellers returneres 0. \\\hline
	\end{tabularx}
	\caption{Metodebeskrivelse for \texttt{testChecksum}}
\end{table}

\begin{table}[H]
	\begin{tabularx}{\textwidth}{| l | Z |} \hline
		Funktion 	& \texttt{updateChecksum} \\\hline
		Parametre 	& \_package: struct controlBox\_package* \\\hline
		Returværdi 	& void \\\hline
		Beskrivelse & Opdaterer \texttt{checksum} \\\hline
	\end{tabularx}
	\caption{Metodebeskrivelse for \texttt{updateChecksum}}
\end{table}


