\subsubsection{Display} \label{sec:SWimp_display}

\begin{figure}[H]
	\centering
	\includegraphics[width=0.6\textwidth]{../fig/Klassebeskrivelser/i2cDisp/i2cdisp.pdf}
	\caption{Klassediagram over I2CDisp}
	\label{fig:i2cdisp}
\end{figure} 

\textbf{Variabel beskrivelse}
\begin{table}[H]
	\begin{tabularx}{\textwidth}{| L{2.5cm} | l | Z |} \hline
		Navn 						& Type 						& Beskrivelse \\\hline
		\texttt{statusByte}         & \texttt{unsigned char}	& Gemmer indstillingerne som er sat på displayet, herunder, curser, blink, on/off osv. \\ \hline
	\end{tabularx}
	\caption{Attributter for klassen i2cDisp}
\end{table}

\textbf{Funktionsbeskrivelser}
\begin{table}[H]
	\begin{tabularx}{\textwidth}{| l | Z |} \hline
		Funktion 	& \texttt{writeChar} 	\\ \hline
		Parametre 	& \_data: char 			\\ \hline
		Returværdi 	& int8\_t 				\\ \hline
		Beskrivelse & Skriver en karakter til displayet. Returner 0 hvis skrivning er sket fejlfrit og 1 hvis der er opstået en fejl. \\ \hline
	\end{tabularx}
	\caption{Metodebeskrivelse for \texttt{writeChar}}
\end{table}

\begin{table}[H]
	\begin{tabularx}{\textwidth}{| l | Z |} \hline
		Funktion 	& \texttt{writeString} 	\\ \hline
		Parametre 	& \_stringPtr: char* 	\\ \hline
		Returværdi 	& int8\_t\\\hline
		Beskrivelse & Skriver en en string til displayet. Returner 0 hvis skrivning er sket fejlfrit og 1 hvis der er opstået en fejl. \\ \hline
	\end{tabularx}
	\caption{Metodebeskrivelse for \texttt{writeString}}
\end{table}


\begin{table}[H]
	\begin{tabularx}{\textwidth}{| l | Z |} \hline
		Funktion 	& \texttt{writeInt} \\\hline
		Parametre 	& \_number: uint32\_t \\\hline
		Returværdi 	& int8\_t\\\hline
		Beskrivelse & Skriver et heltal til displayet. Returner 0 hvis skrivning er sket fejlfrit og 1 hvis der er opstået en fejl. \\\hline
	\end{tabularx}
	\caption{Metodebeskrivelse for \texttt{writeInt}}
\end{table}

\begin{table}[H]
	\begin{tabularx}{\textwidth}{| l | Z |} \hline
		Funktion 	& \texttt{curserOn} \\\hline
		Parametre 	& ingen \\\hline
		Returværdi 	& void\\\hline
		Beskrivelse & Sætter curseren på displayet on \\\hline
	\end{tabularx}
	\caption{Metodebeskrivelse for \texttt{curserOn}}
\end{table}

\begin{table}[H]
	\begin{tabularx}{\textwidth}{| l | Z |} \hline
		Funktion 	& \texttt{curserOff} \\\hline
		Parametre 	& ingen \\\hline
		Returværdi 	& void\\\hline
		Beskrivelse & Sætter curseren på displayet off \\\hline
	\end{tabularx}
	\caption{Metodebeskrivelse for \texttt{curserOff}}
\end{table}


\begin{table}[H]
	\begin{tabularx}{\textwidth}{| l | Z |} \hline
		Funktion 	& \texttt{blinkOn} \\\hline
		Parametre	& ingen \\\hline
		Returværdi 	& void\\\hline
		Beskrivelse & Sætter curseren på displayet til at blinke \\\hline
	\end{tabularx}
	\caption{Metodebeskrivelse for \texttt{blinkOn}}
\end{table}


\begin{table}[H]
	\begin{tabularx}{\textwidth}{| l | Z |} \hline
		Funktion 	& \texttt{blinkOff} \\\hline
		Parametre 	& ingen \\\hline
		Returværdi 	& void\\\hline
		Beskrivelse & Sætter curseren på displayet til at være konstant on \\\hline
	\end{tabularx}
	\caption{Metodebeskrivelse for \texttt{blinkOff}}
\end{table}


\begin{table}[H]
	\begin{tabularx}{\textwidth}{| l | Z |} \hline
		Funktion 	& \texttt{moveCursor} \\\hline
		Parametre 	& \_digit: char \ \_line: char \\\hline
		Returværdi 	& void\\\hline
		Beskrivelse & Flytter curseren til en bestemt position på displayet. \\\hline
	\end{tabularx}
	\caption{Metodebeskrivelse for \texttt{moveCursor}}
\end{table}


\begin{table}[H]
	\begin{tabularx}{\textwidth}{| l | Z |} \hline
		Funktion 	& \texttt{on} \\\hline
		Parametre 	& Ingen \\\hline
		Returværdi 	& void\\\hline
		Beskrivelse & Gør karakterene på displayet synlige. \\\hline
	\end{tabularx}
	\caption{Metodebeskrivelse for \texttt{on}}
\end{table}


\begin{table}[H]
	\begin{tabularx}{\textwidth}{| l | Z |} \hline
		Funktion 	& \texttt{off} \\\hline
		Parametre 	& Ingen \\\hline
		Returværdi 	& void\\\hline
		Beskrivelse & Gør karakterene på displayet usynlige. \\\hline
	\end{tabularx}
	\caption{Metodebeskrivelse for \texttt{off}}
\end{table}

\begin{table}[H]
	\begin{tabularx}{\textwidth}{| l | Z |} \hline
		Funktion 	& \texttt{clear} \\\hline
		Parametre 	& Ingen \\\hline
		Returværdi 	& void\\\hline
		Beskrivelse & Sletter alt hvad der står på displayet og flytter cursoren til position 0,0. \\\hline
	\end{tabularx}
	\caption{Metodebeskrivelse for \texttt{clear}}
\end{table}


\begin{table}[H]
	\begin{tabularx}{\textwidth}{| l | Z |} \hline
		Funktion 	& \texttt{clearAC} \\\hline
		Parametre 	& Ingen \\\hline
		Returværdi 	& void\\\hline
		Beskrivelse & Clearer adresse counteren således curseren flyttes til position 0,0. \\\hline
	\end{tabularx}
	\caption{Metodebeskrivelse for \texttt{clearAC}}
\end{table}

\begin{table}[H]
	\begin{tabularx}{\textwidth}{| l | Z |} \hline
		Funktion 	& \texttt{lightOn} \\\hline
		Parametre 	& Ingen \\\hline
		Returværdi 	& void\\\hline
		Beskrivelse & Tænder for baggrundslyset på displayet. \\\hline
	\end{tabularx}
	\caption{Metodebeskrivelse for \texttt{lightOn}}
\end{table}

\begin{table}[H]
	\begin{tabularx}{\textwidth}{| l | Z |} \hline
		Funktion 	& \texttt{lightOff} \\\hline
		Parametre	& Ingen \\\hline
		Returværdi 	& void\\\hline
		Beskrivelse & Slukker for baggrundslyset på displayet. \\\hline
	\end{tabularx}
	\caption{Metodebeskrivelse for \texttt{lightOff}}
\end{table}

\begin{table}[H]
	\begin{tabularx}{\textwidth}{| l | Z |} \hline
		Funktion	& \texttt{resetHigh} \\\hline
		Parametre 	& Ingen \\\hline
		Returværdi 	& void\\\hline
		Beskrivelse & Sætter reset porten på displayet høj. \\\hline
	\end{tabularx}
	\caption{Metodebeskrivelse for \texttt{resetHigh}}
\end{table}

\begin{table}[H]
	\begin{tabularx}{\textwidth}{| l | Z |} \hline
		Funktion 	& \texttt{resetLow} \\\hline
		Parametre 	& Ingen \\\hline
		Returværdi 	& void\\\hline
		Beskrivelse & Sætter reset porten på displayet lav. \\\hline
	\end{tabularx}
	\caption{Metodebeskrivelse for \texttt{resetLow}}
\end{table}

\begin{table}[H]
	\begin{tabularx}{\textwidth}{| l | Z |} \hline
		Funktion 	& \texttt{i2cStart} \\\hline
		Parametre 	& Ingen \\\hline
		Returværdi 	& void\\\hline
		Beskrivelse & Initierer de perifere enheder på mikroprocessoren som har med I2C at gøre. \\\hline
	\end{tabularx}
	\caption{Metodebeskrivelse for \texttt{i2cStart}}
\end{table}

\begin{table}[H]
	\begin{tabularx}{\textwidth}{| l | Z |} \hline
		Funktion 	& \texttt{i2cStop} \\\hline
		Parametre 	& Ingen \\\hline
		Returværdi 	& void\\\hline
		Beskrivelse & Deaktiverer I2C. \\\hline
	\end{tabularx}
	\caption{Metodebeskrivelse for \texttt{i2cStop}}
\end{table}

\begin{table}[H]
	\begin{tabularx}{\textwidth}{| l | Z |} \hline
		Funktion 	& \texttt{i2cWriteByte} \\\hline
		Parametre 	& Ingen \\\hline
		Returværdi 	& void\\\hline
		Beskrivelse & Skriver en byte ud på I2C bussen \\\hline
	\end{tabularx}
	\caption{Metodebeskrivelse for \texttt{i2cWriteByte}}
\end{table}


\begin{table}[H]
	\begin{tabularx}{\textwidth}{| l | Z |} \hline
		Funktion 	& \texttt{i2cGetStatus} \\\hline
		Parametre 	& Ingen 				\\\hline
		Returværdi 	& unsigned char 		\\\hline
		Beskrivelse & Læser I2C status registret på mikroprocessoren og returnerer status. \\\hline
	\end{tabularx}
	\caption{Metodebeskrivelse for \texttt{i2cGetStatus}}
\end{table}


\begin{table}[H]
	\begin{tabularx}{\textwidth}{| l | Z |} \hline
		Funktion 	& \texttt{dispInit} 	\\\hline
		Parametre 	& Ingen 				\\\hline
		Returværdi 	& void 					\\\hline
		Beskrivelse & Initierer displayet til dets karakterset og standard indstillinger samt enabler DC/DC boost-converteren. Bliver kaldt af \texttt{i2cDispInit}. \\ \hline
	\end{tabularx}
	\caption{Metodebeskrivelse for \texttt{dispInit}}
\end{table}

\textbf{Ekstern funktion }
\begin{table}[H]
	\begin{tabularx}{\textwidth}{| l | Z |} \hline
		Funktion 	& \texttt{i2cDispInit} 	\\\hline
		Parametre 	& \_disp: struct i2cdisp* \\\hline
		Returværdi 	& void \\\hline	
		Beskrivelse & Initierer i2cDisp structen og giver funktionspointere fra .c filen videre. Funktionen initierer også selve displayet og sætter det op til dets standard indstillinger. \\\hline
	\end{tabularx}
	\caption{Metodebeskrivelse for \texttt{i2cDispInit}}
\end{table}