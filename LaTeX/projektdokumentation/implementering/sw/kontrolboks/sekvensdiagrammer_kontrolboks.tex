\subsubsection{Sekvensdiagrammer Kontrolboks} \label{sec:sd_kontrolboks}

%%%%%%%%%%%%%%%%%%%%%%%%%%%%%%%%%%%%%%%%%%%%%%%%%%%%%%%%%%%%%%%%%%%%%%%%%%%%%%%%%%%%%%%%%%%%%%%%%%%%%%%%%%%%%%%%%%
%
%							Sekvensdiagrammer - Kontrolboks
%
%%%%%%%%%%%%%%%%%%%%%%%%%%%%%%%%%%%%%%%%%%%%%%%%%%%%%%%%%%%%%%%%%%%%%%%%%%%%%%%%%%%%%%%%%%%%%%%%%%%%%%%%%%%%%%%%%%

%%%%%%%%%%%%%%%%%%%%%%%%%%%%%%%%%%%%%%%%%%%%%%%%%%%%%%  SD MAIN   %%%%%%%%%%%%%%%%%%%%%%%%%%%%%%%%%%%%%%%%%%%%%%%%%%%%
\textbf{Main}

På Figur \ref{fig:sd_main_kb} ses sekvensdiagram over \texttt{main()}-funktionen i mikroprocessoren på kontrolboksen.

\begin{figure}[H]
	\centering
	\includegraphics[width=0.8\textwidth]{../fig/sekvensdiagrammer/kontrolboks/sd_main.pdf}
	\caption{Sekvensdiagram over \texttt{main()} i kontrolboksen}
	\label{fig:sd_main_kb}
\end{figure} 

Når brugeren tilslutter strøm til kontrolboksen skal main-funktionen vente i 1 sekund før der foretages noget. Dette er til for at sikre at spændingerne på boardet har stabiliseret sig. Efterfølgende skal \texttt{Init()} sørge for at initiere alt hardware, give funktionspointere videre til diverse struct's, samt indlæse indstillingerne fra EEPROM'en. For at sikre at mikroprocessoren aldrig ender i en deadlock enables watch-dog-timer'en. Mikroprocessoren skal altid være parat til at tage input fra brugeren, derfor skal den befinde i et loop og kontinuerligt aflæse om der påtrykket en tast på tasteturet, eller om der er modtaget ny data fra sensoren. Påtrykkes en knap skal lyset i displayet tændes og funktionen \texttt{menu()} skal kaldes. \texttt{menu()} giver brugeren mulighed for at navigere rundt i systemets Use Cases. Når \texttt{menu()} returnerer skal EEPROM'en opdateres med de nye indstillinger som brugeren har konfigureret. Er der blevet modtaget ny data fra sensoren,  skal denne have besked om at data'en er modtaget med succes, og herefter skal displayet opdateres med de seneste værdier. Er der lav batteri på sensoren skal LED'en på tastaturet tændes og teksten "Low battery on sensor" udskrives på displayet. Som sidste led i loop'et skal lyset i displayet slukkes og watch-dog-timeren reset'es.\\
%%%%%%%%%%%%%%%%%%%%%%%%%%%%%%%%%%%%%%%%%%%%%%%%%%%%%%   SD ISR   %%%%%%%%%%%%%%%%%%%%%%%%%%%%%%%%%%%%%%%%%%%%%%%%%%%%

\textbf{Interrupt rutine}
\begin{figure}[H]
	\centering
	\includegraphics[width=0.75\textwidth]{../fig/sekvensdiagrammer/kontrolboks/sd_isr.pdf}
	\caption{Sekvensdiagram over interrupt-rutinen}
	\label{fig:sd_isr_kb}
\end{figure} 

På Figur \ref{fig:sd_isr_kb} ses sekvendiagram over interruptrutinen. Når det eksterne interrupt INT0 aktiveret skal rutinen læse status-registeret i transceiveren. Er der givet interrupt om lav batterispænding skal brugeren informeres ved at teksten "Low battery" skrives ud på display'et samtidig med at LED'en på tastaturet tændes. Herefter skal der ventes i en uendelig while-løkke, for at undgå at motorventilen åbnes og kontrolboksen herefter løber tør for strøm. Er TXRXFIFO interruptet givet er det fordi at transceiveren har modtaget en byte som ligger klar i RXFIRO. Funktionen \texttt{receiveSensorPackage(sensor\_Package*)} kaldes og læser denne byte til \texttt{sensorPackage}. Er TIMER1\_COMPA interruptet aktiveret skal kontrolboksens clock tælles én op. Dette sker hvert sekund. 

%%%%%%%%%%%%%%%%%%%%%%%%%%%%%%%%%%%%%%%%%%%%%%%%%%%%%%   SD UC3   %%%%%%%%%%%%%%%%%%%%%%%%%%%%%%%%%%%%%%%%%%%%%%%%%%%%
\newpage
\textbf{Use Case 3: Indstil ønsket fugtighed i jorden}\\
\begin{figure}[H]
	\centering
	\includegraphics[width=0.8\textwidth]{../fig/sekvensdiagrammer/kontrolboks/sd_uc3.pdf}
	\caption{Sekvensdiagram over UC3: Indstil ønsket fugtighed i jorden}
	\label{fig:sd_uc3_kb}
\end{figure} 

På Figur \ref{fig:sd_uc3_kb} ses sekvensdiagram over Use Case 3. For at denne Use Case aktiveres skal brugeren have trykket på SET-knappen på tastaturet 1 gang. Funktionen skal implementeres i en while løkke og skal som det første clear'e display'	et og skrive teksten "\textit{Soil humidity}" på linje 1 og "\textit{Humidity:}" på linje 2 efterfulgt af værdien af \texttt{soilHumidity}. Efterfølgende afventes der at brugeren påtrykker en tast. Trykkes der på \texttt{Pil-op} skal variablen \texttt{soilHumidity} i struct'en \texttt{settings} tælles én op. Trykkes der på \texttt{Pil-ned} skal \texttt{soilHumidity} tælles én ned. Der skal implementeres grænseværdier således at fugtigheden kun kan indstilles fra 0-50\%. Trykkes der på \texttt{OK} returneres der til hovedmenuen. Trykkes der på \texttt{SET} skal løkken brydes således der navigeres videre til Use Case 4. 

%%%%%%%%%%%%%%%%%%%%%%%%%%%%%%%%%%%%%%%%%%%%%%%%%%%%%%   SD UC4   %%%%%%%%%%%%%%%%%%%%%%%%%%%%%%%%%%%%%%%%%%%%%%%%%%%%
\newpage
\textbf{Use Case 4: Indstil åbningstid for ventil}
\begin{figure}[H]
	\centering
	\includegraphics[width=0.8\textwidth]{../fig/sekvensdiagrammer/kontrolboks/sd_uc4.pdf}
	\caption{Sekvensdiagram over UC4: Indstil åbningstid for ventil}
	\label{fig:sd_uc4_kb}
\end{figure} 

På Figur \ref{fig:sd_uc4_kb} ses sekvensdiagram over Use Case 4. For at denne Use Case aktiveres skal brugeren have trykket på SET-knappen på tastaturet 2 gange. Funktionen skal implementeres i en while løkke og skal som det første clear'e displayet og skrive teksten "\textit{Watering time}" på linje 1 og "\textit{Time:}" på linje 2 efterfulgt af værdien af \texttt{valveTime}. Efterfølgende skal der afventes der tryk på en tast. Trykkes der på \texttt{Pil-op} skal variablen \texttt{valveTime} i struct'en \texttt{settings} tælles én op. Trykkes der på \texttt{Pil-ned} skal \texttt{valveTime} tælles én ned. Der skal implementeres grænseværdier således at åbningstiden kun kan indstilles fra 0-15 min. Trykkes der på \texttt{OK} returneres til hovedmenuen. Trykkes der på \texttt{SET} skal løkken brydes således der navigeres videre til Use Case 5. 


%%%%%%%%%%%%%%%%%%%%%%%%%%%%%%%%%%%%%%%%%%%%%%%%%%%%%%   SD UC5   %%%%%%%%%%%%%%%%%%%%%%%%%%%%%%%%%%%%%%%%%%%%%%%%%%%%
\newpage
\textbf{Use Case 5: Indstil tidsbaseret vandingsinterval}
\begin{figure}[H]
	\centering
	\includegraphics[width=0.8\textwidth]{../fig/sekvensdiagrammer/kontrolboks/sd_uc5.pdf}
	\caption{Sekvensdiagram over UC5: Indstil tidsbaseret vandingsinterval}
	\label{fig:sd_uc5_kb}
\end{figure} 

På Figur \ref{fig:sd_uc5_kb} ses sekvensdiagram over Use Case 5. For at denne Use Case aktiveres skal brugeren have påtrykket SET-knappen på tastaturet 3 gange. Funktionen skal implementeres i en while løkke og skal som det første clear'e displayet og skrive teksten "\textit{Time base}" på linje 1 og "\textit{Open:}" på linje 2 efterfulgt af værdien af \texttt{openTime}. Efterfølgende skal der afventes tryk på en tast. Trykkes der på \texttt{Pil-op} skal variablen \texttt{openTime} i struct'en \texttt{settings} tælles én op. Trykkes der på \texttt{Pil-ned} skal \texttt{valveTime} tælles én ned. Der skal implementeres grænseværdier således at åbningstiden kun kan indstilles fra 0-60 min. Trykkes der på \texttt{OK} returneres til hovedmenuen. Trykkes der på \texttt{SET} skal \texttt{byte} sættes til 1 og løkken startes forfra. Er \texttt{byte} sat til 1 skal teksten på linje 2 på displayet ændres til "\textit{Close:}" efterfulgt af værdien af \texttt{closeTime}. Trykkes der på \texttt{Pil-op} skal variablen \texttt{closeTime} i struct'en \texttt{settings} tælles én op. Trykkes der på \texttt{Pil-ned} skal \texttt{closeTime} tælles én ned. Der skal implementeres grænseværdier således at lukketiden kun kan indstilles fra 0-23 timer. Trykkes der på \texttt{OK} returneres til hovedmenuen. Trykkes der på \texttt{SET} skal løkken brydes således der navigeres videre til Use Case 7.  


%%%%%%%%%%%%%%%%%%%%%%%%%%%%%%%%%%%%%%%%%%%%%%%%%%%%%%   SD UC6   %%%%%%%%%%%%%%%%%%%%%%%%%%%%%%%%%%%%%%%%%%%%%%%%%%%%
\newpage
\textbf{Use Case 6: Åbn/luk ventil manuelt}
\begin{figure}[H]
	\centering
	\includegraphics[width=0.8\textwidth]{../fig/sekvensdiagrammer/kontrolboks/sd_uc6.pdf}
	\caption{Sekvensdiagram over UC6: Åbn/luk ventil manuelt}
	\label{fig:sd_uc6_kb}
\end{figure} 

På Figur \ref{fig:sd_uc6_kb} ses sekvensdiagram over Use Case 6. For at aktivere denne Use Case skal brugeren påtrykke enten \texttt{Pil-op} eller \texttt{Pil-ned} når brugeren befinder sig i hovedmenuen. Trykkes der på \texttt{Pil-op} skal der på displayets linje 1 udskrives "\textit{Open valve?}". Trykkes der efterfølgende på \texttt{OK} skal motorventilen åbnes. Trykkes der på \texttt{SET} skal der returneres til hovedmenuen. Trykkes der på \texttt{Pil-ned} skal der på displayet linje 1 udskrives "\textit{Close valve?}". Trykkes der efterfølgende på \texttt{OK} skal motorventilen lukkes. Trykkes der på \texttt{SET} skal der returneres til hovedmenuen. 


%%%%%%%%%%%%%%%%%%%%%%%%%%%%%%%%%%%%%%%%%%%%%%%%%%%%%%   SD UC7   %%%%%%%%%%%%%%%%%%%%%%%%%%%%%%%%%%%%%%%%%%%%%%%%%%%%
\newpage
\textbf{Use Case 7: Vælg automatisk vandingstidspunkt}
\begin{figure}[H]
	\centering
	\includegraphics[width=0.8\textwidth]{../fig/sekvensdiagrammer/kontrolboks/sd_uc7.pdf}
	\caption{Sekvensdiagram over UC7: Vælg automatisk vandingstidspunkt}
	\label{fig:sd_uc7_kb}
\end{figure}

På Figur \ref{fig:sd_uc7_kb} ses sekvensdiagram over Use Case 7. For at aktivere denne Use Case skal der trykkes på \texttt{SET} på tasteturet 5 gange. Funktionen skal implementeres i en while løkke og skal som det første clear'e displayet og skrive teksten "\textit{Watering period}" på linje 1 og "\textit{Time:}" på linje 2 efterfulgt af settings-teksten af \texttt{waterTime}. Efterfølgende afventes der tryk på en tast. Trykkes der på \texttt{Pil-op} skal variablen \texttt{waterTime} i struct'en \texttt{settings} tælles én op. Trykkes der på \texttt{Pil-ned} skal \texttt{waterTime} tælles én ned. Der skal implementeres grænseværdier således at åbningstiden kun kan indstilles fra 1-5, hvilket vil tilsvare \texttt{ALLDAY}, \texttt{MORNING}, \texttt{EVENING},  \texttt{MORNINGANDEVENING} eller \texttt{AUTOTIME}. Trykkes der på \texttt{OK} returneres til hovedmenuen. Trykkes der på \texttt{SET} skal løkken brydes således der navigeres videre til Use Case 8.


%%%%%%%%%%%%%%%%%%%%%%%%%%%%%%%%%%%%%%%%%%%%%%%%%%%%%%   SD UC8   %%%%%%%%%%%%%%%%%%%%%%%%%%%%%%%%%%%%%%%%%%%%%%%%%%%%
\newpage
\textbf{Use Case 8: Par sensor og kontrolboks}
\begin{figure}[H]
	\centering
	\includegraphics[width=0.8\textwidth]{../fig/sekvensdiagrammer/kontrolboks/sd_uc8.pdf}
	\caption{Sekvensdiagram over UC8: Par sensor og kontrolboks}
	\label{fig:sd_uc8_kb}
\end{figure}

På Figur \ref{fig:sd_uc8_kb} ses sekvensdiagram over Use Case 8. For at aktivere denne Use Case skal der trykkes på \texttt{SET} 6 gange. Funktionen skal implementeres i en while løkke og skal som det første clear'e displayet og skrive teksten "\textit{Connect sensor?}" på linje 1. Trykkes der på \texttt{OK} skal der oprettes forbindelse til sensoren ellers hvis der trykkes på alt andet skal der returneres til hovedmenuen. 