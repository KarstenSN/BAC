\section{HF-Transceiver} 

\begin{figure}[H]
	\centering
	\includegraphics[width=0.8\textwidth]{../fig/Klassebeskrivelser/rfTransciver/rfTransciver.pdf}
	\caption{Klassediagram over RF-Transciveren}
	\label{fig:class_rftransciver}
\end{figure} 

På Figur \ref{fig:class_rftransciver} ses klassediagram over HF-transceiveren.Koden er skrevet i C og det er derfor svært at give en korrekt  standardiseret beskrivelse af klassen som en traditionel C++ klasse, der er i stedet benyttet at realiseret klassen ved en struct. Det vil sige at struct'en oprettes med funktionspointere til alle public funktioner, disse er markeret med + foran funktionen. Ved private members er der sat - foran funktionen. Alle funktioner er implementeret i en .c fil hvor struct'en er implementeret i en .h fil. Det er derfor kun muligt at se de funktioner som struct'en initieres med i init-funktionen. Denne funktionen er en ekstern funktion da den ligger uden for struct'en og bliver kaldt med \texttt{rfTransceiver} som parameter. Funktionen giver pointere fra .c filen videre til struct'en. Private members er funktioner som kun kan ses af selve .c filen selv og disse funktioner kan derfor ikke ses i .h filen.

\textbf{Variabel beskrivelse}
	\begin{table}[H]
		\begin{tabularx}{\textwidth}{| L{2.5cm} | l | Z |} \hline
			Navn 					& Type 				& Beskrivelse \\ \hline
			\texttt{mrf\_Register}	& \texttt{struct}	& Bliver brugt til at gemme værdierne af i alt 17 registre, som er konfigurerbare på transceiveren. Disse register sættes 																og skrives til transceiveren i initierings-funktionen \texttt{rfTransceiverInit}. Når der skal ændres på nogle 																		funktioner i transceiveren er det vigtigt at vide hvad der i forvejen står i et register, for at undgå at overskrive en 															indstilling. Defor opdateres denne struct hver gang et register ændres. \\ \hline
		\end{tabularx}
		\caption{Attributter for klassen MainWindow}
	\label{table:attr_distancesensor}
\end{table}


\textbf{Funktionsbeskrivelser}
	\begin{table}[H]
		\begin{tabularx}{\textwidth}{| l | Z |} \hline
			Funktion 	& \texttt{SendByte} 	\\ \hline
			Parametre 	& \_byte: uint16\_t 	\\ \hline
			Returværdi 	& void					\\ \hline
			Beskrivelse & Kan bruges til at sende en enkelt byte til transceiveren over SPI bussen. Denne funktion er til rådighed for programmøren under udvikling af software og 							bliver derfor ikke brugt direkte. Den bliver oftest kaldt af andre funktioner. \\ \hline
		\end{tabularx}
		\caption{Metodebeskrivelse for \texttt{Sendbyte}}
\end{table}


\begin{table}[H]
	\begin{tabularx}{\textwidth}{| l | Z |} \hline
		Funktion 	& \texttt{enableReceiver} \\\hline
		Parametre 	& Ingen \\\hline
		Returværdi 	& void\\\hline
		Beskrivelse & Bruges til at gøre transceiveren funktionel efter den har været disablet. \\\hline
	\end{tabularx}
	\caption{Metodebeskrivelse for \texttt{enableTransceiver}}
\end{table}


\begin{table}[H]
	\begin{tabularx}{\textwidth}{| l | Z |} \hline
		Funktion 	& \texttt{disableReceiver} \\\hline
		Parametre 	& Ingen \\\hline
		Returværdi 	& void\\\hline
		Beskrivelse & Bruges til at disable transciveren således den ikke har nogen funktion og derved ikke bruger unødig strøm. \\\hline
		\end{tabularx}
	\caption{Metodebeskrivelse for \texttt{disableTransceiver}}
\end{table}


\begin{table}[H]
	\begin{tabularx}{\textwidth}{| l | Z |} 	\hline
		Funktion 	& \texttt{getFifo} 			\\ \hline
		Parametre 	& Ingen 					\\ \hline
		Returværdi 	& uint8\_t					\\ \hline
		Beskrivelse & Modtager hvad der står i FOFOREG når transciveren har givet et interrupt om at registreret er fuldt. \\\hline
		\end{tabularx}
	\caption{Metodebeskrivelse for \texttt{getFifo}}
\end{table}


\begin{table}[H]
	\begin{tabularx}{\textwidth}{| l | Z |} 	\hline
		Funktion 	& \texttt{writeRegister} 	\\\hline
		Parametre 	& \_value: uint16\_t  		\\\hline
		Returværdi 	& void						\\\hline
		Beskrivelse & Bruges til at skrive til et register i transceiveren. Registret OR'ed sammen med dets settings bliver givet med som parametre. \\\hline
		\end{tabularx}
	\caption{Metodebeskrivelse for \texttt{writeRegister}}
\end{table}


\begin{table}[H]
	\begin{tabularx}{\textwidth}{| l | Z |} 	\hline
		Funktion 	& \texttt{getStatus} 		\\\hline
		Parametre 	& Ingen  					\\\hline
		Returværdi 	& uint16\_t 				\\\hline
		Beskrivelse & Kaldes når transceiveren har afgivet interrupt og læser herefter \texttt{STATUSREG}, således mikroprocessoren ved hvad der forårsagede interruptet. \\\hline
		\end{tabularx}
	\caption{Metodebeskrivelse for \texttt{writeRegister}}
\end{table}

\begin{table}[H]
	\begin{tabularx}{\textwidth}{| l | Z |} 	\hline
		Funktion 	& \texttt{resetOn} 			\\\hline
		Parametre 	& Ingen 					\\\hline
		Returværdi 	& void 						\\\hline
		Beskrivelse & Sætter reset porten på transceiveren lav. \\\hline
		\end{tabularx}
	\caption{Metodebeskrivelse for \texttt{resetOn}}
\end{table}

\begin{table}[H]
	\begin{tabularx}{\textwidth}{| l | Z |} 	\hline
		Funktion 	& \texttt{resetOff} 		\\\hline
		Parametre 	& Ingen  					\\\hline
		Returværdi 	& void 						\\\hline
		Beskrivelse & Sætter reset porten på transceiveren høj. \\\hline
		\end{tabularx}
	\caption{Metodebeskrivelse for \texttt{resetOff}}
\end{table}

\begin{table}[H]
	\begin{tabularx}{\textwidth}{| l | Z |} 	\hline
		Funktion 	& \texttt{oscillatorOn} 	\\\hline
		Parametre 	& Ingen 					\\\hline
		Returværdi 	& void 						\\\hline
		Beskrivelse & Enabler krystal oscillatoren på transceiveren  \\\hline
		\end{tabularx}
	\caption{Metodebeskrivelse for \texttt{oscillatorOn}}
\end{table}

\begin{table}[H]
	\begin{tabularx}{\textwidth}{| l | Z |} \hline
		Funktion 	& \texttt{oscillatorOff} 	\\\hline
		Parametre 	& Ingen  					\\\hline
		Returværdi 	& void 						\\\hline
		Beskrivelse & Disabler krystal oscillatoren på transceiveren  \\\hline
		\end{tabularx}
	\caption{Metodebeskrivelse for \texttt{oscillatorOff}}
\end{table}

\begin{table}[H]
	\begin{tabularx}{\textwidth}{| l | Z |} 	\hline
		Funktion 	& \texttt{lowBatDetOn} 		\\\hline
		Parametre 	& voltage: char  			\\\hline
		Returværdi 	& void 						\\\hline
		Beskrivelse & Enabler batterispændings overvågnings funktionen i transceiveren. Den medgivende parameter er spændingen som der sættes for grænsen af hvad 					batterispændingen minimum må være. Når spændingen kommer under dette niveau giver transceiveren et interrupt. \\\hline	
	\end{tabularx}
	\caption{Metodebeskrivelse for \texttt{lowBatDetOn}}
\end{table}

\begin{table}[H]
	\begin{tabularx}{\textwidth}{| l | Z |} 	\hline
		Funktion 	& \texttt{lowBatDetOff} 	\\\hline
		Parametre 	& Ingen  					\\\hline
		Returværdi 	& void 						\\\hline
		Beskrivelse & Disabler batterispændings overvågnings funktionen i transceiveren. \\\hline
		\end{tabularx}
	\caption{Metodebeskrivelse for \texttt{lowBatDetOff}}
\end{table}


\begin{table}[H]
	\begin{tabularx}{\textwidth}{| l | Z |} \hline
		Funktion 	& \texttt{extClockOn} 		\\\hline
		Parametre 	& Ingen  					\\\hline
		Returværdi 	& void 						\\\hline
		Beskrivelse & Enabler den eksterne clock udgang på transceiveren. \\\hline
		\end{tabularx}
	\caption{Metodebeskrivelse for \texttt{extCloclOn}}
\end{table}


\begin{table}[H]
	\begin{tabularx}{\textwidth}{| l | Z |} 	\hline
		Funktion 	& \texttt{extClockOff} 		\\\hline
		Parametre 	& Ingen  					\\\hline
		Returværdi 	& void 						\\\hline
		Beskrivelse & Disabler den eksterne clock udgang på transceiveren. \\\hline
		\end{tabularx}
	\caption{Metodebeskrivelse for \texttt{extCloclOff}}
\end{table}


\begin{table}[H]
	\begin{tabularx}{\textwidth}{| l | Z |} \hline
		Funktion 	& \texttt{sendControlPackage} \\\hline
		Parametre 	& \_package: struct control\_package*   \\\hline
		Returværdi 	& void 									\\\hline
		Beskrivelse & Sender kontrolboks-pakken til sensoren. Som parameter er en pointer til pakken.  \\\hline
		\end{tabularx}
	\caption{Metodebeskrivelse for \texttt{sendControlPackage}}
\end{table}

\begin{table}[H]
	\begin{tabularx}{\textwidth}{| l | Z |} \hline
		Funktion 	& \texttt{receiveSensorPackage} \\\hline
		Parametre 	& \_package: struct sensor\_package* \\\hline
		Returværdi 	& void \\\hline
		Beskrivelse & Modtager pakken fra sensoren. Kaldes når der er blevet givet et interrupt om at der er en pakke på vej. \\\hline
		\end{tabularx}
	\caption{Metodebeskrivelse for \texttt{receiveSensorPackage}}
\end{table}


\begin{table}[H]
	\begin{tabularx}{\textwidth}{| l | Z |} \hline
		Funktion 	& \texttt{mrf\_SS} \\\hline
		Parametre 	& Ingen \\\hline
		Returværdi 	& void \\\hline
		Beskrivelse & Sætter SS porten på transciveren lav. Bruges under SPI kommunikation. \\\hline
	\end{tabularx}
	\caption{Metodebeskrivelse for \texttt{mrf\_SS}}
\end{table}


\begin{table}[H]
	\begin{tabularx}{\textwidth}{| l | Z |} \hline
		Funktion 	& \texttt{mrf\_DS} \\\hline
		Parametre 	& Ingen \\\hline
		Returværdi 	& void \\\hline
		Beskrivelse & Sætter SS porten på transciveren høj. Bruges under SPI kommunikation. \\\hline
	\end{tabularx}
	\caption{Metodebeskrivelse for \texttt{mrf\_DS}}
\end{table}


\begin{table}[H]
	\begin{tabularx}{\textwidth}{| l | Z |} \hline
		Funktion 	& \texttt{mrf\_dataLow} \\\hline
		Parametre 	& Ingen \\\hline
		Returværdi 	& void \\\hline
		Beskrivelse & Sætter data porten på transciveren lav. Bruges under SPI kommunikation når der skal læses fra \texttt{FIFOREG} \\\hline
	\end{tabularx}
	\caption{Metodebeskrivelse for \texttt{mrf\_dataLow}}
\end{table}


\begin{table}[H]
	\begin{tabularx}{\textwidth}{| l | Z |} \hline
		Funktion 	& \texttt{mrf\_dataHigh} \\\hline
		Parametre 	& Ingen \\\hline
		Returværdi 	& void \\\hline
		Beskrivelse & Sætter data porten på transciveren høj. Bruges under SPI kommunikation når der skal læses fra \texttt{FIFOREG} \\\hline
	\end{tabularx}
	\caption{Metodebeskrivelse for \texttt{mrf\_dataHigh}}
\end{table}


\begin{table}[H]
	\begin{tabularx}{\textwidth}{| l | Z |} \hline
		Funktion 	& \texttt{spiInit} \\\hline
		Parametre 	& Ingen \\\hline
		Returværdi 	& void \\\hline
		Beskrivelse & Initierer de perifere enheder på mikroprocessoren der har med SPI at gøre \\\hline
	\end{tabularx}
	\caption{Metodebeskrivelse for \texttt{spiInit}}
\end{table}


\begin{table}[H]
	\begin{tabularx}{\textwidth}{| l | Z |} \hline
		Funktion 	& \texttt{spiSend8} \\\hline
		Parametre 	& \_data: uint8\_t \\\hline
		Returværdi 	& uint8\_t \\\hline
		Beskrivelse & Sender en byte til transceiveren. \\\hline
	\end{tabularx}
\caption{Metodebeskrivelse for \texttt{spiSend8}}
\end{table}


\begin{table}[H]
	\begin{tabularx}{\textwidth}{| l | Z |} \hline
		Funktion 	& \texttt{spiSend} \\\hline
		Parametre 	& \_data: uint16\_t \\\hline
		Returværdi 	& uint16\_t \\\hline
		Beskrivelse & Sender et word til transceiveren. \\\hline
	\end{tabularx}
	\caption{Metodebeskrivelse for \texttt{spiSend}}
\end{table}


\begin{table}[H]
	\begin{tabularx}{\textwidth}{| l | Z |} \hline
		Funktion 	& \texttt{mrf\_resetSynchByteRecognition} \\\hline
		Parametre 	& ingen \\\hline	
		Returværdi 	& void \\\hline	
		Beskrivelse & Resetter den synkrone byte genkendelse når der er blevet modtaget en pakke. På den måde ved transceiveren at den skal have et start word inden den skal modtage 	noget igen. \\\hline
	\end{tabularx}
	\caption{Metodebeskrivelse for \texttt{mrf\_resetSynchByteRecognition}}
\end{table}


\textbf{Ekstern funktion }
\begin{table}[H]
	\begin{tabularx}{\textwidth}{| l | Z |} \hline
		Funktion 	& \texttt{rfTransceiverInit} \\\hline
		Parametre 	& \_transceiver: struct rfTransceiver* \\\hline
		Returværdi 	& void \\\hline
		Beskrivelse & Initierer rfTransceiver structen og giver funktionspointere fra .c filen videre. Funktionen initierer også selve transciveren og sætter den op til at køre på det 	frekvensbånd og med de indstillinger som det er blevet besluttet. \\\hline
	\end{tabularx}
	\caption{Metodebeskrivelse for \texttt{rfTransceiverInit}}
\end{table}

\vfill