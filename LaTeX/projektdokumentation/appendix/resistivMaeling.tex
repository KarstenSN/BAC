\section{Måling af jordfugt}  \label{appendix:jordfugt}

I Listing \ref{lst:matlab_rise_func} ses der MATLAB koden over hvordan $\alpha$ og $\beta$ er bestemt ud fra data fra oscilloscopet. Dataet bliver indlæst i en .csv fil og scriptet laver alle udregningerne automatisk. Dette skript er vist med en måling af en så og plantejord med en fugtighed på 9.09\%.

\begin{lstlisting}[caption={MainWindows constructor},label=lst:matlab_rise_func,frame=single]
clear
clc
close all
format longeng
data = csvread('9p09percent.csv',6,0);
filterlength = 10;

%%filtrer data
datafilt = data;
for i = 1:length(data)-filterlength
    value = 0;
    value2 = 0;
    
    for j = 1:filterlength
        value = value + data(j+i,2);
    end
    
    for j = 1:filterlength
        value2 = value2 + data(j+i,3);
    end
    
    datafilt(i,2)=value/filterlength;
    datafilt(i,3)=value2/filterlength;
end

%%udregninger
meanval = mean(datafilt(:,2));
maxval = max(datafilt(:,2));
maxval2 = max(datafilt(:,3));
minval = 0;%min(datafilt(:,2));


steadyval = maxval;

starttime = 0;
stoptime = 0;
startval = 0;
for i = 1:length(data)-filterlength
  %  if (datafilt(i,2)> (minval) && (data(i,2)< (minval+maxval * 0.01)) && (i > 1) )
  %      starttime = datafilt(i-1,1);
  %      startval = datafilt(i-1,2);
  %  end
    if ((datafilt(i,3)> 0.3) \&\& (startval == 0))
        starttime = datafilt(i,1);
        startval = datafilt(i,2);
    end
  
    if (datafilt(i,2)> steadyval * 0.95) \&\& (datafilt(i,2)< steadyval * 1)
        stoptime = datafilt(i,1);
        stopval = datafilt(i,2);
    end
    
    if (starttime ~= 0) \&\& (stoptime ~= 0) \&\& (starttime < stoptime)
        break
    end
end

if (starttime == 0) || (stoptime == 0)
    error('Data not valid. Frequency may be to low.')
end

beta = maxval/maxval2
alpha = 1/((stoptime - starttime)/3)
fcut = alpha/(2*pi)
rise = (stoptime - starttime)/3

%transferfunction
num = [0 (alpha*beta)];
den = [0 1 alpha];
G = tf(num,den)
%figure(2)
%step(G*5+minval)

%timedomain
simData = zeros(length(data),1);
timeOffset = 0;
timeOffset2 = 0;
for i = 1:length(datafilt)
    if (datafilt(i,3) > (maxval2/2) \&\& (timeOffset == 0))
        timeOffset = datafilt(i,1)-datafilt(1,1);
        timeOffset2 = 0;
    end
    if (timeOffset ~= 0)
        simData(i)=(1-exp(-alpha*((datafilt(i,1)-datafilt(1,1))-timeOffset)))*maxval2*beta+minval;
    end
    
    if ((datafilt(i,3) <= (maxval2/2)) \&\& (timeOffset2 == 0) \&\& (timeOffset ~= 0))
        timeOffset = 0;
        timeOffset2 = datafilt(i,1)-datafilt(1,1);
    end
    
    if (timeOffset2 ~= 0 \&\& timeOffset == 0)
        simData(i)=((exp(-alpha*((datafilt(i,1)-datafilt(1,1))-timeOffset2)))*maxval2*beta+minval);
    end
    
    if (timeOffset2 == 0 \&\& timeOffset == 0)
        simData(i)=minval;
    end
   
end

%%plot data
figure(1)
plot(datafilt(:,1),datafilt(:,3), 'lineWidth',2)
hold on
plot(datafilt(:,1),datafilt(:,2), 'lineWidth',2)
plot(datafilt(:,1),simData, 'lineWidth',2)
title('Plot of measurement')
xlabel('Time (s)')
ylabel('Voltage (v)')
legend('Input', 'Output', 'Approximated system')
line('XData', [starttime starttime data(1,1)], 'YData', [0 startval startval], 'LineWidth', 1,'LineStyle', '-', 'Color', [0 0.6 0])
line('XData', [stoptime stoptime data(1,1)], 'YData', [0 stopval stopval], 'LineWidth', 1,'LineStyle', '-', 'Color', [1 1 0])
dim = [.2 .6 .3 .3];
str = {strcat('alpha = ', num2str(alpha)),strcat('beta = ', num2str(beta)),strcat('fcut = ', num2str(fcut))};
annotation('textbox',dim,'String',str,'FitBoxToText','on');
hold off

% G(s)=(1/R*C)/(s+(1/R*C))
%R = 3700;
%C = 1/(alpha*R)




   
\end{lstlisting}

