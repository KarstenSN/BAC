\section{Resistiv måling af jordfugt}

I Listing \ref{lst:matlab_rise_func1} ses der MATLAB koden over hvordan $\alpha$ og $\beta$ er bestemt ud fra data fra oscilloskopet. Dataet bliver indlæst i en .csv fil og scriptet laver alle udregningerne automatisk. Dette skript er vist med en måling af sand med en fugtighed på 9.09\%.

\begin{lstlisting}[caption={MainWindows constructor},label=lst:matlab_rise_func1,frame=single]
clear
clc
close all
format longeng
data = csvread('9p09percent.csv',6,0);
filterlength = 10;

%%filtrer data
datafilt = data;
for i = 1:length(data)-filterlength
    value = 0;
    value2 = 0;
    
    for j = 1:filterlength
        value = value + data(j+i,2);
    end
    
    for j = 1:filterlength
        value2 = value2 + data(j+i,3);
    end
    
    datafilt(i,2)=value/filterlength;
    datafilt(i,3)=value2/filterlength;
end

%%udregninger
meanval = mean(datafilt(:,2));
maxval = max(datafilt(:,2));
maxval2 = max(datafilt(:,3));
minval = 0;%min(datafilt(:,2));


steadyval = maxval;

starttime = 0;
stoptime = 0;
startval = 0;
for i = 1:length(data)-filterlength
  %  if (datafilt(i,2)> (minval) && (data(i,2)< (minval+maxval * 0.01)) && (i > 1) )
  %      starttime = datafilt(i-1,1);
  %      startval = datafilt(i-1,2);
  %  end
    if ((datafilt(i,3)> 0.3) \&\& (startval == 0))
        starttime = datafilt(i,1);
        startval = datafilt(i,2);
    end
  
    if (datafilt(i,2)> steadyval * 0.95) \&\& (datafilt(i,2)< steadyval * 1)
        stoptime = datafilt(i,1);
        stopval = datafilt(i,2);
    end
    
    if (starttime ~= 0) \&\& (stoptime ~= 0) \&\& (starttime < stoptime)
        break
    end
end

if (starttime == 0) || (stoptime == 0)
    error('Data not valid. Frequency may be to low.')
end

beta = maxval/maxval2
alpha = 1/((stoptime - starttime)/3)
fcut = alpha/(2*pi)
rise = (stoptime - starttime)/3

%transferfunction
num = [0 (alpha*beta)];
den = [0 1 alpha];
G = tf(num,den)
%figure(2)
%step(G*5+minval)

%timedomain
simData = zeros(length(data),1);
timeOffset = 0;
timeOffset2 = 0;
for i = 1:length(datafilt)
    if (datafilt(i,3) > (maxval2/2) \&\& (timeOffset == 0))
        timeOffset = datafilt(i,1)-datafilt(1,1);
        timeOffset2 = 0;
    end
    if (timeOffset ~= 0)
        simData(i)=(1-exp(-alpha*((datafilt(i,1)-datafilt(1,1))-timeOffset)))*maxval2*beta+minval;
    end
    
    if ((datafilt(i,3) <= (maxval2/2)) \&\& (timeOffset2 == 0) \&\& (timeOffset ~= 0))
        timeOffset = 0;
        timeOffset2 = datafilt(i,1)-datafilt(1,1);
    end
    
    if (timeOffset2 ~= 0 \&\& timeOffset == 0)
        simData(i)=((exp(-alpha*((datafilt(i,1)-datafilt(1,1))-timeOffset2)))*maxval2*beta+minval);
    end
    
    if (timeOffset2 == 0 \&\& timeOffset == 0)
        simData(i)=minval;
    end
   
end

%%plot data
figure(1)
plot(datafilt(:,1),datafilt(:,3), 'lineWidth',2)
hold on
plot(datafilt(:,1),datafilt(:,2), 'lineWidth',2)
plot(datafilt(:,1),simData, 'lineWidth',2)
title('Plot of measurement')
xlabel('Time (s)')
ylabel('Voltage (v)')
legend('Input', 'Output', 'Approximated system')
line('XData', [starttime starttime data(1,1)], 'YData', [0 startval startval], 'LineWidth', 1,'LineStyle', '-', 'Color', [0 0.6 0])
line('XData', [stoptime stoptime data(1,1)], 'YData', [0 stopval stopval], 'LineWidth', 1,'LineStyle', '-', 'Color', [1 1 0])
dim = [.2 .6 .3 .3];
str = {strcat('alpha = ', num2str(alpha)),strcat('beta = ', num2str(beta)),strcat('fcut = ', num2str(fcut))};
annotation('textbox',dim,'String',str,'FitBoxToText','on');
hold off

% G(s)=(1/R*C)/(s+(1/R*C))
%R = 3700;
%C = 1/(alpha*R)
  
\end{lstlisting}


Koden laver en figur hvor værdierne af $\alpha$ og $\beta$ kan aflæses. Se Figur \ref{fig:appj1}. Den grønne og gule streg markerer hvor start og stop tidspunktet er blevet sat. Den orange kurve er det approximerede system. Systemet går ikke negativt da koden ikke er skrevet til at løbe mere en en periode igennem. 

\begin{figure}[H]
	\centering
	\includegraphics[width=0.7\textwidth]{../fig/appendix/jordfugt/stepcalc.PNG}
	\caption{Kurve hvor $\alpha$ og $\beta$ er blevet udregnet}
	\label{fig:appj1}
\end{figure} 

$\alpha$ og $\beta$ blev noteret for sandet og der blev skrevet et nyt script til at lave en regression. Det blev forsøgt at samle dataet således at både $\alpha$ og $\beta$ hver i sær kunne bruges til at udregne fugtigheden i sandet. Til sidst blev det forsøgt at lave en regression med begge værdier inkluderet. Denne kurve viste sig dog at være meget følsom for variationer og kunne ved en lille ændring af enten $\alpha$ eller $\beta$ give en negativ værdi. $\alpha$ viste sig at variere meget siden tiden fra foregående måling. $\beta$ værdien ligger dog meget stabil og er derfor den eneste egnede værdi at bestemme fugtigheden. 


\begin{lstlisting}[caption={Matlab kode til hjælp af bestemmelse af $\alpha$ og $\beta$},label=lst:matlab_rise_func2,frame=single]
clc
clear
close all

data = [0 1 1; 9.09 13.9 0.85; 16.7 140 0.585; 23.1 169 0.510; 28.6 198 0.436; 
        35.7 215 0.411; 50 222 0.385; 58 235 0.360; 72.7 244 0.361];
    
percent = data(:,1);
risetimeMs = data(:,2);
beta = data(:,3);

funcBeta = fit(beta, percent,'exp2')
funcRise = fit(risetimeMs, percent,'exp2')
func3d = fit([risetimeMs,beta],percent,'poly13')

figure(1)
plot3(risetimeMs, beta, percent)
grid on
hold on
plot3(risetimeMs,beta,func3d(risetimeMs,beta))

figure(2)
plot(risetimeMs, percent,'linewidth',2)
hold on
plot(funcRise)
title('Plot of measurement and function fit ')
ylabel('Humidity %')
xlabel('Risetime (ms)')
legend('Measured', 'Approximated function')

figure(3)
plot(beta, percent,'linewidth',2)
hold on
plot(funcBeta)
title('Plot of measurement and function fit ')
ylabel('Humidity %')
xlabel('Beta')
legend('Measured', 'Approximated function')

humity = func3d(190,0.502)


\end{lstlisting}


\begin{figure}[H]
	\centering
	\includegraphics[width=0.7\textwidth]{../fig/appendix/jordfugt/funkalpha.PNG}
	\caption{Kurve hvor fugtighed kan aflæses via $\alpha$}
	\label{fig:appj2}
\end{figure} 

\begin{figure}[H]
	\centering
	\includegraphics[width=0.7\textwidth]{../fig/appendix/jordfugt/funkbeta.PNG}
	\caption{Kurve hvor fugtighed kan aflæses via $\beta$}
	\label{fig:appj3}
\end{figure} 

\begin{figure}[H]
	\centering
	\includegraphics[width=0.7\textwidth]{../fig/appendix/jordfugt/funk3d.PNG}
	\caption{3D Kurve hvor fugtighed kan aflæses både via $\alpha$ og $\beta$}
	\label{fig:appj4}
\end{figure} 

Der blev noteret data fra i alt 3 typer jord og disse blev samlet til et nyt script som lavede en regression via $\beta$. Det viste sig dog at der var for stor variation mellem jordtyperne og regressionen kunne derfor ikke ramme præcis. Se Figur \ref{fig:appfuncfit} og \ref{fig:appresplot}.

\begin{lstlisting}[caption={Matlab kode til hjælp af bestemmelse af $\alpha$ og $\beta$},label=lst:matlab_rise_func3,frame=single]
clc
clear
close all

data = [0 0.993; 4 0.928; 8.6 0.764; 8.94 0.811; 9.09 0.83; 12.6 0.657; 16.7 0.585; 
        17.1 0.613; 18.8 0.624; 21.1 0.522; 23.1 0.510; 24.0 0.603; 27.7 0.562; 28.6 0.436;
        28.6 0.506; 30.7 0.553; 35.7 0.411; 50 0.385; 58 0.360;];
    
percent = data(:,1);
beta = data(:,2);

funcBeta = fit(beta, percent,'exp2')

figure(1)
plot(beta, percent,'x','linewidth',2)
axis([0.3 1 0 100])
hold on
plot(funcBeta)
title('Plot of measurement and function fit ')
ylabel('Humidity %')
xlabel('Beta')
legend('Measured', 'Approximated function')

figure(2)
a = plot(funcBeta,beta,percent,'Residuals')
title('Residual plot ')
ylabel('Humidity %')
xlabel('Beta')

\end{lstlisting}

\begin{figure}[H]
	\centering
	\includegraphics[width=0.8\textwidth]{../fig/Jordfugt/functionfit.PNG}
	\caption{Functionfit i MATLAB}
	\label{fig:appfuncfit}
\end{figure} 

\begin{figure}[H]
	\centering
	\includegraphics[width=0.8\textwidth]{../fig/Jordfugt/residualplot.PNG}
	\caption{Plot over residualerne}
	\label{fig:appresplot}
\end{figure} 
\newpage


