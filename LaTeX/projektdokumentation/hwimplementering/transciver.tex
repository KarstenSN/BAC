\section{Transceiver} \label{sec:transceiver}
Der er valgt at benytte en transceiver af typen MRFX49A\cite{lib:trans_MRFX49A_datasheet} fra Microchip, da denne model er SPI-kompatibel og ikke behøver at blive forprogrammeret som mange andre RF-kredse skal. 
Den initialiseres i stedet via SPI-kommunikationen. Chippen er som skræddersyet til dette system da den er designet til batteri-applikationer. Den har en indbygget "low battery voltage detect" som afgive et interrupt til mikroprocessoren når batterispændingen når under et forudbestemt niveau som sættes internt i transceiveren. Den har også en indbygget wake-up timer så chippen kan sættes i sleep-mode og herved minimere strømforbruget til nogle få $\mu A$ indtil den vækkes op igen efter en forudbestemt tid. Transceiveren kan kommunikere ved 433/868/915 MHz og ligger derfor i et frekvens område som ofte bruges i consumer elektronik. Dog er 915MHz mere rettet til det Amerikanske marked.   

\begin{figure}[ht]
	\centering
	\includegraphics[width=0.8\textwidth]{../fig/transciver/mrfx49a_circuit.PNG}
	\caption{Application circuit af MRF49XA}
	\label{fig:mrfx49a_cir}
\end{figure}

MRF49XA har i alt 16 indbyggede registre som der benyttes til at konfigurere IC'en med. For et overblik over disse henvises der til databladet\cite{lib:trans_MRFX49A_datasheet}. Herunder gennemgås hvordan der skrives til og læses fra et enkelt register. \texttt{STSREG} er statusregistret og er det eneste register der kan læses fra. Når statusregistret læses er der oftest blevet givet et eksternt interrupt til mikroprocessoren, dette interrupt kan være forsaget af mange hændelser og disse aflæses i statusregistret. Hændelser der forsager interruptet kan eks. være, at der er for lav batterispænding på sensoren, at der er wake-up timer overflow, eller at der er modtaget en byte fra senderen osv. Overblik over \texttt{STSREG} kan ses i Figur \ref{fig:mrfx49a_stsreg} og strukturen for en fuld læsning ses i Figur \ref{fig:mrfx49a_read}. Hvis bit 10 er "1" når status registret læses betyder det "Low Battery Detected", for yderligere info om statusregistret henvises til databladet \cite{lib:trans_MRFX49A_datasheet}.

\begin{figure}[H]
	\centering
	\includegraphics[width=0.8\textwidth]{../fig/transciver/mrfx49a_stsreg.PNG}
	\caption{Overblik over \texttt{STSREG}}
	\label{fig:mrfx49a_stsreg}
\end{figure}

\begin{figure}[H]
	\centering
	\includegraphics[width=0.8\textwidth]{../fig/transciver/mrfx49a_read.PNG}
	\caption{Læsning af \texttt{STSREG}}
	\label{fig:mrfx49a_read}
\end{figure}

Ved opstart skrives der først til \texttt{GENCREG}-registret som er det generelle konfigurations register. Her vælges hvilken frekvens transceiveren skal kommunikere ved, hvilken load kapacitet krystallet skal bruge, om TX-data registret skal aktiveres og om FIFO-registret skal aktiveres. Overblik over \texttt{GENCREG} kan ses på Figur \ref{fig:mrfx49a_gencreg}.

\begin{figure}[H]
	\centering
	\includegraphics[width=0.8\textwidth]{../fig/transciver/mrfx49a_gencreg.PNG}
	\caption{Overblik over \texttt{GENCREG}-registret}
	\label{fig:mrfx49a_gencreg}
\end{figure}

\texttt{CBB<15:8>} indeholder den første byte som skal sendes til receiveren. Denne byte angiver hvilket register der ønskes skrevet til. For \texttt{GENCREG} er andressen \texttt{0x80H}. Funktionaliteten sættes i bits 7-0. Ønskes der at loade krystallet med $10pF$, at kommunikere ved 433MHz, aktivering af FIFO samt TX registret, skal denne byte indeholde \texttt{0xD3H}. En komplet skrivning skal derfor indeholde \texttt{0x80D3H}. Den eneste forskel fra læsesekvensen i Figur \ref{fig:mrfx49a_read} er at \texttt{SDI} nu indeholder data i stedet for \texttt{SDO}. 