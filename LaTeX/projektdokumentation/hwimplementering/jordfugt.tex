\section{Måling af jordfugt}  \label{sec:jordfugt}
Hovedformålet med jordfugt sensoren er som det fremgår af navnet; at måle jordfugtigheden. Her er der i videnskabelige artikler fundet frem til fire metoder at gøre dette på. En resistiv målemetode \cite{lib:MultiSensorSystem}, en kapasitiv målemetode\cite{lib:MultiSensorSystem}, en optisk målemetode\cite{lib:optical} og en metode hvor der bruges en varmepuls\cite{lib:DPHP}. Den resistive målemetode er den simpleste og billigste men desværre også den mest upræcise. Der er igennem projektet blevet undersøgt om denne målemetode kunne bruges trods dens begrænsninger og om der kunne findes en løsning der på. Da det viste sig at være svært at forbedre den resistive målemetode blev der herefter undersøgt hvordan der kunne laves en kapasitiv måling. Skalaen som der er brugt at at bestemme jordfugten er den volumetriske skala\cite{lib:wiki_waterContent}. 

\begin{equation}
	\theta = \frac{V_w}{V_s + V_w +V_a} \cdot 100\%
	\label{equ:fugtighed}
\end{equation}

Hvor $\theta$ er fugtigheden i procent. $V_w$ er volumen af væsken, $V_s$ er volumen af jorden og $V_a$ er volumen af luften indeholdende i jorden. For at gøre det nemmere at bestemme en jordfugtighed afvejes først 1 liter jord, som der sammenpresses nok til at der ikke er noget luft i prøven. Herefter noteres vægten som $m_{s1L}$ og nu kan ligning \ref{equ:fugtighed} omskrives til ligning \ref{equ:fugtighed2} 

\begin{equation}
	\theta = \frac{m_w}{(m_{wet}-m_w)\cdot \eta + m_w} \cdot 100\% 
	\label{equ:fugtighed2}
\end{equation}

Hvor $m_w$ er massen af væske, $m_{wet}$ er massen af den fugtige jord og $\eta$ er udregnet via vægten af 1 liter jord.

\begin{equation}
	\eta = \frac{1}{m_{s1L}}
\end{equation}

Det er derfor nu muligt at bestemme en jordfugtighed uden at skulle måle volumen af prøven. Til vejning af jordprøver er der blevet brugt en vægt af producenten \texttt{Sartorius} model \texttt{QUINTIX2102-1S} med en præcision på $\pm0.01g$.

\subsubsection{Resistiv måling}
Den resistive målemetode er en metode til at måle den ohmske værdi af jorden. Jo mere vand jorden indeholder jo mere ledende vil jorden være og den ohmske værdi vil herefter falde. Der er dog mange faktor som kan være med til at give en måleusikkerhed, disse usikkerheder er betinget af selve næringsindholdet i jorden, om jorden er forurenet med metaller eller om proben som der måles med, har været udsat for kraftig tæring. Disse betingelser kunne der kalibreres for, men en kalibrering vil være besværlig for en almindelig bruger af systemet, da det vil kræve at brugeren ovntørrer en jordprøve og herefter tilsætter en præcis mængde vand for at vide med sikkerhed hvilken værdi der skal kalibreres ind efter. Det viste sig dog at der sammen med den resistive måling af jorden også opstod en kapasitiv virkning, som funktion af en stigetid. Se Figur\ref{fig:maling1}.  


\begin{figure}[H]
	\centering
	\includegraphics[width=0.8\textwidth]{../fig/Jordfugt/maling1.PNG}
	\caption{Impedans måling af jorden ved 17\% fugtighed}
	\label{fig:maling1}
\end{figure} 

Udfra Figur\ref{fig:maling1} kan der opstilles et ækvivalentdiagram som ses på Figur\ref{fig:akvj1}. I det øjeblik der bliver sat en spænding på proben, vil C1 være totalt afladet og det vil derfor være muligt at aflæse værdien af R1 ved brug af spændingsdeler-formlen. C1 vil herefter begynde at oplade og derfor træder R2 mere og mere i kraft. Det blev  besluttet at modellere jorden som et førsteordens system med overføringsfunktionen:
\begin{equation}
	System = \dfrac{\alpha \cdot \beta}{S + \alpha}
\end{equation}
Ved brug af denne overføringsfunktion ses R1 som værende 0 ohm. 
 
\begin{figure}[H]
	\centering
	\includegraphics[width=0.5\textwidth]{../fig/Jordfugt/akvivalentjord.PNG}
	\caption{Ækvivalentdiagram af jorden}
	\label{fig:akvj1}
\end{figure} 

I matlab blev der skrevet et script, som kan findes i bilagende, til at udregne overføringsfunktionen. Der blev i alt foretaget målinger på tre typer jord, en taget på Helgenæs, en i Randers samt en så og plantejord fra en plantesæk. Aflæsningen blev foretaget ved 3$\tau$ og ikke ved 1 $\tau$ som ellers er normalen. Dette skyldes at R1 er sat til 0 ohm. 

\begin{figure}[H]
	\centering
	\includegraphics[width=0.8\textwidth]{../fig/Jordfugt/maelingafsystem.PNG}
	\caption{Måling af overføringsfunktion i MATLAB}
	\label{fig:msystem}
\end{figure} 

På figur \ref{fig:msystem} ses hvordan den approksimerende overføringsfunktion er blevet målt. Den grønne streg er start tidspunktet og den gule streg er stop tidspunket. i Intervallet herimellem aflæses tiden af tre tidskonstanter. Disse divideres med 3 således at $\alpha$ står tilbage. $\beta$ aflæses blot som forholdet mellem inputtet og outputtet lige før spændingen går negativ. Grunden til at spændingen går negativ skyldes at der med tiden vil opstå slitage på proben, denne slitage vendes således at hver enkelt spyd vil slides lige meget. Det ses at når spændingen fjernes fra proben aflades C1 langsomt. Der blev noteret overføringsfunktioner for alle tre typer jord ved forskellige fugtigheder og tilsidst blev der lavet en regression til at bestemme fugtigheden ud fra $\beta$. $\alpha$ kunne også bruges til at bestemmer fugtigheden, men det viste sig hurtigt at den var meget upræcis at bruge, da den afhang meget af tiden siden den foregående måling. På Figur\ref{fig:funcfit} ses regressionen og på Figur\ref{fig:resplot} ses et residual plot over regressionen. Regressionen blev bestemt via matlabs indbyggede funktion \textit{fit} til at være:

\begin{equation}
	\theta = a \cdot exp(b\cdot x) + c \cdot exp(d\cdot x)
\end{equation}

Hvor a=2.219$\cdot 10^{7}$ b=-39.8 c=151.9 d=-3.473
 
\begin{figure}[H]
	\centering
	\includegraphics[width=0.8\textwidth]{../fig/Jordfugt/functionfit.PNG}
	\caption{Functionfit i MATLAB}
	\label{fig:funcfit}
\end{figure} 

\begin{figure}[H]
	\centering
	\includegraphics[width=0.8\textwidth]{../fig/Jordfugt/residualplot.PNG}
	\caption{Plot over residualerne}
	\label{fig:resplot}
\end{figure} 
 
Residualplottet viser at der er helt op til 8\% afvigelse på regressionen, hvilket må siges ikke at leve op til kravet om 5\% præcision. 
\subsubsection{Kapacitiv måling}
Den kapacitive målemetode benytter sig af et jordspyd stukket ned i jorden \cite{lib:cap_soil}. På jordspyddet er der to kobberplader som vil danne en kapacitet der vil variere afhængig af fugtigheden af den jord det befinder sig i. Se Figur \ref{fig:kap1} og \ref{fig:kap2}.

\begin{figure}[H]
  \centering
  \begin{minipage}[b]{0.4\textwidth}
    \includegraphics[width=1\textwidth]{../fig/Jordfugt/printkapasitet.png}
    \caption{Tværsnit af jordspyd}
    \label{fig:kap1}
  \end{minipage}
  \hfill
  \begin{minipage}[b]{0.4\textwidth}
    \includegraphics[width=1\textwidth]{../fig/Jordfugt/printkapasitet_vad.png}
    \caption{Tværsnit af jordspyd som der er i kontakt med vand. Her skal kobberbanerne dog være belagt med en elektrisk isolerende coating.}
    \label{fig:kap2}
  \end{minipage}
\end{figure}

For en kondensator gælder:

\begin{equation}
	C = \frac{\epsilon_0\cdot \epsilon_r \cdot A}{d}
\end{equation}

Hvor $\epsilon_0 = 8.854\cdot 10^{-12}F/m$ er permativiteten af vaccum, $\epsilon_r$ er permativiteten af materialet mellem kobberet som er 1 for luft og omkring 80 for vand. A er arealet af det kobber som kan se hinanden og d er diameteren mellem banerne. 

Det vil sige at kapaciteten vil blive 80 gange større når jordspyddet er stukket ned i rent vand en hvis den var i tør luft. Hvis vi antager at der er 0.3mm mellem banerne, at de har en længde på 5cm, at kobbertykkelsen er 35$\mu m$ og at jordspyddet befinder sig i vaccum. Vil der være en kapacitet på:

\begin{equation}
	C = \frac{8.854\cdot 10^{-12}\cdot 1 \cdot 35 \cdot 10^{-6} \cdot 5 \cdot 10^{-2}}{0.3 \cdot 10^{-3}} = 0.05pF
\end{equation} 

Antager vi at jordspyddet er stukket ned i vand vil vi have en kapacitet 80 gange størrer:

\begin{equation}
	C = \frac{8.854\cdot 10^{-12}\cdot 80 \cdot 35 \cdot 10^{-6} \cdot 5 \cdot 10^{-2}}{0.3 \cdot 10^{-3}} = 4.00pF
\end{equation} 

Det er derfor nu muligt at måle jordfugten ved at måle variationen af kondensatoren. Til det er der blevet overvejet to muligheder. Enten at måle amplitude ændringen når der sættes et AC signal igennem eller at måle faseændringen. Se Figur \ref{fig:jdig1} og \ref{fig:jdig2}.     

\begin{figure}[H]
  \centering
  \begin{minipage}[b]{0.4\textwidth}
    \includegraphics[width=1.2\textwidth]{../fig/Jordfugt/diagram1.png}
    \caption{Amplitudemåling af kondensator}
    \label{fig:jdig1}
  \end{minipage}
  \hfill
  \begin{minipage}[b]{0.4\textwidth}
    \includegraphics[width=1.2\textwidth]{../fig/Jordfugt/diagram2.png}
    \caption{Amplitudemåling af kondensator med instrumentationsforstærker}
    \label{fig:jdig2}
  \end{minipage}
\end{figure}

På Figur \ref{fig:jdig1} og \ref{fig:jdig2} ses der hvordan amplitudemålingen kan foretages. På Figur \ref{fig:jdig1} ses der et meget simpelt kredsløb som vil oplade en DC på C1 som hvis spænding afhænger af størrelsen af C\_soil. Kredsløbet er dog meget støjfølsomt og er derfor ugenet til dette formål. Kredsløbet kan forbedres med en instrumentationsforstærker som på Figur \ref{fig:jdig2}. Desværre er diodeforspændingen meget varierende med temperatur og strømtræk. Derfor vil dette kredsløb hellere ikke være ideel. Derimod er en måling af fasen mere præcis men samtid også mere avanceret. Det antages i det efterfølgende at kapaciteten vil variere i området 1pF til 60pF \cite{lib:cap_meas}. Ved at gøre denne antagelse kan vi nu udregne den optimale load af kondensatoren for at samlet give en faseændring på 90 grader. Overføringsfunktionen for et RC-højpas-led ses i ligning \ref{equ:j1}. 

\begin{equation}
	G(s) = \dfrac{R}{\dfrac{1}{s\cdot C} \cdot R}
	 \label{equ:j1}
\end{equation}  
På Figur \ref{fig:jf1} ses der et plot af fase ændringen med en modstand som eksperimentelt er bestemt til 160k$\Omega$ og en frekvens på 100kHz. Dette kan dog justeres under implementeringen.  
\begin{figure}[H]
	\centering
	\includegraphics[width=0.8\textwidth]{../fig/Jordfugt/fase_vs_cap.PNG}
	\caption{Faseændring versus kapaciteten}
	\label{fig:jf1}
\end{figure}   
For at måle kapacitetsændringen behøves der derfor en frekvens på 100kHz. Denne kan komme fra mikroprocessoren, men da det vil være en firkantspænding kræves det derfor en hård filtrering. Om filtreringen sker før eller efter kapaciteten, er ligegyldigt matematisk set. Men det vil give et bedre støjimmunt kredsløb ved at lave filtreringen efter. På Figur \ref{fig:jf2} ses diagrammet hvordan kredsløbet kan realiseres. U8B er en spændingsfølger som sidder for at give en høj indgangsimpedans, således fasen kan styrres med R25. Kredsløbet \cite{lib:bandpass} bestående af R22, R21, R16, C33, C34 er et båndpasfilter med øverføringsfunktion som ses i ligning \ref{equ:j2}. Udligning af denne ses i bilag \todo[inline]{reference}.   

\begin{equation}
	G(s) = -\dfrac{R2 C1 R3}{C1 C2 R1 R2 R3 s^2 + C1 R1 R2 s + C2 R1 R2 s + R1+ R2}
	 \label{equ:j2}
\end{equation} 

Hvor R22 = R1, R21 = R2, R16 = R3, C33 = C1, C34 = C2. Det ses at filteret er et 2. ordens system og har derfor en dæmpningsfaktor som er tæt realateret til quality-factor eller Q. Q siger noget om hvor snæver båndbredden er i systemet. Komponentværdierne er bestemt via ligning  \ref{equ:j3}, \ref{equ:j4}, \ref{equ:j5}. Hvor A er forstærkningen $\omega_0$ er $2\pi f$ hvor $f$ er frekvensen og C er værdien af C1 og C2.

\begin{equation}
	R1 = \dfrac{Q}{A\cdot \omega _0 \cdot C}
	 \label{equ:j3}
\end{equation} 

\begin{equation}
	R2 = \dfrac{Q}{(2Q^2-A) \cdot \omega_0 \cdot C}
	 \label{equ:j4}
\end{equation} 

\begin{equation}
	R3 = \dfrac{2Q}{\omega _0 \cdot C}
	 \label{equ:j5}
\end{equation} 
  
Sættes forstærkningen til A=2 og Q=10, $f$=100kHz og C=470pF. Fås komponentværdierne som på Figur \ref{fig:jf2}. På Figur \ref{fig:j6} ses der at forstærkningen ligger på 2 ved 100kHz som forventet og på Figur \ref{fig:j7} ses at der er en faseændring på 0 grader. Dog vil faseændringen være $180^\circ$ da det er en inverterende forstærker. Derfor sættes der en komporator på udgangen af U12A for at vende fasen til $^\circ$ igen. Dog vil der være et propagation delay som vil give en lille faseforskydelse. Ved at køre frekvensen fra mikroprocessoren og udgangen af U8A ind igennem en XOR gate (U10A) vil der blive opladet en DC på C26 som vil afhænge af faseforskydelsen. Ved en faseforskydelse på $180^\circ$ vil der ligge 3.3V og ved en faseforskydelse på $90^\circ$ vil der ligge 1.65V. Da vi kun skal detektere en faseforskydelse på maksimalt $90^\circ$ er der som det sidste led designet en ikke inverterende forstærker, med en forstærkning på 2, for at få fuldt dynamisk range. D5 og D6 sidder for at beskytte mod ESD. På Figur \ref{fig:j8} ses et plot af udgangsspændingen vs. kapaciteten.    
 

\begin{figure}[H]
	\centering
	\includegraphics[width=1\textwidth]{../fig/Jordfugt/diagram3.PNG}
	\caption{Diagram over kredsløb til måling af faseændring}
	\label{fig:jf2}
\end{figure}  

\begin{figure}[H]
  \centering
  \begin{minipage}[b]{0.4\textwidth}
    \includegraphics[width=1.3\textwidth]{../fig/Jordfugt/system_respons.png}
    \caption{Forstærkning vs. frekvens}
    \label{fig:j6}
  \end{minipage}
  \hfill
  \begin{minipage}[b]{0.4\textwidth}
    \includegraphics[width=1.3\textwidth]{../fig/Jordfugt/system_faserespons.png}
    \caption{Fasedrej vs. frekvens}
    \label{fig:j7}
  \end{minipage}
\end{figure}

\begin{figure}[H]
	\centering
	\includegraphics[width=1\textwidth]{../fig/Jordfugt/udgangsvolt.PNG}
	\caption{Udgangsspænding vs. kapacitet}
	\label{fig:j8}
\end{figure}
 

\clearpage