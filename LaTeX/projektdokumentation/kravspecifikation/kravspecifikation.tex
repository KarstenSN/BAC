\chapter{Kravspecifikation} \label{ch:kravspecifikation}
Det automatiske fugtbaseret havevandingsanlæg består samlet af en sensor, en kontrolboks og en motorventil. Dette udgør et automatisk fugtbaseret havevandingsanlæg. 
Systemet henvender sig til privatpersoner med interesse i havearbejde. Disse personer har ofte brug for at få andre til at tilse deres have hvis de er væk fra deres ejendom mere end et par dage ad gangen. Med dette system gives havepasseren mulighed for at vande haven automatisk og kan derfor være væk længere tid ad gangen. Systemet skal baseres på jordfugtigheden således der undgås at vande når planten ikke mangler vand. Systemet vil derfor både kunne spare havepasseren for en del arbejde og samtidig sikre at planterne altid har den optimale fugtighed uden at bruge unødige mængder vand. Systemet virker ved at sensoren er placeret ved et gromedie og sender et trådløst signal indeholdende jordfugtigheden samt en overfladetemperatur, batteristatus og lysintensitet til kontrolboksen. Kontrolboksen åbner eller lukker herefter for den givne ventil afhængig af hvilken fugtighed kontrolboksen er præindstillet til. Brugeren har endvidere mulighed for at vælge at vandingsanlæget kun vander om aftnen, morgnen eller begge. 

%---------------------------------------------------------------------------------------
%									FUNKTIONELLE KRAV
%---------------------------------------------------------------------------------------

\section{Funktionelle krav} \label{sec:funktionelle_krav}
I denne sektion opstilles de funktionelle krav til systemet. Kravende er det som systemet skal kunne gøre når det er fuldt funktionelt. 


\begin{enumerate}\itemsep1pt \parskip0pt \parsep0pt
	\item  \emph{Systemet} skal bestå af en sensor samt en kontrolboks indeholdende en ventil
	\item  \emph{Sensoren} skal måle temperatur, jordfugtighed samt lysintensitet
	\item  \emph{Sensoren} skal være batteridrevet
	\item  \emph{Kontrolboksen} skal udveksle data med sensoren via en trådløsforbindelse
	\item  \emph{Kontrolboksen} skal indeholde en brugergrænseflade i form af keypad og display
	\item  \emph{Kontrolboksen} skal drive en indbygget motorventil til at åbne og lukke for vandet
	\item  \emph{Kontrolboksen} skal kunne måle om slangen er sprunget fra, hvis den er skal der gives en fejlmeddelse
	\item  \emph{Brugeren} skal have mulighed for at åbne og lukke for ventilen manuelt på kontrolboksen
	\item  \emph{Brugeren} skal kunne aktivere/deaktivere systemet på kontrolboksen 
	\item  \emph{Brugeren} skal have mulighed for at indstille en åbningstid for ventilen når der vandes
	\item  \emph{Brugeren} skal have mulighed for at indstille en ønsket jordfugtighed på kontrolboksen som systemet herefter skal opretholde
	\item  \emph{Brugeren} skal have mulighed for at indstille en tidsbaseret vandingssekvens
	\item  \emph{Brugeren} skal kunne vælge automatisk vandingstidsrum, eks. om aftenen, morgenen eller begge
	\item  \emph{Kommunikation} mellem sensor til kontrolboks skal virke ved minimum 50m
\end{enumerate}

\section{Fremtidige krav} \label{sec:fremtige_krav}
Herunder opstilles der nogle krav som i fremtiden kan være med til at forbedre systemet. Disse krav kan tænkes opfyldt i en fremtidig version hvor der allerede er konstrureret et produkt i forvejen og kan altså ses som at være forbedringer til en vertion 2. 

\begin{enumerate}\itemsep1pt \parskip0pt \parsep0pt
	\item  \emph{Systemet} skal kunne betjenes via en smartphone applikation, hvori det er muligt at se grafer over tid for jurdfugtigheden, temperaturen samt lysintensiteten
	\item  \emph{Systemet} skal kunne indstilles til en bestemt afgrøde, således brugeren ikke selv skal vide hvilken fugtighed en bestemt afgrøde kræver
	\item  \emph{Systemet} skal kunne tilsluttes flere sensore

\end{enumerate}
%--------------------------------------------------------------------------------------
%												USE CASES
%--------------------------------------------------------------------------------------
\section{Use Cases} \label{sec:use_cases}
Udfra kravende i sektion \ref{sec:funktionelle_krav} opstilles der følgende use cases som ses i figur \ref{fig:UCsystem}

\begin{figure}[H]
\centering
\includegraphics[width=0.8\textwidth]{../fig/Systemarkitektur/Usecases.pdf}
\caption{Use cases over systemet}
\label{fig:UCsystem}
\end{figure}

% UC1:  Aktiver system
\subsubsection{Use Case 1: Aktiver system}
%-------------------- UC1 --------------------
\begin{table}[H]
\begin{tabularx}{\textwidth}{| L{3.3 cm} | Z |} \hline

\textbf{Navn:} 						 & UC1: Tænd/sluk system						\\ \hline
\textbf{Formål:}					 & Her gives brugeren mulighed for at tænde eller slukke for systemet.  Dette gøres ved at holde power knappen nede i 3 sekunder på tastaturet på kontrolboksen. Se Figur xx for skitse af tastatur 							\\ \hline
\textbf{Initiering:}				 & Bruger 										\\ \hline
\textbf{Aktører:} 					 & Bruger 										\\ \hline
\textbf{Forudsætning:} 				 & Kontrolboksen er tilsluttet forsyningsspænding \\ \hline
\textbf{Resultat:}					 & Systemetet er aktiveret 	\\ \hline
\textbf{Hovedscenarie:}				 & 

\begin{packed_enum}
	\item Brugeren holder power knappen nede i minimum 3 sekunder
	\item Der vises nu temperatur og fugtighed på displayet, målt i jorden hvor sensoren er placeret
		\begin{packed_item}\itemsep1pt \parskip0pt \parsep0pt
		\item {[}Ext 1 : Systemet er i forvejen tændt{]}
		\end{packed_item}
		\begin{packed_item}\itemsep1pt \parskip0pt \parsep0pt
		\item {[}Ext 2 : Det er første gang systemet tændes{]}
		\end{packed_item}
	
\end{packed_enum} 																\\ \hline

\textbf{Udvidelser:}				&  
\textbf{{[}Ext 1 : Systemet er i forvejen tændt{]}}
	\begin{packed_enum}\itemsep1pt \parskip0pt \parsep0pt
		\item Systemet slukker og teksten i displayet forsvinder				
	\end{packed_enum}
\textbf{{[}Ext 2 : Det er første gang systemet tændes eller der er ikke forbindelse til sensoren{]}}
	\begin{packed_enum}\itemsep1pt \parskip0pt \parsep0pt
		\item Teksten "Søger efter sensor" fremkommer på displayet og bliver stående indtil der er forbindelse til en sensor
		\item Der vises nu temperatur og fugtighed på displayet, målt i jorden hvor sensoren er placeret			
	\end{packed_enum}
																				\\ \hline

\end{tabularx}
\caption{UC1: Aktiver system}
\label{tbl:UC1}
\end{table}

% UC2:  Aflæs data
\subsubsection{Use Case 2: Aflæs fugtighed og temperatur}
%-------------------- UC2 --------------------
\begin{table}[H]
\begin{tabularx}{\textwidth}{| L{3.3 cm} | Z |} \hline

\textbf{Navn:} 						 & UC2: Aflæs fugtighed og temperatur											\\ \hline
\textbf{Formål:}					 & Her gives Brugeren mulighed for at aflæse nuværende fugtighed og temperatur	\\ \hline
\textbf{Initiering:}				 & Bruger 																		\\ \hline
\textbf{Aktører:} 					 & Bruger 																		\\ \hline
\textbf{Forudsætning:} 				 & Kontrolboksen er tilsluttet forsyningsspænding, sensoren er tilsluttet batteri, samt UC8: (par sensor og kontrolboks) er kørt med succes \\ \hline
\textbf{Resultat:}					 & Brugeren er informeret om senest målte fugtighed og temperatur 	\\ \hline
\textbf{Hovedscenarie:}				 & 

\begin{packed_enum}
	\item Brugeren kigger på display'et og aflæser den senest målte fugtighed og temperatur
	
\end{packed_enum} 							\\ \hline


	\textbf{Udvidelser:}				& 	\\ \hline

\end{tabularx}
\caption{UC2: Aflæs fugtighed og temperatur}
\label{tbl:UC2}
\end{table}

% UC3:  Indstil ønsket fugtighed i jorden
\subsubsection{Use Case 3: Indstil ønsket fugtighed i jorden}
%-------------------- UC3 --------------------
\begin{table}[H]
\begin{tabularx}{\textwidth}{| L{3.3 cm} | Z |} \hline

\textbf{Navn:} 						 & UC3: Indstil ønsket fugtighed i jorden						\\ \hline
\textbf{Formål:}					 & Her gives brugeren mulighed for at instille den ønskede fugtighed i jorden 							\\ \hline
\textbf{Initiering:}				 & Bruger 										\\ \hline
\textbf{Aktører:} 					 & Bruger 										\\ \hline
\textbf{Forudsætning:} 				 & Kontrolboksen er tilsluttet forsyningsspænding og sensoren er tilsluttet \\ \hline
\textbf{Resultat:}					 & Systemet opretholder en bestemt fugtighed i jorden 	\\ \hline
\textbf{Hovedscenarie:}				 & \todo[inline]{Opdater i forhold til kode}

\begin{packed_enum}
	\item Brugeren trykker på SET knappen på taststuret 
	\item Der vises nu hvad systemet i forvejen er indstillet til
	\item Brugeren trykker nu enten på pil op eller pil ned for at stille fugtigheden op eller ned
	\item Brugeren trykker på OK
	\item Der vises nu temperatur og fugtighed på displayet, målt i jorden hvor sensoren er placeret
	
\end{packed_enum} 																\\ \hline

\textbf{Udvidelser:}				&  

%	\end{packed_enum}
																				\\ \hline

\end{tabularx}
\caption{UC3: Indstil ønsket fugtighed i jorden	}
\label{tbl:UC3}
\end{table}

% UC4:  Indstil åbningstid for ventil
\subsubsection{Use Case 4: Indstil åbningstid for ventil}
%-------------------- UC3 --------------------
\begin{table}[H]
\begin{tabularx}{\textwidth}{| L{3.3 cm} | Z |} \hline

\textbf{Navn:} 						 & UC4: Indstil åbningstid for ventil						\\ \hline
\textbf{Formål:}					 & Her gives brugeren mulighed for at instille åbningstiden for ventilen. Åbningstiden er nødvendig at indstille da kontrolboksen kun modtager data fra sensoren hvert 15. minut.  Åbningstiden kan indstilles fra 1 til 15 minutter  							\\ \hline
\textbf{Initiering:}				 & Bruger 										\\ \hline
\textbf{Aktører:} 					 & Bruger 										\\ \hline
\textbf{Forudsætning:} 				 & Kontrolboksen er tilsluttet forsyningsspænding og sensoren er tilsluttet \\ \hline
\textbf{Resultat:}					 & Åbningstiden for ventilen er sat 	\\ \hline
\textbf{Hovedscenarie:}				 & \todo[inline]{Opdater i forhold til kode}

\begin{packed_enum}
	\item Brugeren trykker 2 gange på SET knappen på taststuret 
	\item Der vises nu hvad åbningstiden i forvejen er indstillet til
	\item Brugeren trykker nu enten på pil op eller pil ned for at stille åbningstiden op eller ned
	\item Brugeren trykker på OK
	\item Der vises nu temperatur og fugtighed på displayet, målt i jorden hvor sensoren er placeret
\end{packed_enum} 																\\ \hline

\textbf{Udvidelser:}				&  

%	\end{packed_enum}
																				\\ \hline
\end{tabularx}
\caption{UC4: Indstil åbningstid for ventil	}
\label{tbl:UC4}
\end{table}

% UC5:Indstil tidsbaseret vandingsinterval
\subsubsection{Use Case 5: Indstil tidsbaseret vandingsinterval}
%-------------------- UC5 --------------------
\begin{table}[H]
\begin{tabularx}{\textwidth}{| L{3.3 cm} | Z |} \hline

\textbf{Navn:} 						 & UC5: Indstil tidsbaseret vandingsinterval						\\ \hline
\textbf{Formål:}					 & Her gives brugeren mulighed for at indstille et tidsbaseret vandingsinterval 							\\ \hline
\textbf{Initiering:}				 & Bruger 										\\ \hline
\textbf{Aktører:} 					 & Bruger 										\\ \hline
\textbf{Forudsætning:} 				 & Kontrolboksen er tilsluttet forsyningsspænding \\ \hline
\textbf{Resultat:}					 & Der er indstillet et tidsbaseret vandingsinterval 	\\ \hline
\textbf{Hovedscenarie:}				 & 

\begin{packed_enum}
	\item Brugeren trykker 3 gange på \texttt{SET} på tastaturet
	\item Der vises nu hvad åbningstiden i forvejen er indstillet til
	\item Brugeren trykker nu enten på pil op eller pil ned for at stille åbningstiden op eller ned. Denne kan indstilles i et interval på 0 til 60 minutter. 
	\item Brugeren trykker på \texttt{SET} på tastaturet
	\item Brugeren trykker nu enten på pil op eller pil ned for at stille lukketiden op eller ned. Denne kan indstilles i et interval på 0 til 23 timer.
	\item Brugeren trykker på \texttt{OK} på tastaturet
	\item Der vises nu temperatur og fugtighed på displayet, målt i jorden hvor sensoren er placeret
	
\end{packed_enum} 																\\ \hline

\textbf{Udvidelser:}				&  

%	\end{packed_enum}
																				\\ \hline
\end{tabularx}
\caption{UC5: Indstil tidsbaseret vandingsinterval}
\label{tbl:UC5}
\end{table}

% UC6:Åbn/luk for ventil
\subsubsection{Use Case 6: Åbn/luk ventil manuelt}
%-------------------- UC6 --------------------
\begin{table}[H]
\begin{tabularx}{\textwidth}{| L{3.3 cm} | Z |} \hline

\textbf{Navn:} 						 & UC6: Åbn/luk ventil manuelt						\\ \hline
\textbf{Formål:}					 & Her gives brugeren mulighed for manuelt at åbne eller lukke ventilen  \\ \hline
\textbf{Initiering:}				 & Bruger 										\\ \hline
\textbf{Aktører:} 					 & Bruger 										\\ \hline
\textbf{Forudsætning:} 				 & Kontrolboksen er tilsluttet forsyningsspænding \\ \hline
\textbf{Resultat:}					 & Ventilen er enten åbnet eller lukket 	\\ \hline
\textbf{Hovedscenarie:}				 & 

\begin{packed_enum}
	\item Brugeren trykker på \texttt{Pil-op} eller \texttt{Pil-ned} på tastaturet
	\item Teksten "Open valve?" forekommer hvis der er trykket \texttt{Pil-op}. Er der trykket \texttt{Pil-ned} forekommer teksten "Close valve?"
	\item Brugeren trykker på \texttt{OK} på tastaturet 
	\item Ventilen åbner eller lukker afhængig af brugerens valg i punkt 1.
		\begin{packed_item}\itemsep1pt \parskip0pt \parsep0pt
		\item {[}Ext 1 : Ventilen er allerede i den ønskede position{]}
		\end{packed_item}
	
	
\end{packed_enum} 																\\ \hline

\textbf{Udvidelser:}				&  
\textbf{{[}Ext 1 : Ventilen er allerede i den ønskede position{]}}
	\begin{packed_enum}\itemsep1pt \parskip0pt \parsep0pt
		\item Teksten "Already open" eller "Already closed" vises på displayet.				
	\end{packed_enum}

%	\end{packed_enum}
																				\\ \hline
\end{tabularx}
\caption{UC6: Åbn/luk ventil manuelt}
\label{tbl:UC6}
\end{table}

% UC7:Vælg automatisk vandingstidspunkt 
\begin{longtable}{| l | >{\raggedright}X | >{\raggedright}X | >{\raggedright}X | >{\raggedright\arraybackslash}p{2.3cm} |} \hline
	\multicolumn{2}{|l|}{\textbf{Use case under test}}  & \multicolumn{3}{l|}{UC7: Indstil vandingstidspunkt} \\ \hline
	\multicolumn{2}{|l|}{\textbf{Scenarie}} 			& \multicolumn{3}{l|}{Hovedscenarie} \\ \hline
	\multicolumn{2}{|l|}{\textbf{Forudsætning}} 		& \multicolumn{3}{p{10.2cm}|}{UC1 og UC8 er gennemført samt sensoren er placeret i tør luft og dagslys. Fugtigheden er indstillet til 10\% i UC3. Åbningstiden er indstillet til 1 minut i UC4\hfill} \\ \hline
	%\multicolumn{5}{|l|}{}\\ \hline
	\textbf{Step} 	& \textbf{Handling} & \textbf{Forventet Resultat} & \textbf{Resultat} & \textbf{Godkendt / Kommentar} \\ \hline
	
	1   & Der trykkes på \texttt{SET} 5 gange
		& Teksten "Watering period" fremkommer på displayes linje 1. På linje 2 vises hvad indstillingen i forvejen er sat til
		& resultat
		& resultat\\ \hline
		
	2   & Der trykkes på enten \texttt{Pil-op} eller \texttt{Pil-ned} for at vælge \texttt{Evening}
		& Vandingstidspunktet er sat til \texttt{Evening}
		& resultat
		& resultat\\ \hline
		
	3   & Der trykkes på \texttt{OK}
		& Der returneres til hovedmenuen
		& resultat
		& resultat\\ \hline
		
	4   & Sensoren ligges i en papkasse således intet lys kan komme igennem
		& Sensoren ligger i en papkasse hvor intet lys kommer igennem
		& resultat
		& resultat\\ \hline
		
	5   & Der ventes i op til 15 min
		& Ventilen åbner
		& resultat
		& resultat\\ \hline
		
	6   & Der ventes i 1 min
		& Ventilen lukker
		& resultat
		& resultat\\ \hline	

	7   & Der trykkes på \texttt{SET} 5 gange
		& Teksten "Watering period" fremkommer på displayes linje 1. På linje 2 vises "Time:Evening"
		& resultat
		& resultat\\ \hline
		
	8   & Der trykkes på \texttt{Pil-op} for at vælge \texttt{Morning}
		& Vandingstidspunktet er sat til \texttt{Morning}
		& resultat
		& resultat\\ \hline
		
	9   & Der trykkes på \texttt{OK}
		& Der returneres til hovedmenuen
		& resultat
		& resultat\\ \hline
		
	10  & Sensoren tages ud af papkassen og ligges i dagslys
		& Sensoren ligger i dagslys
		& resultat
		& resultat\\ \hline
		
	11  & Der ventes i op til 15 min
		& Ventilen åbner
		& resultat
		& resultat\\ \hline		

	12  & Der ventes i 1 min
		& Ventilen lukker
		& resultat
		& resultat\\ \hline	
		
	13  & Der trykkes på \texttt{SET} 5 gange
		& Teksten "Watering period" fremkommer på displayes linje 1. På linje 2 vises "Time:Morning"
		& resultat
		& resultat\\ \hline
		
	14  & Der trykkes på \texttt{Pil-ned} to gange for at vælge \texttt{Both}
		& Vandingstidspunktet er sat til \texttt{Both}
		& resultat
		& resultat\\ \hline
	
	15  & Der trykkes på \texttt{OK}
		& Der returneres til hovedmenuen
		& resultat
		& resultat\\ \hline	
		
	16  & Sensoren ligges i en papkasse således intet lys kan komme igennem
		& Sensoren ligger i en papkasse hvor intet lys kommer igennem
		& resultat
		& resultat\\ \hline
		
	17  & Der ventes i op til 15 min
		& Ventilen åbner
		& resultat
		& resultat\\ \hline
		
	18  & Der ventes i 1 min
		& Ventilen lukker
		& resultat
		& resultat\\ \hline	

	19  & Der ventes i op til 15 min
		& intet sker
		& resultat
		& resultat\\ \hline		
		
	20  & Sensoren tages ud af papkassen og ligges i dagslys
		& Sensoren ligger i dagslys
		& resultat
		& resultat\\ \hline
		
	21  & Der ventes i op til 15 min
		& Ventilen åbner
		& resultat
		& resultat\\ \hline		

	22  & Der ventes i 1 min
		& Ventilen lukker
		& resultat
		& resultat\\ \hline	
				
	
\caption{Accepttest for UC7: Brems Bil }\label{tbl:acceptuc7}
\end{longtable}

\clearpage