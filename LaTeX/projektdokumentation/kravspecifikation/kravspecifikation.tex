\chapter{Kravspecifikation} \label{ch:kravspecifikation}
I det følgende afsnit opstilles de funktionelle og ikke-funktionelle krav til systemt. De funktionelle krav er prioriteret efter MoSCoW-modellen. 
% \TODO{lave henvisning til MoSCoW i litteraturlisten}
Denne model benyttes til at sikre en målrettet projektudvikling. Hvert individuelt krav tildeles en prioritering, hermed kan opnås størst fokus på de mest systemkritiske krav, metoden er beskrevet nærmere i afsnit \ref{sec:moscow}. De ikke-funktionelle krav er arrangeret efter FURPS+ modellen.


%---------------------------------------------------------------------------------------
%									FUNKTIONELLE KRAV
%---------------------------------------------------------------------------------------
\newpage
\section{Funktionelle krav} \label{sec:funktionelle_krav}
I denne sektion opstilles de funktionelle krav til systemet. Disse krav skal opfyldes for at systemet betragtes som fuldt funktionelt. I tabel \ref{tbl:funk_krav} listes de funktionelle krav med unik ID så der i systemet er fuld traceability. kolonnen textit{prioritet} henviser til MoSCoW-prioriteringen der er beskrevet i afsnit \ref{sec:moscow}.

\begin{table}[H]
	\begin{tabularx}{\textwidth}{| m{1.5 cm} | Z | L{1.9 cm}| m{2.9 cm} |} \hline
	\textbf{ID:} 		& \textbf{Krav:}		& \textbf{Prioritet:} 	& \textbf{Kommentarer:} 								\\ \hline
	
MC\_01	& Systemet skal bestå af en sensor samt en kontrolboks indeholdende mikroprocessor og en ventil.	& \textit{Must}	 & 	\\ \hline
MC\_02	& Systemet skal kunne pre-indstilles til en specifik afgrøde 										& \textit{Could} &  \\ \hline
MC\_03	& Systemet skal kunne betjenes fra en smartphone-applikation										& \textit{Would} &  \\ \hline
MC\_04	& Systemet skal kunne tilsluttes flere sensor														& \textit{Would} &  \\ \hline
MC\_05	& Sensoren skal måle temperatur, jordfugtighed samt lysintensitet									& \textit{Must}  & 	\\ \hline
EL\_01	& Sensoren skal være batteridrevet																	& \textit{Must}	 & 	\\ \hline
EL\_02	& Kontrolboksen skal drive en indbygget motorventil til at åbne og lukke for vandet					& \textit{Must}	 & 	\\ \hline
CM\_01	& Kontrolboksen skal kunne udveksle data med sensoren via en trådløs forbindelse					& \textit{Must}	 & 	\\ \hline
CM\_02	& Kommunikation fra sensor til kontrolboks skal virke ved minimum 50m								& \textit{Should}& 	\\ \hline
UX\_01	& Kontrolboksen skal indeholde UI i form af keypad og display										& \textit{Must}	 & 	\\ \hline
UX\_02	& Kontrolboksen skal kunne måle om slangen er sprunget fra, og give fejlmelding 					& \textit{Should}& 	\\ \hline
UX\_03	& Brugeren skal have mulighed for at åbne/lukke for ventilen manuelt på kontrolboksen				& \textit{Must}	 & 	\\ \hline
UX\_04	& Brugeren skal kunne aktivere/deaktivere systemet på kontrolboksen 								& \textit{Must}	 & 	\\ \hline
UX\_05	& Brugeren skal have mulighed for at indstille et ønsket vandingsinterval							& \textit{Must}	 & 	\\ \hline
UX\_06	& Brugeren skal have mulighed for at indstille en ønsket jordfugtighed på kontrolboksen.			& \textit{Must}	 & 	\\ \hline
UX\_07	& Brugeren skal have mulighed for at indstille en tidsbaseret vandingssekvens						& \textit{Must}  & 	\\ \hline
UX\_08	& Brugeren skal kunne vælge automatisk vandingstidsrum, eks. om aftenen, morgenen eller begge		& \textit{Must}	 & 	\\ \hline
UX\_09	& Brugeren skal have mulighed for at indstille en tidsbaseret vandingssekvens 						& \textit{Should}& 	\\ \hline		

	\end{tabularx}
	\caption{Funktionelle krav}
	\label{tbl:funk_krav}
\end{table}



\subsection{MoSCoW} \label{sec:moscow}
I de følgende tabeller herunder ses de samme funktionelle krav som i tabel \ref{tbl:funk_krav}, dog er de her arrangeret efter MoSCoW-modellen. Under \textit{Must} placeres de krav som skal være opfyldt for at systemet kan siges at være minimum funktionsdygtigt. \textit{Should} indeholder de krav der bør være implementeret for at systemet kan kaldes fuldt funktionsdygtigt, disse krav implementeres hvis tid og ressourcer tillader det. \textit{Could} indeholder de krav som systemet ville ha kunne opfylde hvis tid og ressourcer ikke er en begrænsning. textit{Would} er fremtidige krav til systemets efterfølgende iterationer, disse er derfor ikke implementeret i denne prototype.   

\textbf{Must}
\begin{table}[H]
	\begin{tabularx}{\textwidth}{ m{1.5 cm}  Z  L{1.9 cm} m{2.9 cm} }
		MC\_01 	& Systemet skal bestå af en sensor samt en kontrolboks indeholdende mikroprocessor og en motorventil.  	\\
		MC\_05	& Sensoren skal måle temperatur, jordfugtighed samt lysintensitet										\\
		EL\_01	& Sensoren skal være batteridrevet																		\\
		EL\_02	& Kontrolboksen skal drive en indbygget motorventil til at åbne og lukke for vandet						\\
		CM\_01	& Kontrolboksen skal kunne udveksle data med sensoren via en trådløs forbindelse 						\\
		UX\_01	& Kontrolboksen skal indeholde UI i form af keypad og display											\\ 
		UX\_03	& Brugeren skal have mulighed for at åbne/lukke for ventilen manuelt på kontrolboksen					\\
		UX\_04	& Brugeren skal kunne aktivere/deaktivere systemet på kontrolboksen										\\
		UX\_05	& Brugeren skal have mulighed for at indstille et ønsket vandingsinterval								\\
		UX\_06	& Brugeren skal have mulighed for at indstille en ønsket jordfugtighed på kontrolboksen					\\
		UX\_07	& Brugeren skal have mulighed for at indstille en tidsbaseret vandingssekvens							\\
		UX\_08	& Brugeren skal kunne vælge automatisk vandingstidsrum, eks. om aftenen, morgenen eller begge			\\
	\end{tabularx}
	\caption{Must krav}
	\label{tbl:must_krav}
\end{table}

\textbf{Should}
\begin{table}[H]
	\begin{tabularx}{\textwidth}{ m{1.5 cm}  Z  L{1.9 cm} m{2.9 cm} }
		CM\_02	& Kommunikation fra sensor til kontrolboks skal virke ved minimum 50m				\\
		UX\_02	& Kontrolboksen skal kunne måle om slangen er sprunget fra, og give fejlmelding 	\\
		UX\_09	& Brugeren skal have mulighed for at indstille en tidsbaseret vandingssekvens		\\
	\end{tabularx}
	\caption{Should krav}
	\label{tbl:should_krav}
\end{table}

\textbf{Could}
\begin{table}[H]
	\begin{tabularx}{\textwidth}{ m{1.5 cm}  Z  L{1.9 cm} m{2.9 cm} }
		MC\_02	& Systemet skal kunne pre-indstilles til en specifik afgrøde 	\\
	\end{tabularx}
	\caption{Could krav}
	\label{tbl:could_krav}
\end{table}

\textbf{Would}
\begin{table}[H]
	\begin{tabularx}{\textwidth}{ m{1.5 cm}  Z  L{1.9 cm} m{2.9 cm} }
		MC\_03	& Systemet skal kunne betjenes fra en smartphone-applikation	\\
		MC\_04	& Systemet skal kunne tilsluttes flere sensor					\\
	\end{tabularx}
	\caption{Would krav}
	\label{tbl:would_krav}
\end{table}


%---------------------------------------------------------------------------------------
%									IKKE-FUNKTIONELLE KRAV
%---------------------------------------------------------------------------------------
\newpage
\section{Ikke-Funktionelle krav} \label{sec:ikke_funktionelle_krav}
\subsection{FURPS+} \label{sec:furps}
De ikke funktionelle krav til systemet opstilles efter FURPS+ modellen udviklet af Hewlett-Packard. 

% FIND en fornuftig reference 

Denne metode benyttes til at klassificere og administrere funktionelle og ikke-funktionelle krav, her benyttes modellen primært til håndtering af sidstnævnte.
Modellen inddeler kravsættet i følgende undergrupper.

\begin{itemize}
	\item 	Functionality (funktionalitet)
	\item	Usability (Brugervenlighed)
	\item	Reliability (Driftsikkerhed)
	\item	Performance (Performance)
	\item	Supportability (Vedligeholdelse)
	\item	+ (Design and Fysiske begrænsninger, Interfaces)
\end{itemize}

Funktionalitetskravene er der taget hånd om i afsnit \ref{sec:funktionelle_krav}. De ikke funktionelle krav er indekseret som det ses på de følgende tabeller, som med de funktionelle krav tildeles her unik krav ID. 

\newpage
\textbf{Brugervenlighed}
\begin{table}[H]
	\begin{tabularx}{\textwidth}{| m{1.5 cm} | Z | m{2.9 cm} |} \hline
		\textbf{ID:} & \textbf{Krav:}														& \textbf{Kommentarer:} 	\\ \hline
		US\_01		 & Keypad'en skal bestå af 4 taktile trykknapper						&	\\ \hline
		US\_02		 & Keypad'en skal indeholde 1 rød LED til indikation af fejl 			&	\\ \hline
		US\_03		 & Displayet skal kunne vise jordfugtighed, temperatur					&	\\ \hline
		US\_04		 & Displayet skal kunne vise systemets menuer							&	\\ \hline
	\end{tabularx}
	\caption{Krav til brugervenlighed}
	\label{tbl:brugervenlighed_krav}
\end{table}

\textbf{Driftsikkerhed}
\begin{table}[H]
	\begin{tabularx}{\textwidth}{| m{1.5 cm} | Z | m{2.9 cm} |} \hline
		\textbf{ID:} & \textbf{Krav:}														& \textbf{Kommentarer:} \\ \hline
		RI\_01		 & Systemfejl skal indikeres med rød LED på keypad						& \\ \hline
		RI\_02		 & Systemfejlmelding skal vises på display 								& \\ \hline
		RI\_03		 & Systemet skal ha en MTBF (MeanTime Between Failures) på ????			& \\ \hline
		RI\_04		 & System skal ha en MTTR (MeanTime To Restore) på ????					& \\ \hline
	\end{tabularx}
	\caption{Krav til driftsikkerhed}
	\label{tbl:driftsikkerhed_krav}
\end{table}

\textbf{Performance}
\begin{table}[H]
	\begin{tabularx}{\textwidth}{| m{1.5 cm} | Z | m{2.9 cm} |} \hline
		\textbf{ID:} & \textbf{Krav:}														& \textbf{Kommentarer:} \\ \hline
		PF\_01		 & Batteritype (AAA ???)												& \\ \hline
		PF\_02		 & Sensorens batterilevetid (min 1 år)  								& \\ \hline
		PF\_03		 & Hvor mange instanser af systemet skal kunne køre samtidig			& \\ \hline
		PF\_04		 & Temperatur range (1-70 graderC)										& \\ \hline
		PF\_05		 & Målepræcision (5\%)													& \\ \hline
		PF\_06		 & Systemrespons fra brugerinteraktion til aktuator skal være 1<sek		& \\ \hline
	\end{tabularx}
	\caption{Krav til performance}
	\label{tbl:performance_krav}
\end{table}

\textbf{Vedligeholdelse}
\begin{table}[H]
	\begin{tabularx}{\textwidth}{| m{1.5 cm} | Z | m{2.9 cm} |} \hline
		\textbf{ID:} & \textbf{Krav:}														& \textbf{Kommentarer:} \\ \hline
		SP\_01		 & Systemet skal kun vedligeholdes i forbindelse med batteriskift 		& \\ \hline
		SP\_02		 & Batteriet skal være let tilgængelig 									& \\ \hline
		SP\_03		 & Brugeren skal selv kunne skifte batteri på sensor-enheden			& \\ \hline
	\end{tabularx}
	\caption{Krav til vedligeholdelse}
	\label{tbl:vedligeholdelse_krav}
\end{table}


%--------------------------------------------------------------------------------------
%												USE CASES
%--------------------------------------------------------------------------------------
\newpage
\section{Use Cases} \label{sec:use_cases}
Ud fra de funktionelle krav i sektion \ref{sec:funktionelle_krav} opstilles følgende use cases som ses i use case-diagrammet på figur \ref{fig:UCdiagram}

\begin{figure}[H]
\centering
\includegraphics[width=0.8\textwidth]{../fig/Systemarkitektur/Usecases.pdf}
\caption{Use cases over systemet}
\label{fig:UCdiagram}
\end{figure}

\newpage
% UC1:  Aktiver system
\subsubsection{Use Case 1: Aktiver system}
%-------------------- UC1 --------------------
\begin{table}[H]
\begin{tabularx}{\textwidth}{| L{3.3 cm} | Z |} \hline

\textbf{Navn:} 						 & UC1: Tænd/sluk system						\\ \hline
\textbf{Formål:}					 & Her gives brugeren mulighed for at tænde eller slukke for systemet.  Dette gøres ved at holde power knappen nede i 3 sekunder på tastaturet på kontrolboksen. Se Figur xx for skitse af tastatur 							\\ \hline
\textbf{Initiering:}				 & Bruger 										\\ \hline
\textbf{Aktører:} 					 & Bruger 										\\ \hline
\textbf{Forudsætning:} 				 & Kontrolboksen er tilsluttet forsyningsspænding \\ \hline
\textbf{Resultat:}					 & Systemetet er aktiveret 	\\ \hline
\textbf{Hovedscenarie:}				 & 

\begin{packed_enum}
	\item Brugeren holder power knappen nede i minimum 3 sekunder
	\item Der vises nu temperatur og fugtighed på displayet, målt i jorden hvor sensoren er placeret
		\begin{packed_item}\itemsep1pt \parskip0pt \parsep0pt
		\item {[}Ext 1 : Systemet er i forvejen tændt{]}
		\end{packed_item}
		\begin{packed_item}\itemsep1pt \parskip0pt \parsep0pt
		\item {[}Ext 2 : Det er første gang systemet tændes{]}
		\end{packed_item}
	
\end{packed_enum} 																\\ \hline

\textbf{Udvidelser:}				&  
\textbf{{[}Ext 1 : Systemet er i forvejen tændt{]}}
	\begin{packed_enum}\itemsep1pt \parskip0pt \parsep0pt
		\item Systemet slukker og teksten i displayet forsvinder				
	\end{packed_enum}
\textbf{{[}Ext 2 : Det er første gang systemet tændes eller der er ikke forbindelse til sensoren{]}}
	\begin{packed_enum}\itemsep1pt \parskip0pt \parsep0pt
		\item Teksten "Søger efter sensor" fremkommer på displayet og bliver stående indtil der er forbindelse til en sensor
		\item Der vises nu temperatur og fugtighed på displayet, målt i jorden hvor sensoren er placeret			
	\end{packed_enum}
																				\\ \hline

\end{tabularx}
\caption{UC1: Aktiver system}
\label{tbl:UC1}
\end{table}

% UC2:  Aflæs data
\subsubsection{Use Case 2: Aflæs fugtighed og temperatur}
%-------------------- UC2 --------------------
\begin{table}[H]
\begin{tabularx}{\textwidth}{| L{3.3 cm} | Z |} \hline

\textbf{Navn:} 						 & UC2: Aflæs fugtighed og temperatur											\\ \hline
\textbf{Formål:}					 & Her gives Brugeren mulighed for at aflæse nuværende fugtighed og temperatur	\\ \hline
\textbf{Initiering:}				 & Bruger 																		\\ \hline
\textbf{Aktører:} 					 & Bruger 																		\\ \hline
\textbf{Forudsætning:} 				 & Kontrolboksen er tilsluttet forsyningsspænding, sensoren er tilsluttet batteri, samt UC8: (par sensor og kontrolboks) er kørt med succes \\ \hline
\textbf{Resultat:}					 & Brugeren er informeret om senest målte fugtighed og temperatur 	\\ \hline
\textbf{Hovedscenarie:}				 & 

\begin{packed_enum}
	\item Brugeren kigger på display'et og aflæser den senest målte fugtighed og temperatur
	
\end{packed_enum} 							\\ \hline


	\textbf{Udvidelser:}				& 	\\ \hline

\end{tabularx}
\caption{UC2: Aflæs fugtighed og temperatur}
\label{tbl:UC2}
\end{table}

% UC3:  Indstil ønsket fugtighed i jorden
\subsubsection{Use Case 3: Indstil ønsket fugtighed i jorden}
%-------------------- UC3 --------------------
\begin{table}[H]
\begin{tabularx}{\textwidth}{| L{3.3 cm} | Z |} \hline

\textbf{Navn:} 						 & UC3: Indstil ønsket fugtighed i jorden						\\ \hline
\textbf{Formål:}					 & Her gives brugeren mulighed for at instille den ønskede fugtighed i jorden 							\\ \hline
\textbf{Initiering:}				 & Bruger 										\\ \hline
\textbf{Aktører:} 					 & Bruger 										\\ \hline
\textbf{Forudsætning:} 				 & Kontrolboksen er tilsluttet forsyningsspænding og sensoren er tilsluttet \\ \hline
\textbf{Resultat:}					 & Systemet opretholder en bestemt fugtighed i jorden 	\\ \hline
\textbf{Hovedscenarie:}				 & \todo[inline]{Opdater i forhold til kode}

\begin{packed_enum}
	\item Brugeren trykker på SET knappen på taststuret 
	\item Der vises nu hvad systemet i forvejen er indstillet til
	\item Brugeren trykker nu enten på pil op eller pil ned for at stille fugtigheden op eller ned
	\item Brugeren trykker på OK
	\item Der vises nu temperatur og fugtighed på displayet, målt i jorden hvor sensoren er placeret
	
\end{packed_enum} 																\\ \hline

\textbf{Udvidelser:}				&  

%	\end{packed_enum}
																				\\ \hline

\end{tabularx}
\caption{UC3: Indstil ønsket fugtighed i jorden	}
\label{tbl:UC3}
\end{table}

% UC4:  Indstil åbningstid for ventil
\subsubsection{Use Case 4: Indstil åbningstid for ventil}
%-------------------- UC3 --------------------
\begin{table}[H]
\begin{tabularx}{\textwidth}{| L{3.3 cm} | Z |} \hline

\textbf{Navn:} 						 & UC4: Indstil åbningstid for ventil						\\ \hline
\textbf{Formål:}					 & Her gives brugeren mulighed for at instille åbningstiden for ventilen. Åbningstiden er nødvendig at indstille da kontrolboksen kun modtager data fra sensoren hvert 15. minut.  Åbningstiden kan indstilles fra 1 til 15 minutter  							\\ \hline
\textbf{Initiering:}				 & Bruger 										\\ \hline
\textbf{Aktører:} 					 & Bruger 										\\ \hline
\textbf{Forudsætning:} 				 & Kontrolboksen er tilsluttet forsyningsspænding og sensoren er tilsluttet \\ \hline
\textbf{Resultat:}					 & Åbningstiden for ventilen er sat 	\\ \hline
\textbf{Hovedscenarie:}				 & \todo[inline]{Opdater i forhold til kode}

\begin{packed_enum}
	\item Brugeren trykker 2 gange på SET knappen på taststuret 
	\item Der vises nu hvad åbningstiden i forvejen er indstillet til
	\item Brugeren trykker nu enten på pil op eller pil ned for at stille åbningstiden op eller ned
	\item Brugeren trykker på OK
	\item Der vises nu temperatur og fugtighed på displayet, målt i jorden hvor sensoren er placeret
\end{packed_enum} 																\\ \hline

\textbf{Udvidelser:}				&  

%	\end{packed_enum}
																				\\ \hline
\end{tabularx}
\caption{UC4: Indstil åbningstid for ventil	}
\label{tbl:UC4}
\end{table}

% UC5:Indstil tidsbaseret vandingsinterval
\subsubsection{Use Case 5: Indstil tidsbaseret vandingsinterval}
%-------------------- UC5 --------------------
\begin{table}[H]
\begin{tabularx}{\textwidth}{| L{3.3 cm} | Z |} \hline

\textbf{Navn:} 						 & UC5: Indstil tidsbaseret vandingsinterval						\\ \hline
\textbf{Formål:}					 & Her gives brugeren mulighed for at indstille et tidsbaseret vandingsinterval 							\\ \hline
\textbf{Initiering:}				 & Bruger 										\\ \hline
\textbf{Aktører:} 					 & Bruger 										\\ \hline
\textbf{Forudsætning:} 				 & Kontrolboksen er tilsluttet forsyningsspænding \\ \hline
\textbf{Resultat:}					 & Der er indstillet et tidsbaseret vandingsinterval 	\\ \hline
\textbf{Hovedscenarie:}				 & 

\begin{packed_enum}
	\item Brugeren trykker 3 gange på \texttt{SET} på tastaturet
	\item Der vises nu hvad åbningstiden i forvejen er indstillet til
	\item Brugeren trykker nu enten på pil op eller pil ned for at stille åbningstiden op eller ned. Denne kan indstilles i et interval på 0 til 60 minutter. 
	\item Brugeren trykker på \texttt{SET} på tastaturet
	\item Brugeren trykker nu enten på pil op eller pil ned for at stille lukketiden op eller ned. Denne kan indstilles i et interval på 0 til 23 timer.
	\item Brugeren trykker på \texttt{OK} på tastaturet
	\item Der vises nu temperatur og fugtighed på displayet, målt i jorden hvor sensoren er placeret
	
\end{packed_enum} 																\\ \hline

\textbf{Udvidelser:}				&  

%	\end{packed_enum}
																				\\ \hline
\end{tabularx}
\caption{UC5: Indstil tidsbaseret vandingsinterval}
\label{tbl:UC5}
\end{table}

% UC6:Åbn/luk for ventil
\subsubsection{Use Case 6: Åbn/luk ventil manuelt}
%-------------------- UC6 --------------------
\begin{table}[H]
\begin{tabularx}{\textwidth}{| L{3.3 cm} | Z |} \hline

\textbf{Navn:} 						 & UC6: Åbn/luk ventil manuelt						\\ \hline
\textbf{Formål:}					 & Her gives brugeren mulighed for manuelt at åbne eller lukke ventilen  \\ \hline
\textbf{Initiering:}				 & Bruger 										\\ \hline
\textbf{Aktører:} 					 & Bruger 										\\ \hline
\textbf{Forudsætning:} 				 & Kontrolboksen er tilsluttet forsyningsspænding \\ \hline
\textbf{Resultat:}					 & Ventilen er enten åbnet eller lukket 	\\ \hline
\textbf{Hovedscenarie:}				 & 

\begin{packed_enum}
	\item Brugeren trykker på \texttt{Pil-op} eller \texttt{Pil-ned} på tastaturet
	\item Teksten "Open valve?" forekommer hvis der er trykket \texttt{Pil-op}. Er der trykket \texttt{Pil-ned} forekommer teksten "Close valve?"
	\item Brugeren trykker på \texttt{OK} på tastaturet 
	\item Ventilen åbner eller lukker afhængig af brugerens valg i punkt 1.
		\begin{packed_item}\itemsep1pt \parskip0pt \parsep0pt
		\item {[}Ext 1 : Ventilen er allerede i den ønskede position{]}
		\end{packed_item}
	
	
\end{packed_enum} 																\\ \hline

\textbf{Udvidelser:}				&  
\textbf{{[}Ext 1 : Ventilen er allerede i den ønskede position{]}}
	\begin{packed_enum}\itemsep1pt \parskip0pt \parsep0pt
		\item Teksten "Already open" eller "Already closed" vises på displayet.				
	\end{packed_enum}

%	\end{packed_enum}
																				\\ \hline
\end{tabularx}
\caption{UC6: Åbn/luk ventil manuelt}
\label{tbl:UC6}
\end{table}

% UC7:Vælg automatisk vandingstidspunkt 
\begin{longtable}{| l | >{\raggedright}X | >{\raggedright}X | >{\raggedright}X | >{\raggedright\arraybackslash}p{2.3cm} |} \hline
	\multicolumn{2}{|l|}{\textbf{Use case under test}}  & \multicolumn{3}{l|}{UC7: Indstil vandingstidspunkt} \\ \hline
	\multicolumn{2}{|l|}{\textbf{Scenarie}} 			& \multicolumn{3}{l|}{Hovedscenarie} \\ \hline
	\multicolumn{2}{|l|}{\textbf{Forudsætning}} 		& \multicolumn{3}{p{10.2cm}|}{UC1 og UC8 er gennemført samt sensoren er placeret i tør luft og dagslys. Fugtigheden er indstillet til 10\% i UC3. Åbningstiden er indstillet til 1 minut i UC4\hfill} \\ \hline
	%\multicolumn{5}{|l|}{}\\ \hline
	\textbf{Step} 	& \textbf{Handling} & \textbf{Forventet Resultat} & \textbf{Resultat} & \textbf{Godkendt / Kommentar} \\ \hline
	
	1   & Der trykkes på \texttt{SET} 5 gange
		& Teksten "Watering period" fremkommer på displayes linje 1. På linje 2 vises hvad indstillingen i forvejen er sat til
		& resultat
		& resultat\\ \hline
		
	2   & Der trykkes på enten \texttt{Pil-op} eller \texttt{Pil-ned} for at vælge \texttt{Evening}
		& Vandingstidspunktet er sat til \texttt{Evening}
		& resultat
		& resultat\\ \hline
		
	3   & Der trykkes på \texttt{OK}
		& Der returneres til hovedmenuen
		& resultat
		& resultat\\ \hline
		
	4   & Sensoren ligges i en papkasse således intet lys kan komme igennem
		& Sensoren ligger i en papkasse hvor intet lys kommer igennem
		& resultat
		& resultat\\ \hline
		
	5   & Der ventes i op til 15 min
		& Ventilen åbner
		& resultat
		& resultat\\ \hline
		
	6   & Der ventes i 1 min
		& Ventilen lukker
		& resultat
		& resultat\\ \hline	

	7   & Der trykkes på \texttt{SET} 5 gange
		& Teksten "Watering period" fremkommer på displayes linje 1. På linje 2 vises "Time:Evening"
		& resultat
		& resultat\\ \hline
		
	8   & Der trykkes på \texttt{Pil-op} for at vælge \texttt{Morning}
		& Vandingstidspunktet er sat til \texttt{Morning}
		& resultat
		& resultat\\ \hline
		
	9   & Der trykkes på \texttt{OK}
		& Der returneres til hovedmenuen
		& resultat
		& resultat\\ \hline
		
	10  & Sensoren tages ud af papkassen og ligges i dagslys
		& Sensoren ligger i dagslys
		& resultat
		& resultat\\ \hline
		
	11  & Der ventes i op til 15 min
		& Ventilen åbner
		& resultat
		& resultat\\ \hline		

	12  & Der ventes i 1 min
		& Ventilen lukker
		& resultat
		& resultat\\ \hline	
		
	13  & Der trykkes på \texttt{SET} 5 gange
		& Teksten "Watering period" fremkommer på displayes linje 1. På linje 2 vises "Time:Morning"
		& resultat
		& resultat\\ \hline
		
	14  & Der trykkes på \texttt{Pil-ned} to gange for at vælge \texttt{Both}
		& Vandingstidspunktet er sat til \texttt{Both}
		& resultat
		& resultat\\ \hline
	
	15  & Der trykkes på \texttt{OK}
		& Der returneres til hovedmenuen
		& resultat
		& resultat\\ \hline	
		
	16  & Sensoren ligges i en papkasse således intet lys kan komme igennem
		& Sensoren ligger i en papkasse hvor intet lys kommer igennem
		& resultat
		& resultat\\ \hline
		
	17  & Der ventes i op til 15 min
		& Ventilen åbner
		& resultat
		& resultat\\ \hline
		
	18  & Der ventes i 1 min
		& Ventilen lukker
		& resultat
		& resultat\\ \hline	

	19  & Der ventes i op til 15 min
		& intet sker
		& resultat
		& resultat\\ \hline		
		
	20  & Sensoren tages ud af papkassen og ligges i dagslys
		& Sensoren ligger i dagslys
		& resultat
		& resultat\\ \hline
		
	21  & Der ventes i op til 15 min
		& Ventilen åbner
		& resultat
		& resultat\\ \hline		

	22  & Der ventes i 1 min
		& Ventilen lukker
		& resultat
		& resultat\\ \hline	
				
	
\caption{Accepttest for UC7: Brems Bil }\label{tbl:acceptuc7}
\end{longtable}

\clearpage