\subsubsection{Use Case 3: Indstil ønsket fugtighed i jorden}
%-------------------- UC3 --------------------
\begin{table}[H]
\begin{tabularx}{\textwidth}{| L{3.3 cm} | Z |} \hline

\textbf{Navn:} 						 & UC3: Indstil ønsket fugtighed i jorden						\\ \hline
\textbf{Formål:}					 & Her gives brugeren mulighed for at instille den ønskede fugtighed i jorden 							\\ \hline
\textbf{Initiering:}				 & Bruger 										\\ \hline
\textbf{Aktører:} 					 & Bruger 										\\ \hline
\textbf{Forudsætning:} 				 & Kontrolboksen er tilsluttet forsyningsspænding og sensoren er tilsluttet batteri, samt parret via use-case 8\\ \hline
\textbf{Resultat:}					 & Systemet opretholder en bestemt fugtighed i jorden 	\\ \hline
\textbf{Hovedscenarie:}				 & 

\begin{packed_enum}
	\item Brugeren trykker på \texttt{SET} knappen på taststuret 
	\item Der vises nu hvad systemet i forvejen er indstillet til
	\item Brugeren trykker nu enten på pil op eller pil ned for at stille fugtigheden op eller ned. Denne kan indstilles fra 0 - 50\%
	\item Brugeren trykker på \texttt{OK}
	\item Der vises nu temperatur og fugtighed på displayet, målt i jorden hvor sensoren er placeret
	
\end{packed_enum} 																\\ \hline

\textbf{Udvidelser:}				&  

%	\end{packed_enum}
																				\\ \hline

\end{tabularx}
\caption{UC3: Indstil ønsket fugtighed i jorden	}
\label{tbl:UC3}
\end{table}