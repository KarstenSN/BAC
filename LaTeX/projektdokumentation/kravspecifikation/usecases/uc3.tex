\subsubsection{Use Case 3: Indstil ønsket fugtighed}
%-------------------- UC3 --------------------
\begin{table}[H]
\begin{tabularx}{\textwidth}{| L{3.3 cm} | Z |} \hline

\textbf{Navn:} 						 & UC3: Indstil ønsket fugtighed												\\ \hline
\textbf{Formål:}					 & Her gives Brugeren mulighed for at indstille den ønskede fugtighed i jorden 	\\ \hline
\textbf{Initiering:}				 & Bruger 																		\\ \hline
\textbf{Aktører:} 					 & Bruger 																		\\ \hline
\textbf{Forudsætning:} 				 & Kontrolboksen er tilsluttet forsyningsspænding, sensoren er tilsluttet batteri, samt UC:8 (par sensor og kontrolboks) er kørt med succes \\ \hline
\textbf{Resultat:}					 & Systemet opretholder en bestemt fugtighed i jorden 							\\ \hline
\textbf{Hovedscenarie:}				 & 

\begin{packed_enum}
	\item Brugeren trykker på \texttt{SET} knappen på tastaturet 
	\item Systemets nuværende indstillinger vises nu på display'et
	\item Brugeren trykker nu enten på \texttt{Pil op} eller \texttt{Pil ned} for at stille fugtigheden op eller ned. Fugtigheden kan indstilles i intervallet fra 0 - 50\%
	\item Brugeren trykker på \texttt{OK}
	\item Den nuværende temperatur og fugtighed vises på displayet
	
\end{packed_enum} 																\\ \hline

\textbf{Udvidelser:}				&  

%	\end{packed_enum}
																				\\ \hline
\end{tabularx}
\caption{UC3: Indstil ønsket fugtighed}
\label{tbl:UC3}
\end{table}