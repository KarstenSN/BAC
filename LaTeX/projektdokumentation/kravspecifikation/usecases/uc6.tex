\subsubsection{Use Case 6: Åbn/luk ventil manuelt}
%-------------------- UC6 --------------------
\begin{table}[H]
\begin{tabularx}{\textwidth}{| L{3.3 cm} | Z |} \hline

\textbf{Navn:} 						 & UC6: Åbn/luk ventil manuelt						\\ \hline
\textbf{Formål:}					 & Her gives brugeren mulighed for manuelt at åbne eller lukke ventilen  \\ \hline
\textbf{Initiering:}				 & Bruger 										\\ \hline
\textbf{Aktører:} 					 & Bruger 										\\ \hline
\textbf{Forudsætning:} 				 & Kontrolboksen er tilsluttet forsyningsspænding \\ \hline
\textbf{Resultat:}					 & Ventilen er enten åbnet eller lukket 	\\ \hline
\textbf{Hovedscenarie:}				 & 

\begin{packed_enum}
	\item Brugeren trykker på \texttt{Pil-op} eller \texttt{Pil-ned} på tastaturet
	\item Teksten "Open valve?" forekommer hvis der er trykket \texttt{Pil-op}. Er der trykket \texttt{Pil-ned} forekommer teksten "Close valve?"
	\item Brugeren trykker på \texttt{OK} på tastaturet 
	\item Ventilen åbner eller lukker afhængig af brugerens valg i punkt 1.
		\begin{packed_item}\itemsep1pt \parskip0pt \parsep0pt
		\item {[}Ext 1 : Ventilen er allerede i den ønskede position{]}
		\end{packed_item}
	
	
\end{packed_enum} 																\\ \hline

\textbf{Udvidelser:}				&  
\textbf{{[}Ext 1 : Ventilen er allerede i den ønskede position{]}}
	\begin{packed_enum}\itemsep1pt \parskip0pt \parsep0pt
		\item Teksten "Already open" eller "Already closed" vises på displayet.				
	\end{packed_enum}

%	\end{packed_enum}
																				\\ \hline
\end{tabularx}
\caption{UC6: Åbn/luk ventil manuelt}
\label{tbl:UC6}
\end{table}