\subsubsection{Use Case 1: Aktiver system}
%-------------------- UC1 --------------------
\begin{table}[h]
\begin{tabularx}{\textwidth}{| L{3.3 cm} | Z |} \hline

\textbf{Navn:} 						 & UC1: Tænd/sluk system						\\ \hline
\textbf{Formål:}					 & Her gives brugeren mulighed for at tænde eller slukke for systemet.  Dette gøres ved at holde power knappen nede i 3 sekunder på tastaturet på kontrolboksen. Er systemet slukket vil systemet tænde og omvendt. Se Figur xx for skitse af tastatur 							\\ \hline
\textbf{Initiering:}				 & Bruger 										\\ \hline
\textbf{Aktører:} 					 & Bruger 										\\ \hline
\textbf{Forudsætning:} 				 & Kontrolboksen er tilsluttet forsyningsspænding \\ \hline
\textbf{Resultat:}					 & Systemetet er aktiveret 	\\ \hline
\textbf{Hovedscenarie:}				 & 

\begin{packed_enum}
	\item Brugeren holder power knappen nede i minimum 3 sekunder
	\item Der vises nu temperatur og fugtighed på displayet, målt i jorden hvor sensoren er placeret
		\begin{packed_item}\itemsep1pt \parskip0pt \parsep0pt
		\item {[}Ext 1 : Systemet er i forvejen tændt{]}
		\end{packed_item}
	
\end{packed_enum} 																\\ \hline

\textbf{Udvidelser:}				&  
\textbf{{[}Ext 1 : Systemet er i forvejen tændt{]}}
	\begin{packed_enum}\itemsep1pt \parskip0pt \parsep0pt
		\item Systemet slukker og teksten i displayet forsvinder				
	\end{packed_enum}
																				\\ \hline

\end{tabularx}
\caption{UC1: Aktiver system}
\label{tbl:UC1}
\end{table}