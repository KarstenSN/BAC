\subsubsection{Use Case 8: Par sensor og kontrolboks}
%-------------------- UC8 --------------------
\begin{table}[H]
\begin{tabularx}{\textwidth}{| L{3.3 cm} | Z |} \hline

\textbf{Navn:} 						 & UC8: Par sensor og kontrolboks															\\ \hline
\textbf{Formål:}					 & Her gives brugeren mulighed for at parre kontrolboks og sensor  							\\ \hline
\textbf{Initiering:}				 & Bruger 																					\\ \hline
\textbf{Aktører:} 					 & Bruger 																					\\ \hline
\textbf{Forudsætning:} 				 & Kontrolboksen er tilsluttet forsyningsspænding og sensor er tilsluttet batteri 			\\ \hline
\textbf{Resultat:}					 & Der er forbindelse mellem sensor og kontrolboks 											\\ \hline
\textbf{Hovedscenarie:}				 & 

\begin{packed_enum}
	\item Brugeren trykker på \texttt{SET} på tastaturet 6 gange 
	\item Teksten "\texttt{Connect sensor?}" vises nu på display'et
	\item Brugeren trykker på \texttt{OK} på tastaturet
	\item Kontrolboksen skaber forbindelse til sensoren
	\item Teksten "\texttt{Connection established}" vises på display'et.
		\begin{packed_item}\itemsep1pt \parskip0pt \parsep0pt
		\item {[}Ext 1 : Der kunne ikke oprettes forbindelse{]}
		\end{packed_item}
	\item Der returneres til hovedmenu og den nuværende temperatur og fugtighed vise på display'et
	
\end{packed_enum} 		\\ \hline

\textbf{Udvidelser:}	&  
\textbf{{[}Ext 1 : Der kunne ikke oprettes forbindelse{]}}
	\begin{packed_enum}\itemsep1pt \parskip0pt \parsep0pt
		\item Teksten "\texttt{Connection failed}" vises på display'et
		\item Der returneres til hovedmenu og den nuværende temperatur og fugtighed vise på display'et				
	\end{packed_enum}

%	\end{packed_enum}
						\\ \hline
\end{tabularx}
\caption{UC8: Par sensor og kontrolboks}
\label{tbl:UC8}
\end{table}