\section{Keypad} 
Hele systemet betjenes via en keypad som monteres som en del af kontrolboksen. Keypad'en består af 4 knapper som brugeren kan bruge til at navigere rundt i menuen. Se Figur \ref{fig:keypad}. \texttt{SET} knappen bruges til at skifte mellem indstillingerne men bruges samtidig også som tænd og sluk knap. For enten at tænde eller slukke systemet holdes knappen nede i minimum 3 sekunder. \texttt{Pil-op} og \texttt{Pil-ned} bruges til at stille en værdi enten op eller ned og \texttt{OK} knappen bruges til at gemme indstillingen. For fuld overblik over menuen henvises til usecasene i sektion \ref{ch:kravspecifikation}. På keypad'en er der også et vindue til displayet og en rød LED diode som lyser hvis systemet har registret en fejl. Fx. lav batteri eller ingen forbindelse til sensoren. Keypad'en skal fremstilles hos en professionel leverandør og en prototype vil derfor være dyr at fremstille. Der er derfor fundet en keypad på Ebay som har 4 knapper som vil blive brugt som prototype.

\begin{figure}[ht]
	\centering
	\includegraphics[width=0.8\textwidth]{../fig/Keypad/keypad.png}
	\caption{Skitse af keypad'en}
	\label{fig:keypad}
\end{figure} 

\begin{figure}[ht]
	\centering
	\includegraphics[width=0.8\textwidth]{../fig/Keypad/diagram.png}
	\caption{Diagram over keypad'en}
	\label{fig:keypad_diagram}
\end{figure} 