\section{Lysmåler} 
Låsmåleren er blot en fototransistor som vil give en strøm afhængig af hvor meget lys der indstråles. Der er valgt at bruge typen TEMT7000X01. Kurve over strøm vs. indstrålet effekt ses i Figur \ref{fig:fotokurve}.  

\begin{figure}[H]
  \centering
  \begin{minipage}[b]{0.4\textwidth}
	\includegraphics[width=1\textwidth]{../fig/Lysmaeler/fototrans.jpg}
	\caption{Fototransistor}
	\label{fig:fototrans}
  \end{minipage}
  \hfill
  \begin{minipage}[b]{0.4\textwidth}
	\includegraphics[width=1\textwidth]{../fig/Lysmaeler/kurve.jpg}
	\caption{Strøm vs. Indstråling}
	\label{fig:fotokurve}
  \end{minipage}
\end{figure}


\begin{figure}[H]
  \centering
  \begin{minipage}[b]{0.4\textwidth}
	\includegraphics[width=0.5\textwidth]{../fig/Lysmaeler/diagram.jpg}
	\caption{Diagram over lysmåleren}
	\label{fig:resplot}
  \end{minipage}
  \hfill
  \begin{minipage}[b]{0.4\textwidth}
	\includegraphics[width=1\textwidth]{../fig/Lysmaeler/wavelength.jpg}
	\caption{Fototransistorens sensitivitet vs. bølgelængde}
	\label{fig:resplot}
  \end{minipage}
\end{figure}

