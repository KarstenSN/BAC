\section{Lysmåler} 
Låsmåleren er blot en fototransistor som vil give en strøm afhængig af hvor meget lys der indstråles. Der er valgt at bruge typen TEMT7000X01. Kurve over strøm vs. indstrålet effekt ses i Figur \ref{fig:fotokurve}.  

\begin{figure}[H]
  \centering
  \begin{minipage}[b]{0.4\textwidth}
	\includegraphics[width=1\textwidth]{../fig/Lysmaeler/fototrans.jpg}
	\caption{Fototransistor}
	\label{fig:fototrans}
  \end{minipage}
  \hfill
  \begin{minipage}[b]{0.4\textwidth}
	\includegraphics[width=1\textwidth]{../fig/Lysmaeler/kurve.jpg}
	\caption{Strøm vs. Indstråling}
	\label{fig:fotokurve}
  \end{minipage}
\end{figure}

\begin{figure}[H]
  \centering
  \begin{minipage}[b]{0.4\textwidth}
	\includegraphics[width=0.5\textwidth]{../fig/Lysmaeler/diagram.jpg}
	\caption{Diagram over lysmåleren}
	\label{fig:lysdiagram}
  \end{minipage}
  \hfill
  \begin{minipage}[b]{0.4\textwidth}
	\includegraphics[width=1\textwidth]{../fig/Lysmaeler/wavelength.jpg}
	\caption{Fototransistorens sensitivitet vs. bølgelængde}
	\label{fig:lysspectrum}
  \end{minipage}
\end{figure}

\begin{figure}[H]
	\centering
	\includegraphics[width=0.6\textwidth]{../fig/Lysmaeler/solspektrum.jpg}
	\caption{Spektrum over energien i sollys. Kilde: \cite{lib:wiki_sunlight}}
	\label{fig:solspectrum}
\end{figure} 

I følge \cite{lib:wiki_sunlight} er definition på fuld solskind 120$W/M^2$ hvilket omregnet bliver til 12$mW/cm^2$. Men da fototransistoren ikke vil opfange lige meget energi i frekvensområdet fra solen, vil den tilsvarende strøm være mindre end hvad der vises i \ref{fig:fotokurve}. I Figur \ref{fig:solspectrum} ses der hvordan energien er fordelt i sollys, samlignes den med Figur \ref{fig:lysspectrum} ses der at der bliver omdannet mest energi til elektricitet, ved en bølgelængde på 850nM. Antages der at fotodioden skal give en spænding på 3.3V ved fuld solskind og at det kun er $1/4$ del af energien der bliver omdannet, aflæses der 1.7mA ved 2.5mW på Figur \ref{fig:fotokurve} og der vil være brug for en modstand i størrelsen:

\begin{equation}
	R13=\dfrac{3.3V} {1.7mA}=1.9k\Omega
\end{equation}

Dog ønsker vi at detektere solopgang og solnedgang hvor effekten er mindre. R13 bestemmes til at være 4.7$k\Omega$ da den tidligere er brugt på designet og at en test må vise hvad spændingen bliver ved sol op og nedgang.  