\newpage
\section{Antenne} \label{sec:antenne}
\subsection{Valg af transmissionsfrekvens}

Valget af transmissionsfrekvens afhænger af hvilken effekt der ønskes ved receiver-antennen og dermed hvor lang rækkevidde der kan opnås, 
dette sættes i forhold til den effekt transmitter-antennen fødes med. \\
Sammenhængen imellem disse parametre fås ved \textit{Friis transmissions formel} 
2-117 på side 95 i \cite{lib:balanis}

\begin{align} \label{eq:ant_friis}
	\frac{P_{r}}{P_{t}} = e_{t}e_{r} \frac{\lambda^{2}D_{t}(\theta_{t},\phi_{t})D_{r}(\theta_{r},\phi_{r})}{(4 \pi R)^{2}}
\end{align}

\begin{table}[H]
	\begin{tabularx}{\textwidth}{ m{1.5 cm}  m{8 cm}  Z } \\
		$ e_{r} $	& Radiation efficiency of receiver		&	\\
		$ e_{t} $	& Radiation efficiency of transmitter	&	\\
		$\lambda$ 	& Wavelength in Meters					&	\\
		$ P_{r} $ 	& Received Power in dBm 				&	\\
		$ P_{t} $	& Transmited Power in dBm				&	\\
		$ D_{t} $	& Directivity of Transmitter			&	\\
		$ D_{r} $	& Directivity of Receiver				&	\\
		$ R 	$ 	& Distance between Antennas in Meters	&	\\
	\end{tabularx}
	\caption{Friis parametre}
	\label{tbl:friis_param}
\end{table}

Følgende parametre kan opstilles for systemet, her antages det at udstrålingseffektiviteten for transmitter og receiver, 
samt $ e_{cd} $ til $ 0.9 $. Dette er optimistisk sat, men det er grundet et ønske om at få en målestok for antennens maksimale rækkevidde. 

\begin{table}[H]
	\begin{tabularx}{\textwidth}{ m{1.5 cm}  m{4 cm}  Z } 	\\										
		$ P_{t}	 $ 	& $ 5 \times 10^{-3} 		$	& Transmit Power in $ dBm $					\\
		$ D_{t}  $ 	& $ 1 						$	& Directivity of transmitter 				\\
		$ D_{r}  $ 	& $ 1 						$	& Directivity of receiver 					\\
		$ e_{r}  $	& $ 0.9 					$	& Radiation efficiency of receiver			\\
		$ e_{t}  $	& $ 0.9 					$	& Radiation efficiency of transmitter		\\
		$ e_{cd} $	& $ 0.9 					$	& Radiation efficiency						\\
		$ f  	 $	& $ 433						$	& Transmit frequency in $ Hz $				\\
		$ v_{0}  $	& $ 2.99 \times 10^8 m/s $		& Free-space velocity of light in $ m/s $	\\
		
	\end{tabularx}
	\caption{system parametre}
	\label{tbl:system_param}
\end{table}

Først omregnes transmissionsfrekvensen til bølgelængder: 

\begin{align} \label{eq:ant_lambda}
	\lambda = \frac{v_{0}}{f} \Rightarrow \frac{2.99 \times 10^8 m/s}{f} = 0,69053
\end{align}

Herefter benyttes $ \lambda $ til at beregne antennens maksimum effektiv areal $ A_{r} $.\\
Da transmitter og receiver er af samme type gælder denne parameter for begge antenner. 

\begin{align} \label{eq:ant_ar}
	A_{r} = e_{r} D_{r}(\theta_{r},\phi_{r}) \frac{\lambda^{2}}{4 \pi} \Rightarrow  0.9 \times 1 \times \frac{0,6905^{2}}{4 \pi} = 0.03415
\end{align}

Nu kendes faktoren for hvor meget at den udstrålede effekt der kan omsættes, og $ P_{r} $ kan nu beregnes ved at benytte formel 2-114 og 2-116 på side 95 i \cite{lib:balanis}

\begin{align} \label{eq:ant_Wt}
	W_{t} = e_{t} \frac{P_{t} D_{t}(\theta_{t},\phi_{t})}{4 \pi R^{2}}
\end{align}

Nu fås den modtagne effektiv som ligning \ref{eq:ant_Wt} ganget på ligning \ref{eq:ant_ar}

\begin{align} \label{eq:Pr}
	P_{r} = e_{t} \frac{P_{t} D_{t}(\theta_{t},\phi_{t})}{4 \pi R^{2}} \times e_{r} D_{r}(\theta_{r},\phi_{r}) \frac{\lambda^{2}}{4 \pi}
\end{align}

Denne ligning kan nu løses for $ R $ for at finde den maksimale rækkevidde for antennen.  


\newpage
%%% HERTIL %%%%









\subsection{Valg af antenne}

Til første iteration af kontrolboksen blev det valgt at benytte en through-hole monteret helical-antenne, dette valg blev truffet på baggrund af ønsket om hurtigt at komme i gang med en simple og effektiv løsning uden de store krav til plads og beregninger. En antenne af typen W3127 ISM 433MHz Helical Antenna fra \textit{Pulse Electronics}. Denne antenne er var den mest isotropiske af kandidaterne med en Max Gain på $ G_{max}=-2.9 dBi$. jvf. Fig 3 i databladet !!!!!! INDSÆT REF TIL LITTERATURLISTE / Bilag!!!!!

Efterfølgende blev det besluttet at finde en mere pladsvenlig løsning, da sensorboksen og PCB'et heri stiller nogle formfaktor krav der umuliggøre en sådan antenne. Der er ganske enkelt ikke plads til den på sensor-PCB'et og i sensorboksen.
Der blev derfor valgt at forsøge sig med beregninger på en patch-antenne med samme båndbredde.

\subsection{Patch-antenne}
For i første omgang at få et overblik over det fysiske omfang af en patchantenne ved 433Mhz, blev der foretaget en række beregninger. Beregninger og designmetode baserer sig på fremgangsmåden fra kapitel.14 om Microstrip Antennas i "Antenna Theory
Analisys and Design 3.ed" \cite{lib:balanis} . 

Patchantennen opbygges med "line feed" efter transmissionlinje-modellen som er den model der giver et hurtigt overblik over de fysiskeparametre, men dog er mindre præcis og er sværere at modellere kobling på. Men i første omgang ønske jo netop bare en formfaktor. 

\begin{figure}[H]
	\centering
	\includegraphics[width=0.6\textwidth]{../fig/antenner/Balanis_fig_14_1a_MicrostripAntenna.png}
	\caption{Koncept for en transmissionlinje-model af patchantenne}
	\label{fig:antenne1}
\end{figure} 

Ved denne model opnås udstråling fra 2 slots for enderne af patch'en og input-resistance kan kontrolleres ved at varierer reccessen. Men først fastlægges $w$ og $l$. 

\subsubsection{beregninger på patch-antenne}
Følgende er kendt på forhånd: 

\begin{table}[H]
	\begin{tabularx}{\textwidth}{| m{1.5 cm} | m{4 cm} | Z |} \hline 										
		$\varepsilon_{r}$ 	& $ 4.3 $ 							& Dielectric constant of the substrate (FR4)		 \\ \hline
		$ f_{r}$ 			& $ 466 \times 10^{6} Hz $ 			& Resonant frequency 								 \\ \hline
		$ h $				& $ 0.1588 \times 10^{-3}  m $ 		& Hight of the substrate							 \\ \hline
		$ v_{0} $			& $ 2.99 \times 10^8 m/s $			& Free-space velocity of light						 \\ \hline
		$ \mu_{0} $			& $ 4\pi \times 10^{-7} H/m$		& Free-space Permeability							 \\ \hline
		$ \varepsilon_{0}$ 	& $ 8.854 \times 10^{-12} F/m $ 	& Free-space Permittivity							 \\  \hline
	\end{tabularx}
	\caption{Info til beregninger}
	\label{tbl:Info_beregninger}
\end{table}

Først beregnes bredden $ W $ ved at benytte formel 14-6 i \cite[side.819]{lib:balanis}

\begin{align} \label{eq:ant_width}
		W &= \frac{1}{2fr\sqrt{\mu_{0} \varepsilon_{0}}} = \frac{v0}{2fr} \sqrt{\frac{2}{\varepsilon_{r}+1}} \\
		  &= \frac{30 \times 10^9 m/s}{2\times 466 \times 10^{6} Hz} \sqrt{\frac{2}{1+4.3}}= 0.2128 m (21.28cm)
\end{align}

Herefter findes den effektive permeabilitet for substratet. Da patch ligger på et yder lag, deles udstrålingen imellem luft på den ene side og substratet på den anden. Derfor kan substratets $ \varepsilon_{r} $ alene ikke benyttes, da der også skal tages højde for luftens indvirkning. $ \epsilon_{reff} $ findes ved at benytte formel 14-1 i \cite[side.817]{lib:balanis}

\begin{align} \label{eq:ant_ereff}
		\epsilon_{reff} &= \frac{er+1}{2}  + \frac{er-1}{2}  \bigg[ 1+12 \frac{h}{W} \bigg] ^{- \frac{1}{2}} \\
		  				&= \frac{4.3+1}{2} + \frac{4.3-1}{2} \bigg[ 1+12 \frac{0.1588}{21.28} \bigg] ^{- \frac{1}{2}} = 4.2305 
\end{align}

Ved denne type antenner opstår fænomenet \textit{fringing effect}, dette skyldes at feltet der står imellem GND og patch'en også udstråler som vist på \ref{fig:patch_sideview} på side \pageref{fig:patch_sideview}.
Dette får patch'ens længde til at virke større end der fysisk er, denne forlængelse betegnes $ \Delta L$ og ses på fig. \ref{fig:patch_topview} .

\begin{figure}[H]
\centering
	\begin{minipage}{0.5\textwidth}
	  \centering
	  	\includegraphics[width=0.7\linewidth]{../fig/antenner/Balanis_fig_14_7a_lengths.png}
	  	\caption{figure}{Patch-antenne: Top view}
	  	\label{fig:patch_topview}
	\end{minipage}%
	\begin{minipage}{0.5\textwidth}
  		\centering
  		\includegraphics[width=0.9\linewidth]{../fig/antenner/Balanis_fig_14_7b_lengths.png}
  		\caption{figure}{Patch-antenne: Side view}
  		\label{fig:patch_sideview}
	\end{minipage}
\end{figure}

For at beregne $ \Delta L$ benyttes formel 14-2 i \cite[side.818]{lib:balanis}

\begin{align} \label{eq:ant_delta_l}
		\Delta L &= \frac{( \epsilon_{reff} + 0.3) \bigg( \frac{W}{h} + 0.264 \bigg)}
						 {( \epsilon_{reff} - 0.258) \bigg( \frac{W}{h} + 0.8 \bigg)} \times 0.412 \\
				 &= \frac{( 4.2305 + 0.3) \bigg( \frac{21.28}{0.1588} + 0.264 \bigg)}
				 		 {( 4.2305 - 0.258) \bigg( \frac{21.28}{0.1588} + 0.8 \bigg)} \times 0.412 = 0.074318 cm
\end{align}

Med disse parametre på plads kan den aktuelle længde på antennen beregnes ved at benytte formel 14-3 i \cite[side.819]{lib:balanis}

\begin{align} \label{eq:ant_L}
		L &= \frac{1}{2fr\sqrt{\epsilon_{reff}}\sqrt{\mu_{0} \varepsilon_{0}}}-2 \Delta L \\
		  &= \frac{1}{2 \times 466 \times 10^{6} \sqrt{4\pi \times 10^{-7} 8.854 \times 10^{-12}}}-2\times 0.074318 cm = 16.638cm
\end{align}

Efter disse hurtige beregninger kan det konstateres at dimensionerne til patch-antennen vil blive $ W=21.28cm, L=16.638cm $, dette vil, ligesom helical-antennen overskride pladskravene i sensorboksen, derfor må der kigges efter en 3. mulighed.


\subsection{smd-antenne}



% beregninger omkring smd-antenne

% valg af smd-antenne